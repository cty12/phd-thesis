%\documentclass[usenames,dvipsnames,12pt,handout]{beamer}
\documentclass[usenames,dvipsnames,12pt]{beamer}

\usecolortheme{dove}
\usefonttheme{professionalfonts}
\usefonttheme{serif}

\usepackage{bm}
\usepackage{mathtools}
\usepackage{amsmath,amssymb,amsfonts,amsthm}
\usepackage{mathrsfs}
\usepackage{mathabx}
\usepackage{fontspec}
\usepackage{multicol}
\usepackage{xcolor}
\usepackage{pifont}
\usepackage{tabularx}
\usepackage{colortbl}
\usepackage{graphicx}
\usepackage{semantic}
\usepackage{xspace}
\usepackage[all]{xy}
\usepackage{listings}
\usepackage{lstautogobble}
\usepackage{fancybox}
\usepackage{stmaryrd}
\usepackage{rotating}
\usepackage{wasysym}
\usepackage{ulem}
\usepackage{soul}
\usepackage{newunicodechar}
\usepackage[super]{nth}
\usepackage{tikz, tikz-qtree, tikz-qtree-compat}
\usetikzlibrary{shapes, arrows, calc, quotes, tikzmark, decorations.pathreplacing, decorations.markings}
\usepackage{makecell}
\usepackage{epigraph}

\setlength{\multicolsep}{3pt}
\setlength{\columnseprule}{0.5pt}

\newcommand\mytikzmark[2]{\tikz[overlay,remember picture, anchor=base] \node (#1) {#2};}

% colored underline, with Beamer overlay support
% usage: \cul{x} or \cul[blue]{x} or \cul<2->{x} or \cul<2->[blue]{x}
\newcommand<>{\cul}[2][Red]{
  \fontdimen8\textfont3=0.75pt
  \alt#3
      {\color{#1}\underline{{\color{black}#2}}\color{black}}
      {\transparent{0.0}\underline{{\transparent{1.0}#2}}\transparent{1.0}}
}
\newcommand{\smalltt}[1]{{\small \texttt{#1}}}
\newcommand{\McL}{\mathscr{L}}
\newcommand{\yes}{\textcolor{Green}{\ding{51}}}
\newcommand{\no}{\textcolor{Red}{\ding{55}}}
\newcommand{\maybe}{\ding{82}}
\newcommand{\redtext}[1]{\textcolor{Maroon}{#1}}
\newcommand{\bluetext}[1]{\textcolor{NavyBlue}{#1}}
\newcommand{\orangetext}[1]{\textcolor{BurntOrange}{#1}}
\newcommand{\greentext}[1]{\textcolor{PineGreen}{#1}}
\newcommand{\graytext}[1]{\textcolor{gray}{#1}}
\newcommand{\purpletext}[1]{\textcolor{Plum}{#1}}
\newcommand{\bl}[1]{\ensuremath{\orangetext{#1}}}
\newcommand{\key}[1]{\ensuremath{\mathtt{#1}}}
\newcommand{\MID}{\;\mid\;}
\newcommand{\Surface}{\ensuremath{\lambda_{\mathtt{IFC}}^\star}\xspace}
\newcommand{\SurfaceOld}{\ensuremath{\lambda_{\mathtt{SEC}}^\star}\xspace}
\newcommand{\CCOld}{\ensuremath{\lambda_{\mathtt{SEC}}^{\Rightarrow}}\xspace}
\newcommand{\CC}{\ensuremath{\lambda_{\mathtt{IFC}}^{c}}\xspace}
\newcommand{\GSLRef}{\ensuremath{\mathrm{GSL}_\mathsf{Ref}}\xspace}
\newcommand{\GSLRefEps}{\ensuremath{\mathrm{GSL}_\mathsf{Ref}^\epsilon}\xspace}
\newcommand{\SSLRef}{\ensuremath{\mathrm{SSL}_\mathsf{Ref}}\xspace}
\newcommand{\lamgif}{\ensuremath{\mathit{\lambda_{gif}}}\xspace}
\newcommand{\WHILEG}{WHILE\textsuperscript{G}\xspace}
\newcommand{\high}{\textcolor{OrangeRed}{\key{high}}\xspace}
\newcommand{\low}{\textcolor{PineGreen}{\key{low}}\xspace}
\newcommand{\unk}{\textcolor{Maroon}{\key{\star}}\xspace}
\newcommand{\Bool}{\key{Bool}}
\newcommand{\Int}{\key{Int}}
\newcommand{\Unit}{\key{Unit}}
\newcommand{\Fun}[3]{\ensuremath{{#1}\xrightarrow{{#2}}{#3}}}
\newcommand{\Refer}[1]{\ensuremath{\key{Ref}\;{#1}}}
\newcommand{\true}{\key{true}}
\newcommand{\false}{\key{false}}
\newcommand{\unit}{\key{unit}}
\newcommand{\pc}{\ensuremath{\mathit{pc}}\xspace}
\newcommand{\PC}{\ensuremath{\mathit{PC}}\xspace}
\newcommand{\gc}{\ensuremath{\mathit{gc}}\xspace}
\newcommand{\syntax}[1]{\text{\texttt{\textcolor{Purple}{#1}}}}
\newcommand{\ccsyntax}[1]{\text{\texttt{{#1}}}}
\newcommand{\const}[2]{\ensuremath{\syntax{(\$}\;{#1}\syntax{)}_{#2}}}
\newcommand{\lam}[5]{\ensuremath{\syntax{(λ}^{#1}{#2}\syntax{:}{#3}\syntax{.}\,{#4}\syntax{)}_{#5}}}
\newcommand{\app}[3]{\ensuremath{\syntax{(}{#1}\;{#2}\syntax{)}^{\bl{#3}}}}
\newcommand{\ifexp}[4]{\ensuremath{\syntax{(if}\;{#1}\;\syntax{then}\;{#2}\;\syntax{else}\;{#3}\syntax{)}^{\bl{#4}}}}
\newcommand{\refexp}[3]{\ensuremath{\syntax{(ref}\;{#1}\;{#2}\syntax{)}^{\bl{#3}}}}
\newcommand{\deref}[1]{\ensuremath{\syntax{!}\;{#1}}}
\newcommand{\assign}[3]{\ensuremath{\syntax{(}{#1}\;\syntax{:=}\;{#2}\syntax{)}^{\bl{#3}}}}
\newcommand{\ann}[3]{\ensuremath{\syntax{(}{#1}\;\syntax{:}\;{#2}\syntax{)}^{\bl{#3}}}}
\newcommand{\letexp}[3]{\ensuremath{\syntax{let}\;{#1}={#2}\;\syntax{in}\;{#3}}}
\newcommand{\ccconst}[1]{\ensuremath{\ccsyntax{\$}\;{#1}}}
\newcommand{\ccaddr}[1]{\ensuremath{\ccsyntax{addr}\;{#1}}}
\newcommand{\cclam}[2]{\ensuremath{\ccsyntax{λ}{#1}\ccsyntax{.}\,{#2}}}
\newcommand{\ccif}[5]{\ensuremath{\ccsyntax{if}\;{#1}\;{#2}\;{#3}\;{#4}\;{#5}}}
\newcommand{\ccifstar}[4]{\ensuremath{\ccsyntax{if}{\star}\,{#1}\,{#2}\,{#3}\,{#4}}}
\newcommand{\ccref}[2]{\ensuremath{\ccsyntax{ref}\;{#1}\;{#2}}}
\newcommand{\ccrefproj}[3]{\ensuremath{\ccsyntax{ref?}^{\bl{#3}}\;{#1}\;{#2}}}
\newcommand{\ccderef}[3]{\ensuremath{\ccsyntax{!}\;{#1}\;{#2}\;{#3}}}
\newcommand{\ccderefstar}[2]{\ensuremath{\ccsyntax{!}{\star}\,{#1}\,{#2}}}
\newcommand{\ccassign}[5]{\ensuremath{\ccsyntax{assign}\;{#1}\;{#2}\;{#3}\;{#4}\;{#5}}}
\newcommand{\ccassignproj}[5]{\ensuremath{\ccsyntax{assign?}^{\bl{#5}}\;{#1}\;{#2}\;{#3}\;{#4}}}
\newcommand{\cccast}[2]{\ensuremath{{#1}\,\ccsyntax{⟨}\,{#2}\,\ccsyntax{⟩}}}
\newcommand{\ccprot}[4]{\ensuremath{\ccsyntax{prot}\;{#1}\;{#2}\;{#3}\;{#4}}}
\newcommand{\cccastpc}[2]{\ensuremath{\ccsyntax{cast}_{\ccsyntax{pc}}\;{#1}\;{#2}}}
\newcommand{\cclet}[4]{\ensuremath{\ccsyntax{let}\;{#1}\ccsyntax{=}{#2}\ccsyntax{:}{#3}\;\ccsyntax{in}\;{#4}}}
\newcommand{\ccapp}[5]{\ensuremath{\ccsyntax{app}\;{#1}\;{#2}\;{#3}\;{#4}\;{#5}}}
\newcommand{\ccappstar}[4]{\ensuremath{\ccsyntax{app}{\star}\,{#1}\,{#2}\,{#3}\,{#4}}}
\newcommand{\ccopaque}{\ensuremath{\ccsyntax{●}}}
\newcommand{\lconsisjoin}{\ensuremath{\,\widetilde{\curlyvee}\,}}  % label consistent join
\newcommand{\consisjoin}{\ensuremath{\,\widetilde{\vee}\,}}        % type consistent join
\newcommand{\lconsismeet}{\ensuremath{\,\widetilde{\curlywedge}\,}}  % label consistent meet
\newcommand{\consismeet}{\ensuremath{\,\widetilde{\wedge}\,}}        % type consistent meet
\newcommand{\Cast}[3]{\ensuremath{{#1}\Rightarrow^{\bl{#3}}{#2}}}
\newcommand{\Type}{\textit{Type}}
\newcommand{\RawType}{\textit{RawType}}
\newcommand{\Label}{\textit{Label}}
\newcommand{\compile}[1]{\ensuremath{\mathcal{C}\;{#1}}}
\newcommand{\applycast}[3]{\ensuremath{\mathbf{Cast}\;{#1}\,,\,{#2}\leadsto{#3}}}
\newcommand{\blame}[1]{\ensuremath{\key{blame}\;\bl{#1}}}
\newcommand{\reduce}[5]{\ensuremath{{#1}\mid{#2}\mid{#3}\longrightarrow{#4}\mid{#5}}}
\newcommand{\Active}[1]{\ensuremath{\mathbf{Active}\;{#1}}}
\newcommand{\Inert}[1]{\ensuremath{\mathbf{Inert}\;{#1}}}
\newcommand{\bigstep}[5]{\ensuremath{{#1}\mid{#2}\vdash{#3}\Downarrow{#4}\mid{#5}}}
\newcommand{\bigstepe}[5]{\ensuremath{{#1}\mid{#2}\vdash{#3}\Downarrow_\epsilon{#4}\mid{#5}}}
%% coercions
\newcommand{\id}[1]{\ensuremath{\mathit{\bold{id}}(#1)}}
\newcommand{\up}{\ensuremath{\boldsymbol{\uparrow}}}
\newcommand{\inj}[1]{\ensuremath{{#1}\,\boldsymbol{!}}}
\newcommand{\seq}{\ensuremath{\,\boldsymbol{;}\,}}
\newcommand{\err}[3]{\ensuremath{\boldsymbol{\bot}^{\bl{#3}}\;{#1}\;{#2}}}
\newcommand{\proj}[2]{\ensuremath{{#1}\,\boldsymbol{?}^{\bl{#2}}}}
\newcommand{\coerc}[2]{\ensuremath{{#1}\boldsymbol{,}\,{#2}}}
\newcommand{\refco}[2]{\ensuremath{\mathbf{Ref}\;{#1}\;{#2}}}
\newcommand{\funco}[3]{\ensuremath{{#1}\boldsymbol{,}\,{#2}\boldsymbol{\rightarrow}{#3}}}
\newcommand{\precctx}[8]{\ensuremath{{#1};{#2};{#3};{#4};{#5};{#6};{#7};{#8}}}
\newcommand{\ccprec}[4]{\ensuremath{\vdash{#1}\sqsubseteq{#2}\Leftarrow{#3}\sqsubseteq{#4}}}

\newcommand{\highlight}[2]{\colorbox{#1}{\ensuremath{#2}}}
\newcommand{\highlightblue}[1]{\highlight{White!90!NavyBlue}{#1}}
\newcommand{\highlightred}[1]{\highlight{White!90!Maroon}{#1}}

\definecolor{highlight}{gray}{0.9}

\newunicodechar{ℒ}{\ensuremath{\McL}}
\newunicodechar{𝕋}{\ensuremath{\mathbb{T}}}
\newunicodechar{𝕊}{\ensuremath{\mathbb{S}}}

\newtheorem{conjecture}[theorem]{\translate{Conjecture}}
%\newtheorem{theorem}{Theorem}
%\newtheorem{lemma}[theorem]{Lemma}
%\newtheorem{corollary}[theorem]{Corollary}
\newtheorem{proposition}[theorem]{Proposition}
\newtheorem{constraint}[theorem]{Constraint}
%\newtheorem{definition}[theorem]{Definition}
%\newtheorem{example}[theorem]{Example}

%% Configurations:
\setmainfont{EB Garamond}
\setmonofont[Scale=0.9]{Iosevka}
\setbeamertemplate{navigation symbols}{}
\setbeamertemplate{footline}[frame number]
\setbeamerfont{footnote}{size=\tiny}
\setlength{\multicolsep}{3pt}
\setbeamertemplate{itemize item}{$\blacktriangleright$}
\setbeamertemplate{itemize subitem}{$\circ$}
\setbeamertemplate{itemize/enumerate subbody begin}{\footnotesize}

\lstdefinestyle{mystyle}{
  commentstyle=\color{Green},
  keywordstyle=\color{Magenta},
  numberstyle=\tiny\color{Gray},
  stringstyle=\color{Purple},
  basicstyle=\ttfamily\footnotesize,
  breakatwhitespace=false,
  breaklines=true,
  captionpos=b,
  keepspaces=true,
  numbers=left,
  numbersep=5pt,
  showspaces=false,
  showstringspaces=false,
  showtabs=false,
  tabsize=2
}

\lstset{style=mystyle}

\setlength{\epigraphwidth}{\textwidth}


\title{The Holy Grail of Gradual Security \\
  {\small\textcolor{NavyBlue}{Ph.D Thesis Proposal Presentation}} \vspace{-15pt}}
\author{Tianyu Chen}
\institute{\small Computer Science, Indiana University}
\date{}

\begin{document}

%%%%%%%%%% Cover Page %%%%%%%%%%%
\begin{frame}
  \titlepage
  \begin{tikzpicture}[remember picture,overlay]
    \node [anchor=north east] at ([xshift=-15pt, yshift=5pt]current page.north east)
          {\includegraphics[height=30pt]{iu_tab_web}};
  \end{tikzpicture}

  \vspace{-45pt}

  \begin{center}
    \includegraphics[width=4in]{acanthus}
  \end{center}

  \footnotetext{Acanthus textiles. William Morris Gallery}
\end{frame}

%%%%%%%%%% Road Map %%%%%%%%%%%
\begin{frame}
  \frametitle{Road Map}

  \begin{itemize}
  \item[\textcolor{Red}{\ding{43}}] \cul{Background:}
    \begin{itemize}
    \item Explicit flow and implicit flow
    \item Information flow control: static, dynamic, gradual
    \item The tension between noninterference and the gradual guarantee
    \end{itemize}
  \item \Surface in Action
    \begin{itemize}
    \item[\ding{79}] Solving the Tension Between Noninterference and the Gradual Guarantee
    \item Type-Based Reasoning in \Surface
    \end{itemize}
  \item Coercion-based Semantics for Gradual Security
  \item Meta-theoretical Results of \Surface
  \end{itemize}
\end{frame}

%%%%%%%%%% Explicit Information Flow %%%%%%%%%%%
\begin{frame}[fragile]
  \frametitle{Explicit Information Flow}

  Can we infer output from input in the following program?

\begin{verbatim}
let input = private-input () in publish (¬ input)
\end{verbatim}

\onslide<2-> {
  \begin{itemize}
  \item[\yes] Yes!
  \item Witness at least two executions
  \item Output is the negation of input
  \item \purpletext{Explicit flow}
  \end{itemize}
}

\end{frame}

%%%%%%%%%% Implicit Information Flow %%%%%%%%%%%
\begin{frame}[fragile]
  \frametitle{Implicit Information Flow}

  Can we infer output from input in the following program?
\begin{verbatim}
  let input = private-input () in
      publish (if input then false else true)
\end{verbatim}

\onslide<2-> {
\begin{itemize}
  \item[\yes] Also yes
  \item Again, output is the negation of input
  \item \purpletext{Implicit flow}: input influences output through {\large\textit{branching}}
\end{itemize}
}

\end{frame}

%%%%%%%%%% IFC %%%%%%%%%%%
\begin{frame}
  \frametitle{Information-Flow Control (IFC)}

  \begin{itemize}
  \item Ensures that information transfers adhere to a security policy
    \item For example, \high input must not flow to \low output
    \item Propagate and check the security labels
    \item IFC in PL
      \(
      \left\{
      \begin{minipage}[c]{0.75\linewidth}
      \item[] \bluetext{\Large static} using a type system
      \item[] \redtext{\Large dynamic} using runtime monitoring
      \end{minipage}
      \right.
      \)
  \end{itemize}
\end{frame}

%%%%%%%%%% Static IFC %%%%%%%%%%%
\begin{frame}[containsverbatim]
  \frametitle{\bluetext{Static IFC} Allows \textcolor{Green}{Legal} Explicit Flows}

  \begin{lstlisting}[mathescape,autogobble,numbers=none]
    private-input : $\Unit_{\low}$ → $\Bool_{\high}$  publish : $\Bool_{\low}$ → $\Unit_{\low}$
  \end{lstlisting}
  \vspace{10pt}

  Statically-typed program:
  \begin{lstlisting}[mathescape,autogobble,xleftmargin=.15\textwidth,xrightmargin=.15\textwidth]
    let fconst = λ b : $\Bool_{\high}$. false in
    let input  = private-input () in
    let result = fconst input in
      publish result
  \end{lstlisting}

  \begin{itemize}
  \item[\yes] \textcolor{Green}{Well-typed} \qquad~and~\qquad
    \yes \; \textcolor{Green}{Runs successfully} to \texttt{unit}
  \item Why? The return value of \texttt{fconst} is
    \(
    \left\{
    \begin{minipage}[c]{0.3\linewidth}
    \item[] always \texttt{false}
    \item[] of \low-security
    \end{minipage}
    \right.
    \)
  \item Accepted by type-checker. No runtime check
  \end{itemize}
\end{frame}


\begin{frame}[containsverbatim]
  \frametitle{\bluetext{Static IFC} Guards Against \textcolor{Red}{Illegal} Explicit Flows}

  Replace \texttt{fconst} with \texttt{fid} that takes a low-security Bool:

  \begin{lstlisting}[mathescape,autogobble,xleftmargin=.1\textwidth]
    let fid    = λ b : $\highlightblue{\Bool_{\low}}$. b in
    let input  = private-input () in
    let result = fid input in  $\graytext{\text{// compilation error}}$
      publish result
  \end{lstlisting}
  \vspace{15pt}

  \begin{itemize}
  \item[\no] \textcolor{Red}{Ill-typed.} Illegal explicit flow:
  \begin{itemize}
  \item input is \high
  \item \texttt{fid} expects \low argument
  \end{itemize}
  \item Rejected by type-checker. Again no runtime check
  \end{itemize}
\end{frame}

\begin{frame}[containsverbatim]
  \frametitle{Guarding Against Illegal \ul{Implicit Flows}, \bluetext{Statically}}

  Different observable behaviors in different branches:

  \begin{lstlisting}[mathescape,autogobble,xleftmargin=.1\textwidth,xrightmargin=.1\textwidth,frame=lines]
    let flip : $\Bool_{\high}$ → $\Bool_{\low}$ =
      λ b : $\Bool_{\high}$. $\highlightblue{\text{if b then false else true}}$ in
    let input  = private-input () in
    let result = flip input in
      publish result
  \end{lstlisting}

  \begin{itemize}
  \item[\no] \textcolor{Red}{Ill-typed}
  \item Security label on the type of \texttt{if} is the \ul{join} of its branches (both \low)
    and the branch condition (\high).
  \begin{itemize}
  \footnotesize
  \item Expected: \low \;\;\quad (from type annotation $\Bool_{\low}$)
  \item Actual: \quad \high \quad (because of conditional)
  \end{itemize}
  \item[\ding{82}] $\high \npreccurlyeq \low$, thus rejected by type-checker.
    Illegal implicit flow ruled out \bluetext{at compile time}
  \end{itemize}

\end{frame}

\begin{frame}[containsverbatim]
  \frametitle{Guarding Against Illegal Explicit Flows, \redtext{Dynamically}}

  Consider the following dynamically-typed \texttt{fid} example that could
  potentially leak information through explicit flow:

  \begin{lstlisting}[mathescape,autogobble,xleftmargin=.15\textwidth,xrightmargin=.15\textwidth,frame=lines]
    let fid    = λ b : $\highlightred{\Bool_{\unk}}$. b in
    let input  = private-input () in
    let result = fid input in    $\graytext{\text{// error}}$
      publish result
  \end{lstlisting}

  \begin{itemize}
  \item[\yes] \textcolor{Green}{Well-typed} \qquad \textit{but} \qquad
    \no \; \textcolor{Red}{Fails at runtime}
  \item The program errors regardless of input
  \item A runtime happens before the call to \texttt{publish} and checks
    whether \high can flow to \low (of course, no)
  \item[\ding{82}] Illegal explicit flow ruled out \redtext{at runtime}
  \end{itemize}

  \footnotetext{We annotate $\Bool_{\unk}$ explicitly, which conforms with the syntax of \Surface}

\end{frame}


%% \begin{frame}[containsverbatim]
%%   \frametitle{\textit{Coercions} as Runtime IFC Monitor}

%%   The program is compiled to an IR, \CC.
%%   Call \CC a \textit{cast calculus}, because all casts,
%%   represented using \textit{coercions}, are made explicit:

%%   \begin{lstlisting}[mathescape,autogobble,xleftmargin=.15\textwidth,xrightmargin=.15\textwidth,frame=lines]
%%     let fid    = λ b. b in
%%     let input  = private-input () in
%%     let result = fid (input $\highlight{highlight}{\texttt{⟨\,\inj{\high}\,⟩}}$) in
%%       publish (result $\highlight{highlight}{\texttt{⟨\,\proj{\low}{p}\,⟩}}$)
%%   \end{lstlisting}

%%   Running the program with input \texttt{true} (running with \texttt{false} is similar)
%%   results in the following reduction sequence of \CC:

%%   \vspace{-15pt}
%%   {\footnotesize
%%   \begin{align}
%%     \longrightarrow^{*} &
%%     \quad \key{publish} \; (\cccast{\cccast{\key{true}}{\inj{\high}}}{\proj{\low}{p}}) \\[1ex]
%%     \longrightarrow^{*} &
%%     \quad \key{publish} \; (\cccast{\key{true}}{\inj{\high} \seq \proj{\low}{p}}) \\[1ex]
%%     \longrightarrow^{*} &
%%     \quad \blame{p}
%%   \end{align}}

%%   \vspace{-5pt}
%%   The coercion sequence $\inj{\high} \seq \proj{\low}{p}$ captures an explicit flow
%%   from \high through \unk to \low, which is prohibited and triggers a blame.

%% \end{frame}


\begin{frame}[containsverbatim]
  \frametitle{Against Illegal Implicit Flows, \redtext{Dynamically}}

  Consider the following dynamically-typed \texttt{flip} example that
  could potentially leak information through implicit flow:

  \begin{lstlisting}[mathescape,autogobble,xleftmargin=.1\textwidth,xrightmargin=.1\textwidth,frame=lines]
    let flip : $\highlightred{\Bool_{\unk}}$ → $\highlightred{\Bool_{\unk}}$ =
      λ b : $\highlightred{\Bool_{\unk}}$. if b then false else true in
    let input  = private-input () in
    let result = flip input in
      publish result
  \end{lstlisting}

  \begin{itemize}
  \item[\yes] \textcolor{Green}{Well-typed} \qquad \textit{but} \qquad
    \no \; \textcolor{Red}{Fails at runtime}
  \item The program (again) errors regardless of input
  \item \texttt{flip} produces a \high value because of \high branch condition
  \item A runtime happens before the call to \texttt{publish} and checks
    whether \high can flow to \low (of course, no)
  \item[\ding{82}] Illegal implicit flow ruled out \redtext{at runtime}
  \end{itemize}

\end{frame}


%% \begin{frame}[containsverbatim]
%%   \frametitle{\textit{Stamping} on Coercions Enforces Implicit Flow}

%%   \begin{lstlisting}[mathescape,autogobble,xleftmargin=.1\textwidth,xrightmargin=.1\textwidth,frame=lines]
%%     let flip = λ b . if b then (false $\highlight{highlight}{\texttt{⟨\,\inj{\low}\,⟩}}$)
%%     else (true $\highlight{highlight}{\texttt{⟨\,\inj{\low}\,⟩}}$) in
%%     let input  = private-input () in
%%     let result = flip (input $\highlight{highlight}{\texttt{⟨\,\inj{\high}\,⟩}}$) in
%%     publish (result $\highlight{highlight}{\texttt{⟨\,\proj{\low}{p}\,⟩}}$)
%%   \end{lstlisting}

%%   \vspace{-12pt}
%%   {\footnotesize
%%   \begin{align}
%%     \longrightarrow^{*}
%%     \begin{split}
%%       &\; \texttt{let result = ((λ b. if b then (\cccast{\false}{\inj{\low}}) else ...)} \\[-3pt]
%%       &\; \texttt{\qquad\qquad\qquad\quad (\cccast{\true}{\inj{\high}})) in} \\[-3pt]
%%       &\; \texttt{\qquad \key{publish}\;(\cccast{\key{result}}{\proj{\low}{p}})}
%%     \end{split} \\[1ex]
%%     \longrightarrow^{*}
%%     \begin{split}
%%       &\; \texttt{let result = prot \low (if (\cccast{\true}{\highlight{highlight}{\inj{\high}}})} \\[-3pt]
%%       &\; \texttt{\qquad\qquad\qquad\qquad\qquad\qquad then (\cccast{\false}{\inj{\low}}) else ...) in} \\[-3pt]
%%       &\; \texttt{\qquad \key{publish}\;(\cccast{\key{result}}{\proj{\low}{p}})}
%%     \end{split} \\[1ex]
%%     \longrightarrow^{*}
%%     \begin{split}
%%       &\; \texttt{let result = prot \low (prot \highlight{highlight}{\high} (\cccast{\false}{\inj{\low}})) in} \\[-3pt]
%%       &\; \texttt{\qquad \key{publish}\;(\cccast{\key{result}}{\proj{\low}{p}})}
%%     \end{split} \\[1ex]
%%     \longrightarrow^{*}
%%     \begin{split}
%%       &\; \texttt{let result = prot \low (\cccast{\false}{\highlight{highlight}{\up \seq \inj{\high}}}) in} \\[-3pt]
%%       &\; \texttt{\qquad \key{publish}\;(\cccast{\key{result}}{\proj{\low}{p}})}
%%     \end{split} \\[1ex]
%%     \longrightarrow^{*} &
%%     \; \texttt{publish (\cccast{\cccast{\false}{\up \seq \inj{\high}}}{\proj{\low}{p}})} \\[1ex]
%%     \longrightarrow^{*} &
%%     \; \texttt{publish (\cccast{\false}{\up \seq \highlightred{\inj{\high} \seq \proj{\low}{p}}})} \\[1ex]
%%     \longrightarrow^{*} &
%%     \; \texttt{\blame{p}}
%%   \end{align}}

%%   \footnotetext{The information flow violation is detected before the call to \texttt{publish}.}

%% \end{frame}


%% \begin{frame}
%%   \frametitle{Noninterference: the Meta-theory of Information-Flow Security}

%%   Noninterference: information is only allowed to flow between
%%   entities in a system that are associated with labels related by
%%   the flow relation

%%   \begin{itemize}
%%   \item Noninterference is a generalization over various information-
%%     flow properties
%%   \item Conversely, an information-flow property is an instantiation of noninterference
%%     with a particular system, set of labels, and flow relation
%%   \end{itemize}

%%   \footnotetext{Elisavet Kozyri, Stephen Chong and Andrew C. Myers. 2022.
%%     Expressing Information Flow Properties}
%% \end{frame}

%% \begin{frame}
%%   \frametitle{Security Policies as Information-Flow Properties}

%%   Computer security can be expressed as information-flow properties
%%   by choosing appropriate labels and flow relations: \\
%%   \begin{itemize}
%%   \item[+] \ul{Confidentiality:}
%%     $\{\greentext{\key{Public}}, \orangetext{\key{Secret}}\}$,
%%     $\begin{cases}
%%      \greentext{\key{Public}} \preccurlyeq \orangetext{\key{Secret}} \\
%%      \orangetext{\key{Secret}} \npreccurlyeq \greentext{\key{Public}}
%%      \end{cases}$
%%   \item[+] \ul{Integrity:}
%%     $\{\greentext{\key{Trusted}}, \orangetext{\key{Untrusted}}\}$,
%%     $\begin{cases}
%%      \greentext{\key{Trusted}} \preccurlyeq \orangetext{\key{Untrusted}} \\
%%      \orangetext{\key{Untrusted}} \npreccurlyeq \greentext{\key{Trusted}}
%%      \end{cases}$
%%   \item[+] \ul{Availability:}
%%     $\{\greentext{\key{HighAvailability}}, \orangetext{\key{LowAvailability}}\}$,
%%     $\begin{cases}
%%      \greentext{\key{HighAvailability}} \preccurlyeq \orangetext{\key{LowAvailability}} \\
%%      \orangetext{\key{LowAvailability}} \npreccurlyeq \greentext{\key{HighAvailability}}
%%      \end{cases}$
%%   \end{itemize}

%%   For simplicity, we consider $\{\low, \high \}$ in this talk.

%%   \footnotetext{Zella G. Ruthberg and Robert G. McKenzie. 1977.
%%     Audit and Evaluation of Computer Security}
%%   \footnotetext{Peng Li, Yun Mao and Steve Zdancewic. 2003. Information Integrity Policies}

%% \end{frame}


\begin{frame}[containsverbatim]
  \frametitle{\purpletext{Gradual Typing} Bridges Static and Dynamic IFC}

  Consider the following \purpletext{partially annotated} version of \texttt{flip}.
  The return value must be \low, because we intend to output the result:

  \begin{lstlisting}[mathescape,autogobble,xleftmargin=.1\textwidth,xrightmargin=.1\textwidth,frame=lines]
    let flip : $\highlightred{\Bool_{\unk}}$ → $\highlightblue{\Bool_{\low}}$ =
      λ b : $\highlightred{\Bool_{\unk}}$. if b then false else true in
    let input  = private-input () in
    let result = flip input in
      publish result
  \end{lstlisting}

  \begin{itemize}
  \item[\yes] \textcolor{Green}{Well-typed} \quad \textit{but} \quad
    \no \; \textcolor{Red}{Fails at runtime} {\scriptsize (for both \texttt{true} and \texttt{false})}
    \begin{itemize}
      \item thus preventing the leak through implicit flow
    \end{itemize}
  \item The information flow violation is detected \textit{ealier} than the dynamic
    version, as \texttt{flip} returns
  \item[\ding{82}] Checking happens on the boundaries between statically- and
      dynamically-typed code fragments
  \end{itemize}

\end{frame}


\begin{frame}
  \frametitle{\purpletext{The Gradual Guarantee}}
  \begin{itemize}
    \item[\bluetext{\ding{82}}] \bluetext{{\large Removing} annotations} from a correctly running program:
      \begin{itemize}
      \item[] Example: {\small $(42 : \low : \high) \sqsupseteq (42 : \unk : \high) \sqsupseteq (42 : \unk : \unk)$}
      \item[$\rightarrow$] ... results in the same runtime behavior ($42$)
      \end{itemize}
    \vspace{15pt}
    \item[\redtext{\ding{82}}] \redtext{{\large Adding} annotations} may introduce more errors:
      \begin{itemize}
      \item[] Example: {\small $(42 : \unk : \unk : \unk) \sqsubseteq (42 : \high : \unk : \low)$}
      \item {\small $(42 : \unk : \unk : \unk) \Downarrow 42$} but
        {\small $(42 : \high : \unk : \low) \Downarrow \key{error}$ }
      \end{itemize}
  \end{itemize}
\end{frame}

\begin{frame}
  \frametitle{Satisfying Noninterference and the Gradual Guarantee in One Programming Language}

  ... is {\Large \redtext{hard}} according to the literature:
  \vspace{-10pt}

  \epigraph{``We believe that there might be an \redtext{\large inherent incompatibility}
    between the strictness required to enforce a hyper-property like noninterference,
    and the optimistic flexibility dictated by the dynamic gradual guarantee.''}
           {\scriptsize Matías Toro, Ronald Garcia, and Éric Tanter. 2018.
             Type-Driven Gradual Security with References}
  \vspace{-15pt}
  \epigraph{``There is some recent evidence that the dynamic gradual guarantee –
    which some see as essential to gradual typing – is \redtext{\large incompatible} with
    various hyperproperties, like noninterference and parametricity.''}
           {\scriptsize Michael Greenberg. 2019.
             The Dynamic Practice and Static Theory of Gradual Typing}

\end{frame}


\begin{frame}[containsverbatim]
  \frametitle{Review: No-Sensitive-Upgrade Checking}

  \begin{itemize}
  \item No-sensitive-upgrade (NSU) {\scriptsize (Austin and Flanagan 2009)} prevents
    \ul{implicit flow} leaks through \ul{writes} to mutable references
  \item For gradual typing, NSU happens at \redtext{\large runtime}, when type information
    is insufficient in deciding if a heap write is secure
  \item Program that potentially leaks information through the heap:
  \end{itemize}

  \begin{lstlisting}[mathescape,autogobble,xleftmargin=.1\textwidth,xrightmargin=.1\textwidth,frame=lines]
  let input : $\Bool_{\unk}$ = private-input () in
  let a     = ref $\low$ true in
    if input then a := false else a := true ;
    publish (! a)
  \end{lstlisting}

  \begin{itemize}
  \item[\yes] \textcolor{Green}{Well-typed} \quad \textit{but} \quad
    \no \; \textcolor{Red}{Fails at runtime} {\scriptsize (for both \texttt{true} and \texttt{false})}
  \item NSU checking \redtext{\large terminates} this program, because it attempts to write to a
    \low memory location under a \high execution context (PC), thus preventing
    the leak through heap
  \end{itemize}
\end{frame}

\begin{frame}[containsverbatim]
  \frametitle{The Tension (in a Nutshell)}

  Toro et al. [2018] discover a tension between noninterference and the gradual
  guarantee in their language design, \GSLRef.

  Counterexample of the gradual guarantee in \GSLRef:

  \begin{multicols}{2}
    \noindent
    \redtext{\textbf{Left:} less precise, more dynamic}
    \begin{lstlisting}[mathescape,autogobble]
      let x = private-input () in
      let y = ref $\Bool_{\unk}$ $\true_{\unk}$ in
        if x then (y := $\false_{\high}$)
             else ()
    \end{lstlisting}
    \columnbreak
    \bluetext{\textbf{Right:} more precise, more static}
    \begin{lstlisting}[mathescape,autogobble,numbers=none]
      let x = private-input () in
      let y = ref $\Bool_{\high}$ $\true_{\high}$ in
        if x then (y := $\false_{\high}$)
             else ()
    \end{lstlisting}
  \end{multicols}

  \begin{itemize}
  \item[\yes] \textcolor{Green}{Both are well-typed}
  \item[\yes] The \bluetext{more precise (Right)} program runs \textcolor{Green}{\large successfully} to \texttt{unit}
  \item[\no]  The \redtext{less precise (Left)} program \textcolor{Red}{\large errors}!
    \begin{itemize}
    \footnotesize
    \item In \GSLRef, \unk corresponds to the interval $[\low, \high]$
    \end{itemize}
  \item[\ding{54}] {\large Violates \purpletext{the gradual guarantee}!}
  \end{itemize}
\end{frame}


\begin{frame}
  \frametitle{Possible Sources of the Tension}

\begin{table}[tbp]
  \scriptsize
  \centering
  \begin{tabularx}{\textwidth}{X|c|c|c|c|c}
  \hline
  \thead{Lang.} & \thead{Noninter-\\ference} & \thead{Gradual\\Guarantee} &
  \thead{Type-guided \\ classification} & \thead{NSU} & \thead{Runtime \\ security labels} \\
  \hline
  \GSLRef    & \mytikzmark{a}{\yes}  & \cellcolor{Red!10} \mytikzmark{b}{\no} & \yes  & \yes & $\{ \low, \high, \unk \}$ \\[1ex]
  \hline
  GLIO      & \yes & \cellcolor{Green!10} \mytikzmark{c}{\yes} & \mytikzmark{d}{\no}  & \yes & $\{ \low, \high \}$ \\[1ex]
  \hline
  \WHILEG & \yes & \cellcolor{Green!10} \mytikzmark{e}{\yes} & \yes   & \mytikzmark{f}{\no} & $\{ \low, \high, \unk \}$ \\[1ex]
  \hline
  \rowcolor{highlight}
  \purpletext{\Surface (ours)} & \yes & \cellcolor{Green!10} \mytikzmark{g}{\yes} & \yes & \yes & \mytikzmark{h}{$\{ \low, \high \}$} \\[1ex]
  \hline
  \end{tabularx}
\begin{tikzpicture}[overlay, remember picture, yshift=.25\baselineskip, shorten >=.5pt, shorten <=.5pt]
  \draw[dashed,thick] (a) to [bend right=15] (b);
  \draw[dashed,thick] (c) to [bend right=15] (d);
  \draw[dashed,thick] (e) to [bend right=10] (f);
  \draw[thick]        (g) to [bend right=8 ] (h);
\end{tikzpicture}
  \label{tab:cc-features}
\end{table}
\end{frame}


\begin{frame}
  \frametitle{Road Map}

  \begin{itemize}
  \item \textcolor{gray}{Background}
    %% \begin{itemize}
    %% \item Information flow properties and noninterference
    %% \item Information flow control: static, dynamic, gradual
    %% \item The gradual guarantee
    %% \item The tension between noninterference and the gradual guarantee
    %% \end{itemize}
  \item[\textcolor{Red}{\ding{43}}] \cul{\Surface in Action:}
    \begin{itemize}
    \item[\ding{79}] Solving the Tension Between Noninterference and the Gradual Guarantee
    \item Type-Based Reasoning in \Surface
    \end{itemize}
  \item Coercion-based Semantics for Gradual Security
  \item Meta-theoretical Results of \Surface
  \end{itemize}
\end{frame}


\begin{frame}[fragile]
  \frametitle{Solution to the Tension, in \Surface}

  \begin{multicols}{2}
    \noindent
    \redtext{\textbf{Left:} less precise, more dynamic}
    \begin{lstlisting}[mathescape,autogobble]
      let x = private-input () in
      let y : $\highlightred{(\Refer{\Bool_{\unk}})_{\unk}}$ =
          ref $\high$ $\true_{\high}$ in
        if x then (y := $\false_{\high}$)
             else ()
    \end{lstlisting}
    \columnbreak
    \bluetext{\textbf{Right:} more precise, more static}
    \begin{lstlisting}[mathescape,autogobble,numbers=none]
      let x = private-input () in
      let y : $\highlightblue{(\Refer{\Bool_{\high}})_{\high}}$ =
          ref $\high$ $\true_{\high}$ in
      if x then (y := $\false_{\high}$)
           else ()
    \end{lstlisting}
  \end{multicols}

  \begin{itemize}

  \item[\yes] \textcolor{Green}{Both are well-typed}
  \item[\yes] The \bluetext{more precise (Right)} program runs \textcolor{Green}{\large successfully} to \texttt{unit}
  \item[\yes] The \redtext{less precise (Left)} one also runs \textcolor{Green}{\large successfully} to \texttt{unit}
  \item[\greentext{\ding{82}}] Does \purpletext{\textit{\Large not}} violate \purpletext{the gradual guarantee}!
  \onslide<2-> {
  \item[] Problem solved!
  }
  \onslide<3-> {
  \item[] {\Large But why?}
  }
  \end{itemize}


  \onslide<2> {
  \begin{tikzpicture}[remember picture,overlay]
    \node [anchor=south east] at ([xshift=-30pt, yshift=10pt]current page.south east)
          {\includegraphics[height=90pt]{yay}};
  \end{tikzpicture}
  }
\end{frame}


\begin{frame}[containsverbatim]

  \begin{multicols}{2}
    \noindent
    \redtext{Less precise in \GSLRef:}
    \begin{lstlisting}[mathescape,autogobble]
      let x = private-input () in
      let y = ref $\highlightred{\Bool_{\unk}}$ $\highlightred{\true_{\unk}}$ in
        if x then (y := $\false_{\high}$)
             else ()
    \end{lstlisting}
    \redtext{Less precision in \Surface:}
    \begin{lstlisting}[mathescape,autogobble]
      let x = private-input () in
      let y : $\highlightred{(\Refer{\Bool_{\unk}})_{\unk}}$ =
          ref $\highlight{highlight}{\high}$ $\highlight{highlight}{\true_{\high}}$ in
        if x then (y := $\false_{\high}$)
             else ()
    \end{lstlisting}
    \columnbreak
    \bluetext{More precise in \GSLRef:}
    \begin{lstlisting}[mathescape,autogobble,numbers=none,breaklines=false]
      let x = private-input () in
      let y = ref $\highlightblue{\Bool_{\high}}$ $\highlightblue{\true_{\high}}$ in
        if x then (y := $\false_{\high}$)
             else ()
    \end{lstlisting}
    \bluetext{More precise in \Surface:}
    \begin{lstlisting}[mathescape,autogobble,numbers=none]
      let x = private-input () in
      let y : $\highlightblue{(\Refer{\Bool_{\high}})_{\high}}$ =
          ref $\highlight{highlight}{\high}$ $\highlight{highlight}{\true_{\high}}$ in
        if x then (y := $\false_{\high}$)
             else ()
    \end{lstlisting}
  \end{multicols}

  In \Surface,
  Security labels on type annotations can be \highlightblue{\bluetext{\text{\large specific}}} or
  \highlightred{\unk}, but those on literals and memory locations
  \highlight{highlight}{\text{\large stay specific}}.

\end{frame}


\begin{frame}[containsverbatim]

  Omitted security label annotations on literals default to \low:

  \begin{multicols}{2}
    \noindent
    \text{Less precise in \GSLRef:}
    \begin{lstlisting}[mathescape,autogobble]
      let x = private-input () in
      let y = ref $\Bool_{\unk}$ $\true_{\unk}$ in
        if x then (y := $\false_{\high}$)
             else ()
    \end{lstlisting}
    \columnbreak
    \purpletext{Less precision in \Surface:}
    \begin{lstlisting}[mathescape,autogobble,numbers=none]
      let x = private-input () in
      let y : $(\Refer{\Bool_{\unk}})_{\unk}$ =
          ref $\high$ true in
        if x then (y := false)
             else ()
    \end{lstlisting}
  \end{multicols}

\end{frame}


\begin{frame}
  \frametitle{Solving the Tension in \Surface (Summary)}

  Design choices of \GSLRef:
  \begin{itemize}
    \item Security labels on both types and literals can be \unk
    \item Runtime security labels can also be \unk (due to \unk on literals)
    \item Runtime has to ``guess'' conservatively
    \begin{itemize}
    \footnotesize
    \item[$\rightarrow$] more runtime errors when moving toward less precise
    \item[$\rightarrow$] violates the gradual guarantee!
    \end{itemize}
  \end{itemize}
  \onslide<2-> {
  % cross out
  \begin{tikzpicture}[remember picture, overlay]
    \draw[ultra thick] (0,0) -- (11,3) (11,0) -- (0,3);
  \end{tikzpicture}

  \purpletext{Design choices of \Surface:}
  \begin{itemize}
  \item Security labels on \purpletext{\large type annotations} may decrease in precision
    \vspace{-10pt}
    {\footnotesize
     \[
     \highlightred{(\Refer{\Bool_{\unk}})_{\unk}} \sqsubseteq \highlightblue{(\Refer{\Bool_{\high}})_{\high}}
     \]}
    {\footnotesize
    \begin{itemize}
      \item NSU checking happens. Heap IFC policy enforced at runtime
    \end{itemize}}
  \item Labels on {\large literals} and {\large memory locations} remain specific
    \begin{itemize}
      {\footnotesize
      \item security of data: only the programmer knows;
        must not be inferred
      \item[$\rightarrow$] runtime security levels remain specific during
        program execution
      }
    \end{itemize}
  \end{itemize}
  }

\end{frame}


\begin{frame}[containsverbatim]
  \frametitle{Security Coercions as Runtime IFC Monitor}

  Revisit the dynamically-typed \Surface program:
  \begin{lstlisting}[mathescape,autogobble,xleftmargin=.1\textwidth,frame=lines]
    let flip : $\highlightred{\Bool_{\unk}}$ → $\highlightred{\Bool_{\unk}}$ =
        λ b : $\highlightred{\Bool_{\unk}}$. if b then false else true in
    let input  = private-input () in
    let result = flip input in
      publish result
  \end{lstlisting}

  Compile the \Surface program to the following cast calculus \CC term,
  by making $\highlight{highlight}{\text{all casts}}$ explicit:

  \begin{lstlisting}[mathescape,autogobble,xleftmargin=.1\textwidth,frame=lines]
    let flip = λ b . if b then (false $\highlight{highlight}{\texttt{⟨\,\inj{\low}\,⟩}}$)
                          else (true $\highlight{highlight}{\texttt{⟨\,\inj{\low}\,⟩}}$) in
    let input  = private-input () in
    let result = flip (input $\highlight{highlight}{\texttt{⟨\,\inj{\high}\,⟩}}$) in
      publish (result $\highlight{highlight}{\texttt{⟨\,\proj{\low}{p}\,⟩}}$)
  \end{lstlisting}

\end{frame}

\begin{frame}

  Reducing the \CC term blames the projection {\footnotesize (before calling \texttt{publish})}:

  \vspace{-12pt}
  {\footnotesize
  \begin{align}
    \longrightarrow^{*}
    \begin{split}
      &\; \texttt{let result = ((λ b. if b then (\cccast{\false}{\inj{\low}}) else ...)} \\[-3pt]
      &\; \texttt{\qquad\qquad\qquad\quad (\cccast{\true}{\inj{\high}})) in} \\[-3pt]
      &\; \texttt{\qquad \key{publish}\;(\cccast{\key{result}}{\proj{\low}{p}})}
    \end{split} \\[1ex]
    \longrightarrow^{*}
    \begin{split}
      &\; \texttt{let result = prot \low (if (\cccast{\true}{\highlight{highlight}{\inj{\high}}})} \\[-3pt]
      &\; \texttt{\qquad\qquad\qquad\qquad\qquad\qquad then (\cccast{\false}{\inj{\low}}) else ...) in} \\[-3pt]
      &\; \texttt{\qquad \key{publish}\;(\cccast{\key{result}}{\proj{\low}{p}})}
    \end{split} \\[1ex]
    \longrightarrow^{*}
    \begin{split}
      &\; \texttt{let result = prot \low (prot \highlightblue{\high} (\cccast{\false}{\inj{\low}})) in} \\[-3pt]
      &\; \texttt{\qquad \key{publish}\;(\cccast{\key{result}}{\proj{\low}{p}})}
    \end{split} \\[1ex]
    \longrightarrow^{*}
    \begin{split}
      &\; \texttt{let result = prot \low (\cccast{\false}{\highlight{highlight}{\up \seq \inj{\high}}}) in} \\[-3pt]
      &\; \texttt{\qquad \key{publish}\;(\cccast{\key{result}}{\proj{\low}{p}})}
    \end{split} \\[1ex]
    %% \longrightarrow^{*} &
    %% \; \texttt{publish (\cccast{\cccast{\false}{\up \seq \inj{\high}}}{\proj{\low}{p}})} \\[1ex]
    \longrightarrow^{*} &
    \; \texttt{publish (\cccast{\false}{\up \seq \highlightred{\inj{\high} \seq \proj{\low}{p}}})} \\[1ex]
    \longrightarrow^{*} &
    \; \texttt{\blame{p}}
  \end{align}}

  \purpletext{
    Sequencing models explicit flow. Stamping models implicit flow.
    Checking by reducing coercion sequences
  }
\end{frame}


\begin{frame}
  \frametitle{Type-Based Reasoning in \Surface}

  \begin{itemize}
  \item Type-based reasoning: Toro et al. [2018] observe that security
    typing induces ``free theorems'' about noninterference
  \item Type-based reasoning is the synergy of two design choices:
    \begin{enumerate}
    \item Vigilance
    \item Type-Guided Classification
    \end{enumerate}
  \item GLIO {\scriptsize (Azevedo de Amorim et al. 2020)} satisfies
    the gradual guarantee by sacrificing type-guide classification,
    which they claim to be the reason \GSLRef {\scriptsize (Toro et al. 2018)}
    violates the gradual guarantee
  \item \Surface supports type-based reasoning just like \GSLRef
  \end{itemize}
\end{frame}


\begin{frame}[containsverbatim]
  \frametitle{Vigilance: Type-Based Reasoning for Explicit Flows}

  Consider the example from Toro et al. [2018]:

  \begin{lstlisting}[mathescape,autogobble,frame=lines]
    let mix : $\Int_{\low}$ → $\Int_{\high}$ → $\Int_{\low}$ =
        λ pub priv .
          if pub < (priv : $\Int_{\unk}$ : $\Int_{\low}$) then 1 else 2 in
      mix 1$_{\low}$ 5$_{\low}$
  \end{lstlisting}

  \ul{\Large Free theorem:} Either \textcircled{1} the \low result of \texttt{mix}
  never depends on the \high \texttt{priv} argument or
  \redtext{\textcircled{2} \texttt{mix} produces a runtime error}.

  \bigskip

  {\footnotesize
  (GLIO: not vigilant $\rightarrow$ does not produce an error
  $\rightarrow$ violates the free theorem)}

  \bigskip

  \purpletext{In \Surface,} $\cccast{\key{5}}{\up \seq \inj{\high} \seq \proj{\low}{p}} \Downarrow \blame{p}$
\end{frame}


\begin{frame}[containsverbatim]
  \frametitle{Type-Guided Classification: \\ Type-Based Reasoning for Implicit Flows}

  Another example from Toro et al. [2018]:
  \begin{lstlisting}[mathescape,autogobble,frame=lines]
    let mix : $\Int_{\low}$ → $\Int_{\unk}$ → $\Int_{\low}$ =
        λ pub priv. if pub < priv then 1 else 2 in
    let smix : $\Int_{\low}$ → $\Int_{\high}$ → $\Int_{\low}$ =
        λ pub priv. mix pub priv in
      smix 1$_{\low}$ 5$_{\low}$
  \end{lstlisting}

  \ul{\Large Free theorem:} The \texttt{smix} function either
  \textcircled{1} returns a value that does not depend on
  \texttt{priv} or \redtext{\textcircled{2}
    produces a runtime error}

  \bigskip

  {\footnotesize
    (GLIO: \textcircled{1} not vigilant \textcircled{2} does not classify values using types
    $\rightarrow$ does not produce an error
    $\rightarrow$ violates the free theorem)}

\end{frame}


\begin{frame}[containsverbatim]

  \begin{lstlisting}[mathescape,autogobble,frame=lines]
    let mix = λ pub priv.
      (if (pub ⟨$\,\inj{\low}\,$⟩) < priv
          then (1 ⟨$\,\inj{\low}\,$⟩)
          else (2 ⟨$\,\inj{\low}\,$⟩)) ⟨$\,\proj{\low}{p}\,$⟩ in
    let smix = λ pub priv. mix pub (priv ⟨$\,\inj{\high}\,$⟩) in
      smix 1 (5 ⟨$\,\up\,$⟩)
  \end{lstlisting}

  {\scriptsize
    \begin{align}
      \longrightarrow^{*} &
      \quad \cccast{\texttt{(if (\cccast{\key{1}}{\inj{\low}} < \cccast{\key{5}}{\up\seq\inj{\high}}) then \cccast{\key{1}}{\inj{\low}}{} else ...)}}{\proj{\low}{p}} \\[1ex]
      \longrightarrow^{*} &
      \quad \cccast{\texttt{(if \colorbox{highlight}{(\cccast{\true}{\up\seq\inj{\high}})} then \cccast{\key{1}}{\inj{\low}}{} else ...)}}{\proj{\low}{p}} \\[1ex]
      \longrightarrow^{*} &
      \quad \cccast{\ccsyntax{(}\ccprot{}{\highlightred{\high}}{\ccsyntax{(}\cccast{\key{1}}{\inj{\low}}}{}\ccsyntax{)}\ccsyntax{)}}{\proj{\low}{p}} \\[1ex]
      \longrightarrow^{*} &
      \quad \cccast{\cccast{\key{1}}{\up\seq\inj{\high}}}{\proj{\low}{p}} \\
      \longrightarrow^{*} &
      \quad \blame{p}
  \end{align}}

  \purpletext{In \Surface, the program errors, thus satisfying the free theorem}

\end{frame}


\begin{frame}
  \frametitle{Road Map}

  \begin{itemize}
  \item \textcolor{gray}{Background}
  \item \textcolor{gray}{\Surface in Action}
  \item[\textcolor{Red}{\ding{43}}] \cul{Coercion-based Semantics for Gradual Security}
  \item Meta-theoretical Results of \Surface
  \end{itemize}
\end{frame}


\begin{frame}
  \frametitle{Coercion Calculus for Security Labels}

  Syntax and typing for security coercions and
  coercion sequences:

  \vspace{-20pt}
  \begin{figure}[tbp]
  \raggedright
  \[
  \begin{array}{rcll}
    \text{specific security labels} & \ell & \in & \{ \low , \high \} \\
    \text{gradual security labels}  & g    & ::= & \unk \MID \ell \\
    \text{blame labels}         & \bl{p}, \bl{q}     &      & \\
    \text{security coercions}            & c, d     & ::=  & \id{g} \MID \up \MID \inj{\ell} \MID \proj{\ell}{p} \MID \bot^{\bl{p}} \\
    \text{coercion sequences} & \bar{c}, \bar{d} & ::=  & \id{g} \MID \err{g_1}{g_2}{p} \MID \bar{c} \seq c
  \end{array}
  \]
  \fbox{$\vdash c : g_1 \Rightarrow g_2$}
  \begin{gather*}
    \inference{}{\vdash \id{g} : g \Rightarrow g}
    \quad
    \inference{}
              {\vdash \,\up\, : \low \Rightarrow \high}
    \quad
    \inference{}
              {\vdash \inj{\ell} : \ell \Rightarrow \unk}
    \\[1ex]
    \inference{}
              {\vdash \proj{\ell}{p} : \unk \Rightarrow \ell}
    \quad
    \inference{}
              {\vdash \bot^{\bl{p}} : \high \Rightarrow \low}
  \end{gather*}
  \end{figure}
\end{frame}


\begin{frame}
  Reduction semantics and normal forms of the coercion calculus on security labels:

\begin{figure}[tbp]
  \footnotesize
\raggedright
  \fbox{$\mathbf{NF}\; \bar{c}$}
  \begin{gather*}
  \inference{}{\mathbf{NF}\; \id{g}}
  \quad
  \inference{}{\mathbf{NF}\; \id{\unk} \seq \proj{\ell}{p}}
  \quad
  \inference{\mathbf{NF}\; \bar{c}}{\mathbf{NF}\; \bar{c}\seq\inj{\ell}}
  \quad
  \inference{\mathbf{NF}\; \bar{c}}{\mathbf{NF}\; \bar{c}\seq\up}
  \end{gather*}
  \fbox{$c \mathrel{;} c \longrightarrow c$}
  \begin{gather*}
  \textit{?-id}~
  \inference{}
            {\inj{\ell} \seq \proj{\ell}{p} \longrightarrow \id{\ell}}
  \quad
  \textit{?-}\uparrow~
  \inference{}
            {\inj{\low} \seq \proj{\high}{p} \longrightarrow \;\;\up}
  \\[1ex]
  \textit{?-}\bot~
  \inference{}
            {\inj{\high} \seq \proj{\low}{p} \longrightarrow \bot^{\bl{p}}}
  \end{gather*}
  \fbox{$\bar{c} \longrightarrow \bar{d}$}
  \begin{gather*}
  \textit{id}~
  \inference{\mathbf{NF}\; \bar{c}}
            {\bar{c} \seq \id{g} \longrightarrow \bar{c}}
  \qquad
  \bot~
  \inference{\mathbf{NF}\; \bar{c} & \vdash \bar{c} : g_1 \Rightarrow g_2}
            {\bar{c} \seq \bot^{\bl{p}} \longrightarrow \err{g_1}{\low}{p} }
  \\[1ex]
  \xi\textit{-}\bot~
  \inference{\vdash c : g_2 \Rightarrow g_3}
    {\err{g_1}{g_2}{p} \seq c \longrightarrow \err{g_1}{g_3}{p}}
  \\[1ex]
  \xi_L~
  \inference{\bar{c} \longrightarrow \bar{d}}{\bar{c} \seq c \longrightarrow \bar{d} \seq c}
  \qquad
  \xi_R~
  \inference{\mathbf{NF}\; \bar{c} & c \seq d \longrightarrow c'}
            {\bar{c} \seq c \seq d \longrightarrow \bar{c}; c'}
  \end{gather*}
\end{figure}
\end{frame}

\begin{frame}
  \frametitle{(A Glimpse of) the Cast Calculus \CC}

  \begin{itemize}
    \item Representation of PC: label expressions
      {\small $e, \PC ::= \ell \MID \blame{p} \MID \cccast{e}{\bar{c}}$}
    \item Coercions on values of \CC:
      \vspace{-5pt}
      \[
      \footnotesize
      \begin{array}{rcll}
        \text{base types}               & \iota     & ::= & \Unit \MID \Bool \\
        \text{raw types}                & T, S      & ::= & \iota \MID \Fun{A}{gc}{B} \mid \Refer{(T_g)} \\
        \text{types}                    & A, B      & ::= & T_g \\
        \text{raw coercions}            & c_r, d_r  & ::=  & \id{\iota} \MID \refco{\bm{c}}{\bm{d}} \MID \left( \funco{\bar{d}}{\bm{c}}{\bm{d}} \right) \\
        \text{coercions}                & \bm{c}, \bm{d} & ::= & \coerc{c_r}{\bar{c}}
      \end{array}
      \]
    \item NSU checking: reducing label expressions
      \vspace{-5pt}
      {\scriptsize
      \begin{gather*}
      \inference{n \; \mathbf{FreshIn} \; \mu(\ell) & \highlightred{\cccast{\PC}{\unk \Rightarrow^{\bl{p}} \ell} \longrightarrow^{*} \PC'} }
                {\reduce{\ccrefproj{\ell}{V}{p}}{\mu}{\PC}{\ccaddr{n}}{(\mu , \ell \mapsto n \mapsto V)}}
      \\[1ex]
      \begin{split}
      \inference{\mathbf{NF}\;\bar{c} & \highlightred{\cccast{(\mathit{stamp!}\;\PC\;|\bar{c}|)}{\unk \Rightarrow^{\bl{p}} \hat{\ell}} \longrightarrow^{*} \PC'} & \cccast{V}{\bm{c}} \longrightarrow^{*} W}
                {\reduce{\ccassignproj{\left(\cccast{\ccaddr{n}}{\coerc{\refco{\bm{c}}{\bm{d}}}{\bar{c}}}\right)}{V}{T}{g}{p}}{\mu}{\PC}{\ccconst{\unit}}{[\hat{\ell} \mapsto n \mapsto W] \; \mu}} \\
      \vdash \bm{c} : T_g \Rightarrow S_{\hat{\ell}} , \vdash \bm{d} : S_{\hat{\ell}} \Rightarrow T_g
      \end{split}
      \end{gather*}
      }
  \end{itemize}
\end{frame}


\begin{frame}
  \frametitle{Road Map}

  \begin{itemize}
  \item \textcolor{gray}{Background}
  \item \textcolor{gray}{\Surface in Action}
  \item \textcolor{gray}{Coercion-based Semantics for Gradual Security}
  \item[\textcolor{Red}{\ding{43}}] \cul{Meta-theoretical Results of \Surface}
  \end{itemize}
\end{frame}


\begin{frame}
  \begin{theorem}[Compilation preserves types]
    \label{thm:compile-pres}
    If $\Gamma ; g \vdash M : A$, then $\Gamma ; \emptyset ; g ; \low \vdash \compile{M} : A$.
  \end{theorem}

  \begin{theorem}[Progress]
    \label{thm:progress}
    Suppose \PC is well-typed: $\vdash \PC \Leftarrow g$,
    $M$ is well-typed: $\emptyset ; \Sigma ; g ; | \PC | \vdash M \Leftarrow A$,
    and the heap $\mu$ is well-typed: $\Sigma \vdash \mu$. \\
    Then either (1) $M$ is a value or (2) $M$ is a blame
    or (3) $M$ can take a reduction step:
    $\reduce{M}{\mu}{\PC}{N}{\mu'}$ for some $N$ and $\mu'$.
  \end{theorem}

  \begin{theorem}[Preservation]
    \label{thm:preservation}
    Suppose \PC is well-typed:  $\vdash \PC \Leftarrow g$,
    $M$ is well-typed: $\emptyset ; \Sigma ; g ; |\PC| \vdash M \Leftarrow A$
    and the heap $\mu$ is well-typed: $\Sigma \vdash \mu$. \\
    If $\reduce{M}{\mu}{\PC}{N}{\mu'}$, there exists $\Sigma'$ s.t
    $\Sigma' \supseteq \Sigma$, $\emptyset ; \Sigma' ; g ; |\PC| \vdash N \Leftarrow A$,
    and $\Sigma' \vdash \mu'$.
  \end{theorem}
\end{frame}

\begin{frame}
  \begin{theorem}[The gradual guarantee]
    Suppose $M$, $M'$ are related by precision:
    $$\precctx{\emptyset}{\emptyset}{\emptyset}{\emptyset}{\low}{\low}{\low}{\low}\ccprec{M}{M'}{A}{A'}$$
    If $M'$ evaluations to a value:
    $$M'\mid\emptyset\mid\low \longrightarrow^{*} V'\mid\mu'$$
    there exists $V$ and $\mu$ s.t. $M$ evaluates to $V$:
    $$M\mid\emptyset\mid\low \longrightarrow^{*} V\mid\mu$$
    and the resulting values are related by precision for some $\Sigma$, $\Sigma'$:
    $$\precctx{\emptyset}{\emptyset}{\Sigma}{\Sigma'}{\low}{\low}{\low}{\low}\ccprec{V}{V'}{A}{A'}$$
  \end{theorem}
\end{frame}

\begin{frame}

  The noninterference of \Surface is conjectured by that of \SurfaceOld:
  \bigskip

  \Surface performs type-guided classification but \SurfaceOld does not,
  so the value that a \Surface program produces is at least as secure as
  the value produced by the same program in \SurfaceOld.
  \bigskip

  \begin{theorem}[Noninterference of \SurfaceOld]
    \label{thm:NI}
    If $M$ is well-typed
    $(x{:}\Bool_{\high}) ; \emptyset ; \low ; \low \vdash M : \Bool_{\low}$
    and
    \begin{equation*}
    \begin{split}
        \bigstep{\emptyset}{\low}{M [ x := (b_1)_{\high}]}{V_1}{\mu_1} \\
        \bigstep{\emptyset}{\low}{M [ x := (b_2)_{\high}]}{V_2}{\mu_2}
    \end{split}
    \end{equation*}
    then $V_1 = V_2$.
  \end{theorem}
\end{frame}

\begin{frame}
  \frametitle{Code and Data Availability}

  \purpletext{\large \url{https://github.com/Gradual-Typing/LambdaSecStar}}
  \bigskip

  \includegraphics[width=\textwidth]{repo_stats}
\end{frame}

\begin{frame}
  \frametitle{Main Takeaways}

  \begin{enumerate}
    \item It is possible to satisfy both noninterference and the gradual guarantee
      in a gradual security-typed language,
      provided that the security level of data remains specific at runtime
      \smallskip
    \item Gradual information flow can be represented as coercions. In particular,
      NSU checking is a special projection that casts PC to the security of the memory
      location to modify
      \smallskip
    \item The key to the semantics design of of a gradual security-typed language is
      identifying injections ($\inj{\ell}$) and projections ($\proj{\ell}{p}$)
  \end{enumerate}
\end{frame}

\begin{frame}
  \centering \Large
  \redtext{Thank you} \bluetext{for your attention!}
\end{frame}

\end{document}
