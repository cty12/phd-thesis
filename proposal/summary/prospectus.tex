\documentclass[10pt, letterpaper]{article}

\usepackage{mathtools}
\usepackage{amsmath,amssymb,amsfonts,amsthm}
\usepackage{mathabx}
\usepackage{mathrsfs}
\usepackage[onehalfspacing]{setspace}
\usepackage{hyperref}
\usepackage{libertine}
\usepackage{fontspec}
\usepackage{semantic}
\usepackage{minted}
\usepackage{multicol}
\usepackage[dvipsnames]{xcolor}
\usepackage[numbers]{natbib}
\usepackage{pifont}
\usepackage{tabularx}
\usepackage{booktabs}
\usepackage{xspace}
\usepackage{circledsteps}
\usepackage[justification=centering]{caption}

\newcommand{\key}[1]{\ensuremath{\mathtt{#1}}}
\newcommand{\Surface}{\ensuremath{\lambda_{\mathtt{IFC}}^\star}\xspace}
\newcommand{\SurfaceOld}{\ensuremath{\lambda_{\mathtt{SEC}}^\star}\xspace}
\newcommand{\CCOld}{\ensuremath{\lambda_{\mathtt{SEC}}^{\Rightarrow}}\xspace}
\newcommand{\CC}{\ensuremath{\lambda_{\mathtt{IFC}}^{c}}\xspace}
\newcommand{\DynIFC}{\ensuremath{\lambda_{\mathtt{SEC}}}\xspace}
\newcommand{\GSLRef}{\ensuremath{\mathrm{GSL}_\mathsf{Ref}}\xspace}
\newcommand{\unk}{\key{\star}\xspace}

\setmonofont[Scale=0.9]{Iosevka}

\title{PhD Dissertation Prospectus: \\
  The Holy Grail of Gradual Security}

\author{Tianyu Chen \\ Computer Science, Indiana University}
\date{March 2024}

\begin{document}

\maketitle

\fbox{
	\parbox{0.9\linewidth}{
    \textbf{My thesis:} \\
    \Surface satisfies both noninterference and the gradual
    guarantee without sacrificing type-based reasoning,
    by excluding the statically unknown label \unk
    from runtime security labels.
	}
}
\vspace{10pt}

In my PhD dissertation, I will introduce a gradual security-typed programming
language, \Surface, which \Circled{1} enforces information flow security
by satisfying noninterference, \Circled{2} satisfies the gradual guarantee,
\Circled{3} enjoys type-guided classification, and \Circled{4}
utilizes NSU checking to enforce implicit flows through the heap
with no static analysis required.
%% The semantics of \Surface is
%% given by translation to a novel security cast calculus \CC,
%% which is the first to be specified using the coercion calculus
%% of~\citet{Henglein:1994nz}.

\section{What Gradual Security Is and Why It Matters}

Information-flow control (IFC) ensures that information transfers within a
program adhere to a security policy, e.g., by preventing high-security data from
flowing to a low-security channel. This adherence can be enforced statically
using a type system~\citep{volpano1996sound,Myers:1997aa,myers1999jflow}, or
dynamically using runtime
monitoring~\citep{Askarov:2009vq,austin2009efficient,Devriese:2010up,
  stefan2011flexible,Austin:2017uh,Xiang:2021ub}, or using static analysis to
pre-compute information that facilitates runtime monitoring
~\citep{le2005monitoring,le2007automaton,Chandra:2007we,Shroff:2007tg,russo2010dynamic,moore2011static}.
The static and dynamic approaches have complementary strengths and weaknesses;
the dynamic approach requires less effort from the programmer while the static
approach provides stronger guarantees and less runtime overhead. Taking
inspiration from gradual typing~\citep{Siek:2006bh,Siek:2007qy}, researchers
have explored how to give programmers seamless control over which parts of the
program are secured statically versus
dynamically~\citep{Disney:2011fv,Fennell:2013ab,Fennell:2015aa}.

Gradual languages should satisfy the gradual guarantee: programs that only
differ in the precision of their type annotations should behave the same modulo
cast errors. Unfortunately, \citet{Toro:2018aa} identify a tension between the
gradual guarantee and information security; they were unable to satisfy both
properties in the language \GSLRef and had to settle for only satisfying
information-flow security (noninterference). \citet{Amorim:2020aa} conjecture
one possible source of the tension: the \textit{type-guided classification}
performed in \GSLRef~\citep{Toro:2018aa}. They propose a new gradually typed language,
GLIO, which sacrifices type-guided classification and prove both noninterference and
the gradual guarantee. \citet{bichhawat2021gradual} conjecture that \textit{NSU checking}
could be another possible source of the tension.
They leverage static analysis ahead of program execution to determine
the write effects in untaken branches. They study a
simple imperative language with first-order stores and prove
noninterference and the gradual guarantee.

\section{Key Insight and Planned Contributions}

Contrary to the prior work, I show that one does not need
to give up on type-guided classification or NSU checking to resolve the tension.
Instead, the tension can be resolved by walking back a design choice in \GSLRef,
which was to allow \unk as a runtime security label.
Beside the insight, I plan to make the following technical contributions:

\begin{itemize}
%% \item Identify the real cause of the tension between information flow security
%%   and the gradual guarantee.
\item A coercion calculus for security labels and a coercion calculus for secure values.
  The two coercion calculi serve as the runtime monitor for IFC.
\item A cast calculus \CC with IFC that defines the dynamic semantics of \Surface.
\item A pen-and-paper proof of noninterference for \Surface through the simulation
  between its cast calculus \CC and a dynamic IFC programming language \DynIFC.
\item The first design of a gradual security-typed language with type-guided classification
  that satisfies the gradual guarantee, namely \Surface.
\item A mechanization of \Surface with the gradual guarantee proved in the Agda~\citep{Bove:2009aa}.
\end{itemize}

\section{Dissertation Timeline}

I have finished the technical development of \Surface and plan to finish my degree requirements
before Dec. 31\textsuperscript{th}, 2024.

\clearpage
\bibliographystyle{ACM-Reference-Format}
\bibliography{all.bib}

\end{document}
