\documentclass[10pt, letterpaper]{article}

\usepackage{mathtools}
\usepackage{amsmath,amssymb,amsfonts,amsthm}
\usepackage{mathabx}
\usepackage{mathrsfs}
\usepackage[onehalfspacing]{setspace}
\usepackage{hyperref}
\usepackage{libertine}
\usepackage{fontspec}
\usepackage{semantic}
\usepackage{minted}
\usepackage{multicol}
\usepackage[dvipsnames]{xcolor}
\usepackage[numbers]{natbib}
\usepackage{pifont}
\usepackage{tabularx}
\usepackage{booktabs}
\usepackage{xspace}
\usepackage{circledsteps}
\usepackage[justification=centering]{caption}

\newcommand{\key}[1]{\ensuremath{\mathtt{#1}}}
\newcommand{\Surface}{\ensuremath{\lambda_{\mathtt{IFC}}^\star}\xspace}
\newcommand{\SurfaceOld}{\ensuremath{\lambda_{\mathtt{SEC}}^\star}\xspace}
\newcommand{\CCOld}{\ensuremath{\lambda_{\mathtt{SEC}}^{\Rightarrow}}\xspace}
\newcommand{\CC}{\ensuremath{\lambda_{\mathtt{IFC}}^{c}}\xspace}
\newcommand{\DynIFC}{\ensuremath{\lambda_{\mathtt{SEC}}}\xspace}
\newcommand{\unk}{\key{\star}\xspace}

\setmonofont[Scale=0.9]{Iosevka}

\title{PhD Dissertation Prospectus: \\
  The Holy Grail of Gradual Security}

\author{Tianyu Chen \\ Computer Science, Indiana University}
\date{March 2024}

\begin{document}

\maketitle

\fbox{
	\parbox{0.9\linewidth}{
    \textbf{My thesis:} \\
    \Surface satisfies both noninterference and the gradual
    guarantee without sacrificing type-based reasoning,
    by excluding the statically unknown label \unk
    from runtime security labels.
	}
}
\vspace{10pt}

In my PhD dissertation, I will introduce a gradual security-typed programming
language, \Surface, which \Circled{1} enforces information flow security
by satisfying noninterference, \Circled{2} satisfies the gradual guarantee,
\Circled{3} enjoys type-guided classification, and \Circled{4}
utilizes NSU checking to enforce implicit flows through the heap
with no static analysis required. The semantics of \Surface is
given by translation to a novel security cast calculus \CC,
which is the first to be specified using the coercion calculus
of~\citet{Henglein:1994nz}.

\section{What Gradual Security Is and Why It Matters}

\bibliographystyle{ACM-Reference-Format}
\bibliography{all.bib}

\end{document}
