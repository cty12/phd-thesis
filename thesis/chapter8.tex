\chapter{Conclusion and Future Work}
\label{ch:conclusion}

To conclude my dissertation, I discuss possible future work in
Section~\ref{sec:future} and summarize the entire dissertation in
Section~\ref{sec:conclusion}.

\section{Future Work}
\label{sec:future}

There are four main research questions about \Surface that I would like to
answer.

\subsection{RQ1: How to mechanize the noninterference proof for \Surface?}

To answer RQ1, we plan to mechanize the noninterference proof for
\Surface by extending the Agda model of \Surface. In prior
work~\parencite{chen2024quest}, we modeled the semantics of \Surface in
Agda and mechanized the proofs of type safety and the gradual
guarantee. We also mechanized the noninterference proof for the
dynamic extreme of \Surface called \DynIFC. We proved on paper that
\Surface satisfies noninterference by a simulation between \Surface
and \DynIFC. Given the sensitive nature of information-flow control,
we plan to verify that \Surface is indeed secure by mechanizing its
noninterference proof.

Following our pen-and-paper proof, we plan to take a three-step
approach. In the first step, we define a relation between terms in the
\CC cast calculus and terms in \DynIFC. This relation captures the
semantic invariant that the value produced by a \CC term is always at
least as secure as the one produced by its related term in \DynIFC. We
define this simulation relation as a datatype in Agda. Second, we
prove a simulation lemma between \CC and \DynIFC: if a \CC term and a
\DynIFC term are related and the \CC side takes one step, the two
sides are able to step to terms that are related again. Finally in the
last step, we prove the noninterference lemma of \CC as a corollary of
the simulation lemma, and then the noninterference theorem of \Surface
as a corollary of both the noninterference lemma of \CC and
that compilation from \Surface to \CC preserves types.

\subsection{RQ2: Is it possible to make \Surface space efficient?}

%% The following commented sentences are not introductory, so they
%% don't belong here. They belongs in a place where we talk about tail
%% recursion, but it looks like we don't talk anywhere else about tail
%% recursion, so this should just be deleted. -Jeremy
%%
%% Representing casts as coercions does not automatically enable space efficiency.
%% A na\"ive implementation of coercions breaks the tail call property, because two
%% mutually recursive functions (where the recursive calls are in tail position)
%% may cause the mediating casts to build up and takes up space proportional to the
%% number of calls.

In prior work, we design two coercion-based cast calculi for GTLC,
$\lambda\textup{\textsf{C}}$ and $\lambda\textup{\textsf{S}}$, and
establish a bisimulation between the
two~\parencite{Siek:2015ab}. $\lambda\textup{\textsf{C}}$ is not space
efficient, while $\lambda\textup{\textsf{S}}$ is space efficient. The
key design that enables space efficiency for
$\lambda\textup{\textsf{S}}$ is that the semantics compresses
coercions into a compact form using the composition operator. However,
the IFC cast calculus \CC does not compress security coercions, so the
current semantics of \Surface is not space efficient.

We plan to develop a space-efficient semantics for \Surface, thereby
presenting a positive answer to the research question. Inspired by
$\lambda\textup{\textsf{S}}$, we will define a composition operator on
security coercions and use it to define a space-efficient
semantics. In addition, we plan to mechanize our space efficiency
proof by extending the Agda model of
\Surface. Our mechanization is
expected to follow the one in our prior work about
GTLC~\parencite{siek_chen_2021}, by proving that the size of any cast is
bounded.

\subsection{RQ3: Is \SurfacePlus expressive?}
\label{sec:case-studies}

We plan to show that \SurfacePlus is expressive by (1) implementing
\SurfacePlus in the Grift compiler and (2) exploring software
prototypes of smart contracts, mobile applications, and university
education systems using this compiler.  The \SurfacePlus language will
augment \Surface with numbers, strings, tuples, vectors, and recursive
types.

\paragraph{Compiler development} In the first step, we plan to build a compiler
for \SurfacePlus, namely GriftIFC, by modifying the compiler passes of Grift to
incorporate security types and security coercions. First, we extend the
typechecker of Grift to accept programs with security type annotations. For the
cast insertion pass, we are going to implement the cast insertion algorithm in
Section 6.2 of our prior work~\parencite{chen2024quest}. The algorithm inserts
security coercions whenever there is insufficient static information to enforce
IFC. After this step, the source program is translated to an intermediate
representation where all casts are made explicit. For the cast lowering pass, we
are going to modify Grift to expose runtime functions that implement the
semantics of security coercions. We plan to use the same closure conversion and
memory allocation / reclamation techniques as Grift.

\paragraph{Case studies} In the second step, we plan to demonstrate
the expressiveness of \SurfacePlus via four case studies using
GriftIFC. In each case study, we will experiment with multiple
versions of the same program that vary in precision, to demonstrate
that gradual typing is indeed able to enable programmer-controlled
migration between static and dynamic IFC in real-world applications.
\begin{itemize}
\item[] \textbf{Smart contracts}~ We are going to implement a prototype in
  \SurfacePlus that simulates the Parity Wallet smart contract attack and
  demonstrate that \SurfacePlus is able to protect the integrity of important
  state variables. We will also explore the prevention techniques against common
  smart contract security threats such as confused deputy attacks and
  re-entrancy attacks, by encoding the defense against the two types of attacks
  as information-flow properties in \SurfacePlus
  following~\textcite{cecchetti2021compositional} and~\textcite{yaoscif}.
\item[] \textbf{E-voting}~ We plan to reimplement part of the Civitas e-voting
  system~\parencite{clarkson2008civitas} in \SurfacePlus. This case study will
  demonstrate that \SurfacePlus is expressive enough for existing programs that
  take advantage of static and dynamic IFC to enforce their security policies.
\item[] \textbf{Mobile computing}~ We plan to implement the two case studies of
  Lifty~\parencite{polikarpova2020liquid}, ``AirBnB'' and ``Instagram''. ``AirBnB''
  models an information leak of AirBnB which displays uncensored phone numbers
  in message previews. We will follow Lifty and implement the information-flow
  policies that (1) unredacted phone numbers are only visible to message
  senders and system administrators (2) redacted phone numbers are
  visible to message recipients using \SurfacePlus. ``Instagram'' models the
  information leakage of Instagram that exposes the ``following'' relation of
  private accounts through followers whose accounts are public. We plan to
  enforce the information-flow policy that the ``following'' relation between A
  and B is visible to C if the accounts of both A and B are visible to C.
\item[] \textbf{Education}~ We plan to implement a prototype student information
  management system in \SurfacePlus that is compliant with FERPA. The system
  should protect confidential information, such as a student's GPA, transcript,
  SSN, and banking information, against public disclosure. On the other hand,
  the student's ``directory information'' such as name, enrollment status,
  campus, and field of study are publicly visible by default, unless restricted
  by the student.
\end{itemize}

\subsection{RQ4: Is it possible to implement \SurfacePlus with high
        performance?}

We plan to demonstrate that it is possible to implement \SurfacePlus with high
performance. We are going to evaluate the performance of GriftIFC by (1)
measuring the execution time of versions of a program that vary in type
precision and (2) comparing the runtime overhead of GriftIFC against fully
static IFC programming languages, such as Jif~\parencite{Myers:2006aa} and Flow
Caml~\parencite{simonet-flowcaml-manual}, and fully dynamic systems, such as
LIO~\parencite{STEFAN:2017ta}.

\paragraph{Benchmark suite and sampling} To evaluate the implementation of
gradually-typed languages, one needs to consider many versions of the same
program obtained by adding or removing type annotations~\parencite{Takikawa:2016aa}.
We plan to build a benchmark suite for gradual IFC using excerpts from the case
study programs of Section~\ref{sec:case-studies}.
After that, we plan to extend the type sampling algorithm of Grift. The type
sampling algorithm takes a statically-typed program and generates gradual
versions of the original program, by inserting gradual types at the source
locations where static types were originally found. The algorithm constrains the
overall program’s type precision, so that the percentage of the unknown type
falls in a certain range. The algorithm selects an equal number of samples for
each range until the desired number of samples have been generated.

\paragraph{Runtime cost of gradual IFC under different type precision}
We will explore the runtime overhead of gradual IFC with respect to type
precision. For each program in the benchmark suite, we plan to measure the
execution time of configurations that vary in precision and show the result in a
scatter graph, where x-axis is type precision and y-axis is program execution
time.

\paragraph{Gradual IFC overhead compared to static and dynamic IFC}
To measure the overhead of gradual IFC compared with fully static IFC, we plan
to compare the execution time of programs in \SurfacePlus to the same
programs adapted to Jif and Flow Caml. To measure the overhead compared with
fully dynamic IFC, we will compare \SurfacePlus to LIO. In both
experiments, we plan to display the results as bar charts where x-axis is
different benchmark programs in different IFC languages and y-axis is their
execution time.

\section{Conclusion}
\label{sec:conclusion}

TBA
