\chapter{Compiling From \Surface to \CC}
\label{ch:compile}

{\color{NavyBlue} %%% new text

In this section, I define the coerce functions that produce coercions between
security labels and types (Section~\ref{sec:coerce-func}). After that, I define
the compilation function from \Surface to \CC (Section~\ref{sec:compile-func}),
so that the semantics of \Surface can be given by \CC.

} %%% end new text

\section{Coerce Functions}
\label{sec:coerce-func}

\begin{figure}[tbp]
  \raggedright
  \begin{align*}
    \unk \Rightarrow^{\bl{p}} \unk &= \id{\unk} & \fbox{$g\Rightarrow^{\bl{p}}g=\bar{c}$} \\
    \ell \Rightarrow^{\bl{p}} \unk &= \id{\ell} \seq \inj{\ell} \\
    \unk \Rightarrow^{\bl{p}} \ell &= \id{\unk} \seq \proj{\ell}{\bl{p}} \\
    \low \Rightarrow^{\bl{p}} \low &= \id{\low} \\
    \low \Rightarrow^{\bl{p}} \high &= \id{\low} \seq \up \\
    \high \Rightarrow^{\bl{p}} \high &= \id{\high} \\
    \iota \Rightarrow^{\bl{p}} \iota &= \id{\iota} & \fbox{$S \Rightarrow^{\bl{p}} T = c_r$} \\
    (\Refer{A}) \Rightarrow^{\bl{p}} (\Refer{B}) &= \refco{(B \Rightarrow^{\bl{p}} A)}{(A \Rightarrow^{\bl{p}} B)} \\
    (\Fun{A}{g_1}{B}) \Rightarrow^{\bl{p}} (\Fun{C}{g_2}{D}) &= (\funco{g_2 \Rightarrow^{\bl{p}} g_1}{(C \Rightarrow^{\bl{p}} A)}{(B \Rightarrow^{\bl{p}} D)}) \\
    S_{g_1} \Rightarrow^{\bl{p}} T_{g_2} &= (\coerc{S \Rightarrow^{\bl{p}} T}{g_1 \Rightarrow^{\bl{p}} g_2}) & \fbox{$A \Rightarrow^{\bl{p}} B = \bm{c}$}
  \end{align*}
  \caption{Coerce functions of security labels and types}
  \label{fig:coerce}
\end{figure}

The coerce function between security labels is defined in
Figure~\ref{fig:coerce}. Function $g_1\Rightarrow^{\bl{p}}g_2=\bar{c}$ takes two
security labels that satisfies $g_1 \precsim g_2$ and a blame label, and
generates a coercion sequence. The source of the coercion sequence is $g_1$, and
the target is $g_2$. The blame label goes to projections inside the generated
coercion sequence.

The coerce functions between security types is also defined in
Figure~\ref{fig:coerce}. Function $S \Rightarrow^{\bl{p}} T = c_r$ takes two raw
types satisfying $S \lesssim T$ and generates a raw coercion from $S$ to $T$.
Function $A \Rightarrow^{\bl{p}} B = \bm{c}$ takes two types satisfying $A
\lesssim B$ and generates a coercion on values from $A$ to $B$. The coerce
functions of types call the coerce function of security labels on each pair of
labels inside the two types. For example
\[
\Bool_{\low} \Rightarrow^{\bl{p}} \Bool_{\unk} = \coerc{\id{\Bool}}{\id{\low}\seq\inj{\low}}
\]
and
\begin{align*}
(\Fun{\Bool_{\unk}}{\low}{\Bool_{\unk}})_{\low} \Rightarrow^{\bl{p}} (\Fun{\Bool_{\low}}{\low}{\Bool_{\high}})_{\high} = &\\
  \coerc{\funco{\id{\low}}{(\coerc{\id{\Bool}}{\id{\low} \seq \inj{\low}})}{(\coerc{\id{\Bool}}{\id{\unk} \seq \proj{\high}{p}})}}{\id{\low} \seq \up} &
\end{align*}

\section{Compilation}
\label{sec:compile-func}

\begin{figure}[tbp]
\raggedright
\fbox{$\compile{M} = N$}
{\footnotesize
\begin{align*}
    \small
    % Const
    \compile{\const{k}{\ell}} =& \ccconst{k} \\
    % Var
    \compile{x} =& x \\
    % Lam
    \compile{\lam{g}{x}{A}{N}{\ell}} =& \cclam{x}{\compile{N}} \\
    % App
    \intertext{Let $L, M$ be well-typed $\Gamma; g' \vdash L : (\Fun{A}{\gc}{B})_g, \Gamma; g' \vdash M : A'$}
    \compile{\app{L}{M}{p}} =&
    \begin{cases}
      \ccapp{(\cccast{(\compile{L})}{\bm{c_1}})}{(\cccast{(\compile{M})}{\bm{c_2}})}{A}{B}{g} \qquad\text{if $g$, $g'$ and $\gc$ are all specific} \\
      \qquad\text{where}~ \bm{c_1} = (\Fun{A}{\gc}{B})_g \Rightarrow^{\bl{p}} (\Fun{A}{g' \curlyvee g}{B})_g, \bm{c_2} = A' \Rightarrow^{\bl{p}} A \\
      \cccast{(\ccappstar{(\cccast{(\compile{L})}{\bm{c_1}})}{(\cccast{(\compile{M})}{\bm{c_2}})}{A}{T})}{\bm{d}} \quad~\text{otherwise} \\
      \qquad\text{where}~ B = T_{g''}, \bm{c_1} = (\Fun{A}{\gc}{T_{g''}})_g \Rightarrow^{\bl{p}} (\Fun{A}{\unk}{T_{\unk}})_{\unk} \\
      \qquad\qquad~ \bm{c_2} = A' \Rightarrow^{\bl{p}} A, \bm{d} = T_{\unk} \Rightarrow^{\bl{p}} (\mathit{stamp}\;T_{g''}\;g)
    \end{cases}
    % If
    \intertext{Let $L, M, N$ be well-typed $\Gamma; g' \vdash L : \Bool_g, \Gamma; g' \lconsisjoin g \vdash M : A,
      \Gamma; g' \lconsisjoin g \vdash N : B$}
    \compile{\ifexp{L}{M}{N}{p}} =&
    \begin{cases}
      \ccif{(\compile{L})}{(A \consisjoin B)}{g}{(\cccast{(\compile{M})}{\bm{c_1}})}{(\cccast{(\compile{N})}{\bm{c_2}})} \quad\text{if $g, g'$ are both specific} \\
      \cccast{(\ccifstar{(\cccast{(\compile{L})}{\bm{d_1}})}{T}{(\cccast{\cccast{(\compile{M})}{\bm{c_1}}}{\bm{d_2}})}{(\cccast{\cccast{(\compile{N})}{\bm{c_2}}}{\bm{d_3}})})}{\bm{d_4}} \\
      \qquad\text{where}~ A \consisjoin B = T_{g''}, \bm{d_1} = \Bool_g \Rightarrow^{\bl{p}} \Bool_{\unk}, \bm{d_2} = T_{g''} \Rightarrow^{\bl{p}} T_{\unk} \\
      \qquad \bm{d_3} = T_{g''} \Rightarrow^{\bl{p}} T_{\unk}, \bm{d_4} = (\mathit{stamp}\;(A \consisjoin B)\;\unk) \Rightarrow^{\bl{p}} (\mathit{stamp}\;(A \consisjoin B)\;g)
    \end{cases} \\
    &\text{where}~
    \bm{c_1} = A \Rightarrow^{\bl{p}} A \consisjoin B, \bm{c_2} = B \Rightarrow^{\bl{p}} A \consisjoin B
    % Ann
    \intertext{Let $M$ be well-typed $\Gamma; g' \vdash M : A'$}
    \compile{\ann{M}{A}{p}} =& \cccast{(\compile{M})}{A' \Rightarrow^{\bl{p}} A}
    % Let
    \intertext{Let $M, N$ be well-typed $\Gamma; g' \vdash M : A, (\Gamma , x{:}A) ; g' \vdash N : B$}
    \compile{(\letexp{x}{M}{N})} =& \cclet{x}{(\compile{M})}{A}{(\compile{N})}
    % Ref
    \intertext{Let $M$ be well-typed $\Gamma; g' \vdash M : T_g$}
    \compile{\refexp{\ell}{M}{p}} =&
    \begin{cases}
      \ccref{\ell}{(\cccast{(\compile{M})}{T_g \Rightarrow^{\bl{p}} T_{\ell}})} & ~\text{if $g'$ is specific} \\
      \ccrefproj{\ell}{(\cccast{(\compile{M})}{T_g \Rightarrow^{\bl{p}} T_{\ell}})}{p} & ~\text{otherwise}
    \end{cases}
    % Deref
    \intertext{Let $M$ be well-typed $\Gamma; g' \vdash M : (\Refer{A})_g$}
    \compile{(\deref{M}{p})} =&
    \begin{cases}
      \ccderef{(\compile{M})}{A}{g} &~\text{if $g$ is specific} \\
      \ccderefstar{(\cccast{(\compile{M})}{\bm{c}})}{T} &~\text{otherwise} \\
      \qquad\text{where}~ A = T_{g''}, \bm{c} = (\Refer{T_{g''}})_g \Rightarrow^{\bl{p}} (\Refer{T_{\unk}})_{\unk}
    \end{cases}
    % Assign
    \intertext{Let $L, M$ be well-typed $\Gamma; g' \vdash L : (\Refer{T_{\hat{g}}})_g, \Gamma; g' \vdash M : A$}
    \compile{\assign{L}{M}{p}} =&
    \begin{cases}
    \ccassign{(\compile{L})}{(\cccast{(\compile{M})}{\bm{c_2}})}{T}{\hat{g}}{g} & ~\text{if $g$, $g'$ and $\hat{g}$ are all specific} \\
    \ccassignproj{(\cccast{(\compile{L})}{\bm{c_1}})}{(\cccast{(\compile{M})}{\bm{c_2}})}{T}{\hat{g}}{\bl{p}} & ~\text{otherwise}
    \end{cases} \\
    &\text{where}~
    \bm{c_1} = (\Refer{T_{\hat{g}}})_g \Rightarrow^{\bl{p}} (\Refer{T_{\hat{g}}})_{\unk} , \bm{c_2} = A \Rightarrow^{\bl{p}} T_{\hat{g}}
\end{align*}}
\caption{Compilation from \Surface to \CC}
\label{fig:compile}
\end{figure}

{\color{NavyBlue} %%% new text

The compile function takes the form $\compile{M} = M'$, where the input $M$ is a
\Surface program and the output $M'$ is a \CC term. We discuss six interesting
cases: function application, if-conditional, annotation, reference creation,
dereferencing, and reference assignment.

\paragraph{Function Application.}
If the top-level labels of the function type, $g$ and \gc, as well as the
current PC ($g'$) are all specific, the constraint on PC during function
application can be statically justified. As a result, we recursively compile $L$
and $M$ and generate \texttt{app}, the static variant of function application.
We cast compiled $L$ using $\bm{c_1}$, which adjusts the PC part of the function
type. We then cast compiled $M$ using $\bm{c_2}$, which goes from $A'$ (the type
of $M$) to $A$ (the domain type of the function). Otherwise, if at least one of
the three security labels is \unk, we cannot statically justify the constraint
on PC. Consequently, we recursively compile $L$ and $M$ and generate
$\mathtt{app}{\star}$, the dynamic variant of function application. We insert
three casts: the first cast ($\bm{c_1}$) converts both top-level labels of the
function type and the top-level label of the co-domain type to \unk. The second
cast ($\bm{c_2}$) goes from $A'$ (the type of $M$) to the domain type $A$. The
third cast ($\bm{d}$) casts the result of the function application, from
$T_{\unk}$ (where $T$ is the raw type part of the co-domain type) to the result
of stamping the label of the function type onto the co-domain.

\paragraph{If-conditional.}
If the label of the branch condition ($g$) and the current PC ($g'$) are both
specific, we recursively compile the branch condition as well as the two
branches and generate a static \texttt{if}. We insert one cast into each of the
compiled branches, from the type of that branch to the consistent join of the
types of the two branches ($A \consisjoin B$). If at least one label is \unk, we
recursively compile the sub-terms and generate $\mathtt{if}{\star}$. In this
case, we need to insert four additional casts: the first cast $d_1$ converts the
branch condition from $\Bool_g$ to $\Bool_{\unk}$. The second ($d_2$) and the
third cast ($d_3$) convert their respective branches into $T_{\unk}$ (where $T$
is the raw type part of $A \consisjoin B$). The fourth cast ($d_4$) casts the
type of the if-conditional to the result of stamping $g$ onto $A \consisjoin B$
to preserve types.

\paragraph{Annotation.}
We recursively compile the annotated term and insert one cast, which goes from
the type of the term to the type annotation.

\paragraph{Reference Creation.}
Recall that the security level of the newly-created memory location is given in
the syntax of \texttt{ref} (always specific). If the current PC ($g'$) is
specific, we can statically justify the heap-policy check. Consequently, we
recursively compile the sub-term ($M$) and generate a static \texttt{ref}. We
only need to insert one cast, which goes from $M$'s type ($T_g$) to the type of
the memory location ($T_{\ell}$). Otherwise, if the current PC is \unk, we
generate the dynamic version of reference creation (\texttt{ref?}), thus
incurring an NSU check, which enforces the heap policy at runtime.

\paragraph{Dereferencing.}
We case on $g$, which is the label on the type of the sub-term ($M$). If $g$ is
specific, we recursively compile $M$ and generate the static version of
dereference. On the other hand, if $g$ is unknown (\unk), we recursively compile
$M$ and generate the dynamic version of dereference ($\mathtt{!}{\star}$). In
addition, we cast the labels on both the reference type and the referenced type
to \unk.

\paragraph{Reference Assignment.}
If $g$, $g'$, and $\hat{g}$ are specific, the check for heap policy can be
statically justified. We recursively compile both $L$ and $M$ and generate a
static \texttt{assign}. We cast $M'$ using coercion $\bm{c_2}$, which casts from
the type of $M'$ to the type of the memory cell. If at least one of the three
security labels is \unk, the typing information is insufficient to justify the
assignment. The compilation produces an \texttt{assign?} instead, whose
semantics performs NSU checking at runtime.

} %%% end new text
