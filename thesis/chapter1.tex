\chapter{Introduction}

\section{Motivation}

With the development of digital society, people are increasingly
concerned about the confidentiality of their personal data and
the integrity of their online assets. Increasingly relying on
computing devices and the Internet in their daily life, people fear that
sensitive personal information, such as social security numbers,
medical records, bank account balances... may be revealed to malicious
third parties. People also worry that their digital photo albums,
signatures on online legal documents, spreadsheets in cloud storage...
may be tampered and manipulated by potential attackers.

Indeed, the fears are justified by recent new events.
In 2018, the Cambridge Analytica scandal hit the world headlines,
where the data collected from 87 million social media users was misused
without their consent
~\parencite{cadwalladr2018facebook,kitchgaessner2017cambridge,gonzalez2019global,hinds2020wouldn}.
In the healthcare sector, from 2005 to 2019, 249.09 million individuals
were affected by data breaches that caused exposure of sensitive medical
data~\parencite{seh2020healthcare}. In respect of data integrity, researchers
have found ways to tamper the analytics APIs~\parencite{pfeffer2018tampering} and
the metadata (such as the numbers of likes, follows, and views)~\parencite{paquet2017can}
of major social media platforms. To deal with the security and privacy
challenges of the increasingly digitalized world, the European Union introduced
the General Data Protection Regulation (GDPR) to reform and regulate
the collection and processing of personal data. However, studies show
that business entities experience challenges in complying with GDPR
or auditing for compliance~\parencite{smirnova2024understanding},
particularly small-to-medium enterprises~\parencite{sirur2018we,freitas2018gdpr,harting2021impacts}.

\begin{figure}[tbp]
  \small
  \begin{align*}
    \langle RECORD \rangle ::= & {\color{blue} \textbf{\{FirstName=}} \langle ID \rangle {\color{blue} \textbf{;}} \\
                               & \enspace {\color{blue} \textbf{LastName=}} \langle ID \rangle {\color{blue} \textbf{;}} \\
                               & \enspace {\color{blue} \textbf{SSN=}} \langle SSN \rangle {\color{blue} \textbf{\}}} \\
    \langle ID \rangle     ::= & w , w \in \{ {\color{blue} \textbf{A}}, ... {\color{blue} \textbf{Z}}, {\color{blue} \textbf{a}}, ... {\color{blue} \textbf{z}} \}^{+} \\
    \langle SSN \rangle    ::= & \langle D \rangle \langle D \rangle \langle D \rangle {\color{blue} \textbf{-}}
                                 \langle D \rangle \langle D \rangle {\color{blue} \textbf{-}}
                                 \langle D \rangle \langle D \rangle \langle D \rangle \langle D \rangle \\
    \langle D \rangle      ::= & d , d \in \{ {\color{red} \textbf{0}}, ... {\color{red} \textbf{9}} \}
  \end{align*}
  \caption{The user input grammar for a hypothetical application}
  \label{fig:grammar}
\end{figure}

From a technical perspective, ensuring the security and privacy of
personal data typically involves tracking and checking
the flow of information.
Modern software applications often accept user input where
selected fields are sensitive, whose confidentiality is required
during both parsing and processing. To rule out information leaks,
neither the sensitive fields, nor any data that depends on those fields,
is allowed to be revealed to a low-privilege observer.
Consider a web application that receives three fields from its user:
1) first name 2) last name 3) social security number, the grammar of which
is defined in figure \ref{fig:grammar}, where terminals are divided into
low-security and high-security.

\section{Problem Statement}

\section{Thesis Statement}

%\section{\Surface}

\section{Outline}
