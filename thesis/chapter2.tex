\chapter{Gradual Information-Flow Control (IFC) in \Surface}
\label{ch:examples}

In this chapter, I first define the gradual IFC surface language \Surface, by
presenting its syntax, type system, and semantics in
Section~\ref{sec:surface-def}. After that, I put \Surface into action in
Section~\ref{sec:examples}. I present example programs to demonstrate how
\Surface enables a gradual and smooth transition between static and dynamic IFC,
while supporting type-based reasoning and satisfying the gradual guarantee at
the same time. Finally, in Section~\ref{sec:embedding}, I conclude the chapter
by discussing the static and the dynamic extremes of \Surface. \Surface enables
a continuum of IFC enforcement between the static and dynamic extremes that it
embeds.

\section{The Gradual IFC Language \Surface}
\label{sec:surface-def}

This section is organized as follows. I first present the syntax of \Surface in
Section~\ref{sec:surface-syntax}. I then present the type system for \Surface in
Section~\ref{sec:surface-typing}. Finally, I discuss the semantics of \Surface
by defining the \texttt{eval} function in Section~\ref{sec:surface-semantics}.

\subsection{The Syntax of \Surface}
\label{sec:surface-syntax}

\begin{figure}[tbp]
\raggedright
  {\small
  \[
  \begin{array}{rcll}
    \text{specific security labels} & \ell & \in & \{ \low , \high \} \\
    \text{security labels}  & g    & ::= & \unk \MID \ell \\
    \text{base types}               & \iota     & ::= & \Unit \MID \Bool \\
    \text{raw types}                & T, S      & ::= & \iota \MID \Fun{A}{\gc}{B} \mid \Refer{(T_g)} \\
    \text{blame labels}         & \bl{p}, \bl{q}     &      & \\
    \text{types}                    & A, B      & ::= & T_g \\
    \text{terms}                & L , M , N & ::=  & x \MID \const{k}{\text{\hl{$\ell$}}} \MID \lam{g}{x}{A}{N}{\text{\hl{$\ell$}}} \MID \app{L}{M}{p} \\
    &                                       & \MID & \ifexp{L}{M}{N}{p} \MID \letexp{x}{M}{N} \\
    &                                       & \MID & \refexp{\text{\hl{$\ell$}}}{M}{p} \MID \deref{M}{p} \MID \assign{L}{M}{p} \MID \ann{M}{A}{p}
  \end{array}
  \]}
  \caption{Syntax of \Surface (highlighted security labels \hl{$\ell$} default to \low if omitted)}
  \label{fig:surface-syntax}
\end{figure}

The syntax of the gradual language \Surface is shown in
Figure~\ref{fig:surface-syntax}. The figure also includes the definitions for
security labels, security types, and blame labels.

For simplicity, I will use a two-point security lattice $\langle \{\high, \low\}
, \preccurlyeq, \curlyvee , \curlywedge \rangle$, where $\high$ is for private,
sensitive data and $\low$ is for public, disclosable data. Of course, any
lattice of security labels could be used in place of low and high. The ordering
is standard: $\low \preccurlyeq \high$ and $\high \npreccurlyeq \low$. So
information is allowed to flow from public sources to private sinks but not the
other way around. I refer to $\{\high,\low\}$ as \emph{specific security
labels}.

Types in \Surface have security labels associated with them, for example,
$\Bool_{\high}$ is the type for booleans with high security, $\Unit_{\low}$ is
the type for the unit value with low security, and $\Bool_{\unk}$ is the type of
a boolean whose security level is unknown at compile time. I refer to
$\{\high,\low,\unk\}$ as \emph{security labels}. A function type
$(\Fun{A}{\gc}{B})_g$ carries an additional security label \gc, which is the
type of the program counter (PC) to evaluate the body of the function.

Some terms in \Surface are annotated with an identifier called a blame label
(\bl{p}). When compiled to the intermediate representation, those terms generate
runtime checking (casts) that may fail. In case a check fails, it raises a cast
error, called \textit{blame}, that contains its blame label. In this way, the
programmer knows which cast is causing the problem and which part of the program
generates that cast.

To enable information-flow control, \Surface allows the programmer to annotate
constants, mutable references, and $\lambda$-abstractions with a specific
security label. \Surface ensures that if a value is annotated with $\high$, it
will not flow into a sink that is $\low$ security. If the programmer does not
annotate a value with a label, \Surface defaults the value's label to \low. So
\texttt{true} is shorthand for $\mathtt{true}_{\low}$. \Surface supports
higher-order functions, mutable references, and explicit type annotations. One
thing to note is that compared with literals, $\lambda$-abstractions in \Surface
carry an addition security label annotation $g$, which is the type of the PC
label expression used to evaluate the body of the $\lambda$. Same as the
security labels in type annotations, $g$ defaults to \unk if omitted. For
readers familiar with \GSLRef, the syntax of \Surface is similar to that of
\GSLRef. The main syntactic difference is that in \Surface, the security labels
of literals and newly created memory cells default to a specific label such as
\low, while in \GSLRef they default to a runtime unknown security level \unk. I
am going to show in Section~\ref{sec:example2} that defaulting to a specific
security label helps us resolve the tension between noninterference and the
gradual guarantee.

\subsection{The Type System of \Surface}
\label{sec:surface-typing}

I first define operators and relations on security labels and types that are
used by the typing rules in Section~\ref{sec:ops-and-rels}. I then introduce
the typing rules of \Surface in Section~\ref{sec:surface-typing-rules}.

\subsubsection{Operators and Relations on Security Labels and Types}
\label{sec:ops-and-rels}

\begin{figure}[tbp]
\begin{align*}
\ell \sqcup \ell &= \ell & \fbox{$g \sqcup g$}\\
\unk \sqcup g &= g \\
g \sqcup \unk &= g \\[1ex]
\iota \sqcup \iota &= \iota & \fbox{$T \sqcup T$} \\
(\Refer{A}) \sqcup (\Refer{B}) &= \Refer{(A \sqcup B)} \\
(\Fun{A}{g_1}{B}) \sqcup (\Fun{C}{g_2}{D}) &= \Fun{(A \sqcup C)}{g_1 \sqcup g_2}{(B \sqcup D)} \\[1ex]
S_{g_1} \sqcup T_{g_2} &= (S \sqcup T)_{g_1 \sqcup g_2}  & \fbox{$A \sqcup A$}
\\[4ex]
%%%%%%
\ell_1 \lconsisjoin \ell_2 &= \ell_1 \curlyvee \ell_2 &\fbox{$g \lconsisjoin g$}\\
\text{-} \lconsisjoin \unk &= \unk \\
\unk \lconsisjoin \text{-} &= \unk \\[1ex]
\iota \consisjoin \iota &= \iota & \fbox{$T \consisjoin T$}\\
(\Refer{A}) \consisjoin (\Refer{B}) &= \Refer{(A \sqcup B)} \\
(\Fun{A}{g_1}{B}) \consisjoin (\Fun{C}{g_2}{D}) &= \Fun{(A \consismeet C)}{g_1 \lconsismeet g_2}{(B \consisjoin D)} \\[1ex]
S_{g_1} \consisjoin T_{g_2} &= (S \consisjoin T)_{g_1 \lconsisjoin g_2} & \fbox{$A \consisjoin A$}
\\[4ex]
%%%%%%
\ell_1 \lconsismeet \ell_2 &= \ell_1 \curlywedge \ell_2 & \fbox{$g \lconsismeet g$}\\
\text{-} \lconsismeet \unk &= \unk \\
\unk \lconsismeet \text{-} &= \unk \\[1ex]
\iota \consismeet \iota &= \iota & \fbox{$T \consismeet T$} \\
(\Refer{A}) \consismeet (\Refer{B}) &= \Refer{(A \sqcup B)} \\
(\Fun{A}{g_1}{B}) \consismeet (\Fun{C}{g_2}{D}) &= \Fun{(A \consisjoin C)}{g_1 \lconsisjoin g_2}{(B \consismeet D)} \\[1ex]
S_{g_1} \consismeet T_{g_2} &= (S \consismeet T)_{g_1 \lconsismeet g_2}  & \fbox{$A \consismeet A$}
\end{align*}
\\[1ex]
%%%%%%
\[
\mathit{stamp} \; (T_{g_1}) \; g_2 = T_{g_1 \lconsisjoin g_2}
\]
\caption{Auxiliary operators for security labels and types: join w.r.t precision (-$\sqcup$-),
consistent join (-$\lconsisjoin$- for labels and -$\consisjoin$- for types),
and consistent meet (-$\lconsismeet$- for labels and -$\consismeet$- for types).
Stamping for types}
\label{fig:grad-ops}
\end{figure}

Figure~\ref{fig:grad-ops} presents auxiliary operators on security labels and
types. These operators include join w.r.t precision, consistent join, consistent
meet, and the stamping operation on types. The operators are standard by following
those of \GSLRef and GLIO. Join w.r.t precision returns the least upper bound of
the precision of two labels or two types, for example
\[
(\Fun{\Bool_{\low}}{\unk}{\Bool_{\high}})_{\unk} \sqcup
(\Fun{\Bool_{\low}}{\low}{\Bool_{\high}})_{\unk} =
(\Fun{\Bool_{\low}}{\low}{\Bool_{\high}})_{\unk}
\]
Consistent join of security labels resorts to security lattice join if both
labels are statically known; otherwise the operator returns \unk if at least one
label is \unk. Consistent join of types is recursive on the structure of the
types, with the PC labels and the domain types in function types being
contravariant and the referenced types being invariant. The consistent meet
operators of security labels and types are analogous to consistent join. The
stamping operator is a shorthand that computes the consistent join of the
top-level label of the type with another label, while keeping the rest of the
type unchanged.

\begin{figure}[ht]
  \raggedright
  \fbox{$g_1 \precsim g_2$}
  \begin{gather*}
    {\precsim}\unk~
    \inference{}{g \precsim \unk}
    \qquad
    \unk{\precsim}~
    \inference{}{\unk \precsim g}
    \qquad
        {\precsim}\textit{-}\ell~
        \inference{\ell_1 \preccurlyeq \ell_2}{\ell_1 \precsim \ell_2}
  \end{gather*}
  \raggedright
  \fbox{$S \lesssim T$}
  \begin{gather*}
          {\lesssim}\textit{-}\iota~
          \inference{}{\iota \lesssim \iota}
          \quad
              {\lesssim}\textit{-ref}~
              \inference{A \lesssim B & B \lesssim A}{\Refer{A} \lesssim \Refer{B}} \quad
                        {\lesssim}\textit{-fun}~
                        \inference{g_2 \precsim g_1 & C \lesssim A & B \lesssim D}
                                  {\Fun{A}{g_1}{B} \lesssim \Fun{C}{g_2}{D}}
  \end{gather*}
  \raggedright
  \fbox{$A \lesssim B$}
  \begin{gather*}
    {\lesssim}\textit{-}\tau~
    \inference{g_1 \precsim g_2 & S \lesssim T}{S_{g_1} \lesssim T_{g_2}}
  \end{gather*}
  \caption{Consistent subtyping for labels and types}
  \label{fig:consis-sub}
\end{figure}

Figure~\ref{fig:consis-sub} presents the definitions of consistent subtyping for
security labels and types, which will be used in the type system of \Surface.
Consistent subtyping is the composition of consistency and subtyping. The
following are equivalent: (1) $A \lesssim B$, (2) $A \sim C <: B$ for some $C$,
and (3) $A <: D \sim B$ for some $D$. The consistent subtyping relations in
\Surface are standard, similar to those in \GSLRef and GLIO, with the PC and
domain type in a function type being contravariant (rule
${\lesssim}\textit{-fun}$) and the referenced type being invariant
(rule ${\lesssim}\textit{-ref}$).

\begin{figure}[tbp]
  \raggedright
  \fbox{$g_1 \sqsubseteq g_2$}
  \begin{gather*}
    \unk{\sqsubseteq}~
    \inference{}{\unk \sqsubseteq g}
    \qquad\quad
    \ell{\sqsubseteq}\ell~
    \inference{}{\ell \sqsubseteq \ell}
  \end{gather*}
  \fbox{$S \sqsubseteq T$}
  \begin{gather*}
    {\sqsubseteq}\textit{-}\iota~
    \inference{}{\iota \sqsubseteq \iota}
    \quad
        {\sqsubseteq}\textit{-ref}~
        \inference{A \sqsubseteq B}{\Refer{A} \sqsubseteq \Refer{B}}
        \quad
            {\sqsubseteq}\textit{-fun}~
            \inference{g_1 \sqsubseteq g_2 &
              A \sqsubseteq C &
              B \sqsubseteq D}
                      {\Fun{A}{g_1}{B} \sqsubseteq \Fun{C}{g_2}{D}}
  \end{gather*}
  \fbox{$A \sqsubseteq B$}
  \begin{gather*}
    {\sqsubseteq}\textit{-}\tau~
    \inference{g_1 \sqsubseteq g_2 & S \sqsubseteq T}{S_{g_1} \sqsubseteq T_{g_2}}
  \end{gather*}
  \caption{Precision of security labels and types}
  \label{fig:type-prec}
\end{figure}

I define a precision ordering $\sqsubseteq$ on security labels and security
types. For security labels, the statically unknown label \unk is the most
imprecise, so $\unk \sqsubseteq g$ for any label $g$ and $\ell \sqsubseteq \ell$
for any specific security label $\ell$. The precision ordering extends to types
in a natural way, so for example, $\Bool_{\unk} \sqsubseteq \Bool_{\low}$ and
$(\Fun{\Bool_{\low}}{\unk}{\Bool_{\high}})_{\unk} \sqsubseteq $.
Figure~\ref{fig:type-prec} of the Appendix gives the definition of precision on
types.

\subsubsection{Typing Rules for \Surface}
\label{sec:surface-typing-rules}

\begin{figure}[tbp]
\raggedright
  \fbox{$\Gamma ; g \vdash M : A$}
  \begin{gather*}
  \small
    {\vdash}\mathit{var}~
    \inference{\Gamma \ni x : A}
              {\Gamma; g \vdash x : A}
    \quad
    {\vdash}\mathit{const}~
    \inference{k : \iota}
              {\Gamma; g \vdash \const{k}{\ell} : \iota_\ell}
    \\[1ex]
    {\vdash}\mathit{lam}~
    \inference{(\Gamma , x{:}A); g_2 \vdash N : B}
              {\Gamma; g_1 \vdash \lam{g_2}{x}{A}{N}{\ell} : (\Fun{A}{g_2}{B})_\ell}
    \\[1ex]
    {\vdash}\mathit{app}~
    \inference{\Gamma; g \vdash L : (\Fun{A}{g_2}{B})_{g_1} &
               \Gamma; g \vdash M : A' \\
               A' \lesssim A & g \precsim g_2 & g_1 \precsim g_2}
              {\Gamma; g \vdash \app{L}{M}{p} : \textit{stamp}\;B\;g_1}
    \\[1ex]
    {\vdash}\mathit{let}~
    \inference{\Gamma; g \vdash M : A \\
               (\Gamma , x{:}A) ; g \vdash N : B}
              {\Gamma; g \vdash \letexp{x}{M}{N} : B}
    \\[1ex]
    {\vdash}\mathit{if}~
    \inference{\Gamma; g_2 \vdash L : \Bool_{g_1} \\
               \Gamma; g_2 \lconsisjoin g_1 \vdash M : A &
               \Gamma; g_2 \lconsisjoin g_1 \vdash N : B \\
               A \consisjoin B = C}
              {\Gamma; g_2 \vdash \ifexp{L}{M}{N}{p} : \textit{stamp}\;C\;g_1}
    \\[1ex]
    {\vdash}\mathit{ref}~
    \inference{\Gamma; g_2 \vdash M : T_{g_1} \\
               T_{g_1} \lesssim T_\ell & \highlightblue{g_2 \precsim \ell}}
              {\Gamma; g_2 \vdash \refexp{\ell}{M}{p} : (\Refer{T_\ell})_{\low}}
    \quad
    {\vdash}\mathit{deref}~
    \inference{\Gamma; g_2 \vdash M : (\Refer{A})_{g_1}}
              {\Gamma; g_2 \vdash \deref{M}{p} : \textit{stamp}\;A\;g_1}
    \\[1ex]
    {\vdash}\mathit{assign}~
    \inference{\Gamma; g_2 \vdash L : (\Refer{T_{\hat{g}}})_{g_1} &
               \Gamma; g_2 \vdash M : A \\
               A \lesssim T_{\hat{g}} & \highlightblue{g_2 \precsim \hat{g}} & \highlightblue{g_1 \precsim \hat{g}}}
              {\Gamma; g_2 \vdash \assign{L}{M}{p} : \Unit_{\low}}
    \quad
    {\vdash}\mathit{ann}~
    \inference{\Gamma; g \vdash M : A' \\ A' \lesssim A}
              {\Gamma; g \vdash \ann{M}{A}{p} : A}
  \end{gather*}
  \caption{Typing rules of \Surface. Side conditions about the heap policy are
    \highlightblue{\text{highlighted}}}
  \label{fig:surface-typing-full}
\end{figure}

The typing rules for \Surface are shown in Figure~\ref{fig:surface-typing-full}.
They are directly adapted from those of \GSLRef, by changing the security labels
on values to disallow the \unk label. The type system of \GSLRef is derived from
its static counterpart \SSLRef by replacing labels and types as well as their
operators and predicates with the gradual variants, while \SSLRef is in turn an
adaptation of prior security-typed languages such as \textcite{Fennell:2013ab,
  heintze1998slam, zdancewic2002programming}.

For example, in \SSLRef the typing rule of application looks like:
\begin{equation*}
{\vdash}\textit{app-SSLRef}~
\inference{\Gamma; \pc \vdash L : (\Fun{A}{\pc'}{B})_\ell & \Gamma; \pc \vdash M : A' \\
  A' <: A & \pc \preccurlyeq \pc' & \ell \preccurlyeq \pc'}
{\Gamma; \pc \vdash \app{L}{M}{} : \mathit{stamp}\;B\;\ell}
\end{equation*}
where $A' <: A$ is the usual type subsumption of function argument. The side
conditions $\ell \preccurlyeq \pc'$ and $\pc \preccurlyeq \pc'$ restricts the PC
label on the function type so that no information is leaked through side
effects. The type of the application has label that is the join of the label on
$B$ and $\ell$ ($\mathit{stamp}\;B\;\ell$). In \Surface, the typing judgment
takes the form $\Gamma ; g \vdash M : A$, where the static PC label $g$ and the
type $A$ become gradual (may be or contain \unk). Like \GSLRef, I replace label
partial order with label consistent subtyping, type subtyping with type
consistent subtyping, and label join with label consistent join and get rule
${\vdash}\mathit{app}$.

The only major difference from the type system of \GSLRef is that because of the
concrete label restriction on the syntax of constants and
$\lambda$-abstractions, these terms must have concrete labels at the top level
of their respective types (rule ${\vdash}\mathit{const}$ and
${\vdash}\mathit{lam}$). Similarly, the type of the value in a newly allocated
cell (rule ${\vdash}\mathit{ref}$) has a concrete top-level label:
$(\Refer{T_{\text{\hl{$\ell$}}}})_{\low}$. The reference itself has a \low label
because it is newly created and cannot leak information.

\subsection{The Semantics of \Surface}
\label{sec:surface-semantics}

%% foreshadow the semantics by defining the eval function

I define the semantics of \Surface through a compile function that takes a
well-typed \Surface program and returns a cast-calculus (\CC) term. The compile
function takes the form $\compile{M} = M'$, where $M$ is a \Surface program and
$M'$ is a \CC term.

I define the evaluation function for \Surface by (1) compiling from \Surface to
\CC and (2) running the compiled \CC term using its operational semantics.
\begin{align*}
  \mathit{eval}(M) =&\,V\;\text{if}\;\reducemult{(\compile{M})}{\emptyset}{\low}{V}{\mu} \\
  \mathit{eval}(M) =&\,\blame{\bl{p}}\;\text{if}\;\reducemult{(\compile{M})}{\emptyset}{\low}{\blame{\bl{p}}}{\mu}
\end{align*}

I will define the compile function and the operational semantics for \CC in
Chapter~\ref{ch:sem}.

\section{\Surface in Action}
\label{sec:examples}

This section is organized as follows. In Section~\ref{sec:example1}, I review
the basics of gradual IFC using \Surface programs. I show that \Surface enables
a gradual transition between static and dynamic IFC. In
Section~\ref{sec:example2}, I review the counterexamples of
\textcite{Toro:2018aa} and demonstrate that the tension between security and the
gradual guarantee can be solved by removing \unk from the runtime security
labels. Finally, I show that \Surface enables the same type-based reasoning
capabilities through free theorems as \GSLRef, because \Surface is vigilant and
performs type-guided classification.

\subsection{The Gradual Transition Between Static and Dynamic IFC in \Surface}
\label{sec:example1}

In this section I review the basic concepts of gradual information flow control
using \Surface, establishing the intuition that \Surface enables a smooth,
gradual transition between static and dynamic IFC. I start with fully static
\Surface programs and show that \Surface can behave like a static security-typed
language, guarding against both illegal explicit and implicit flows at compile
time. I then replace some security label annotations in types with \unk, so that
the programs become partially typed and the typing information alone is
insufficient to enforce IFC. I show that security coercions, serving as the
runtime security monitor of \Surface, are able to capture both explicit flow and
implicit flow violations at runtime, thereby preventing information leakage and
enforcing security.

I model I/O with two functions, \texttt{user-input} and \texttt{publish}: the
former returns a high-security boolean that represents sensitive input
information; the latter takes a low-security boolean and publishes it into a
publicly visible channel.

\paragraph{Gradual IFC includes static IFC}

For statically typed programs, \Surface behaves just like a statically
typed IFC language. Consider the following well-behaved \Surface
program that takes in a high-security user input, passes it to the
function \texttt{fconst} that ignores the input and
returns \texttt{false}, which is then published.

\begin{lstlisting}[style=tt]
  let fconst = |$\lambda$| b : |$\Bool_{\high}$|. false in
  let input  = user-input () in
  let result = fconst input in
    publish result
\end{lstlisting}

The program type-checks and runs without error, with no need for
runtime checks to enforce security. Indeed, a malicious party cannot
infer anything about the high-security input because (1) the return
value of \texttt{fconst} is always the same value \false~ (2)
the value \false~is of low security, so the explicit flow
into \texttt{publish} is allowed.

If we replace \texttt{fconst} with the identity function \texttt{fid} with
parameter type $\Bool_{\low}$, the program becomes ill-typed as is usual for a
statically typed IFC language: the type system disallows the explicit flow from
the high-security input to \texttt{fid}, which expects a low-security boolean
value.

\begin{lstlisting}[style=tt]
  let fid    = |$\lambda$| b : |\colorbox{highlight}{$\Bool_{\low}$}|. b in
  let input  = user-input () in
  let result = fid input in      |\graytext{// static error}|
    publish result
\end{lstlisting}

Sometimes the observable behaviors of a program can depend on its branching
structure. If some of the branch conditions have a data dependency on
high-security input, a malicious party might be able to infer it from the
observable behaviors, giving rise to illegal \textit{implicit flows}
~\parencite{denning1976lattice}, which must be ruled out to guarantee security.

Consider the following program in which the function \texttt{flip} contains a
conditional expression, whose condition is dependent on a high-security user
input. Its two branches return different low-security booleans, creating a
potential implicit flow from high to low:

\begin{lstlisting}[style=tt]
  let flip : |$\Bool_{\high}$| -> |$\Bool_{\low}$| =
       |$\lambda$| b : |$\Bool_{\high}$|. |\colorbox{highlight}{\texttt{if b then false else true}}| in
  let input  = user-input () in
  let result = flip input in
    publish result
\end{lstlisting}

%% \noindent As is typical of statically typed IFC languages,
%% the type system of \Surface rejects this program, thereby preventing
%% an information leak through an implicit flow.
%% %% Perhaps the programmer
%% %% mistakenly annotated the return type of \texttt{flip} thinking that it
%% %% must return $\Bool_{\low}$, because both branches contain low-security
%% %% values.
%% To see why, note that the branch condition is of high security, so the
%% type of the \texttt{if} expression as a whole is $\Bool_{\high}$. In
%% particular, the type checker computes the security level of
%% an \texttt{if} to be the join of its branches (both \low) and the
%% condition (\high), yielding $\low \curlyvee \high
%% = \high$. The \texttt{flip} function is expected to return
%% $\Bool_{\low}$ according to its type annotation, but returns
%% $\Bool_{\high}$ because of the conditional,
%% $\high \npreccurlyeq \low$, so the program is ill-typed.

\subsection{Implicit Flow, NSU Checks, Unknown Security, and the Gradual Guarantee}
\label{sec:example2}

%% TODO define the precision relations here

\subsection{Type-Based Reasoning in \Surface}
\label{sec:example3}

\section{The Static and Dynamic Extremes of \Surface}
\label{sec:embedding}
