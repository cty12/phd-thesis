\chapter*{Vita}
\addtocontents{toc}{
 \unexpanded{\unexpanded{\renewcommand{\cftchapdotsep}{\cftnodots}}}
}
\addcontentsline{toc}{chapter}{Curriculum Vitae}

\doublespacing

\textbf{\Huge Tianyu Chen}


\section*{\sc Contact Information}
\vspace{.05in}
\begin{tabular}{@{}p{3in}p{3in}}
  3025B. Luddy Hall, Indiana University        & {\it E-mail:} chen512@iu.edu                          \\
  700 N. Woodlawn Ave.    & {\it Website:} \href{https://homes.luddy.indiana.edu/chen512/}{homes.luddy.indiana.edu/chen512}  \\
  Bloomington, IN 47408   &
\end{tabular}


\section*{\sc Education}

\begin{itemize}
\item {\em Ph.D.}, Computer Science, Indiana University \hfill {\bf August 2016 -- May 2025}
  \begin{itemize}
    \item Advisor: Professor Jeremy G. Siek
  \end{itemize}
\item {\em B.Eng.}, Tsinghua University \hfill {\bf August 2012 -- July 2016}
  \begin{itemize}
  \item Major: Computer Science and Technology
  \end{itemize}
\end{itemize}


\section*{\sc Experience}

{\bf Indiana University}, Bloomington, Indiana USA \\
{\em Instructor-of-Record (IOR)} \hfill {\bf August 2024 -- Present} \\
{\em Research Assistant and Associate Instructor} \hfill {\bf August 2016 -- July 2024} \\
{\bf CertiK}, Remote, New York USA \\
{\em Software Verification Intern on Coq and CompCert} \hfill {\bf May 2021 -- August 2021} \\
{\bf Columbia University}, New York City, New York USA \\
{\em Visiting Student on Transparent Paxos} \hfill {\bf August 2015 -- October 2015} \\


\section*{\sc Publications}

{\bf Quest Complete: the Holy Grail of Gradual Security} \\
Tianyu Chen, Jeremy G. Siek. \\
In Proceedings of the 45th ACM SIGPLAN Conference on Programming Language Design and
Implementation (\textbf{PLDI 2024}). \\
{\bf Parameterized Cast Calculi and Reusable Meta-theory for Gradually Typed Lambda Calculi } \\
Jeremy G. Siek, Tianyu Chen. \\
In Journal of Functional Programming (\textbf{JFP}), November 2021. \\
{\bf Mechanized Type Safety for Gradual Information Flow} \\
Tianyu Chen, Jeremy G. Siek. \\
In the 7th Workshop on Language-Theoretic Security (\textbf{LangSec 2021}). \\
{\bf Racing in Hyperspace: Closing Hyper-threading Side Channels on SGX with Contrived Data Races} \\
Guoxing Chen, Wenhao Wang, Tianyu Chen, Sanchuan Chen, Yinqian Zhang, XiaoFeng Wang, Ten-Hwang Lai, Dongdai Lin. \\
In Proceedings of 2018 IEEE Symposium on Security and Privacy (\textbf{Oakland 2018}). \\
{\bf Characterizing Smartwatch Usage in the Wild} \\
Xing Liu, Tianyu Chen, Feng Qian, Zhixiu Guo, Felix Xiaozhu Lin, Xiaofeng Wang, Kai Chen. \\
In Proceedings of the 15th Annual International Conference on Mobile Systems, Applications,
and Services (\textbf{MobiSys 2017}). \\
{\bf Paxos Made Transparent} \\
Heming Cui, Rui Gu, Cheng Liu, Tianyu Chen, Junfeng Yang. \\
In Proceedings of the 25th Symposium on Operating Systems Principles (\textbf{SOSP 2015}). \\


\section*{\sc Drafts and Posters}

{\bf Mechanized Noninterference for Gradual Security } \\
Tianyu Chen, Jeremy G. Siek. \\
Draft, November 2022. \\
{\bf Generic Blame-Subtyping Theorem in Agda Using Abstract Binding Trees } \\
Tianyu Chen. \\
In POPL 2022 Student Research Competition (\textbf{POPL SRC 2022}). \\

\section*{\sc Teaching}

{\bf System Programming With C and Unix} (CSCI-C291 and ENGR-E111) \\
{\em Instructor} \hfill {\bf Spring 2025, Fall 2024} \\
{\bf Data Structures} (CSCI-C343, 200 students) \\
{\em Lead Associate Instructor} \hfill {\bf Spring 2024} \\
{\bf Data Structures, Honors} (CSCI-H343) \\
{\em Associate Instructor} \hfill {\bf Fall 2023} \\
{\bf Secure Protocols} (CSCI-B433 and INFO-I433) \\
{\em Associate Instructor} \hfill {\bf Spring 2020, Spring 2019} \\
{\bf Malware: Threat and Defense} (CSCI-B546 and INFO-I521) \\
{\em Associate Instructor} \hfill {\bf Fall 2019} \\

\section*{\sc Awards}

{\bf Luddy PhD Instructor Award}
\hfill {\bf 2024 -- 2025} \\
{\bf Research Assistant of the Year} (Computer Science)
\hfill {\bf 2023 -- 2024} \\

\section*{\sc Service}

{\bf \href{https://wonks.github.io/}{PL Wonks}} \\
{\em Organizer, Video Chair} \hfill {\bf Fall 2023 -- Present} \\
{\bf \href{https://wonks.github.io/plrg/}{The PL Reading Group (PLRG)}} \\
{\em Organizer} \hfill {\bf Fall 2021 -- Spring 2023}

\section*{\sc Software}

{\bf \href{https://github.com/Gradual-Typing/LambdaIFCStar}{Mechanized Gradual Information-Flow Control}}
\hfill {\bf December 2023}
\begin{itemize}
\item I designed two gradual information-flow control programming languages,
  with and without type-guided classification.
\item I mechanized the proofs of type safety, noninterference, and the gradual
  guarantee in Agda.
\end{itemize}
{\bf \href{https://github.com/jsiek/gradual-typing-in-agda}{Gradual Typing in Agda}}
\hfill {\bf August 2020}
\begin{itemize}
\item I contributed to the mechanized compendium of the gradually-typed
  $\lambda$-calculus (GTLC) and a variety of cast calculi.
\end{itemize}
{\bf \href{https://github.com/Gradual-Typing/lambda-sec/tree/master/glio}{Mechanization of GLIO}}
\hfill {\bf May 2020}
\begin{itemize}
\item I implemented a definitional interpreter for GLIO. GLIO is an experimental
  gradual security programming language proposed by researchers from Carnegie
  Mellon University.
\item I proved type safety for GLIO and mechanized the proof in Agda.
\end{itemize}


\section*{\sc Programming Skills}

{\bf Languages: } C, Java, Python, Racket, Agda, Coq \\
{\bf Tools: } Emacs, Shell scripting, Unix sysadmin, virtualization, Linux containers

\pagenumbering{gobble}
