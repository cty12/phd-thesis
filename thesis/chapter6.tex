\chapter{Gradual Guarantee of \Surface}
\label{ch:gg}

{\color{NavyBlue} %%% new text

In this chapter, I prove that \Surface satisfies the gradual guarantee
(Theorem~\ref{thm:gg}). The gradual guarantee says that if a \Surface program
successfully evaluates to some value, then a variant of the program where the
type annotations are less precise must also evaluate to the same value. In
Section~\ref{sec:sim-cexpr}, I prove a simulation lemma between more and less
precise coercion sequences: if the more precise side takes one step, the less
precise side can take zero or more steps and be related by precision again.
After that, in Section~\ref{sec:simulation}, I prove a similar simulation lemma
for the cast calculus \CC. Finally, in Section~\ref{sec:gg}, I prove the gradual
guarantee for \Surface as a corollary of the simulation lemma for \CC.

The Agda proofs cited in the section are available at:
\begin{center}
  \url{https://github.com/Gradual-Typing/LambdaIFCStar}
\end{center}
} %%% end new text

\section{Simulation Between More and Less Precise Coercion Sequences}
\label{sec:sim-cexpr}

\begin{figure}[tbp]
\raggedright
  \fbox{$\vdash c \sqsubseteq d$, $\vdash c \sqsubseteq g$ , and $\vdash g \sqsubseteq d$}
  {\small
  \begin{gather*}
    {\sqsubseteq}\textit{-c} ~
    \inference{g_1 \sqsubseteq g_1' & g_2 \sqsubseteq g_2' \\ \vdash c : g_1 \Rightarrow g_2 & \vdash d : g_1' \Rightarrow g_2'}
              {\vdash c \sqsubseteq d}
    \\[2ex]
    {\sqsubseteq}\textit{-cl} ~
    \inference{g_1 \sqsubseteq g & g_2 \sqsubseteq g \\ \vdash c : g_1 \Rightarrow g_2}
              {\vdash c \sqsubseteq g}
    \qquad
    {\sqsubseteq}\textit{-cr} ~
    \inference{g \sqsubseteq g_1 & g \sqsubseteq g_2 \\ \vdash d : g_1 \Rightarrow g_2}
              {\vdash g \sqsubseteq d}
  \end{gather*}}
  \fbox{$\vdash \bar{c} \sqsubseteq \bar{d}$}
  {\small
  \begin{gather*}
    {\sqsubseteq}\textit{-id} ~
    \inference{g \sqsubseteq g'}
              {\vdash \id{g} \sqsubseteq \id{g'}}
    \quad
    {\sqsubseteq}\textit{-cast} ~
    \inference{\vdash \bar{c} \sqsubseteq \bar{d} & \vdash c \sqsubseteq d}
              {\vdash \bar{c} \seq c \sqsubseteq \bar{d} \seq d}
    \\[2ex]
    {\sqsubseteq}\textit{-castl} ~
    \inference{\vdash \bar{c} \sqsubseteq \bar{d} & \vdash c \sqsubseteq g_2 \\ \vdash \bar{d} : g_1 \Rightarrow g_2}
              {\vdash \bar{c} \seq c \sqsubseteq \bar{d}}
    \quad
    {\sqsubseteq}\textit{-castr} ~
    \inference{\vdash \bar{c} \sqsubseteq \bar{d} & \vdash g_2 \sqsubseteq d \\ \vdash \bar{c} : g_1 \Rightarrow g_2}
              {\vdash \bar{c} \sqsubseteq \bar{d} \seq d}
    \\[2ex]
    {\sqsubseteq}\textit{-}\bot ~
    \inference{g_1 \sqsubseteq g_3 & g_2 \sqsubseteq g_4 & \vdash \bar{c} : g_1 \Rightarrow g_2}
              {\vdash \bar{c} \sqsubseteq \err{g_3}{g_4}{p}}
  \end{gather*}}
  \fbox{$\vdash \bar{c} \sqsubseteq_l g$}
  {\small
  \begin{gather*}
    {\sqsubseteq}_l\textit{-id} ~
    \inference{g \sqsubseteq g'}
              {\vdash \id{g} \sqsubseteq_l g'}
    \qquad\quad
    {\sqsubseteq}_l\textit{-cast} ~
    \inference{\vdash \bar{c} \sqsubseteq_l g & \vdash c \sqsubseteq_l g}
              {\vdash \bar{c} \seq c \sqsubseteq_l g}
  \end{gather*}}
 \fbox{$\vdash g \sqsubseteq_r \bar{d}$}
 {\small
  \begin{gather*}
    {\sqsubseteq}_r\textit{-id} ~
    \inference{g \sqsubseteq g'}
              {\vdash g \sqsubseteq_r \id{g'}}
    \quad
    {\sqsubseteq}_r\textit{-cast} ~
    \inference{\vdash g \sqsubseteq_r \bar{d} & \vdash g \sqsubseteq_r d}
              {\vdash g \sqsubseteq_r \bar{d} \seq d}
    \\[2ex]
    {\sqsubseteq}_r\textit{-}\bot ~
    \inference{g \sqsubseteq g_1 & g \sqsubseteq g_2}
              {\vdash g \sqsubseteq_r \err{g_1}{g_2}{p}}
  \end{gather*}}
  \caption{Precision relation of the coercion calculus}
  \label{fig:cexpr-prec}
\end{figure}

Our end goal is to prove the gradual guarantee for \Surface.
The proof depends on a simulation lemma between more and less precise terms of
the cast calculus \CC . We use coercion sequences as the IFC monitor in \CC.
Reducing a coercion sequence can result in a blame which errors the program. So
we would like to prove that the simulation lemma holds for the coercion calculus
on security labels.
%
The precision relation on security coercions is defined in
Figure~\ref{fig:cexpr-prec}. The precision relation between two coercion
sequences $\bar{c},\bar{d}$ takes the form $\vdash \bar{c} \sqsubseteq \bar{d}$.
Recall that the gradual guarantee states that replacing type annotations with
\unk (decreasing type precision) should result in the same value for a correctly
running program while adding annotations (increasing type precision) may trigger
more runtime errors. The precision relation is a syntactical characterization of
the runtime behaviors of programs of different type precision. We explain the
intuition with two examples.

\begin{example}
\label{ex:prec}
\normalfont

Consider the following two programs that are related by precision
because the first one has a $\unk$ annotation where the second one
has \high.
\[
\true_{\low} : \Bool_{\unk} : \Bool_{\unk} \qquad\text{ and }\qquad
\true_{\low} : \Bool_{\high} : \Bool_{\unk}
\]
At runtime, the less precise program on the left produces value
$(\cccast{\true}{\id{\low} \seq \inj{\low}})$ and the more precise program
on the right produces $(\cccast{\true}{\id{\low} \seq \up \seq \inj{\high}})$.
The \texttt{true}s are straightforwardly related; we need to show the two
coercion sequences are also related:
\[
\vdash \id{\low} \seq \inj{\low} \sqsubseteq \id{\low} \seq \up \seq \inj{\high}
\]
Starting at the beginning of the
two sequences, we have $\id{\low} \sqsubseteq \id{\low}$ because
$\sqsubseteq$ is reflexive. Next we have
$\inj{\low} \sqsubseteq \;\up$, which makes sense because the source
and targets of the two coercions are related by precision:
$\low \sqsubseteq \low$ and $\unk \sqsubseteq \high$.  Finally, the
coercion $\inj{\high}$ can be added to the end of the more precise
sequence because both its source and target type or more precise than
the target of the left-hand sequence.  That is,
$\unk \sqsubseteq \high$ and $\unk \sqsubseteq \unk$.
%
The injections at the ends of the two sequences, \inj{\low}
and \inj{\high}, cannot be directly related via precision because
$\low \not\sqsubseteq \high$.  Instead, \inj{\low} is related
with \up{}. This underlines the indispensability of explicit subtype
coercion \up{} for the purposes of proving the gradual guarantee.

\end{example}

\begin{example}
\label{ex:prec-error}
\normalfont

Next we consider an example where the less precise program produces a
value but the more precise program encounters an error. This situation
is allowed by the gradual guarantee, but the opposite one is not. We
extend the example with a cast to $\low$ on the more precise side.
\[
\true_{\low} : \Bool_{\unk} : \Bool_{\unk} : \Bool_{\unk}
\qquad\text{ and }\qquad
\true_{\low} : \Bool_{\high} : \Bool_{\unk} : \Bool_{\low}
\]
The first program again produces value $(\cccast{\true}{\id{\low}\seq\inj{\low}})$.
The second, on the other hand, reduces to
$(\cccast{\true}{\id{\low} \seq \up \seq \highlightred{\inj{\high} \seq \proj{\low}{p}}}) \longrightarrow^{*}
(\cccast{\true}{\err{\low}{\low}{p}})$, because of the contradicting
annotations \high and \low (note that \high is in the middle of the sequence and both the
source and target labels of the blame coercion are \low so that types are preserved).
The failure is then propagated out and the term further reduces to \blame{p}.
The precision of coercion sequences relates $\boldsymbol{\bot}$ on the right-hand side
to any coercion sequence on the left so long as the respective source and target types are
related via precision, in this case $\low \sqsubseteq \low$ and $\unk \sqsubseteq \low$.
\end{example}

Consider the security levels (Definition~\ref{def:sec}) of both sides
of Example~\ref{ex:prec}, which are \low on the less precise side and
\high on the more precise side.
We observe that \Surface terms related by precision may produce
values of different security: a less precise value may have lower
security than a more precise value. Indeed, we prove the following for
coercions in normal form:

\begin{lemma}[Security is monotonic with respect to precision]
\label{lem:prec-sec}
Suppose $\mathbf{NF}\; \bar{c}$ and $\mathbf{NF}\; \bar{d}$. If $\vdash \bar{c} \sqsubseteq \bar{d}$, then $| \bar{c} | \preccurlyeq | \bar{d} |$.
\end{lemma}
\begin{proof}
  The proof is mechanized in \texttt{/src/CoercionExpr/SecurityLevel.agda}.
\end{proof}

Next we prove a catch-up lemma for coercion sequences, where the less
precise side catches up with a more precise sequence that is in normal
form. The proof is by casing on $\mathbf{NF}\;\bar{d}$ first and then
performing induction on the precision relation in each case.

\begin{lemma}[Catching up to a more precise coercion sequence]
\label{lem:cexpr-catchup}
If $\mathbf{NF}\; \bar{d}$ and $\vdash \bar{c} \sqsubseteq \bar{d}$,
there exists $\bar{c}'$ such that $\bar{c} \longrightarrow^{*} \bar{c}'$,
$\mathbf{NF}\; \bar{c}'$, and $\vdash \bar{c}' \sqsubseteq \bar{d}$.
\end{lemma}
\begin{proof}
  The proof is mechanized in \texttt{/src/CoercionExpr/CatchUp.agda}.
\end{proof}

\noindent Using Lemma~\ref{lem:cexpr-catchup}, we then prove the following
simulation lemma for coercion sequences:

\begin{lemma}[Simulation between related coercion sequences]
\label{lem:cexpr-sim}
If $\vdash \bar{c} \sqsubseteq \bar{d}$ and $\bar{d} \longrightarrow \bar{d}'$,
there exists $\bar{c}'$ such that $\bar{c} \longrightarrow^{*} \bar{c}'$
and $\vdash \bar{c}' \sqsubseteq \bar{d}'$.
\end{lemma}
\begin{proof}
  The proof is mechanized in \texttt{sim} of \texttt{/src/CoercionExpr/Simulation.agda}.
\end{proof}

We also prove that stamping on coercion sequences preserves precision:

\begin{lemma}[Stamping preserves precision of coercion sequences]
\label{lem:cexpr-stamp-sim}
If $\vdash \bar{c} \sqsubseteq \bar{d}$, then \\ $\vdash \mathit{stamp}\;\bar{c}\;\ell \sqsubseteq \mathit{stamp}\;\bar{d}\;\ell$
and $\vdash \mathit{stamp!}\;\bar{c}\;\ell_1 \sqsubseteq \mathit{stamp!}\;\bar{d}\;\ell_2$ and
$\vdash \mathit{stamp!}\;\bar{c}\;\ell_1 \sqsubseteq \mathit{stamp}\;\bar{d}\;\ell_2$ if $\ell_1 \preccurlyeq \ell_2$.
\end{lemma}
\begin{proof}
  The proof is mechanized in \texttt{/src/CoercionExpr/Stamping.agda}.
\end{proof}

\section{Simulation Between \CC Terms of Different Precision}
\label{sec:simulation}

\begin{figure}[tbp]
\raggedright
  \fbox{$\precctx{\Gamma}{\Gamma'}{\Sigma}{\Sigma'}{g}{g'}{\ell}{\ell'}\ccprec{M}{M'}{A}{A'}$}
  {\small
  \begin{gather*}
  {\sqsubseteq}\textit{-var}~
  \inference{\Gamma \ni x : A & \Gamma' \ni x : A'}
            {\precctx{\Gamma}{\Gamma'}{\Sigma}{\Sigma'}{g}{g'}{\ell}{\ell'}\ccprec{x}{x}{A}{A'}}
  \\[3ex]
  {\sqsubseteq}\textit{-const}~
  \inference{}
            {\precctx{\Gamma}{\Gamma'}{\Sigma}{\Sigma'}{g}{g'}{\ell}{\ell'}\ccprec{\ccconst{k}}{\ccconst{k}}{\iota_{\ell}}{\iota_{\ell}}}
  \\[3ex]
  {\sqsubseteq}\textit{-addr}~
  \inference{\Sigma(\hat{\ell}, n) = T & \Sigma'(\hat{\ell}, n) = T'}
  {\precctx{\Gamma}{\Gamma'}{\Sigma}{\Sigma'}{g}{g'}{\ell}{\ell'}\ccprec{\ccaddr{n}}{\ccaddr{n}}{(\Refer{T_{\hat{\ell}}})_\ell}{(\Refer{T'_{\hat{\ell}}})_\ell}}
  \\[3ex]
    {\sqsubseteq}\textit{-lam}~
    \inference{g_2 \sqsubseteq g_2' & A \sqsubseteq A' \\
               \forall \ell\,\ell'.\, \precctx{(\Gamma,x{:}A)}{(\Gamma',x{:}A')}{\Sigma}{\Sigma'}{g_2}{g_2'}{\ell}{\ell'}\ccprec{N}{N'}{B}{B'}}
              {\precctx{\Gamma}{\Gamma'}{\Sigma}{\Sigma'}{g_1}{g_1'}{\ell_1}{\ell_1'}\ccprec{\cclam{x}{N}}{\cclam{x}{N'}}{(\Fun{A}{g_2}{B})_{\ell_2}}{(\Fun{A'}{g_2'}{B'})_{\ell_2}}}
  \\[3ex]
    {\sqsubseteq}\textit{-app}~
    \inference{\precctx{\Gamma}{\Gamma'}{\Sigma}{\Sigma'}{\ell_1}{\ell_1}{\ell_2}{\ell_2'}\ccprec{L}{L'}{(\Fun{A}{\ell_1 \curlyvee \ell_3}{B})_{\ell_3}}{(\Fun{A'}{\ell_1 \curlyvee \ell_3}{B'})_{\ell_3}} \\
    \precctx{\Gamma}{\Gamma'}{\Sigma}{\Sigma'}{\ell_1}{\ell_1}{\ell_2}{\ell_2'}\ccprec{M}{M'}{A}{A'} \\
    C = \mathit{stamp}\;B\;\ell_3 & C' = \mathit{stamp}\;B'\;\ell_3}
              {\precctx{\Gamma}{\Gamma'}{\Sigma}{\Sigma'}{\ell_1}{\ell_1}{\ell_2}{\ell_2'}\ccprec{\ccapp{L}{M}{A}{B}{\ell_3}}{\ccapp{L'}{M'}{A'}{B'}{\ell_3}}{C}{C'}}
  \\[3ex]
  {\sqsubseteq}\textit{-app}{\star}~
    \inference{\precctx{\Gamma}{\Gamma'}{\Sigma}{\Sigma'}{g}{g'}{\ell}{\ell'}\ccprec{L}{L'}{(\Fun{A}{\unk}{T_{\unk}})_{\unk}}{(\Fun{A'}{\unk}{T'_{\unk}})_{\unk}} \\
      \precctx{\Gamma}{\Gamma'}{\Sigma}{\Sigma'}{g}{g'}{\ell}{\ell'}\ccprec{M}{M'}{A}{A'}}
              {\precctx{\Gamma}{\Gamma'}{\Sigma}{\Sigma'}{g}{g'}{\ell}{\ell'}\ccprec{\ccappstar{L}{M}{A}{T}}{\ccappstar{L'}{M'}{A'}{T'}}{T_{\unk}}{T'_{\unk}}}
  \\[3ex]
    {\sqsubseteq}\textit{-app${\star}$l}~
    \inference{\precctx{\Gamma}{\Gamma'}{\Sigma}{\Sigma'}{g}{\ell_1}{\ell_2}{\ell_2'}\ccprec{L}{L'}{(\Fun{A}{\unk}{T_{\unk}})_{\unk}}{(\Fun{A'}{\ell_1 \curlyvee \ell_3}{B'})_{\ell_3}} \\
                  \precctx{\Gamma}{\Gamma'}{\Sigma}{\Sigma'}{g}{\ell_1}{\ell_2}{\ell_2'}\ccprec{M}{M'}{A}{A'} \\
                  C' = \mathit{stamp}\;B'\;\ell_3}
              {\precctx{\Gamma}{\Gamma'}{\Sigma}{\Sigma'}{g}{\ell_1}{\ell_2}{\ell_2'}\ccprec{\ccappstar{L}{M}{A}{T}}{\ccapp{L'}{M'}{A'}{B'}{\ell_3}}{T_{\unk}}{C'}}
  \\[3ex]
  {\sqsubseteq}\textit{-let}~
  \inference{\precctx{\Gamma}{\Gamma'}{\Sigma}{\Sigma'}{g}{g'}{\ell_1}{\ell_1'}\ccprec{M}{M'}{A}{A'} \\
  \forall \ell_2\,\ell_2'.\, \precctx{(\Gamma,x{:}A)}{(\Gamma',x{:}A')}{\Sigma}{\Sigma'}{g}{g'}{\ell_2}{\ell_2'}\ccprec{N}{N'}{B}{B'}}
  {\precctx{\Gamma}{\Gamma'}{\Sigma}{\Sigma'}{g}{g'}{\ell_1}{\ell_1'}\ccprec{\cclet{x}{M}{A}{N}}{\cclet{x}{M'}{A'}{N'}}{B}{B'}}
  \end{gather*}}
  \caption{Precision rules of \CC (Part I)}
  \label{fig:cc-prec-1}
\end{figure}

\begin{figure}[tbp]
\raggedright
  \fbox{$\precctx{\Gamma}{\Gamma'}{\Sigma}{\Sigma'}{g}{g'}{\ell}{\ell'}\ccprec{M}{M'}{A}{A'}$}
  {\small
  \begin{gather*}
  {\sqsubseteq}\textit{-if}~
  \inference{\precctx{\Gamma}{\Gamma'}{\Sigma}{\Sigma'}{\ell_1}{\ell_1}{\ell_2}{\ell_2'}\ccprec{L}{L'}{\Bool_{\ell_3}}{\Bool_{\ell_3}} \\
      \forall \ell\,\ell'.\,\precctx{\Gamma}{\Gamma'}{\Sigma}{\Sigma'}{\ell_1 \curlyvee \ell_3}{\ell_1 \curlyvee \ell_3}{\ell}{\ell'}\ccprec{M}{M'}{A}{A'} \\
      \forall \ell\,\ell'.\,\precctx{\Gamma}{\Gamma'}{\Sigma}{\Sigma'}{\ell_1 \curlyvee \ell_3}{\ell_1 \curlyvee \ell_3}{\ell}{\ell'}\ccprec{N}{N'}{A}{A'} \\
      B = \mathit{stamp}\;A\;\ell_3 & B' = \mathit{stamp}\;A'\;\ell_3}
            {\precctx{\Gamma}{\Gamma'}{\Sigma}{\Sigma'}{\ell_1}{\ell_1}{\ell_2}{\ell_2'}\ccprec{\ccif{L}{A}{\ell_3}{M}{N}}{\ccif{L'}{A'}{\ell_3}{M'}{N'}}{B}{B'}}
  \\[3ex]
  {\sqsubseteq}\textit{-if}{\star}~
  \inference{\precctx{\Gamma}{\Gamma'}{\Sigma}{\Sigma'}{g}{g'}{\ell_1}{\ell_1'}\ccprec{L}{L'}{\Bool_{\unk}}{\Bool_{\unk}} \\
      \forall \ell_2\,\ell_2'.\,\precctx{\Gamma}{\Gamma'}{\Sigma}{\Sigma'}{\unk}{\unk}{\ell_2}{\ell_2'}\ccprec{M}{M'}{T_{\unk}}{T'_{\unk}} \\
      \forall \ell_2\,\ell_2'.\,\precctx{\Gamma}{\Gamma'}{\Sigma}{\Sigma'}{\unk}{\unk}{\ell_2}{\ell_2'}\ccprec{N}{N'}{T_{\unk}}{T'_{\unk}} }
            {\precctx{\Gamma}{\Gamma'}{\Sigma}{\Sigma'}{g}{g'}{\ell_1}{\ell_1'}\ccprec{\ccifstar{L}{T}{M}{N}}{\ccifstar{L'}{T'}{M'}{N'}}{T_{\unk}}{T'_{\unk}}}
  \\[3ex]
  {\sqsubseteq}\textit{-if${\star}$l}~
  \inference{\precctx{\Gamma}{\Gamma'}{\Sigma}{\Sigma'}{g}{\ell_1}{\ell_2}{\ell_2'}\ccprec{L}{L'}{\Bool_{\unk}}{\Bool_{\ell_3}} \\
    \forall \ell\,\ell'.\,\precctx{\Gamma}{\Gamma'}{\Sigma}{\Sigma'}{\unk}{\ell_1 \curlyvee \ell_3}{\ell}{\ell'}\ccprec{M}{M'}{T_{\unk}}{A'} \\
    \forall \ell\,\ell'.\,\precctx{\Gamma}{\Gamma'}{\Sigma}{\Sigma'}{\unk}{\ell_1 \curlyvee \ell_3}{\ell}{\ell'}\ccprec{N}{N'}{T_{\unk}}{A'} \\
    B' = \mathit{stamp}\;A'\;\ell_3}
            {\precctx{\Gamma}{\Gamma'}{\Sigma}{\Sigma'}{g}{\ell_1}{\ell_2}{\ell_2'}\ccprec{\ccifstar{L}{T}{M}{N}}{\ccif{L'}{A'}{\ell_3}{M'}{N'}}{T_{\unk}}{B'}}
  \\[3ex]
    {\sqsubseteq}\textit{-ref}~
    \inference{\precctx{\Gamma}{\Gamma'}{\Sigma}{\Sigma'}{\ell_1}{\ell_1}{\ell_2}{\ell_2'}\ccprec{M}{M'}{T_{\ell_3}}{T'_{\ell_3}} &
               \ell_1 \preccurlyeq \ell_3}
              {\precctx{\Gamma}{\Gamma'}{\Sigma}{\Sigma'}{\ell_1}{\ell_1}{\ell_2}{\ell_2'}\ccprec{\ccref{\ell_3}{M}}{\ccref{\ell_3}{M'}}{(\Refer{T_{\ell_3}})_{\low}}{(\Refer{T'_{\ell_3}})_{\low}}}
  \\[3ex]
    {\sqsubseteq}\textit{-ref?}~
    \inference{\precctx{\Gamma}{\Gamma'}{\Sigma}{\Sigma'}{\unk}{\unk}{\ell_1}{\ell_1'}\ccprec{M}{M'}{T_{\ell_2}}{T'_{\ell_2}}}
              {\precctx{\Gamma}{\Gamma'}{\Sigma}{\Sigma'}{\unk}{\unk}{\ell_1}{\ell_1'}\ccprec{\ccrefproj{\ell_2}{M}{p}}{\ccrefproj{\ell_2}{M'}{q}}{(\Refer{T_{\ell_2}})_{\low}}{(\Refer{T'_{\ell_2}})_{\low}}}
  \\[3ex]
  {\sqsubseteq}\textit{-ref?l}~
  \inference{\precctx{\Gamma}{\Gamma'}{\Sigma}{\Sigma'}{\unk}{\ell_1}{\ell_2}{\ell_3}\ccprec{M}{M'}{T_\ell}{T'_\ell} &
             \ell_1 \preccurlyeq \ell}
  {\precctx{\Gamma}{\Gamma'}{\Sigma}{\Sigma'}{\unk}{\ell_1}{\ell_2}{\ell_3}\ccprec{\ccrefproj{\ell}{M}{p}}{\ccref{\ell}{M'}}{(\Refer{T_\ell})_{\low}}{(\Refer{T'_\ell})_{\low}}}
  \\[3ex]
  {\sqsubseteq}\textit{-deref}~
  \inference{\precctx{\Gamma}{\Gamma'}{\Sigma}{\Sigma'}{g}{g'}{\ell_1}{\ell_1'}\ccprec{M}{M'}{(\Refer{A})_{\ell_2}}{(\Refer{A'})_{\ell_2}} \\
  B = \mathit{stamp}\;A\;\ell_2 & B' = \mathit{stamp}\;A'\;\ell_2}
  {\precctx{\Gamma}{\Gamma'}{\Sigma}{\Sigma'}{g}{g'}{\ell_1}{\ell_1'}\ccprec{\ccderef{M}{A}{\ell_2}}{\ccderef{M'}{A'}{\ell_2}}{B}{B'}}
  \\[3ex]
    {\sqsubseteq}\textit{-deref}{\star}~
    \inference{\precctx{\Gamma}{\Gamma'}{\Sigma}{\Sigma'}{g}{g'}{\ell}{\ell'}\ccprec{M}{M'}{(\Refer{T_{\unk}})_{\unk}}{(\Refer{T'_{\unk}})_{\unk}}}
              {\precctx{\Gamma}{\Gamma'}{\Sigma}{\Sigma'}{g}{g'}{\ell}{\ell'}\ccprec{\ccderefstar{M}{T}}{\ccderefstar{M'}{T'}}{T_{\unk}}{T'_{\unk}}}
  \\[3ex]
    {\sqsubseteq}\textit{-deref${\star}$l}~
    \inference{\precctx{\Gamma}{\Gamma'}{\Sigma}{\Sigma'}{g}{g'}{\ell}{\ell'}\ccprec{M}{M'}{(\Refer{T_{\unk}})_{\unk}}{(\Refer{A'})_{\ell_2}} \\
               B' = \mathit{stamp}\;A'\;\ell_2}
              {\precctx{\Gamma}{\Gamma'}{\Sigma}{\Sigma'}{g}{g'}{\ell_1}{\ell_1'}\ccprec{\ccderefstar{M}{T}}{\ccderef{M'}{A'}{\ell_2}}{T_{\unk}}{B'}}
  \end{gather*}}
  \caption{Precision rules of \CC (Part II)}
  \label{fig:cc-prec-2}
\end{figure}

\begin{figure}[tbp]
\raggedright
  \fbox{$\precctx{\Gamma}{\Gamma'}{\Sigma}{\Sigma'}{g}{g'}{\ell}{\ell'}\ccprec{M}{M'}{A}{A'}$}
  {\small
  \begin{gather*}
  {\sqsubseteq}\textit{-assign}~
  \inference{\precctx{\Gamma}{\Gamma'}{\Sigma}{\Sigma'}{\ell_1}{\ell_1}{\ell_2}{\ell_2'}\ccprec{L}{L'}{(\Refer{T_{\hat{\ell}}})_\ell}{(\Refer{T'_{\hat{\ell}}})_\ell} \\
             \precctx{\Gamma}{\Gamma'}{\Sigma}{\Sigma'}{\ell_1}{\ell_1}{\ell_2}{\ell_2'}\ccprec{M}{M'}{T_{\hat{\ell}}}{T'_{\hat{\ell}}} \\
             \ell_1 \preccurlyeq \hat{\ell} & \ell \preccurlyeq \hat{\ell}}
            {\begin{align*}\precctx{\Gamma}{\Gamma'}{\Sigma}{\Sigma'}{\ell_1}{\ell_1}{\ell_2}{\ell_2'}
                \vdash \ccassign{L}{M}{T}{\hat{\ell}}{\ell} \sqsubseteq \ccassign{L'}{M'}{T'}{\hat{\ell}}{\ell} \Leftarrow \\
                \Unit_{\low} \sqsubseteq \Unit_{\low}
            \end{align*}}
  \\[4ex]
  {\sqsubseteq}\textit{-assign?}~
    \inference{\precctx{\Gamma}{\Gamma'}{\Sigma}{\Sigma'}{g}{g'}{\ell}{\ell'}\ccprec{L}{L'}{(\Refer{T_{\hat{g}}})_{\unk}}{(\Refer{T'_{\hat{g}'}})_{\unk}} \\
      \precctx{\Gamma}{\Gamma'}{\Sigma}{\Sigma'}{g}{g'}{\ell}{\ell'}\ccprec{M}{M'}{T_{\hat{g}}}{T'_{\hat{g}'}} }
              {\begin{align*}\precctx{\Gamma}{\Gamma'}{\Sigma}{\Sigma'}{g}{g'}{\ell}{\ell'}
                  \vdash \ccassignproj{L}{M}{T}{\hat{g}}{p} \sqsubseteq \ccassignproj{L'}{M'}{T'}{\hat{g}'}{q} \Leftarrow \\
                  \Unit_{\low} \sqsubseteq \Unit_{\low}
              \end{align*}}
  \\[4ex]
    {\sqsubseteq}\textit{-assign?l}~
    \inference{\precctx{\Gamma}{\Gamma'}{\Sigma}{\Sigma'}{g}{\ell_1}{\ell_2}{\ell_2'}\ccprec{L}{L'}{(\Refer{T_{\hat{g}}})_{\unk}}{(\Refer{T'_{\hat{\ell}}})_\ell} \\
      \precctx{\Gamma}{\Gamma'}{\Sigma}{\Sigma'}{g}{\ell_1}{\ell_2}{\ell_2'}\ccprec{M}{M'}{T_{\hat{g}}}{T'_{\hat{\ell}}} \\
      \ell_1 \preccurlyeq \hat{\ell} & \ell \preccurlyeq \hat{\ell}}
              {\begin{align*}\precctx{\Gamma}{\Gamma'}{\Sigma}{\Sigma'}{g}{\ell_1}{\ell_2}{\ell_2'}
                  \vdash \ccassignproj{L}{M}{T}{\hat{g}}{p} \sqsubseteq \ccassign{L'}{M'}{T'}{\hat{\ell}}{\ell} \Leftarrow \\
                  \Unit_{\low} \sqsubseteq \Unit_{\low}
              \end{align*}}
  \\[4ex]
  {\sqsubseteq}\textit{-prot}~
    \inference
        {\precctx{\Gamma}{\Gamma'}{\Sigma}{\Sigma'}{g_1}{g_1'}{|\PC|}{|\PC'|}\ccprec{M}{M'}{A}{A'} &
          \ccprec{\PC}{\PC'}{g_1}{g_1'} \\
          \ell_1 \curlyvee \ell_2 \preccurlyeq |\PC| & \ell'_1 \curlyvee \ell_2 \preccurlyeq |\PC'| &
          B = \mathit{stamp}\;A\;\ell_2 & B' = \mathit{stamp}\;A'\;\ell_2}
        {\precctx{\Gamma}{\Gamma'}{\Sigma}{\Sigma'}{g_2}{g_2'}{\ell_1}{\ell'_1}\ccprec{\ccprot{\PC}{\ell_2}{M}{A}}{\ccprot{\PC'}{\ell_2}{M'}{A'}}{B}{B'}}
  \\[4ex]
  {\sqsubseteq}\textit{-prot!}~
  \inference
  {\precctx{\Gamma}{\Gamma'}{\Sigma}{\Sigma'}{g_1}{g_1'}{|\PC|}{|\PC'|}\ccprec{M}{M'}{T_{\unk}}{T'_{\unk}} &
   \ccprec{\PC}{\PC'}{g_1}{g_1'} \\
   \ell_1 \curlyvee \ell_2 \preccurlyeq |\PC| & \ell'_1 \curlyvee \ell'_2 \preccurlyeq |\PC'| &
   \highlight{highlight}{\ell_2 \preccurlyeq \ell_2'}}
  {\precctx{\Gamma}{\Gamma'}{\Sigma}{\Sigma'}{g_2}{g_2'}{\ell_1}{\ell'_1}\ccprec{\ccprot{\PC}{\ell_2}{M}{T_{\unk}}}{\ccprot{\PC'}{\ell'_2}{M'}{T'_{\unk}}}{T_{\unk}}{T'_{\unk}}}
  \\[4ex]
  {\sqsubseteq}\textit{-prot!l}~
  \inference
      {\precctx{\Gamma}{\Gamma'}{\Sigma}{\Sigma'}{g_1}{g_1'}{|\PC|}{|\PC'|}\ccprec{M}{M'}{T_{\unk}}{A} &
          \ccprec{\PC}{\PC'}{g_1}{g_1'} \\
          \ell_1 \curlyvee \ell_2 \preccurlyeq |\PC| & \ell'_1 \curlyvee \ell'_2 \preccurlyeq |\PC'| &
          B = \mathit{stamp}\;A\;\ell_2' &
          \highlight{highlight}{\ell_2 \preccurlyeq \ell_2'}}
      {\precctx{\Gamma}{\Gamma'}{\Sigma}{\Sigma'}{g_2}{g_2'}{\ell_1}{\ell'_1}\ccprec{\ccprot{\PC}{\ell_2}{M}{T_{\unk}}}{\ccprot{\PC'}{\ell'_2}{M'}{A}}{T_{\unk}}{B}}
  \end{gather*}}
  \caption{Precision rules of \CC (Part III)}
  \label{fig:cc-prec-3}
\end{figure}

\begin{figure}[tbp]
\raggedright
  \fbox{$\precctx{\Gamma}{\Gamma'}{\Sigma}{\Sigma'}{g}{g'}{\ell}{\ell'}\ccprec{M}{M'}{A}{A'}$}
  {\small
  \begin{gather*}
  {\sqsubseteq}\textit{-cast}~
  \inference
  {\precctx{\Gamma}{\Gamma'}{\Sigma}{\Sigma'}{g}{g'}{\ell}{\ell'}\ccprec{M}{M'}{A}{A'} & \bm{c} \sqsubseteq \bm{c'} \\
   \vdash \bm{c} : A \Rightarrow B & \vdash \bm{c'} : A' \Rightarrow B'}
  {\precctx{\Gamma}{\Gamma'}{\Sigma}{\Sigma'}{g}{g'}{\ell}{\ell'}\ccprec{\cccast{M}{\bm{c}}}{\cccast{M'}{\bm{c'}}}{B}{B'}}
  \\[4ex]
  {\sqsubseteq}\textit{-castl}~
  \inference
  {\precctx{\Gamma}{\Gamma'}{\Sigma}{\Sigma'}{g}{g'}{\ell}{\ell'}\ccprec{M}{M'}{A}{A'} &
   \bm{c} \sqsubseteq A' & \vdash \bm{c} : A \Rightarrow B}
  {\precctx{\Gamma}{\Gamma'}{\Sigma}{\Sigma'}{g}{g'}{\ell}{\ell'}\ccprec{\cccast{M}{\bm{c}}}{M'}{B}{A'}}
  \\[4ex]
  {\sqsubseteq}\textit{-castr}~
  \inference
  {\precctx{\Gamma}{\Gamma'}{\Sigma}{\Sigma'}{g}{g'}{\ell}{\ell'}\ccprec{M}{M'}{A}{A'} &
   A \sqsubseteq \bm{c'} & \vdash \bm{c'} : A' \Rightarrow B'}
  {\precctx{\Gamma}{\Gamma'}{\Sigma}{\Sigma'}{g}{g'}{\ell}{\ell'}\ccprec{M}{\cccast{M'}{\bm{c'}}}{A}{B'}}
  \\[4ex]
  {\sqsubseteq}\textit{-blame}~
  \inference
  {\Gamma; \Sigma; g; \ell \vdash M \Leftarrow A & A \sqsubseteq A'}
  {\precctx{\Gamma}{\Gamma'}{\Sigma}{\Sigma'}{g}{g'}{\ell}{\ell'}\ccprec{M}{\blame{p}}{A}{A'}}
  \end{gather*}}
  \caption{Precision rules of \CC (Part IV)}
  \label{fig:cc-prec-4}
\end{figure}

The main simulation lemma says that if two terms are related by
\textit{precision} and the more precise side takes one step, then the less
precise side is able to multi-step and get back in sync. The precision relation
is in form
$\precctx{\Gamma}{\Gamma'}{\Sigma}{\Sigma'}{g}{g'}{\ell}{\ell'}\ccprec{M}{M'}{A}{A'}$,
where $\Gamma;\Sigma;g;\ell$ corresponds to the typing context, heap context,
type of \PC, and security of \PC of the less precise term $M$ and
$\Gamma';\Sigma';g';\ell'$ is for those of the more precise term $M'$. The types
of the two terms, $A$ and $A'$, are related by precision between types. The
intuition between this precision relation is that casts are allowed to appear in
different places between the more precise and the less precise \CC terms.
Moreover, the casts must be in shapes that preserve the precision of \Surface
(more or fewer static type annotations provided by the programmer). According to
the gradual guarantee, the more precise side is allowed to signal more blames,
so there is a rule (${\sqsubseteq}\textit{-blame}$) that relates \blame{p} to
any term $M$ on the less precise side as long as their types are in sync. We
list the precision rules for the cast calculus \CC in
Figure~\ref{fig:cc-prec-1},~\ref{fig:cc-prec-2},~\ref{fig:cc-prec-3},
and~\ref{fig:cc-prec-4}.

With the precision relation of \CC defined, we first state the catch-up lemma,
which catches up to a more-precise value by multi-stepping on the less-precise
side:

\begin{lemma}[Catching up to more precise]
\label{lem:catchup}
If term $M$ and value $V'$ are related by precision:
$$\precctx{\Gamma}{\Gamma'}{\Sigma}{\Sigma'}{g}{g'}{\ell}{\ell'}\ccprec{M}{V'}{A}{A'}$$
then there exists value $V$ s.t $M\mid\mu\mid\PC \longrightarrow^{*} V\mid\mu$ and
$\precctx{\Gamma}{\Gamma'}{\Sigma}{\Sigma'}{g}{g'}{\ell}{\ell'}\ccprec{V}{V'}{A}{A'}$.
\end{lemma}
\begin{proof}
  The proof is mechanized in \texttt{/src/Simulation/CatchUp.agda}.
\end{proof}

It is straightforward to show that substitution preserves precision for \CC
(typing context precision is defined point-wise):

\begin{lemma}[Substitution preserves precision]
  \label{lem:subst-pres-prec}
  Suppose the typing contexts are related by precision: $\Gamma \sqsubseteq
  \Gamma$ and $\Sigma \sqsubseteq \Sigma'$. If
  $$\precctx{(\Gamma,x{:}A)}{(\Gamma',x{:}A')}{\Sigma}{\Sigma'}{g}{g'}{\ell}{\ell'}\ccprec{N}{N'}{B}{B'}$$
  and
  $$\forall g\,g'\,\ell\,\ell'.\, \precctx{\Gamma}{\Gamma'}{\Sigma}{\Sigma'}{g}{g'}{\ell}{\ell'}\ccprec{V}{V'}{A}{A'}$$
  then
  $\precctx{\Gamma}{\Gamma'}{\Sigma}{\Sigma'}{g}{g'}{\ell}{\ell'}\ccprec{N[x:=V]}{N'[x:=V']}{B}{B'}$.
\end{lemma}
\begin{proof}
  The proof is mechanized in \texttt{/src/CC2/SubstPrecision.agda}.
\end{proof}

We then prove the main simulation lemma using Lemma~\ref{lem:catchup}
("catch-up") and Lemma~\ref{lem:subst-pres-prec} (substitution preserves
precision). Heap precision is defined point-wise, similar to the definition of
heap well-typedness.

\begin{lemma}[Simulation between more precise and less precise \CC terms]
\label{lem:sim}
Suppose \PC, $\PC'$ are related by precision: $\ccprec{\PC}{\PC'}{g}{g'}$.
Moreover suppose $M$, $M'$ are related by precision:
$$\precctx{\emptyset}{\emptyset}{\Sigma_1}{\Sigma_1'}{g}{g'}{|\PC|}{|\PC'|}\ccprec{M}{M'}{A}{A'}$$
heap contexts $\Sigma_1$, $\Sigma_1'$ are related by precision:
$\Sigma_1 \sqsubseteq \Sigma_1'$, the initial heaps $\mu_1$, $\mu_1'$
are also related by precision: $\Sigma_1 ; \Sigma_1' \vdash \mu_1 \sqsubseteq \mu_1'$. \\
If \reduce{M'}{\mu_1'}{\PC'}{N'}{\mu_2'}, there exists $\Sigma_2$,
$\Sigma_2'$, $N$, $\mu_2$ s.t $\Sigma_2 \supseteq \Sigma_1$,
$\Sigma_2' \supseteq \Sigma_1'$, $\Sigma_2 \sqsubseteq \Sigma_2'$ ,
$$M\mid\mu_1\mid\PC \longrightarrow^{*} N\mid\mu_2$$
the resulting terms are related by precision:
$\precctx{\emptyset}{\emptyset}{\Sigma_2}{\Sigma_2'}{g}{g'}{|\PC|}{|\PC'|}\ccprec{N}{N'}{A}{A'}$
and the resulting heaps are also related by precision: $\Sigma_2 ; \Sigma_2' \vdash \mu_2 \sqsubseteq \mu_2'$.
\end{lemma}
\begin{proof}
  The proof is mechanized in \texttt{/src/Simulation/Simulation.agda}.
\end{proof}

\section{The Gradual Guarantee of \Surface}
\label{sec:gg}

\begin{figure}[tbp]
  \raggedright
  \fbox{$\vdash M_1 \sqsubseteq M_2$}
  {\small
  \begin{gather*}
    % const
    {\sqsubseteq}^G\textit{-const}~
    \inference{}{\vdash \const{k}{\ell} \sqsubseteq \const{k}{\ell}}
    \qquad
    % var
    {\sqsubseteq}^G\textit{-var}~
    \inference{}{\vdash x \sqsubseteq x}
    \\[2ex]
    % lam
    {\sqsubseteq}^G\textit{-lam}~
    \inference{g_1 \sqsubseteq g_2 & A_1 \sqsubseteq A_2 & \vdash N_1 \sqsubseteq N_2 }
              {\vdash \lam{g_1}{x}{A_1}{N_1}{\ell} \sqsubseteq \lam{g_2}{x}{A_2}{N_2}{\ell}}
    \quad
    % app
    {\sqsubseteq}^G\textit{-app}~
    \inference{\vdash L_1 \sqsubseteq L_2 & \vdash M_1 \sqsubseteq M_2}
              {\vdash \app{L_1}{M_1}{p} \sqsubseteq \app{L_2}{M_2}{p}}
    \\[2ex]
    % if
    {\sqsubseteq}^G\textit{-if}~
    \inference{\vdash L_1 \sqsubseteq L_2 & \vdash M_1 \sqsubseteq M_2 & \vdash N_1 \sqsubseteq N_2}
              {\vdash \ifexp{L_1}{M_1}{N_1}{p} \sqsubseteq \ifexp{L_2}{M_2}{N_2}{p}}
    \\[2ex]
    % ann
    {\sqsubseteq}^G\textit{-ann}~
    \inference{\vdash M_1 \sqsubseteq M_2 & A_1 \sqsubseteq A_2}
              {\vdash \ann{M_1}{A_1}{p} \sqsubseteq \ann{M_2}{A_2}{p}}
    \\[2ex]
    % let
    {\sqsubseteq}^G\textit{-let}~
    \inference{\vdash M_1 \sqsubseteq M_2 & \vdash N_1 \sqsubseteq N_2}
              {\vdash \letexp{x}{M_1}{N_1} \sqsubseteq \letexp{x}{M_2}{N_2}}
    \\[2ex]
    % ref
    {\sqsubseteq}^G\textit{-ref}~
    \inference{\vdash M_1 \sqsubseteq M_2}
              {\vdash \refexp{\ell}{M_1}{p} \sqsubseteq \refexp{\ell}{M_2}{p}}
    \quad
    % deref
    {\sqsubseteq}^G\textit{-deref}~
    \inference{\vdash M_1 \sqsubseteq M_2}
              {\vdash \deref{M_1}{p} \sqsubseteq \deref{M_2}{p}}
    \\[2ex]
    % assign
    {\sqsubseteq}^G\textit{-assign}~
    \inference{\vdash L_1 \sqsubseteq L_2 & \vdash M_1 \sqsubseteq M_2}
              {\vdash \assign{L_1}{M_1}{p} \sqsubseteq \assign{L_2}{M_2}{p}}
  \end{gather*}}
  \caption{Precision rules of \Surface}
  \label{fig:surface-precision}
\end{figure}

The definition of term precision for \Surface is in
Figure~\ref{fig:surface-precision}. We first prove that the compilation from
\Surface to \CC preserves types:

\begin{lemma}[Compilation preserves precision]
  \label{lem:compile-pres-prec}
  Suppose the tying contexts are related by precision: $\Gamma \sqsubseteq
  \Gamma'$, and the PC labels are also related: $g \sqsubseteq g'$. If
  well-typed \Surface terms $M, M'$ are related by precision: $\vdash M
  \sqsubseteq M'$ and $\Gamma ; g \vdash M : A$ and $\Gamma'; g' \vdash M' :
  A'$, then
  $$\forall \ell\, \ell'.\, \precctx{\Gamma}{\Gamma'}{\emptyset}{\emptyset}{g}{g'}{\ell}{\ell'}\ccprec{\compile{M}}{\compile{M'}}{A}{A'}$$
\end{lemma}
\begin{proof}
  The proof is mechanized in \texttt{/src/Compile/Precision/CompilePrecision.agda}.
\end{proof}

{\color{NavyBlue} %%% new text
We prove the following lemma about \Surface that is a corollary of
Lemma~\ref{lem:sim}:

\begin{lemma}
\label{lem:gg}
Suppose $M$ and $M'$ are well-typed terms in \Surface that are related by
precision, that is $\vdash M \sqsubseteq M'$ and $(x{:}\Bool_{\high}); \low \vdash
M : A$ and $(x{:}\Bool_{\high}); \low \vdash M' : A'$, so $A \sqsubseteq A'$.
%
If the compilation of $M'$ substituted with input $b$ reduces to a value:
$$(\compile{M'})[x:=\ccconst{b}]\mid\emptyset\mid\low \longrightarrow^{*} V'\mid\mu'$$
there exists some value $V$ and heap $\mu$ s.t. the compilation of $M$
substituted with $b$ reduces to $V$:
$$(\compile{M})[x:=\ccconst{b}]\mid\emptyset\mid\low \longrightarrow^{*} V\mid\mu$$
and the resulting values are related by precision for some $\Sigma$, $\Sigma'$:
$$\precctx{\emptyset}{\emptyset}{\Sigma}{\Sigma'}{\low}{\low}{\low}{\low}\ccprec{V}{V'}{A}{A'}$$
\end{lemma}
\begin{proof}
  By Lemma~\ref{lem:compile-pres-prec} (compilation preserves precision), we have
  $$\precctx{(x{:}\Bool_{\high})}{(x{:}\Bool_{\high})}{\emptyset}{\emptyset}{\low}{\low}{\low}{\low}\ccprec{\compile{M}}{\compile{M'}}{A}{A'}$$
  Substitution preserves precision (Lemma~\ref{lem:subst-pres-prec}), so
  $$\precctx{\emptyset}{\emptyset}{\emptyset}{\emptyset}{\low}{\low}{\low}{\low}\ccprec{(\compile{M})[x:=\ccconst{b}]}{(\compile{M'})[x:=\ccconst{b}]}{A}{A'}$$
  We then proceed by induction on the reduction of $(\compile{M'})[x:=\ccconst{b}]$ to a value
  $V'$, using Lemma~\ref{lem:sim} to show that $(\compile{M})[x:=\ccconst{b}]$ reduces to a
  corresponding term at each step. So we have $(\compile{M})[x:=\ccconst{b}] \longrightarrow^{*}
  N$ where $N \sqsubseteq V'$ for some $N$. We then apply
  Lemma~\ref{lem:catchup} to show that $N$ reduces to a value $V$ where $V
  \sqsubseteq V'$.
\end{proof}

Finally, we state and prove the gradual guarantee of \Surface as a corollary of Lemma~\ref{lem:gg}:

\begin{theorem}[Gradual guarantee of \Surface]
  \label{thm:gg}
  Suppose $M$ and $M'$ are \Surface programs related by precision: $\vdash M \sqsubseteq M'$.
  If $\mathit{eval}(M', b_1)=b_2$, then $\mathit{eval}(M, b_1)=b_2$.
\end{theorem}
\begin{proof}
  By Definition~\ref{def:surface-program} (whole programs of \Surface),
  $(x{:}\Bool_{\high});\low\vdash M : \Bool_{\low}$ and
  $(x{:}\Bool_{\high});\low\vdash M' : \Bool_{\low}$. By $\mathit{eval}(M',
  b_1)=b_2$ and inversion on the definition of \textit{eval}
  (Figure~\ref{fig:eval}), we have
  $\reducemult{(\compile{M'})[x:=\ccconst{b_1}]}{\emptyset}{\low}{\ccconst{b_2}}{\mu'}$
  for some $\mu'$. By Lemma~\ref{lem:gg}, there exists some $V, \mu$ such that
  $$(\compile{M})[x:=\ccconst{b_1}]\mid\emptyset\mid\low \longrightarrow^{*} V\mid\mu$$
  and
  $$\precctx{\emptyset}{\emptyset}{\Sigma}{\Sigma'}{\low}{\low}{\low}{\low}\ccprec{V}{\ccconst{b_2}}{\Bool_{\low}}{\Bool_{\low}}$$
  By inversion on the precision of \CC and by the canonical form of a value of $\Bool_{\low}$, $V = \ccconst{b_2}$.
  By the definition of \textit{eval}, $\mathit{eval}(M, b_1)=b_2$.
\end{proof}
} %%% end new text
