\chapter{Noninterference of \Surface}
\label{ch:noninterference}

%{\color{NavyBlue} %%% new text


In this chapter, I prove that \Surface satisfies noninterference
(Theorem~\ref{thm:NI-Surface}). Noninterference says that high-security input
never flows into low-security output.

I show that security checks can be modeled by reducing security coercion
sequences to their normal form in Section~\ref{sec:norm-IF}. After that, I
present the full proof of noninterference for \Surface. This proof employs a
three-step approach.

First, in Section~\ref{sec:NI-dynifc}, I prove that \DynIFC (the dynamic extreme
of \Surface) satisfies noninterference. This proof is a straightforward
adaptation of the noninterference proof from \textcite{Chen:2022aa}. The proof
(Lemma~\ref{lem:NI-old}) uses a standard erasure-based
approach~\parencite{LI20101974, stefan2011flexible, stefan2012flexible,
  Fennell:2013ab, STEFAN:2017ta}, where the high-security parts of a program are
erased to an opaque value.

Second, in Section~\ref{sec:sim-cc-dynifc}, I prove a simulation lemma between
\CC and \DynIFC (Lemma~\ref{lem:sim-leq}). The simulation relation is defined in
Figures~\ref{fig:sim-rel-1} and~\ref{fig:sim-rel-2}; the intuition is that a \CC
term always produces a value that is as secure as the one produced by its
related \DynIFC term. I translate \CC terms to \DynIFC by (1) getting rid of all
the casts and (2) converting static heap enforcement (\texttt{ref} and
\texttt{assign}) to dynamic enforcement (NSU). The translation is defined in
Figure~\ref{fig:cast-erase}. The noninterference property of \CC
(Lemma~\ref{lem:NI-CC}) follows directly from the multi-step simulation lemma
(Lemma~\ref{lem:sim-mult}) and the noninterference result of \DynIFC.

Third, in Section~\ref{sec:NI}, I prove the noninterference theorem of \Surface
(Theorem~\ref{thm:NI-Surface}) as a corollary of the noninterference property of
\CC.

%} %%% end new text

Similar to \GSLRef and GLIO, the statements of noninterference for both \DynIFC
and \CC are \textit{termination-insensitive}. Thus, I will only consider
successful executions that produce values. In other words, I will not consider
reduction rules that trigger IFC monitor failures, such as NSU errors in \DynIFC
or cast errors in \CC. As I explained in Section~\ref{sec:example1}, the
programming language runtime forces the program to diverge whenever blame is
detected, possibly sending a private error message to the software developer.

The Agda proofs cited in the section are available at:
\begin{center}
  \url{https://github.com/Gradual-Typing/LambdaIFCStar}
\end{center}

\section{The Normalization of Coercions Checks Information Flow}
\label{sec:norm-IF}

I show that reducing coercion sequences to normal form models IFC checks because
the normalization either succeeds or fails. If a coercion sequences successfully
reduces to normal form, the IFC check succeeds and the flow is justified. If it
reduces to a failure, then an illegal flow is detected and the program errors.

\begin{lemma}[Strong normalization of coercion sequences]
If $\vdash \bar{c} : g_1 \Rightarrow g_2$, then either
(1) $\bar{c} \longrightarrow^{*} \bar{d}$ and $\mathbf{NF}\;\bar{d}$ or
(2) $\bar{c} \longrightarrow^{*} \err{g_1}{g_2}{p}$.
\end{lemma}
\begin{proof}
  The proof is in \texttt{cexpr-sn} of
  \texttt{/src/CoercionExpr/CoercionExpr.agda}
\end{proof}

\noindent Normalization of coercion sequences is deterministic:

\begin{lemma}[Reduction of coercion sequences is deterministic]
If $\bar{c} \longrightarrow \bar{d}_1$ and $\bar{c} \longrightarrow \bar{d}_2$,
then $\bar{d}_1 = \bar{d}_2$.
\end{lemma}
\begin{proof}
  The proof is in \texttt{det} of \texttt{/src/CoercionExpr/CoercionExpr.agda}.
\end{proof}

\begin{lemma}[Normalization of coercion sequences is deterministic]
Suppose $\bar{c} \longrightarrow^{*} \bar{d}_1$
and $\bar{c} \longrightarrow^{*} \bar{d}_2$.
If $\mathbf{NF}\;\bar{d}_i$ or $\bar{d}_i = \err{g_1}{g_2}{p}$,
then $\bar{d}_1 = \bar{d}_2$.
\end{lemma}
\begin{proof}
  The proof is in \texttt{det-mult} of \texttt{/src/CoercionExpr/CoercionExpr.agda}.
\end{proof}

Recall that in Section~\ref{sec:cexpr-comp-stamp}, we mention that coercion
composition models explicit flows, while coercion stamping models implicit flow.
Here we prove lemmas that formalize that intuition.

We reason about explicit flows first. If we compose one coercion
sequence with another and then reduce the result to normal form, the
security of the resulting coercion sequence should be greater than or
equal to that of the first sequence:

\begin{lemma}[Composition models explicit flow]\ \\
  \label{lem:comp-explicit}
  If $\mathbf{NF} \; \bar{c}$
  and $\bar{c} \mdoubleplus \bar{d} \longrightarrow^{*} \bar{c}'$
  and $\mathbf{NF} \; \bar{c}'$,
  then $|\bar{c}| \preccurlyeq |\bar{c}'|$.
\end{lemma}
\begin{proof}
  The proof is in \texttt{comp-security} of \texttt{/src/CoercionExpr/SecurityLevel.agda}.
\end{proof}

Next we show that stamping models implicit flow correctly, promoting
the security of the stamped coercion by joining it with the stamped
label:

\begin{lemma}[Stamping models implicit flow]\ \\
  \label{lem:stamp-implicit}
  If $\;\mathbf{NF}\; \bar{c}$,
  then $| \mathit{stamp} \; \bar{c} \; \ell| = |\bar{c}| \curlyvee \ell$
  and $| \mathit{stamp!} \; \bar{c} \; \ell| = |\bar{c}| \curlyvee \ell$.
\end{lemma}
\begin{proof}
  The proof is in \texttt{stamp$_{\mathtt{l}}$-security} of \texttt{/src/CoercionExpr/Stamping.agda}.
\end{proof}

\section{Noninterference of \DynIFC}
\label{sec:NI-dynifc}

\begin{figure}[tbp]
  \raggedright
  \fbox{\bigstep{\mu}{\pc}{M}{V}{\mu'}}
  {\small
  \begin{gather*}
  {\Downarrow}\textit{-val}~
  \inference{}{\bigstep{\mu}{\pc}{V}{V}{\mu}}
  \qquad
  {\Downarrow}\textit{-app}~
  \inference
  {\bigstep{\mu}{\pc}{L}{\dynlam{x}{N}{\ell}}{\mu_1} \\
   \bigstep{\mu_1}{\pc}{M}{V}{\mu_2} \\
   \bigstep{\mu_2}{\pc \curlyvee \ell}{N[ x := V ]}{W}{\mu_3}}
  {\bigstep{\mu}{\pc}{L \; M}{W \curlyvee \ell}{\mu_3}}
  \\[2ex]
  {\Downarrow}\textit{-if-true}~
  \inference
  {\bigstep{\mu}{\pc}{L}{\dynconst{\true}{\ell}}{\mu_1} \\
   \bigstep{\mu_1}{\pc \curlyvee \ell}{M}{V}{\mu_2}}
  {\bigstep{\mu}{\pc}{\dynif{L}{M}{N}}{V \curlyvee \ell}{\mu_2}}
  \quad
  {\Downarrow}\textit{-if-false}~
  \inference
  {\bigstep{\mu}{\pc}{L}{\dynconst{\false}{\ell}}{\mu_1} \\
   \bigstep{\mu_1}{\pc \curlyvee \ell}{N}{V}{\mu_2}}
  {\bigstep{\mu}{\pc}{\dynif{L}{M}{N}}{V \curlyvee \ell}{\mu_2}}
  \\[2ex]
    {\Downarrow}\textit{-ref?}~
    \inference
        {\bigstep{\mu}{\pc}{M}{V}{\mu_1} \\
          n \; \mathbf{FreshIn} \; \mu_1(\ell) & \highlightred{\pc \preccurlyeq \ell}}
        {\bigstep{\mu}{\pc}{\dynref{?}{\ell}{M}}{\dynaddr{n_{\ell}}{\low}}{(\mu_1, \ell \mapsto n \mapsto (V \curlyvee \ell))}}
  \\[2ex]
  {\Downarrow}\textit{-deref}~
  \inference
  {\bigstep{\mu}{\pc}{M}{\dynaddr{n_{\hat{\ell}}}{\ell}}{\mu_1} \\
    \mathit{\mu_1(\hat{\ell},n) = V}}
  {\bigstep{\mu}{\pc}{\dynderef{M}}{V \curlyvee \ell}{\mu_1}}
  \\[2ex]
  {\Downarrow}\textit{-assign?}~
  \inference
  {\bigstep{\mu}{\pc}{L}{\dynaddr{n_{\hat{\ell}}}{\ell}}{\mu_1} \\
   \bigstep{\mu_1}{\pc}{M}{V}{\mu_2} &
   \highlightred{\pc \curlyvee \ell \preccurlyeq \hat{\ell}}}
  {\bigstep{\mu}{\pc}{\dynassign{L}{?}{M}}{\dynconst{\unit}{\low}}{[\hat{\ell} \mapsto n \mapsto (V \curlyvee \hat{\ell})] \mu_2}}
  \end{gather*} }
  \caption{Big-step operational semantics (successful cases) of \DynIFC}
  \label{fig:big-step-dyn}
\end{figure}


%% \begin{figure}[tbp]
%%   \raggedright
%%   \fbox{$\reduce{M}{\mu}{\pc}{N}{\mu'}$}
%%   {\small
%%   \begin{gather*}
%%   \xi~
%%   \inference
%%   {\reduce{M}{\mu}{\pc}{M'}{\mu'}}
%%   {\reduce{\mathit{plug}\;M\;F}{\mu}{\pc}{\mathit{plug}\;M'\;F}{\mu'}}
%%   \\[1ex]
%%   \textit{prot-val}~
%%   \inference{}{\reduce{\dynprot{\ell}{V}}{\mu}{\pc}{V \curlyvee \ell}{\mu}}
%%   \quad
%%   \textit{prot-ctx}~
%%   \inference
%%   {\reduce{M}{\mu}{\pc \curlyvee \ell}{M'}{\mu'}}
%%   {\reduce{\dynprot{\ell}{M}}{\mu}{\pc}{\dynprot{\ell}{M'}}{\mu'}}
%%   \\[1ex]
%%   \beta~
%%   \inference{}{\reduce{\dynlam{x}{N}{\ell} \; V}{\mu}{\pc}{\dynprot{\ell}{(N [x := V])}}{\mu}}
%%   \\[1ex]
%%   \beta\textit{-if-true}~
%%   \inference{}{\reduce{\dynif{\dynconst{\true}{\ell}}{M}{N}}{\mu}{\pc}{\dynprot{\ell}{M}}{\mu}}
%%   \\[1ex]
%%   \beta\textit{-if-false}~
%%   \inference{}{\reduce{\dynif{\dynconst{\false}{\ell}}{M}{N}}{\mu}{\pc}{\dynprot{\ell}{N}}{\mu}}
%%   \\[1ex]
%%   \textit{ref?-ok}~
%%   \inference
%%   {\highlightred{\pc \preccurlyeq \ell} & n \; \mathbf{FreshIn} \; \mu(\ell)}
%%   {\reduce{\dynref{?}{\ell}{V}}{\mu}{\pc}{\dynaddr{n_{\ell}}{\low}}{(\mu, \ell \mapsto n \mapsto (V \curlyvee \ell))}}
%%   \\[1ex]
%%   \textit{deref}~
%%   \inference
%%   {\mathit{\mu(\hat{\ell},n) = V}}
%%   {\reduce{\dynderef{\dynaddr{n_{\hat{\ell}}}{\ell}}}{\mu}{\pc}{\dynprot{\ell}{V}}{\mu}}
%%   \\[1ex]
%%   \textit{assign?-ok}~
%%   \inference
%%   {\highlightred{\pc \curlyvee \ell \preccurlyeq \hat{\ell}}}
%%   {\reduce{\dynassign{\dynaddr{n_{\hat{\ell}}}{\ell}}{?}{V}}{\mu}{\pc}{\dynconst{\unit}{\low}}{[\hat{\ell} \mapsto n \mapsto (V \curlyvee \hat{\ell})] \mu}}
%%   \end{gather*}}
%%   \caption{Small-step operational semantics (successful cases) of \DynIFC}
%%   \label{fig:sem-succ-dyn}
%% \end{figure}

We obtain the big-step semantics in Figure~\ref{fig:big-step-dyn} by a
mechanical conversion from the successful (non-erroring) cases of the small-step
semantics of \DynIFC in Figure~\ref{fig:dyn-ifc}. I conjecture that small-step
and big-step semantics for \DynIFC coincide; in particular, multi-stepping to a
value should imply big-step:

\begin{lemma}
  \label{lem:mult-step-impl-big-step}
  If $\reducemult{M}{\mu}{\pc}{V}{\mu'}$, then $\bigstep{\mu}{\pc}{M}{V}{\mu'}$
\end{lemma}

\noindent The proof technique for Lemma~\ref{lem:mult-step-impl-big-step} should
be standard. We could first follow~\textcite{Streicher:2006aa} (the first
exercise about PCF on p. 17) and prove a lemma that if
$\reduce{M}{\mu_1}{\pc}{N}{\mu_2}$ and $\bigstep{\mu_2}{\pc}{N}{V}{\mu_3}$, then
$\bigstep{\mu_1}{\pc}{M}{V}{\mu_3}$. We then perform induction on
$\reducemult{M}{\mu}{\pc}{V}{\mu'}$: if $M$ is already a value, the goal is
proved directly; otherwise, $M$ takes at least one reduction step, so we use the
induction hypothesis and apply the aforementioned lemma.

\begin{lemma}[Noninterference of \DynIFC]
\label{lem:NI-old}
If
$\bigstep{\emptyset}{\low}{M [ x := \dynconst{b_1}{\high}]}{\dynconst{b_3}{\low}}{\mu_1}$
and
$\bigstep{\emptyset}{\low}{M [ x := \dynconst{b_2}{\high}]}{\dynconst{b_4}{\low}}{\mu_2}$
then $b_3 = b_4$.
\end{lemma}
\begin{proof}
The proof is fully mechanized in \texttt{/src/Dyn/Noninterference.agda}. The
structure of the proof directly follows the noninterference proof of
\CCOld~\parencite{Chen:2022aa}.
\end{proof}

\section{Simulation Between \CC and \DynIFC}
\label{sec:sim-cc-dynifc}

\begin{figure}[tbp]
  \raggedright
  \fbox{$\simrel{\gc}{M}{N}{A}$}
  \small
  \begin{gather*}
    % var
    {\leq}\textit{-var}~
    \inference{}{\simrel{\gc}{x}{x}{A}}
    \qquad
    % const
    {\leq}\textit{-const}~
    \inference{\ell' \preccurlyeq \ell}
              {\gc \vdash \dynconst{k}{\ell'} \leq \ccconst{k} \Leftarrow \iota_{\ell}}
    \\[2ex]
    {\leq}\textit{-wrapped-const}~
    \inference{\ell' \preccurlyeq |\bar{c}| & \vdash \bar{c} : \ell \Rightarrow g & \mathbf{NF}\;\bar{c} & \ell \neq g}
    {\gc \vdash \dynconst{k}{\ell'} \leq \cccast{\ccconst{k}}{\coerc{\id{\iota}}{\bar{c}}} \Leftarrow \iota_g}
    \\[2ex]
    % lambda
    {\leq}\textit{-lam}~
    \inference{\simrel{g}{N'}{N}{B} & \ell' \preccurlyeq \ell}
              {\simrel{\gc}{\dynlam{x}{N'}{\ell'}}{\cclam{x}{N}}{(\Fun{A}{g}{B})_{\ell}}}
    \\[2ex]
    {\leq}\textit{-wrapped-lam}~
    \inference{\simrel{g_3}{N'}{N}{D} & \ell \preccurlyeq |\bar{c}| \\ \vdash \bar{d} : g_1 \Rightarrow g_3 & \vdash \bm{c} : A \Rightarrow C & \vdash \bm{d} : D \Rightarrow B & \mathbf{NF}\;\bar{c}}
              {\simrel{\gc}{\dynlam{x}{N'}{\ell}}{\cccast{(\cclam{x}{N})}{\coerc{\funco{\bar{d}}{\bm{c}}{\bm{d}}}{\bar{c}}}}{(\Fun{A}{g_1}{B})_{g_2}}}
    \\[2ex]
    % addr
    {\leq}\textit{-addr}~
    \inference{\ell' \preccurlyeq \ell}
              {\simrel{\gc}{\dynaddr{n_{\hat{\ell}}}{\ell'}}{\ccaddr{n}}{\Refer{(T_{\hat{\ell}})}_{\ell}}}
    \\[2ex]
    {\leq}\textit{-wrapped-addr}~
    \inference{\ell \preccurlyeq |\bar{c}| & \vdash \bm{c} : T_{g_1} \Rightarrow S_{\hat{\ell}} & \vdash \bm{d} : S_{\hat{\ell}} \Rightarrow T_{g_1} & \mathbf{NF}\;\bar{c}}
              {\simrel{\gc}{\dynaddr{n_{\hat{\ell}}}{\ell}}{\cccast{(\ccaddr{n})}{\coerc{\refco{\bm{c}}{\bm{d}}}{\bar{c}}}}{(\Refer{(T_{g_1})})_{g_2}}}
    \\[2ex]
    % app
    {\leq}\textit{-app}~
    \inference{\simrel{\ell^c}{M'}{M}{(\Fun{A}{\ell^c \curlyvee \ell}{B})_\ell} & \simrel{\ell^c}{N'}{N}{A}}
              {\simrel{\ell^c}{M'\;N'}{\ccapp{M}{N}{A}{B}{\ell}}{C}}
    \\[2ex]
    % app*
    {\leq}\textit{-app}\star~
    \inference{\simrel{\gc}{M'}{M}{(\Fun{A}{\unk}{T_{\unk}})_{\unk}} & \simrel{\gc}{N'}{N}{A}}
              {\simrel{\gc}{M'\;N'}{\ccappstar{M}{N}{A}{T}{}}{T_{\unk}}}
    \\[2ex]
    % if
    {\leq}\textit{-if}~
    \inference{\simrel{\ell^c}{L'}{L}{\Bool_\ell} & \simrel{\ell^c \curlyvee \ell}{M'}{M}{A} & \simrel{\ell^c \curlyvee \ell}{N'}{N}{A}}
              {\simrel{\ell^c}{\dynif{L'}{M'}{N'}}{\ccif{L}{A}{\ell}{M}{N}}{B}}
    \\[2ex]
    % if*
    {\leq}\textit{-if}\star~
    \inference{\simrel{\gc}{L'}{L}{\Bool_{\unk}} & \simrel{\unk}{M'}{M}{T_{\unk}} & \simrel{\unk}{N'}{N}{T_{\unk}}}
              {\simrel{\gc}{\dynif{L'}{M'}{N'}}{\ccifstar{L}{T}{M}{N}}{T_{\unk}}}
    \\[2ex]
    % cast
    {\leq}\textit{-cast}~
    \inference{\simrel{\gc}{M}{N}{A} & \vdash \bm{c} : A \Rightarrow B}
              {\simrel{\gc}{M}{\cccast{N}{\bm{c}}}{B}}
    \\[2ex]
    % prot
    {\leq}\textit{-prot}~
    \inference{\ell' \preccurlyeq \ell & g_2 \vdash M' \leq M \Leftarrow A & \vdash PC \Leftarrow g_2}
    {g_1 \vdash \dynprot{\ell'}{M'} \leq \ccprot{\PC}{\ell}{M}{A} \Leftarrow B}
  \end{gather*}
  \caption{Simulation relation between \CC and \DynIFC (Part I)}
  \label{fig:sim-rel-1}
\end{figure}

\begin{figure}[tbp]
  \raggedright
  \fbox{$\simrel{\gc}{M}{N}{A}$}
  \small
  \begin{gather*}
    % ref
    {\leq}\textit{-ref}~
    \inference{\simrel{\ell^c}{M'}{M}{T_\ell}}
              {\simrel{\ell^c}{\dynref{?}{\ell}{M'}}{\ccref{\ell}{M}}{(\Refer{T_\ell})_{\low}}}
    \\[2ex]
    % ref?
    {\leq}\textit{-ref?}~
    \inference{\simrel{\unk}{M'}{M}{T_\ell}}
              {\simrel{\unk}{\dynref{?}{\ell}{M'}}{\ccrefproj{\ell}{M}{p}}{(\Refer{T_\ell})_{\low}}}
    \\[2ex]
    % deref
    {\leq}\textit{-deref}~
    \inference{\simrel{\gc}{M'}{M}{(\Refer{A})_\ell}}
              {\simrel{\gc}{\dynderef{M'}}{\ccderef{M}{A}{\ell}}{B}}
    \quad
    % deref*
    {\leq}\textit{-deref}\star~
    \inference{\simrel{\gc}{M'}{M}{(\Refer{T_{\unk}})_{\unk}}}
              {\simrel{\gc}{\dynderef{M'}}{\ccderefstar{M}{T}}{T_{\unk}}}
    \\[2ex]
    % assign
    {\leq}\textit{-assign}~
    \inference{\simrel{\ell^c}{L'}{L}{(\Refer{T_{\hat{\ell}}})_{\ell}} & \simrel{\ell^c}{M'}{M}{T_{\hat{\ell}}}}
              {\simrel{\ell^c}{\dynassign{L'}{?}{M'}}{\ccassign{L}{M}{T}{\hat{\ell}}{\ell}}{\Unit_{\low}}}
    \\[2ex]
    {\leq}\textit{-assign?}~
    \inference{\simrel{\gc}{L'}{L}{(\Refer{T_g})_{\unk}} & \simrel{\gc}{M'}{M}{T_g}}
              {\simrel{\gc}{\dynassign{L'}{?}{M'}}{\ccassignproj{L}{M}{T}{g}{p}}{\Unit_{\low}}}
  \end{gather*}
  \caption{Simulation relation between \CC and \DynIFC (Part II)}
  \label{fig:sim-rel-2}
\end{figure}

\begin{definition}[Simulation between heaps]
\label{def:sim-heap}
$\Sigma \vdash \mu' \leq \mu \triangleq \forall \ell, n.$ if $\mu(\ell,n) = V$, then there exists
$V'$ s.t $\mu'(\ell,n) = V'$ and $\simrel{\low}{V'}{V}{T_{\ell}}$ where $T = \Sigma(\ell,n)$.
\end{definition}

\begin{lemma}[Casting preserves simulation]
  \label{lem:cast-sim}
  If $\simrel{\gc}{W'}{V}{A}$, $\vdash \bm{c} : A \Rightarrow B$, \\
  and $\cccast{V}{\bm{c}} \longrightarrow^{*} W$, then $\simrel{\gc}{W'}{W}{B}$.
\end{lemma}
\begin{proof}
By induction on the multi-step cast reduction.
\begin{description}
\item[Zero step] Directly proved by applying rule ${\leq}\textit{-cast}$.
\item[One or more steps] Casing on the first reduction step yields three sub-cases:
\begin{description}
\item[\textit{cast}]
\begin{align}
\simrel{\gc}{W'}{V_r}{A} \\
\cccast{V_r}{\coerc{c_r}{\bar{c}}} \longrightarrow \cccast{V_r}{\coerc{c_r}{\bar{d}}} \longrightarrow^{*} W
\end{align}
By induction hypothesis, \simrel{\gc}{W'}{W}{B}.
\item[\textit{cast-id}]
\begin{align}
\simrel{\gc}{W'}{V_r}{A} \\
\cccast{V_r}{\coerc{\id{\iota}}{\id{g}}} \longrightarrow V_r
\end{align}
The goal is proved directly.
\item[\textit{cast-comp}]
\begin{align}
\simrel{\gc}{W'}{\cccast{V_r}{\bm{c}}}{A} \\
\cccast{\cccast{V_r}{\bm{c}}}{\bm{d}} \longrightarrow \cccast{V_r}{\bm{c} \mdoubleplus \bm{d}} \longrightarrow^{*} W
\end{align}
We further reason about $\cccast{V_r}{\bm{c} \mdoubleplus \bm{d}} \longrightarrow^{*} W$. The coercion
after composition $\bm{c} \mdoubleplus \bm{d}$ is reducible and not identity, so the reduction must
take one step by \textit{cast} and reduce the top-level coercion sequence to its normal form $\bar{c}_n$:
\[
\cccast{\cccast{V_r}{\coerc{c_{r1}}{\bar{c}}}}{\coerc{c_{r2}}{\bar{d}}} \longrightarrow \cccast{V_r}{\coerc{c_r}{\bar{c} \mdoubleplus \bar{d}}}
\longrightarrow \cccast{V_r}{\coerc{c_r}{\bar{c}_n}} \longrightarrow^{*} W
\]
We know $|\bar{c}| \preccurlyeq |\bar{c}_n|$ because composition models explicit flow (Lemma~\ref{lem:comp-explicit}).
\begin{itemize}
\item If $V_r = \ccconst{k}$ and $\bar{c}_n = \id{\ell}$. $W' = \dynconst{k}{\ell'}$
and $\ell' \preccurlyeq |\bar{c}|$ (Lemma~\ref{lem:sim-wrapped-const}).
We know $\simrel{\gc}{\dynconst{k}{\ell'}}{\ccconst{k}}{\iota_\ell}$ by rule ${\leq}\textit{-const}$,
because $\ell' \preccurlyeq |\bar{c}| \preccurlyeq |\bar{c}_n| = |\id{\ell}| = \ell$.
\item Otherwise, \cccast{V_r}{\coerc{c_r}{\bar{c}_n}} is already a value.
Apply rule ${\leq}\textit{-wrapped-const}$, ${\leq}\textit{-wrapped-lam}$, or ${\leq}\textit{-wrapped-addr}$
depending on whether $V_r$ is a constant, a $\lambda$, or an address.
\end{itemize}
\end{description}
\end{description}
\end{proof}

\begin{lemma}[Substitution preserves simulation]
  \label{lem:subst-sim}
  If $\simrel{g}{N'}{N}{A}$, $(\Gamma, x:B);\Sigma;g;\ell \vdash N \Leftarrow A$, and $\forall g. \simrel{g}{M'}{M}{B}$, then $\simrel{g}{N'[x:=M']}{N[x:=M]}{A}$.
\end{lemma}
\begin{proof}
  The proof is fully mechanized in \texttt{/src/Security/SubstPres.agda}. The
  simulation relation is defined in \texttt{/src/Security/SimRel.agda}.
\end{proof}

\begin{lemma}[Simulation with wrapped constant]
\label{lem:sim-wrapped-const}
If $\simrel{\gc}{M}{\cccast{(\ccconst{k})}{\coerc{\id{\iota}}{\bar{c}}}}{\iota_g}$
and $(\coerc{\id{\iota}}{\bar{c}})$ is irreducible,
then there exists $\ell$ s.t $M = \dynconst{k}{\ell}$, and $\ell \preccurlyeq |\bar{c}|$.
\end{lemma}
\begin{proof}
Inversion on the simulation relation:
\begin{description}
\item[Related by ${\leq}\textit{-wrapped-const}$] The goal is proved directly.
\item[Related by ${\leq}\textit{-cast}$] We know \simrel{\gc}{M}{\ccconst{k}}{\iota_\ell}
              and $\bar{c} : \ell \Rightarrow g$. Note that $(\coerc{\id{\iota}}{\bar{c}})$ is irreducible,
              so $\bar{c}$ can be \up, $\inj{\ell}$, or $\up\seq\inj{\high}$, so $\ell \preccurlyeq |\bar{c}|$.
              By rule ${\leq}\textit{-const}$, we know $M = \dynconst{k}{\ell'}$ for some $\ell'$
              and $\ell' \preccurlyeq \ell$.
              Thus $\ell' \preccurlyeq \ell \preccurlyeq |\bar{c}|$.
\end{description}
\end{proof}

\begin{lemma}[Simulation with reference proxy]
\label{lem:sim-ref-proxy}
If $\simrel{\gc}{M}{\cccast{(\ccaddr{n})}{\coerc{\refco{\bm{c}}{\bm{d}}}{\bar{c}}}}{(\Refer{T_{g_1}})_{g_2}}$
$\vdash \bm{c} : T_{g_1} \Rightarrow S_{\hat{\ell}}$, $\vdash \bm{d} : S_{\hat{\ell}} \Rightarrow T_{g_1}$,
and $\mathbf{NF}\;\bar{c}$, then there exists $\ell$ s.t $M = \dynaddr{n_{\hat{\ell}}}{\ell}$,
and $\ell \preccurlyeq |\bar{c}|$.
\end{lemma}
\begin{proof}
Analogous to Lemma~\ref{lem:sim-wrapped-const}. The only extra case to consider
is $\bar{c} = \id{\ell}$, which also satisfies $\ell \preccurlyeq |\bar{c}|$.
\end{proof}

\begin{lemma}[Simulation with function proxy]
\label{lem:sim-func-proxy}
If $\simrel{\gc}{M}{\cccast{(\cclam{x}{N})}{\coerc{\funco{\bar{d}}{\bm{c}}{\bm{d}}}{\bar{c}}}}{(\Fun{A}{g_1}{B})_{g_2}}$,
$\vdash \bar{d} : g_1 \Rightarrow g_3$,
and $\mathbf{NF}\;\bar{c}$, then there exists $N', \ell$ s.t $M = \dynlam{x}{N'}{\ell}$,
$\simrel{g_3}{N'}{N}{B}$, and $\ell \preccurlyeq |\bar{c}|$.
\end{lemma}
\begin{proof}
Analogous to Lemma~\ref{lem:sim-wrapped-const} and Lemma~\ref{lem:sim-ref-proxy}.
\end{proof}

\begin{lemma}[Simulation with \CC value]
\label{lem:sim-val}
If $\simrel{\gc}{M}{V}{A}$, then $M$ is a value.
\end{lemma}
\begin{proof}
Inversion on the value $V$ and the simulation relation $\simrel{\gc}{M}{V}{A}$ yields 6 cases.
We consider the two cases for constants; the cases for $\lambda$s and addresses are analogous.
\begin{description}
\item[Related by ${\leq}\textit{-const}$] We know $M = \dynconst{k}{\ell}$ for some $k, \ell$,
so $M$ is a value.
\item[Related by ${\leq}\textit{-wrapped-const}$] By Lemma~\ref{lem:sim-wrapped-const},
$M = \dynconst{k}{\ell}$ for some $k, \ell$, so $M$ is a value.
\end{description}
\end{proof}

\begin{lemma}[Stamping preserves simulation]
\label{lem:sim-stamp}
If $\simrel{\gc}{V}{W}{A}$ and $\ell' \preccurlyeq \ell$, then
$\simrel{\gc}{V \curlyvee \ell'}{\mathit{stamp}\;W\;\ell}{\mathit{stamp}\;A\;\ell}$.
\end{lemma}
\begin{proof}
Casing on $\ell' \preccurlyeq \ell$:
\begin{description}
\item[Case 1] $\ell'=\low, \ell=\low$: Straightforward because stamping low returns the same value.
\item[Case 2] $\ell'=\low, \ell=\high$: Casing on value $W$ and the simulation relation $\simrel{\gc}{V}{W}{A}$
produces 6 sub-cases. We consider the two cases for constants below (corresponding to ${\leq}\textit{-const}$ and
${\leq}\textit{-wrapped-const}$), because the cases for $\lambda$s and addresses are analogous:
\begin{itemize}
\item $\simrel{\gc}{\dynconst{k}{\ell_1'}}{\ccconst{k}}{\iota_{\ell_1}}$
      \begin{itemize}
      \item[-] If $\ell_1 = \low$, $\simrel{\gc}{\dynconst{k}{\ell_1'}}{\cccast{\ccconst{k}}{\coerc{\id{\iota}}{\up}}}{\iota_{\high}}$
      because $\ell_1' \preccurlyeq |\up| = \high$ (rule ${\leq}\textit{-wrapped-const}$).
      \item[-] If $\ell_1 = \high$, $\simrel{\gc}{\dynconst{k}{\ell_1'}}{\ccconst{k}}{\iota_{\high}}$
      because $\ell_1' \preccurlyeq \high$ (rule ${\leq}\textit{-const}$).
      \end{itemize}
\item $\simrel{\gc}{\dynconst{k}{\ell_1'}}{\cccast{\ccconst{k}}{\coerc{\id{\iota}}{\bar{c}}}}{-}$
      \begin{itemize}
      \item[-]
        $\simrel{\gc}{\dynconst{k}{\ell_1'}}{\cccast{k}{\coerc{\id{\iota}}{\mathit{stamp}\;\bar{c}\;\high}}}{-}$
        because \newline $\ell_1' \preccurlyeq |\mathit{stamp}\;\bar{c}\;\high|
        = |\bar{c}| \curlyvee \high = \high$ by Lemma~\ref{lem:stamp-implicit}
        (Stamping models implicit flow) and rule
        ${\leq}\textit{-wrapped-const}$.
      \end{itemize}
\end{itemize}
\item[Case 3] $\ell'=\high, \ell=\high$: Analogous to Case 2; the only difference is that
           instead of $\dynconst{k}{\ell_1'}$, the left side is always $\dynconst{k}{\high}$
           because it is stamped with $\ell' = \high$.
\end{description}
\end{proof}

\begin{lemma}[Casting a label expression models explicit flow]
\label{lem:cast-lexpr-explicit}
If $\mathbf{NF}\;e_1$ and $\cccast{e_1}{\bar{c}} \longrightarrow^{*} e_2$ and $\mathbf{NF}\;e_2$,
$|e_1| \preccurlyeq |e_2|$.
\end{lemma}
\begin{proof}
The proof is in \texttt{cast-security} of \texttt{/src/LabelExpr/Security.agda}.
The proof is by inversion on the multi-step reduction.
\end{proof}

\begin{lemma}[Stamping a label expression models implicit flow]
\label{lem:stamp-lexpr-implicit}
If $\mathbf{NF}\;e$, $|\mathit{stamp}\;e\;\ell| = |e| \curlyvee \ell$ and
$|\mathit{stamp!}\;e\;\ell| = |e| \curlyvee \ell$.
\end{lemma}
\begin{proof}
The proof is fully mechanized in $\mathtt{stamp_e\texttt{-}security}$ and
$\mathtt{stamp!_e\texttt{-}security}$ of \texttt{/src/LabelExpr/Security.agda}.
The proof is by casing on $\mathbf{NF}\;e$.
\end{proof}

\begin{lemma}
\label{lem:cast-leq}
Suppose $\Gamma ; \Sigma ; g ; \ell \vdash V \Leftarrow S_{\ell_1}$ and
$\vdash \bm{c} : S_{\ell_1} \Rightarrow T_{\ell_2}$. If $\cccast{V}{\bm{c}} \longrightarrow^{*} W$,
then $\ell_1 \preccurlyeq \ell_2$.
\end{lemma}
\begin{proof}
By induction on the multi-step reduction.
\begin{description}
\item[Zero step] $\bm{c} = (\coerc{c_r}{\bar{c}})$ is irreducible, so
$\bar{c} : \ell_1 \Rightarrow \ell_2$ is in its normal form. Thus $\ell_1 \preccurlyeq \ell_2$.
\item[One or more steps] Case on the first reduction step:
\begin{itemize}
\item Rule \textit{cast}: $\cccast{V_r}{\coerc{c_r}{\bar{c}}} \longrightarrow \cccast{V_r}{\coerc{c_r}{\bar{d}}} \longrightarrow^{*} W$.
  By preservation of coercion sequences, $\bar{d} : \ell_1 \Rightarrow \ell_2$. By induction hypothesis, $\ell_1 \preccurlyeq \ell_2$.
\item Rule \textit{cast-id}: $\cccast{V_r}{\coerc{\id{\iota}}{\id{g}}} \longrightarrow V_r$. We directly know $\ell_1 = g = \ell_2$.
\item Rule \textit{cast-comp}: $\cccast{\cccast{V_r}{\bm{c}}}{\bm{d}} \longrightarrow \cccast{V_r}{\bm{c} \mdoubleplus \bm{d}} \longrightarrow^{*} W$.
      Suppose $\bm{c} = (\coerc{-}{\bar{c}}), \bar{c} : \ell_0 \Rightarrow \ell_1$ and
      $\bm{d} = (\coerc{-}{\bar{d}}), \bar{d} : \ell_1 \Rightarrow \ell_2$.
      We know $\cccast{V_r}{\coerc{-}{\bar{c} \mdoubleplus \bar{d}}} \longrightarrow \cccast{V_r}{\bar{c}_n} \longrightarrow^{*} W$.
      By Lemma~\ref{lem:comp-explicit} (Composition models explicit flow),
      $\ell_1 = |\bar{c}| \preccurlyeq |\bar{c}_n| = \ell_2$.
\end{itemize}
\end{description}
\end{proof}

\begin{lemma}
\label{lem:leq-value-pc}
If $\simrel{g_1}{M}{V}{A}$ then $\simrel{g_2}{M}{V}{A}$.
\end{lemma}
\begin{proof}
The proof is fully mechanized in Agda and can be found in the
supplementary material. The proof is by casing on value and the
simulation relation.
\end{proof}


\begin{lemma}[Simulation between \CC and \DynIFC]
  \label{lem:sim-leq}
  Suppose $M_1$ is a well-typed \CC term: $\Gamma ; \Sigma_1 ; \gc ; |\PC| \vdash M_1 \Leftarrow A$,
  $PC$ is a well-typed label expression: $\vdash \PC \Leftarrow \gc$, and $\mu_1$ is a well-typed
  heap: $\Sigma_1 \vdash \mu_1$.
  Suppose $\gc \vdash M_1' \leq M_1 \Leftarrow A$, $\Sigma_1 \vdash \mu_1' \leq \mu_1$, and $\ell \preccurlyeq |\PC|$.
  If $\reduce{M_1}{\mu_1}{\PC}{M_2}{\mu_2}$, then there exists $M_3, \mu_3, M_2', \mu_2'$ s.t
  $\reducemult{M_2}{\mu_2}{\PC}{M_3}{\mu_3}$,
  $\reducemult{M_1'}{\mu_1'}{\ell}{M_2'}{\mu_2'}$,
  $\gc \vdash M_2' \leq M_3 \Leftarrow A$,
  and $\Sigma_2 \vdash \mu_2' \leq \mu_3$ for some $\Sigma_2$.
\end{lemma}
\begin{proof}
  The proof is by induction on the reduction relation
  $\reduce{M_1}{\mu_1}{\PC}{M_2}{\mu_2}$
  and inversion on the simulation relation
  $\vdash M_1' \leq M_1 \Leftarrow A$.
  We only consider successful cases of the reduction (which do not produce errors),
  because the theorem statement of noninterference (Theorem~\ref{thm:NI-Surface})
  for \Surface
  is termination-insensitive.
  \begin{description}
    \item[Case~$\xi$:]
      \begin{align}
      \reduce{N_1}{\mu_1}{\PC}{N_2}{\mu_2} \\
      \mathit{plug}\;N_1\;F = M_1 \\
      \mathit{plug}\;N_2\;F = M_2
      \end{align}
      We case on the frame $F$. We consider $F=\ccapp{\Box}{M}{A}{B}{\ell}$ below;
      the cases for other frames all follow the same pattern.
      By inversion of the simulation relation, the left side must be a function application:
      \[
      \simrel{\ell^c}{N_1'\;M'}{\ccapp{N_1}{M}{A}{B}{\ell}}{C}
      \]
      We know $\simrel{\ell^c}{N_1'}{N_1}{(\Fun{A}{\ell^c \curlyvee \ell}{B})_\ell}$ and $\simrel{\ell^c}{M'}{M}{A}$.
      By the induction hypothesis, there exists $N_3, \mu_3, N_2', \mu_2'$ such that
      $\reducemult{N_2}{\mu_2}{\PC}{N_3}{\mu_3}$, $\reducemult{N_1'}{\mu_1'}{\ell}{N_2'}{\mu_2'}$,
      $\simrel{\ell^c}{N_2'}{N_3}{(\Fun{A}{\ell^c \curlyvee \ell}{B})_\ell}$, and $\Sigma_2 \vdash \mu_2' \leq \mu_3$ for some $\Sigma_2$.
      Choose $M_3 = \mathit{plug}\;N_3\;F = \ccapp{N_3}{M}{A}{B}{\ell}$, $\mu_3 = \mu_3$, $M_2' = N_2'\;M'$,
      and $\mu_2' = \mu_2'$.
      We know $\reducemult{\ccapp{N_2}{M}{A}{B}{\ell}}{\mu_2}{\PC}{\ccapp{N_3}{M}{A}{B}{\ell}}{\mu_3}$,
      $\reducemult{N_1'\;M'}{\mu_1'}{\ell}{N_2'\;M'}{\mu_2'}$, and $\Sigma_2 \vdash \mu_2' \leq \mu_3$.
      We still need to show $\simrel{\ell^c}{N_2'\;M'}{\ccapp{N_3}{M}{A}{B}{\ell}}{C}$, which is proved by applying
      rule ${\leq}\mathit{-app}$.
  \item[Case~\textit{prot-val}:]
    By inversion on the simulation relation, the left side must be a protection term.
    By Lemma~\ref{lem:sim-val}, the body $V'$ must be a value:
    \begin{align}
    g_1 \vdash \dynprot{\ell'}{V'} &\leq \ccprot{\PC}{\ell}{V}{A} \Leftarrow B  \\
    g_2 \vdash V' &\leq V \Leftarrow A \label{eqn:117} \\
    \ell' &\preccurlyeq \ell \label{eqn:118} \\
    \vdash \PC &\Leftarrow g_2
    \end{align}
    We take a step by \textit{prot-val} on the left side, we need to relate:
    \[
    \simrel{g_1}{V' \curlyvee \ell'}{\mathit{stamp}\;V\;\ell}{B}
    \]
    which is proved by applying Lemma~\ref{lem:sim-stamp} on~\eqref{eqn:117} (with Lemma~\ref{lem:leq-value-pc} applied)
    and~\eqref{eqn:118}.
  \item[Case~\textit{prot-ctx}:]
  \[
  \reduce{M}{\mu_1}{\PC_2}{N}{\mu_2}
  \]
  By inversion on the simulation relation, the left side must be protection:
  \begin{align}
    g_1 \vdash \dynprot{\ell'}{M'} &\leq \ccprot{\PC_2}{\ell}{M}{A} \Leftarrow B  \\
    g_2 \vdash M' &\leq M \Leftarrow A \\
    \ell' &\preccurlyeq \ell \\
    \vdash \PC_2 &\Leftarrow g_2
  \end{align}
  By inversion on the typing of the protection term on the right:
  \begin{equation}
  \label{eqn:119}
  |\PC_1| \curlyvee \ell \preccurlyeq |\PC_2|
  \end{equation}
  where $\PC_1$ is the original PC to reduce the protection term on the right.
  To get the induction hypothesis, we need to show $\ell'' \curlyvee \ell' \preccurlyeq |PC_2|$
  where $\ell''$ is the PC to reduce the protection term on the left.
  We know $\ell'' \preccurlyeq |PC_1|$, thus:
  \[
  \ell'' \curlyvee \ell' \preccurlyeq \ell' \curlyvee \ell \preccurlyeq |\PC_1| \curlyvee \ell \preccurlyeq |\PC_2|
  \]
  The induction hypothesis shows that there exists $L, \mu_3, N', \mu_2'$ s.t
  $\reducemult{N}{\mu_2}{\PC_2}{L}{\mu_3}$,
  $\reducemult{M'}{\mu_1'}{\ell'' \curlyvee \ell'}{N'}{\mu_2'}$,
  $\simrel{g_2}{N'}{L}{A}$, and $\Sigma_2 \vdash \mu_2' \leq \mu_3$ for some $\Sigma_2$.
  We step the left side to $\dynprot{\ell'}{N'}$ and step the right side to
  $\ccprot{\PC_2}{\ell}{L}{A}$ using the congruence of \textit{prot-ctx}.
  We then relate the protection terms on both sides using rule ${\leq}\mathit{-prot}$.
  \item[Case~\textit{cast}:] There are three sub-cases: \textit{cast}, \textit{cast-id},
  and \textit{cast-comp}.
    \begin{description}
    \item[Sub-case~\textit{cast}:]
    We know $\simrel{\gc}{M'}{\cccast{V_r}{\coerc{c_r}{\bar{c}}}}{B}$. \cccast{V_r}{\coerc{c_r}{\bar{c}}}
    is not a value, so $\simrel{\gc}{M'}{V_r}{A}$. We need to relate
    $\simrel{\gc}{M'}{\cccast{V_r}{\coerc{c_r}{\bar{d}}}}{B}$, which is directly proved by applying
    ${\leq}\textit{-cast}$.
    \item[Sub-case~\textit{cast-id}:]
    We know $\simrel{\gc}{M'}{\cccast{V_r}{\coerc{\id{\iota}}{\id{g}}}}{A}$. Again, note that
    \cccast{V_r}{\coerc{\id{\iota}}{\id{g}}} is not a value, so $\simrel{\gc}{M'}{V_r}{A}$.
    \item[Sub-case~\textit{cast-comp}:]
    We know $\simrel{\gc}{M'}{\cccast{\cccast{V_r}{\bm{c}}}{\bm{d}}}{B}$. By inversion using
    rule ${\leq}\mathit{-cast}$, $\simrel{\gc}{M'}{\cccast{V_r}{\bm{c}}}{A}$; $\bm{c}$ is irreducible,
    by Lemma~\ref{lem:sim-val} $M'$ is a value. We need to show there exists $M$ such that
    $\reducemult{\cccast{V_r}{\bm{c}\mdoubleplus\bm{d}}}{\mu_1}{\PC}{M}{\mu_1}$ and $\simrel{\gc}{M'}{M}{B}$.
    Casing on $V_r$:
    \begin{itemize}
    \item \textbf{If $V_r = \ccconst{k}$.} Suppose
    $\bm{c}\mdoubleplus\bm{d} = \coerc{d_r}{\bar{c} \mdoubleplus \bar{d}}$ for some $d_r$.
    We take a step using \textit{cast} by reducing the composed coercion sequence to its
    normal form: $\bar{c} \mdoubleplus \bar{d} \longrightarrow^{+} \bar{c}_n$.
    By Lemma~\ref{lem:comp-explicit} (Composition models explicit flow), $|\bar{c}| \preccurlyeq |\bar{c}_n|$.
    By Lemma~\ref{lem:sim-wrapped-const}, we know $M' = \dynconst{k}{\ell'}$ and $\ell' \preccurlyeq |\bar{c}|$.
    By transitivity, $\ell' \preccurlyeq |\bar{c}_n|$.
    (1) If $\bar{c}_n$ is not an identity coercion, we relate
    $\simrel{\gc}{\dynconst{k}{\ell'}}{\cccast{\ccconst{k}}{\coerc{d_r}{\bar{c}_n}}}{B}$ directly using
    rule ${\leq}\textit{-wrapped-const}$. (2) If $\bar{c}_n = \id{\ell}$, we take one additional step by
    \textit{cast-id}. We know $\ell' \preccurlyeq |\id{\ell}| = \ell$,
    thus $\simrel{\gc}{\dynconst{k}{\ell'}}{\ccconst{k}}{\iota_\ell}$
    by rule ${\leq}\textit{-const}$.
    \item If $V_r = \cclam{x}{N}$. Similar to the constant case above.
    The only difference is that we do not need to specially handle $\bar{c}_n=\id{\ell}$,
    because the function coercion $(\coerc{\funco{\bar{d}}{\bm{c}}{\bm{d}}}{\id{\ell}})$
    is already irreducible.
    \item If $V_r = \ccaddr{n}$. Same as the lambda case above.
    \end{itemize}
    \end{description}
  \item[Case~$\beta$:]
      By inversion on the simulation relation, the left side must be a function application:
      \[
      \simrel{\ell^c}{L\;M}{\ccapp{(\cclam{x}{N})}{V}{A}{B}{\ell}}{C}
      \]
      where $C=\textit{stamp}\;B\;\ell$.
      We know that $L$ must be a $\lambda$ by rule ${\leq}\textit{-lam}$ and
      $M$ must be a value by Lemma~\ref{lem:sim-val}:
      \begin{align}
      \ell^c \vdash (\dynlam{x}{N'}{\ell'})\;V' &\leq \ccapp{(\cclam{x}{N})}{V}{A}{B}{\ell}
      \Leftarrow C \\
      \ell' &\preccurlyeq \ell       \label{eqn:105} \\
      \ell^c \curlyvee \ell \vdash N' &\leq N \Leftarrow B \label{eqn:106} \\
      \ell^c \vdash V' &\leq V \Leftarrow A \label{eqn:107}
      \end{align}
      We take one step by $\beta$ on the left side of the simulation relation:
      \[
        \reduce{(\dynlam{x}{N'}{\ell'})\;V'}{\mu'}{\ell''}{\dynprot{\ell'}{(N'[x:=V'])}}{\mu'}
      \]
      Now we need to show
      \[
      \ell^c \vdash \dynprot{\ell'}{(N'[x:=V'])} \leq \ccprot{(\mathit{stamp}\;\PC\;\ell)}{\ell}{(N[x:=V])}{B}
      \Leftarrow C
      \]
      Applying ${\leq}\textit{-prot}$ yields three sub-goals:
      (1) $\ell' \preccurlyeq \ell$ is directly proved by~\eqref{eqn:105}
      (2) $\simrel{\ell^c \curlyvee \ell}{N'[x:=V']}{N[x:=V]}{B}$ is proved by applying Lemma~\ref{lem:subst-sim}
          on~\eqref{eqn:106} and~\eqref{eqn:107}.
      (3) $\vdash \mathit{stamp}\;\PC\;\ell \Leftarrow \ell^c \curlyvee \ell$ because stamping is well-typed.
  \item[Case~\textit{app-cast}:]
      \begin{align}
      \cccast{(\mathit{stamp}\;\PC_1\;\ell)}{\bar{d}} &\longrightarrow^{*} \PC_2  \label{eqn:108} \\
      \cccast{V}{\bm{c}} &\longrightarrow^{*} W  \label{eqn:109}
      \end{align}
      By inversion on the simulation relation, the left side must be a function application:
      \[
      \ell^c \vdash L\;M \leq \ccapp{(\cccast{\cclam{x}{N}}{\coerc{\funco{\bar{d}}{\bm{c}}{\bm{d}}}{\bar{c}}})}{V}{C}{D}{\ell}
      \Leftarrow E
      \]
      where $E = \textit{stamp}\;D\;\ell$, $\vdash \bar{d} : \ell^c \curlyvee \ell \Rightarrow \gc$,
      $\vdash \bm{c} : C \Rightarrow A$, and $\vdash \bm{d} : B \Rightarrow D$.

      We know that $L$ must be a $\lambda$ by Lemma~\ref{lem:sim-func-proxy} and
      $M$ must be a value by Lemma~\ref{lem:sim-val}:
      \begin{align}
      \ell^c \vdash (\dynlam{x}{N'}{\ell'})\;V' &\leq \ccapp{(\cccast{\cclam{x}{N}}{\coerc{\funco{\bar{d}}{\bm{c}}{\bm{d}}}{\bar{c}}})}{V}{C}{D}{\ell}
      \Leftarrow E \\
      \ell' &\preccurlyeq |\bar{c}| \label{eqn:110} \\
      \gc \vdash N' &\leq N \Leftarrow B \label{eqn:111} \\
      \ell^c \vdash V' &\leq V \Leftarrow C \label{eqn:112}
      \end{align}
      We take a step by $\beta$ on the left side of simulation relation:
      \[
        \reduce{(\dynlam{x}{N'}{\ell'})\;V'}{\mu'}{\ell''}{\dynprot{\ell'}{(N'[x:=V'])}}{\mu'}
      \]
      Now we need to show
      \[
      \ell^c \vdash \dynprot{\ell'}{(N'[x:=V'])} \leq \ccprot{\PC_2}{\ell}{(\cccast{(N[x:=W])}{\bm{d}})}{D}
      \Leftarrow E
      \]
      By rule ${\leq}$\textit{-prot}, it is equivalent to showing:
      \begin{align}
      \ell' &\preccurlyeq \ell \label{eqn:115} \\
      \gc \vdash N'[x:=V'] &\leq \cccast{(N[x:=W])}{\bm{d}} \Leftarrow D \label{eqn:116} \\
      \vdash \PC_2 &\Leftarrow \gc
      \end{align}
      We know $\vdash \PC_2 \Leftarrow \gc$ because reducing label expressions preserves types.
      By~\eqref{eqn:110} and $|\bar{c}| = \ell$ we prove \eqref{eqn:115}.
      Apply Lemma~\ref{lem:cast-sim} on~\eqref{eqn:112} and~\eqref{eqn:109}, we get
      $\simrel{\ell^c}{V'}{W}{A}$.
      By Lemma~\ref{lem:subst-sim}, Lemma~\ref{lem:leq-value-pc}, and~\eqref{eqn:111},
      $\simrel{\gc}{N'[x:=V']}{N[x:=W]}{B}$.
      Apply rule ${\leq}$\textit{-cast} and we prove~\eqref{eqn:116}.
  \item[Case~\textit{app$\star$-cast}:] Similar to \textit{app-cast}.
  The only noteworthy difference is that on the right side, we use $|\bar{c}|$ instead of the
  $\ell$ from the syntax of the function application, both in the protection term
  and when stamping the PC.
  \item[Cases~$\beta$\textit{-if-true} and~$\beta$\textit{-if-false}:]
  By inversion on the simulation relation, the left side must also
  be an if-conditional:
  \[
  \simrel{\ell^c}{\dynif{\dynconst{\true}{\ell'}}{M'}{N'}}{\ccif{\ccconst{\true}}{A}{\ell}{M}{N}}{B}
  \]
  We know:
  \begin{align}
  \ell' &\preccurlyeq \ell \\
  \ell^c \curlyvee \ell \vdash M' &\leq M \Leftarrow A \\
  \ell^c \curlyvee \ell \vdash N' &\leq N \Leftarrow A
  \end{align}
  Take one step on the left side. We need to relate
  \[
  \simrel{\ell^c}{\dynprot{\ell'}{M'}}{\ccprot{(\mathit{stamp}\;\PC\;\ell)}{\ell}{M}{A}}{B}
  \]
  The goal is directly proved by applying rule ${\leq}\textit{-prot}$.
  The case for $\beta$\textit{-if-false} is analogous.
  \item[Cases \textit{if-true-cast} and \textit{if-false-cast}:]
  Note that the label partial order trivially holds because $\ell' \preccurlyeq \high$
  for any $\ell'$. The rest is analogous to $\beta$\textit{-if-true} and~$\beta$\textit{-if-false}.
  \item[Cases \textit{if$\star$-true-cast} and \textit{if$\star$-false-cast}:]
  By Lemma~\ref{lem:sim-wrapped-const} we know $\ell' \preccurlyeq |\bar{c}|$.
  The rest is analogous to $\beta$\textit{-if-true} and~$\beta$\textit{-if-false}.
  \item[Case~\textit{ref}:]
  %% \[
  %% n\;\mathbf{FreshIn}\;\mu(\ell)
  %% \]
  From the typing derivation (rule ${\vdash}\textit{-ref}$), we also know:
  \begin{align}
  \vdash \PC &\Leftarrow \ell^c \\
  \ell^c &\preccurlyeq \ell
  \end{align}
  which implies
  \begin{equation}
  |\PC| \preccurlyeq \ell
  \end{equation}
  We know $\ell' \preccurlyeq |\PC| \preccurlyeq \ell$, where $\ell'$ is the PC that the left side is reduced with.
  We take a step by \textit{ref?-ok} on the left side by choosing the same fresh address $n$. We need to show:
  \begin{align}
  \simrel{\ell^c}{\dynaddr{n_\ell}{\low}}{\ccaddr{n}}{(\Refer{T_\ell})_{\low}}  \label{eqn:120} \\
  (\Sigma_1, \ell \mapsto n \mapsto T) \vdash (\mu', \ell \mapsto n \mapsto (V' \curlyvee \ell)) \leq (\mu, \ell \mapsto n \mapsto V)  \label{eqn:121}
  \end{align}

  \eqref{eqn:120} is proved directly by ${\leq}\textit{-addr}$.
  To prove \eqref{eqn:121} we need to show:
  \begin{equation}
  \label{eqn:122}
  \simrel{\low}{V' \curlyvee \ell}{V}{T_\ell}
  \end{equation}
  By inversion on the simulation relation and Lemma~\ref{lem:leq-value-pc},
  we know $\simrel{\low}{V'}{V}{T_\ell}$.
  From the definition of stamping we know $\textit{stamp}\;V\;\ell = V$
  if $V \Leftarrow T_\ell$;
  \eqref{eqn:122} is thus proved by Lemma~\ref{lem:sim-stamp}.
  \item[Case~\textit{ref?}:]
  \[
  \cccast{\PC_1}{\unk \Rightarrow^{\bl{p}} \ell} \longrightarrow^{*} \PC_2
  \]
  Reducing label expressions preserves types, so $\vdash \PC_2 \Leftarrow \ell$.
  We know $|\PC_2| = \ell$ and $\ell' \preccurlyeq |\PC_1|$, where $\ell'$ is the
  PC that the left side is reduced with. Casting models explicit flow
  (Lemma~\ref{lem:cast-lexpr-explicit}), so $|\PC_1| \preccurlyeq |\PC_2|$.
  We thus have $\ell' \preccurlyeq |\PC_1| \preccurlyeq |\PC_2| = \ell$. Take one step
  by \textit{ref?-ok} on the left side by choosing the same fresh address $n$.
  The rest of the proof follows that of \textit{ref}.
  %% We need to show:
  %% \begin{align}
  %% \simrel{\dynaddr{n_\ell}{\low}}{\ccaddr{n}}{(\Refer{T_\ell})_{\low}}  \\
  %% (\mu', \ell \mapsto n \mapsto (V' \curlyvee \ell)) \leq (\mu, \ell \mapsto n \mapsto V)
  %% \end{align}
  \item[Case~\textit{deref}:]
  \[
  \mu(\hat{\ell},n) = V
  \]
  By $\Sigma \vdash \mu' \leq \mu$ and Definition~\ref{def:sim-heap},
  there exists $V'$ s.t $\mu'(\hat{\ell}, n) = V'$ and $\simrel{\low}{V'}{V}{T_{\hat{\ell}}}$
  where $T = \Sigma(\hat{\ell}, n)$.
  By inversion on the simulation relation, the left side must be a dereference:
  \[
  \simrel{\gc}{\dynderef{\dynaddr{n_{\hat{\ell}}}{\ell'}}}{\ccderef{(\ccaddr{n})}{T_{\hat{\ell}}}{\ell}}{B}
  \]
  where $\ell' \preccurlyeq \ell$. Take one step on the left using rule \textit{deref},
  we need to relate:
  \[
  \simrel{\gc}{\dynprot{\ell'}{V'}}{\ccprot{\high}{\ell}{V}{T_{\hat{\ell}}}}{B}
  \]
  which is directly proved by rule ${\leq}\textit{-prot}$ and Lemma~\ref{lem:leq-value-pc}.
  \item[Case~\textit{deref$\star$-cast}:]
  \[
  \mu(\hat{\ell}, n) = V
  \]
  By $\Sigma \vdash \mu' \leq \mu$ and Definition~\ref{def:sim-heap},
  there exists $V'$ s.t $\mu'(\hat{\ell}, n) = V'$ and $\simrel{\low}{V'}{V}{S_{\hat{\ell}}}$
  where $S = \Sigma(\hat{\ell}, n)$.
  By inversion on the simulation relation, the left side must be a dereference:
  \[
  \simrel{\gc}{\dynderef{\dynaddr{n_{\hat{\ell}}}{\ell}}}{\ccderefstar{(\cccast{\ccaddr{n}}{\coerc{\refco{\bm{c}}{\bm{d}}}{\bar{c}}})}{T}}{T_{\unk}}
  \]
  where $\vdash \bm{c} : T_{\unk} \Rightarrow S_{\hat{\ell}} , \vdash \bm{d} : S_{\hat{\ell}} \Rightarrow T_{\unk}$.
  By Lemma~\ref{lem:sim-ref-proxy}, $\ell \preccurlyeq |\bar{c}|$.
  We take a step on the left side using \textit{deref}. We need to relate:
  \[
  \simrel{\gc}{\dynprot{\ell}{V'}}{\ccprot{\high}{|\bar{c}|}{(\cccast{V}{\bm{d}})}{T_{\unk}}}{B}
  \]
  which is proved directly by applying Lemma~\ref{lem:leq-value-pc}, ${\leq}\textit{-prot}$,
  and then ${\leq}\textit{-cast}$.
  \item[Case~\textit{deref-cast}:] Analogous to \textit{deref$\star$-cast}.
  \item[Case~\textit{$\beta$-assign}:]
  By inversion on the simulation relation, the left side is also an assignment:
  \[
  \simrel{\ell^c}{\dynassign{\dynaddr{n_{\hat{\ell}}}{\ell'}}{?}{V'}}{\ccassign{(\ccaddr{n})}{V}{T}{\hat{\ell}}{\ell}}{\Unit_{\low}}
  \]
  where $\ell' \preccurlyeq \ell$. From the typing derivation we know $\ell^c \curlyvee \ell \preccurlyeq \hat{\ell}$,
  $\vdash \PC \Leftarrow \ell^c$. We know $\ell'' \preccurlyeq |\PC| = \ell^c$ where $\ell''$ is the PC on the left,
  thus $\ell'' \curlyvee \ell' \preccurlyeq \hat{\ell}$. We take one step on the left side using \textit{assign?-ok}.
  The \texttt{unit}s on both sides relate straightforwardly. We need to relate:
  \[
  \Sigma \vdash [\hat{\ell} \mapsto n \mapsto (V' \curlyvee \hat{\ell})] \mu' \leq [\hat{\ell} \mapsto n \mapsto V] \mu
  \]
  Given $\Sigma(\hat{\ell}, n) = T$, we need to show:
  \[
  \simrel{\low}{V' \curlyvee \hat{\ell}}{V}{T_{\hat{\ell}}}
  \]
  which can be proved similar to~\eqref{eqn:122} because $\vdash V \Leftarrow T_{\hat{\ell}}$.
  \item[Case~\textit{assign-cast}:]
  \[
  \cccast{V}{\bm{c}} \longrightarrow^{*} W
  \]
  where $\vdash \bm{c} : T_{\hat{\ell}_2} \Rightarrow S_{\hat{\ell}_1}$. By Lemma~\ref{lem:cast-leq},
  $\hat{\ell}_2 \preccurlyeq \hat{\ell}_1$. From the typing derivation, $\vdash \PC \Leftarrow \ell^c$ and
  $\ell^c \curlyvee \ell \preccurlyeq \hat{\ell}_2$. By inversion on the simulation relation,
  the left side is also an assignment:
  \[
  \simrel{\ell^c}{\dynassign{\dynaddr{n_{\hat{\ell}_1}}{\ell'}}{?}{V'}}{\ccassign{(\cccast{\ccaddr{n}}{\coerc{\refco{\bm{c}}{\bm{d}}}{\bar{c}}})}{V}{T}{\hat{\ell}_2}{\ell}}{\Unit_{\low}}
  \]
  We know $\ell' \preccurlyeq |\bar{c}| = \ell$. $\ell'' \preccurlyeq |\PC| = \ell^c$ where $\ell''$
  is the PC that the left side is reduced with. Thus we have
  $\ell'' \curlyvee \ell' \preccurlyeq \hat{\ell}_2 \preccurlyeq \hat{\ell}_1$.
  The check succeeds so we take one step on the left side by \textit{assign?-ok}.
  The \texttt{unit}s on both sides relate straightforwardly. We need to relate:
  \[
  \Sigma \vdash [\hat{\ell}_1 \mapsto n \mapsto (V' \curlyvee \hat{\ell}_1)] \mu' \leq [\hat{\ell}_1 \mapsto n \mapsto W] \mu
  \]
  Need to show:
  \[
  \simrel{\low}{V' \curlyvee \hat{\ell}_1}{W}{S_{\hat{\ell}_1}}
  \]
  which can be proved similar to~\eqref{eqn:122}.

  \item[Case~\textit{assign?-cast}:]
  \begin{align}
  \cccast{(\mathit{stamp!}\;\PC_1\;|\bar{c}|)}{\unk \Rightarrow^{\bl{p}} \hat{\ell}} &\longrightarrow^{*} \PC_2 \\
  \cccast{V}{\bm{c}} &\longrightarrow^{*} W
  \end{align}

  By inversion on the simulation relation, the left side is also an assignment:
  \[
  \simrel{\gc}{\dynassign{\dynaddr{n_{\hat{\ell}}}{\ell}}{?}{V'}}{\ccassignproj{\left(\cccast{\ccaddr{n}}{\coerc{\refco{\bm{c}}{\bm{d}}}{\bar{c}}}\right)}{V}{T}{g}{p}}{\Unit_{\low}}
  \]
  where $\vdash \bm{c} : T_g \Rightarrow S_{\hat{\ell}}, \vdash \bm{d} : S_{\hat{\ell}} \Rightarrow T_g$.
  By Lemma~\ref{lem:sim-ref-proxy}, $\ell \preccurlyeq |\bar{c}|$.
  Stamping models implicit flow (Lemma~\ref{lem:stamp-lexpr-implicit}),
  so $|\mathit{stamp}\;\PC_1\;|\bar{c}|| = |\PC_1| \curlyvee |\bar{c}|$;
  casting models explicit flow (Lemma~\ref{lem:cast-lexpr-explicit}), so
  $|\PC_1| \curlyvee |\bar{c}| \preccurlyeq |\PC_2|$.
  Reduction of label expressions preserves types, so $\vdash \PC_2 \Leftarrow \hat{\ell}$.
  Thus $|\PC_2| = \hat{\ell}$, $|\PC_1| \curlyvee |\bar{c}| \preccurlyeq \hat{\ell}$.
  We know $\ell' \preccurlyeq |\PC_1|$ where $\ell'$ is the PC the left side reduces with.
  Thus $\ell' \curlyvee \ell \preccurlyeq \hat{\ell}$; the check succeeds so we take one
  step on the left side by \textit{assign?-ok}.
  The \texttt{unit}s on both sides relate straightforwardly. We need to relate:
  \[
  \Sigma \vdash [\hat{\ell} \mapsto n \mapsto (V' \curlyvee \hat{\ell})] \mu' \leq [\hat{\ell} \mapsto n \mapsto W] \mu
  \]
  Need to show:
  \[
  \simrel{\low}{V' \curlyvee \hat{\ell}}{W}{S_{\hat{\ell}}}
  \]
  which again can be proved similar to~\eqref{eqn:122}.

  \end{description}
\end{proof}

\begin{lemma}[Multi-step simulation]
\label{lem:sim-mult}
Suppose $M$ is a well-typed \CC term $\Gamma ; \Sigma ; \gc ; |\PC| \vdash M \Leftarrow A$,
$PC$ is a well-typed label expression: $\vdash \PC \Leftarrow \gc$, and $\mu_1$ is a well-typed
heap: $\Sigma \vdash \mu_1$.
Suppose $\simrel{\gc}{M'}{M}{A}$, $\Sigma \vdash \mu_1' \leq \mu_1$ ,and $\ell \preccurlyeq |\PC|$.
If $\reducemult{M}{\mu_1}{\PC}{V}{\mu_2}$, then there exists $V',\mu_2'$ such that
$\reducemult{M'}{\mu_1'}{\ell}{V'}{\mu_2'}$
and $\simrel{\gc}{V'}{V}{A}$.
\end{lemma}
\begin{proof}
By induction on multi-step reduction $\reducemult{M}{\mu_1}{\PC}{V}{\mu_2}$.
\begin{description}
\item[Zero step] If $M$ is already a value, choose $V'$ to be $M'$.
By Lemma~\ref{lem:sim-val}, $M'$ is also a value.
The values are in sync because $\simrel{\gc}{M'}{M}{A}$.
\item[One or more steps]
\begin{align}
\reduce{M}{\mu_1}{\PC}{N}{\mu_3} \\
\reducemult{N}{\mu_3}{\PC}{V}{\mu_2}
\end{align}
Apply Lemma~\ref{lem:sim-leq} and then use the induction hypothesis.
\end{description}
\end{proof}

\begin{figure}[tbp]
  \raggedright
  \fbox{$\epsilon\;M\;A = M'$}
  {\small
  \begin{align*}
  \epsilon\;x\;{-} &= x \\
  \epsilon\;(\ccconst{k})\;(\iota_\ell) &= \dynconst{k}{\ell} \\
  \epsilon\;(\cclam{x}{N})\;((\Fun{A}{-}{B})_{\ell}) &= \dynlam{x}{\epsilon\;N\;B}{\ell} \\
  \epsilon\;(\ccaddr{n})\;(\Refer{(T_{\hat{\ell}})}_{\ell}) &= \dynaddr{n_{\hat{\ell}}}{\ell} \\
  \epsilon\;(\ccapp{M}{N}{A}{B}{\ell})\;{-} &= (\epsilon\;M\;(\Fun{A}{\unk}{B})_\ell)\;(\epsilon\;N\;A) \\
  \epsilon\;(\ccappstar{M}{N}{A}{T}{})\;{-} &= (\epsilon\;M\;(\Fun{A}{\unk}{T_{\unk}})_{\unk})\;(\epsilon\;N\;A) \\
  \epsilon\;(\ccif{L}{A}{\ell}{M}{N})\;{-} &= \dynif{(\epsilon\;L\;\Bool_\ell)}{(\epsilon\;M\;A)}{(\epsilon\;N\;A)} \\
  \epsilon\;(\ccifstar{L}{T}{M}{N})\;{-} &= \dynif{(\epsilon\;L\;\Bool_{\unk})}{(\epsilon\;M\;T_{\unk})}{(\epsilon\;N\;T_{\unk})} \\
  \epsilon\;(\ccref{\ell}{M})\;(\Refer{T_\ell})_{\low} &= \dynref{?}{\ell}{(\epsilon\;M\;T_\ell)} \\
  \epsilon\;(\ccrefproj{\ell}{M}{p})\;(\Refer{T_\ell})_{\low} &= \dynref{?}{\ell}{(\epsilon\;M\;T_\ell)} \\
  \epsilon\;(\ccderef{M}{A}{\ell})\;{-} &= \dynderef{(\epsilon\;M\;(\Refer{A})_\ell)} \\
  \epsilon\;(\ccderefstar{M}{T})\;{-} &= \dynderef{(\epsilon\;M\;(\Refer{T_{\unk}})_{\unk})} \\
  \epsilon\;(\ccassign{L}{M}{T}{\hat{\ell}}{\ell})\;{-} &= \dynassign{(\epsilon\;L\;(\Refer{T_{\hat{\ell}}})_\ell)}{?}{(\epsilon\;M\;T_{\hat{\ell}})} \\
  \epsilon\;(\ccassignproj{L}{M}{T}{\hat{g}}{p})\;{-} &= \dynassign{(\epsilon\;L\;(\Refer{T_{\hat{g}}})_{\unk})}{?}{(\epsilon\;M\;T_{\hat{g}})} \\
  \epsilon\;(\cccast{N}{\bm{c}})\;{-} &= \epsilon\;N\;A \quad \text{, where }\vdash c : A \Rightarrow B \\
  \epsilon\;(\ccprot{\PC}{\ell}{M}{A})\;{-} &= \dynprot{\ell}{(\epsilon\;M\;A)}
  \end{align*}}
  \caption{Erasure from \CC to \DynIFC}
  \label{fig:cast-erase}
\end{figure}

We then prove that \CC satisfies termination-insensitive noninterference. The
statement of termination-insensitive noninterference says that if we run a
program with different high-security inputs in two executions, then their
low-security output values should be the related (e.g the same boolean):

\begin{lemma}[Noninterference for \CC]
\label{lem:NI-CC}
If $M$ is well-typed:
$(x{:}\Bool_{\high}) ; \emptyset ; \low ; \low \vdash M \Leftarrow \Bool_{\low}$
and
{\normalfont
\begin{equation*}
\reducemult{M [ x := \ccconst{b_1}]}{\emptyset}{\low}{V_1}{\mu_1}
\quad\text{ and }\quad
\reducemult{M [ x := \ccconst{b_2}]}{\emptyset}{\low}{V_2}{\mu_2}
\end{equation*}}
then $V_1 = V_2$.
\end{lemma}
\begin{proof}
From the definition of $\epsilon$ (Figure~\ref{fig:cast-erase}), we know
$\simrel{\low}{\epsilon(M)[x:= \dynconst{b_i}{\high}]}{M [ x :=
    \ccconst{b_i}]}{\Bool_{\low}}$. By Lemma~\ref{lem:sim-mult}, there exists
$V_i', \mu_i'$ s.t $\reducemult{\epsilon(M)[x:=
    \dynconst{b_i}{\high}]}{\emptyset}{\low}{V_i'}{\mu_i'}$ and
$\simrel{\low}{V_i'}{V_i}{\Bool_{\low}}$. The simulation relation must be of
form $\simrel{\low}{\dynconst{a_i}{\low}}{\ccconst{a_i}}{\Bool_{\low}}$ where
$a_i$ is the output boolean. By Lemma~\ref{lem:mult-step-impl-big-step} and
Lemma~\ref{lem:NI-old} , $a_1 = a_2$, thus $V_1 = \ccconst{a_1} = \ccconst{a_2}
= V_2$.
\end{proof}

\section{Noninterference of \Surface}
\label{sec:NI}

The noninterference lemma of \Surface is a straightforward corollary of the
noninterference lemma of \CC and compilation preserves types:

\begin{lemma}
\label{lem:NI-Surface}
Suppose a \Surface term $M$ is well-typed:
\[
(x{:}\Bool_{\high}) ; \low \vdash M : \Bool_{\low}
\]
If for any boolean inputs $b_1,b_2$
{\normalfont
\begin{equation*}
\reducemult{(\compile{M}) [ x := \ccconst{b_1}]}{\emptyset}{\low}{V_1}{\mu_1}
\quad\text{ and }\quad
\reducemult{(\compile{M}) [ x := \ccconst{b_2}]}{\emptyset}{\low}{V_2}{\mu_2}
\end{equation*}}
then the resulting values $V_1 = V_2$.
\end{lemma}
\begin{proof}
By Lemma~\ref{lem:compile-pres} (compilation preserves types),
$(x{:}\Bool_{\high}) ; \emptyset ; \low ; \low \vdash M' \Leftarrow \Bool_{\low}$.
By Lemma~\ref{lem:NI-CC} (noninterference for \CC), $V_1 = V_2$.
\end{proof}

%{\color{NavyBlue} %%% new text

Finally, we prove noninterference for \Surface as a direct corollary of
Lemma~\ref{lem:NI-Surface}. Noninterference says that for any high-security,
sensitive user input $b_1, b_2$, the low-security, publicly visible output will
always be the same:

\begin{theorem}[Noninterference for \Surface]
  \label{thm:NI-Surface}
  If $M$ is a \Surface program and $\mathit{eval}(M,b_1)=b_1'$ and $\mathit{eval}(M,b_2)=b_2'$,
  then $b_1' = b_2'$.
\end{theorem}
\begin{proof}
  By Definition~\ref{def:surface-program} (whole programs of \Surface),
  $(x{:}\Bool_{\high}) ; \low \vdash M : \Bool_{\low}$. By the definition of
  \textit{eval} and the canonical form of a value of $\Bool_{\low}$,
  $\reducemult{(\compile{M}) [ x := \ccconst{b_1}]}{\emptyset}{\low}{\ccconst{b_1'}}{\mu_1}$ and
  $\reducemult{(\compile{M}) [ x := \ccconst{b_2}]}{\emptyset}{\low}{\ccconst{b_2'}}{\mu_2}$.
  Applying Lemma~\ref{lem:NI-Surface}, $b_1'=b_2'$.
\end{proof}

%} %%% end new text
