%%%%%%%%%%%%%%%%%%%%%%%%%%%%%%%%%%%%%%%%%%%%%%%%%%%%
% Document type, global settings, and packages
%%%%%%%%%%%%%%%%%%%%%%%%%%%%%%%%%%%%%%%%%%%%%%%%%%%%

\documentclass[12pt]{report}   %12 point font for Times New Roman
\usepackage{graphicx}  %for images and plots
\usepackage[letterpaper, left=1.5in, right=1in, top=1in, bottom=1in]{geometry}
\usepackage[doublespacing]{setspace}  %use this package to set linespacing as desired
\usepackage{times}  %set Times New Roman as the font
\usepackage[explicit]{titlesec}  %title control and formatting
\usepackage[titles]{tocloft}  %table of contents control and formatting
\usepackage[backend=bibtex, sorting=none]{biblatex}  %reference manager
\usepackage[bookmarks=true, hidelinks]{hyperref}
\usepackage[page]{appendix}  %for appendices
\usepackage{rotating}  %for rotated, landscape images
\usepackage{amsmath}
\usepackage{mathabx}
\usepackage[dvipsnames]{xcolor}
\usepackage{listings}
\usepackage{lstautogobble}
\usepackage{bbold}
\usepackage{semantic}
\usepackage{soul}
\usepackage{enumitem}
\usepackage{multicol}
\usepackage{stmaryrd}
\usepackage{ebproof}
\usepackage{newunicodechar}
\usepackage{xspace}
\usepackage{bm}
\usepackage{empheq}
\usepackage{pifont}
\usepackage{tabularx}
\usepackage{colortbl}
\usepackage{makecell}
\usepackage{tikz, tikz-qtree, tikz-qtree-compat}
\usepackage{circledsteps}
%\usepackage{newunicodechar}
%\usepackage{minted}
\usepackage{booktabs}
\usepackage{todonotes}
\usepackage{caption}
\captionsetup[table]{skip=10pt}

\usetikzlibrary{shapes,arrows}
\usetikzlibrary{decorations.markings}

\setlength{\multicolsep}{3pt}

\lstdefinestyle{mystyle}{
  commentstyle=\color{Green},
  keywordstyle=\color{Magenta},
  numberstyle=\tiny\color{Gray},
  stringstyle=\color{Purple},
  basicstyle=\ttfamily\small,
  breakatwhitespace=false,
  breaklines=true,
  captionpos=b,
  keepspaces=true,
  numbers=left,
  numbersep=5pt,
  showspaces=false,
  showstringspaces=false,
  showtabs=false,
  tabsize=2
}
\lstset{style=mystyle}

\newcommand{\cmark}{\ding{51}}
\newcommand{\xmark}{\ding{55}}
%\newcommand\doubleplus{+\kern-1.3ex+\kern0.8ex}
%\newcommand\mdoubleplus{\ensuremath{\mathbin{+\mkern-5mu+}}}
\DeclareSymbolFont{bbsymbol}{U}{bbold}{m}{n}
\DeclareMathSymbol{\bbsemicolon}{\mathbin}{bbsymbol}{"3B}
\newcommand\mdoubleplus{\ensuremath{\bbsemicolon}}
%\newcommand{\done}{\usym{2611}\xspace}
\newcommand\tikzmark[2]{\tikz[overlay,remember picture, anchor=base] \node (#1) {#2};}
\renewcommand\theadfont{\bfseries}
\renewcommand{\ttdefault}{cmtt}

\newtheorem{theorem}{Theorem}
\newtheorem{lemma}[theorem]{Lemma}
\newtheorem{corollary}[theorem]{Corollary}
\newtheorem{proposition}[theorem]{Proposition}
\newtheorem{constraint}[theorem]{Constraint}
\newtheorem{definition}[theorem]{Definition}
\newtheorem{example}[theorem]{Example}

\newcommand{\yes}{\textcolor{Green}{\ding{51}} Yes}
\newcommand{\no}{\textcolor{Red}{\ding{55}} No}
\newcommand{\maybe}{\ding{82} Maybe}
\newcommand{\redtext}[1]{\textcolor{Maroon}{#1}}
\newcommand{\bluetext}[1]{\textcolor{NavyBlue}{#1}}
\newcommand{\purpletext}[1]{\textcolor{Plum}{#1}}
\newcommand{\orangetext}[1]{\textcolor{BurntOrange}{#1}}
\newcommand{\graytext}[1]{\textcolor{gray}{#1}}
\newcommand{\bl}[1]{\ensuremath{\orangetext{#1}}}
\newcommand{\key}[1]{\ensuremath{\mathtt{#1}}}
\newcommand{\MID}{\;\mid\;}
\newcommand{\Surface}{\ensuremath{\lambda_{\mathtt{IFC}}^\star}\xspace}
\newcommand{\SurfaceOld}{\ensuremath{\lambda_{\mathtt{SEC}}^\star}\xspace}
\newcommand{\CCOld}{\ensuremath{\lambda_{\mathtt{SEC}}^{\Rightarrow}}\xspace}
\newcommand{\CC}{\ensuremath{\lambda_{\mathtt{IFC}}^{c}}\xspace}
\newcommand{\DynIFC}{\ensuremath{\lambda_{\mathtt{IFC}}^{\mathtt{DYN}}}\xspace}
\newcommand{\GSLRef}{\ensuremath{\mathrm{GSL}_\mathsf{Ref}}\xspace}
\newcommand{\GSLRefEps}{\ensuremath{\mathrm{GSL}_\mathsf{Ref}^\epsilon}\xspace}
\newcommand{\SSLRef}{\ensuremath{\mathrm{SSL}_\mathsf{Ref}}\xspace}
\newcommand{\lamgif}{\ensuremath{\mathit{\lambda_{gif}}}\xspace}
\newcommand{\WHILEG}{WHILE\textsuperscript{G}\xspace}
\newcommand{\laminfo}{\ensuremath{\lambda^{\textit{info}}}\xspace}
\newcommand{\high}{\textcolor{OrangeRed}{\key{high}}\xspace}
\newcommand{\low}{\textcolor{PineGreen}{\key{low}}\xspace}
\newcommand{\unk}{\textcolor{Maroon}{\key{\star}}\xspace}
\newcommand{\Bool}{\key{Bool}}
\newcommand{\Int}{\key{Int}}
\newcommand{\Unit}{\key{Unit}}
\newcommand{\Fun}[3]{\ensuremath{{#1}\xrightarrow{{#2}}{#3}}}
\newcommand{\Refer}[1]{\ensuremath{\key{Ref}\;{#1}}}
\newcommand{\true}{\key{true}}
\newcommand{\false}{\key{false}}
\newcommand{\unit}{\key{unit}}
\newcommand{\pc}{\ensuremath{\mathit{pc}}\xspace}
\newcommand{\PC}{\ensuremath{\mathit{PC}}\xspace}
\newcommand{\gc}{\ensuremath{\mathit{gc}}\xspace}
\newcommand{\syntax}[1]{\text{\texttt{\textcolor{Purple}{#1}}}}
\newcommand{\ccsyntax}[1]{\text{\texttt{{#1}}}}
\newcommand{\const}[2]{\ensuremath{\syntax{(\$}\;{#1}\syntax{)}_{#2}}}
\newcommand{\lam}[5]{\ensuremath{\syntax{(${\lambda}$}^{#1}{#2}\syntax{:}{#3}\syntax{.}\,{#4}\syntax{)}_{#5}}}
\newcommand{\app}[3]{\ensuremath{\syntax{(}{#1}\;{#2}\syntax{)}^{\bl{#3}}}}
\newcommand{\ifexp}[4]{\ensuremath{\syntax{(if}\;{#1}\;\syntax{then}\;{#2}\;\syntax{else}\;{#3}\syntax{)}^{\bl{#4}}}}
\newcommand{\refexp}[3]{\ensuremath{\syntax{(ref}\;{#1}\;{#2}\syntax{)}^{\bl{#3}}}}
\newcommand{\deref}[2]{\ensuremath{\syntax{!}^{\bl{#2}}\;{#1}}}
\newcommand{\assign}[3]{\ensuremath{\syntax{(}{#1}\;\syntax{:=}\;{#2}\syntax{)}^{\bl{#3}}}}
\newcommand{\ann}[3]{\ensuremath{\syntax{(}{#1}\;\syntax{:}\;{#2}\syntax{)}^{\bl{#3}}}}
\newcommand{\letexp}[3]{\ensuremath{\syntax{let}\;{#1}={#2}\;\syntax{in}\;{#3}}}
\newcommand{\ccconst}[1]{\ensuremath{\ccsyntax{\$}\;{#1}}}
\newcommand{\ccaddr}[1]{\ensuremath{\ccsyntax{addr}\;{#1}}}
\newcommand{\cclam}[2]{\ensuremath{\ccsyntax{λ}{#1}\ccsyntax{.}\,{#2}}}
\newcommand{\ccif}[5]{\ensuremath{\ccsyntax{if}\;{#1}\;{#2}\;{#3}\;{#4}\;{#5}}}
\newcommand{\ccifinj}[4]{\ensuremath{\ccsyntax{if!}\;{#1}\;{#2}\;{#3}\;{#4}}}
\newcommand{\ccref}[2]{\ensuremath{\ccsyntax{ref}\;{#1}\;{#2}}}
\newcommand{\ccrefproj}[3]{\ensuremath{\ccsyntax{ref?}^{\bl{#3}}\;{#1}\;{#2}}}
\newcommand{\ccderef}[3]{\ensuremath{\ccsyntax{!}\;{#1}\;{#2}\;{#3}}}
\newcommand{\ccderefinj}[2]{\ensuremath{\ccsyntax{!!}\;{#1}\;{#2}}}
\newcommand{\ccassign}[5]{\ensuremath{\ccsyntax{assign}\;{#1}\;{#2}\;{#3}\;{#4}\;{#5}}}
\newcommand{\ccassignproj}[5]{\ensuremath{\ccsyntax{assign?}^{\bl{#5}}\;{#1}\;{#2}\;{#3}\;{#4}}}
\newcommand{\cccast}[2]{\ensuremath{{#1}\,\ccsyntax{\textlangle}\,{#2}\,\ccsyntax{\textrangle}}}
\newcommand{\ccprot}[4]{\ensuremath{\ccsyntax{prot}\;{#1}\;{#2}\;{#3}\;{#4}}}
\newcommand{\ccprotinj}[4]{\ensuremath{\ccsyntax{prot!}\;{#1}\;{#2}\;{#3}\;{#4}}}
\newcommand{\cccastpc}[2]{\ensuremath{\ccsyntax{cast}_{\ccsyntax{pc}}\;{#1}\;{#2}}}
\newcommand{\cclet}[4]{\ensuremath{\ccsyntax{let}\;{#1}\ccsyntax{=}{#2}\ccsyntax{:}{#3}\;\ccsyntax{in}\;{#4}}}
\newcommand{\ccapp}[5]{\ensuremath{\ccsyntax{app}\;{#1}\;{#2}\;{#3}\;{#4}\;{#5}}}
\newcommand{\ccappinj}[4]{\ensuremath{\ccsyntax{app!}\;{#1}\;{#2}\;{#3}\;{#4}}}
\newcommand{\ccopaque}{\ensuremath{\ccsyntax{●}}}
\newcommand{\lconsisjoin}{\ensuremath{\,\widetilde{\curlyvee}\,}}  % label consistent join
\newcommand{\consisjoin}{\ensuremath{\,\widetilde{\vee}\,}}        % type consistent join
\newcommand{\lconsismeet}{\ensuremath{\,\widetilde{\curlywedge}\,}}  % label consistent meet
\newcommand{\consismeet}{\ensuremath{\,\widetilde{\wedge}\,}}        % type consistent meet
\newcommand{\Cast}[3]{\ensuremath{{#1}\Rightarrow^{\bl{#3}}{#2}}}
\newcommand{\Type}{\textit{Type}}
\newcommand{\RawType}{\textit{RawType}}
\newcommand{\Label}{\textit{Label}}
\newcommand{\compile}[1]{\ensuremath{\mathcal{C}\;{#1}}}
\newcommand{\applycast}[3]{\ensuremath{\mathbf{Cast}\;{#1}\,,\,{#2}\leadsto{#3}}}
\newcommand{\blame}[1]{\ensuremath{\key{blame}\;\bl{#1}}}
\newcommand{\reduce}[5]{\ensuremath{{#1}\mid{#2}\mid{#3}\longrightarrow{#4}\mid{#5}}}
\newcommand{\reducemult}[5]{\ensuremath{{#1}\mid{#2}\mid{#3}\longrightarrow^{*}{#4}\mid{#5}}}
\newcommand{\Active}[1]{\ensuremath{\mathbf{Active}\;{#1}}}
\newcommand{\Inert}[1]{\ensuremath{\mathbf{Inert}\;{#1}}}
\newcommand{\bigstep}[5]{\ensuremath{{#1}\mid{#2}\vdash{#3}\Downarrow{#4}\mid{#5}}}
\newcommand{\bigstepe}[5]{\ensuremath{{#1}\mid{#2}\vdash{#3}\Downarrow_\epsilon{#4}\mid{#5}}}
%% coercions
\newcommand{\id}[1]{\ensuremath{\mathit{\mathbf{id}}(#1)}}
\newcommand{\up}{\ensuremath{\boldsymbol{\uparrow}}}
\newcommand{\inj}[1]{\ensuremath{{#1}\,\boldsymbol{!}}}
\newcommand{\seq}{\ensuremath{\boldsymbol{;}\,}}
\newcommand{\err}[3]{\ensuremath{\boldsymbol{\bot}^{\bl{#3}}\;{#1}\;{#2}}}
\newcommand{\proj}[2]{\ensuremath{{#1}\,\boldsymbol{?}^{\bl{#2}}}}
\newcommand{\coerc}[2]{\ensuremath{{#1}\boldsymbol{,}\,{#2}}}
\newcommand{\refco}[2]{\ensuremath{\mathbf{Ref}\;{#1}\;{#2}}}
\newcommand{\funco}[3]{\ensuremath{{#1}\boldsymbol{,}\,{#2}\boldsymbol{\rightarrow}{#3}}}
\newcommand{\precctx}[8]{\ensuremath{{#1};{#2};{#3};{#4};{#5};{#6};{#7};{#8}}}
\newcommand{\ccprec}[4]{\ensuremath{\vdash{#1}\sqsubseteq{#2}\Leftarrow{#3}\sqsubseteq{#4}}}

% DynIFC: dynamic IFC with a labeled heap
\newcommand{\dynconst}[2]{\ensuremath{\ccsyntax{(\$}\;{#1}\ccsyntax{)}_{#2}}}
\newcommand{\dynaddr}[2]{\ensuremath{\ccsyntax{(addr}\;{#1}\ccsyntax{)}_{#2}}}
\newcommand{\dynlam}[3]{\ensuremath{\ccsyntax{(${\lambda}$}{#1}\ccsyntax{.}\,{#2}\ccsyntax{)}_{#3}}}
\newcommand{\dynif}[3]{\ensuremath{\ccsyntax{if}\;{#1}\;{#2}\;{#3}}}
\newcommand{\dynlet}[3]{\ensuremath{\ccsyntax{let}\;{#1}={#2}\;\ccsyntax{in}\;{#3}}}
\newcommand{\dynref}[3]{\ensuremath{\ccsyntax{ref}^{\ccsyntax{#1}}\;{#2}\;{#3}}}
\newcommand{\dynderef}[1]{\ensuremath{\ccsyntax{!}\;{#1}}}
\newcommand{\dynassign}[3]{\ensuremath{{#1}\;\ccsyntax{:=}^{\ccsyntax{#2}}\;{#3}}}
\newcommand{\dynprot}[2]{\ensuremath{\ccsyntax{prot}\;{#1}\;{#2}}}

\newcommand{\highlight}[2]{\colorbox{#1}{\ensuremath{#2}}}
\newcommand{\highlightblue}[1]{\highlight{White!90!NavyBlue}{#1}}
\newcommand{\highlightred}[1]{\highlight{White!90!Maroon}{#1}}

\definecolor{highlight}{gray}{0.9}
\sethlcolor{highlight}

\colorlet{red}{OrangeRed}
\colorlet{green}{PineGreen}

\lstdefinestyle{tt}{
  basicstyle=\ttfamily\small,
  autogobble,
  escapechar=|
}

%%%%%%%%%%%%%%%%%%%%%%%%%%%%%%%%%%%
% Bibliography
%%%%%%%%%%%%%%%%%%%%%%%%%%%%%%%%%%%

%Add your bibliography file here
\bibliography{references,all}
% prevent certain fields in references from printing in bibliography
\AtEveryBibitem{\clearfield{issn}}
\AtEveryBibitem{\clearlist{issn}}

\AtEveryBibitem{\clearfield{language}}
\AtEveryBibitem{\clearlist{language}}

\AtEveryBibitem{\clearfield{doi}}
\AtEveryBibitem{\clearlist{doi}}

\AtEveryBibitem{\clearfield{url}}
\AtEveryBibitem{\clearlist{url}}

\AtEveryBibitem{%
  \ifentrytype{online}
    {}
    {\clearfield{urlyear}\clearfield{urlmonth}\clearfield{urlday}}}

%%%%%%%%%%%%%%%%%%%%%%
% Start of Document
%%%%%%%%%%%%%%%%%%%%%%

\begin{document}
%\doublespacing  %set line spacing

%%%%%%%%%%%%%%%%%%%%%%%%%%%%%%%%%%%%%
% Title Page
%%%%%%%%%%%%%%%%%%%%%%%%%%%%%%%%%%%%%
\currentpdfbookmark{Title Page}{titlePage}  %add PDF bookmark for this page
\input{titlePage.tex}

%%%%%%%%%%%%%%%%%%%%%%%%%%%%%%%%%%%%%
% Approval Page
%%%%%%%%%%%%%%%%%%%%%%%%%%%%%%%%%%%%%
\pagenumbering{roman}
\setcounter{page}{2} % set the page number appropriately based on the number of intro pages
\newpage
%% Define your committee members. If you have less than 6, simple delete/comment the unused lines

\newcommand{\committeeChairpersonTypedName}{Jeremy G. Siek}
\newcommand{\committeeChairpersonPostNominalInitials}{Ph.D.}

\newcommand{\committeeMemberTwoTypedName}{Amr Sabry}
\newcommand{\committeeMemberTwoPostNominalInitials}{Ph.D.}

\newcommand{\committeeMemberThreeTypedName}{Chung-chieh Shan}
\newcommand{\committeeMemberThreePostNominalInitials}{Ph.D.}

\newcommand{\committeeMemberFourTypedName}{Sam Tobin-Hochstadt}
\newcommand{\committeeMemberFourPostNominalInitials}{Ph.D.}

% Uncomment to add another committee member
%\newcommand{\committeeMemberFiveTypedName}{Name}
%\newcommand{\committeeMemberFivePostNominalInitials}{Post-Nominal Initials}

\newcommand{\myRule}{\rule{0.5\textwidth}{0.4pt}}

\newcommand{\approvalDay}{02}
\newcommand{\approvalMonth}{05}
\newcommand{\approvalYear}{2025}

%%%%%%%%%%%%%%%%%%%%%%%%%%%%%%%%%%%%%%%%%%%%%%%%%%%%%%%%%
% Do not edit these lines unless you wish to customize
% the template
%%%%%%%%%%%%%%%%%%%%%%%%%%%%%%%%%%%%%%%%%%%%%%%%%%%%%%%%%


\newgeometry{left=1in}

\begin{center}
 
Accepted by the Graduate Faculty, Indiana University, in partial fulfillment of the requirements for the degree of Doctor of Philosophy.

\end{center}

\vspace{2\baselineskip}

\ifdefined\committeeMemberFourTypedName
Doctoral Committee\\

\null\hfill \myRule\\
\null\hfill \committeeChairpersonTypedName, \committeeChairpersonPostNominalInitials\\
\null\hfill \myRule\\
\null\hfill \committeeMemberTwoTypedName, \committeeMemberTwoPostNominalInitials\\
\null\hfill \myRule\\
\null\hfill \committeeMemberThreeTypedName, \committeeMemberThreePostNominalInitials\\
\null\hfill \myRule\\
\null\hfill \committeeMemberFourTypedName, \committeeMemberFourPostNominalInitials\\

\ifdefined\committeeMemberFiveTypedName
\null\hfill \myRule\\
\null\hfill \committeeMemberFiveTypedName, \committeeMemberFivePostNominalInitials\\
\fi

\fi
\vfill
Date of Defense: \approvalMonth/\approvalDay/\approvalYear
\restoregeometry


%%%%%%%%%%%%%%%%%%%%%%%%%%%%%%%%%%%%%
% Copyright
%%%%%%%%%%%%%%%%%%%%%%%%%%%%%%%%%%%%%
\newpage
\input{copyright.tex}

%%%%%%%%%%%%%%%%%%%%%%%%%%%%%%%%%%%%%
% Dedication
%%%%%%%%%%%%%%%%%%%%%%%%%%%%%%%%%%%%%
\newpage
% Define your dedication statement here

\newcommand{\yourDedication}{To my parents and my grandmother}

%%%%%%%%%%%%%%%%%%%%%%%%%%%%%%%%%%%%%%%%%%%%%%%%%%%%%%%%%
% Do not edit these lines unless you wish to customize
% the template
%%%%%%%%%%%%%%%%%%%%%%%%%%%%%%%%%%%%%%%%%%%%%%%%%%%%%%%%%

\begin{center}

\vspace*{\fill}
\yourDedication\\
\vspace*{\fill}

\end{center}


%%%%%%%%%%%%%%%%%%%%%%%%%%%%%%%%%%%%%
% Acknowledgments
%%%%%%%%%%%%%%%%%%%%%%%%%%%%%%%%%%%%%
\newpage
\phantomsection
\addcontentsline{toc}{chapter}{Acknowledgements}
\begin{centering}
\textbf{ACKNOWLEDGEMENTS}\\
\vspace{\baselineskip}
\end{centering}

First and foremost, I would like to thank my advisor, Prof. Jeremy Siek for his
outstanding guidance and unwavering support. Prof. Siek is my role model for
being a researcher. He encourages me to think in a deep and rigorous way while
presenting the results in a lucid and accessible way. Prof. Siek is also my role
model for being an educator. He shows me how to present the course materials in
an incremental and easily accessible way that is relatable to the experiences of
the students. I owe all the achievements during my graduate studies to Prof.
Siek.

I would like to thank the professors in my Ph.D. advisory committee (in no
particular order): Prof. Amr Sabry, Prof. Chung-chieh Shan, and Prof. Sam
Tobin-Hochstadt for providing feedback and suggestions to my research and this
dissertation. In addition, I would like to thank Prof. Dan Friedman: like many
students at Indiana University, I was introduced to the world of programming
languages by Prof. Friedman's ``little books'' and the lectures of C311/B521
``Programming Language Principles.''

I would like to thank the exceptional community of programming language
researchers at Indiana University. I am especially grateful to the following
professors, students, and postdoctoral researchers (in no particular order):
Prof. Carlo Angiuli, Prof. Ryan Newton, Dr. David Christiansen, Joshua Crotts,
Caner Derici, Chenchao Ding, Fred Fu, Aria Givens, Ethan Hawk, Artem Iurchenko,
Sanad Kadu, Caleb Schultz Kisby, Darshal Shetty, Zixiu Su, Jifeng Wu, Yafei
Yang, Andre Kuhlenschmidt, Ryan Scott, Tulip Amalie, Kartik Sabharwal, Annie
Pompa, Vikraman Choudhury, Sam Bowman, Chaitanya Koparkar, Rajan Walia, Sarah
Spall, Weixi Ma, Matthew Heimerdinger, Chao-Hong Chen, Michael Vollmer (now
professor at the University of Kent), Victoria Vollmer, Kuang-Chen Lu, Joshua
Larkin, Deyaaeldeen Almahallawi, Aaron Hsu, Paulette Koronkevich, Andrew Kent,
and Jason Hemann (now professor at Seton Hall University). \textit{Thank you!}

I thank Zihan Chen, Che Jingyin, Katharine Khamhaengwong, Jacob Striebel,
Xiaorui Pan, and Kan Yuan at Indiana University for friendship.

I am grateful to all the students whom I taught. In particular, I would like to
thank Calvin Josenhans, Gautam Hari, Sparsh Nair, Oleksandra Tkachuk, Zeshawn
Zahid, Jacob Herbert, Dylan Jacoby, Solomon Zinn Krulewitch, and Yuntian Zeng. I
also thank Dr. Akesha Horton for her lectures on pedagogy.

I thank Dr. Vilhelm Sj\"{o}berg for hosting me as a summer intern on software
verification at CertiK. I thank Prof. Kristopher Micinski, Prof. Sergey Bratus,
and Prof. Gang Tan for advice and discussion.

I would like to thank the operating system research group at Columbia University
led by Prof. Junfeng Yang, who provided a valuable summer research opportunity
to me when I was an undergrad. I appreciate the friendship and help from (in no
particular order) Dr. Heming Cui (now professor at the University of Hong Kong),
Rui Gu, Yang Tang (now professor at New York University), Gang Hu, Xinhao Yuan,
Lingmei Weng, and Chang Lou (now professor at the University of Virginia). I
thank Prof. Yu-Ping Wang for lecturing me on symbolic execution while I was
doing undergraduate independent study at Tsinghua. Also, I express special
gratitude to the friends and roommates from my undergraduate days at Tsinghua:
Hongyin Luo, Yuan Yang, Hao Wang, Hongyi Wen (now professor at New York
University Shanghai), Xintian Li, Hanxuan Yu, and S\^{e}kai Zhou.

I would like to thank Hattie Fu for her love, support, and company during the
final (and one of the hardest) stage of my dissertation writing.

I thank my grandmother, who taught physics and electrical engineering before
retirement, for introducing me to mathematics and applied sciences. I thank my
aunts for buying me the first computer in my life: a used graphic workstation
with an Intel 80486 processor. Lastly, and most importantly, I would like to say
thank you to my parents. I could not have completed this quest without your
support and understanding.

(This dissertation is based upon work supported by the National Science
Foundation under Grant No. 1763922 ``Performant Sound Gradual Typing.'')


%%%%%%%%%%%%%%%%%%%%%%%%%%%%%%%%%%%%%
% Abstract
%%%%%%%%%%%%%%%%%%%%%%%%%%%%%%%%%%%%%
\newpage
\phantomsection
\addcontentsline{toc}{chapter}{Abstract}
%%%%%%%%%%%%%%%%%%%%%%%%%%%%%%%%%%%%%%%%%%%%%%%%%%%%%%%%%
% Do not edit these lines unless you wish to customize
% the template
%%%%%%%%%%%%%%%%%%%%%%%%%%%%%%%%%%%%%%%%%%%%%%%%%%%%%%%%%
\newgeometry{left=1in}

\begin{center}

\yourName\\
\MakeUppercase{\thesisTitle}

\end{center}

\vspace{1.5\baselineskip}

Ensuring the security and privacy of personal data typically involves tracking
and checking the flow of information, which can be performed either statically
using a type system or dynamically using runtime monitoring. The dynamic
approach of information-flow control (IFC) requires less effort from the
programmer while the static approach provides stronger guarantees and less
runtime overhead. Languages with gradual IFC combine static and dynamic
techniques to prevent security leaks, so the programmer can choose when it is
appropriate to increase the precision of the type annotations and put in the
effort to pass the static checks and when it is appropriate to reduce the
precision of type annotations, deferring the enforcement to runtime. Gradual
programming languages should satisfy the gradual guarantee: programs that only
differ in the precision of their type annotations should behave the same modulo
cast errors. Unfortunately, Toro et al. [2018] identify a tension between the
gradual guarantee and information security. They conjecture that it is not
possible to enforce noninterference and satisfy the gradual guarantee.

In my PhD dissertation, I harmoniously combine static and dynamic enforcement of
IFC in one programming language, \Surface, which satisfies both noninterference
and the gradual guarantee at the same time without making any
sacrifices. \Surface (1) enforces information flow security, (2) satisfies the
gradual guarantee, (3) supports type-based reasoning, and (4) requires no extra
static analysis prior to program execution. The key to the design of \Surface is
to exclude the unknown label from runtime security labels. On the technical
side, the semantics of \Surface is the first gradual information-flow control
language to be specified using coercion calculi (a la Henglein). Casts
in \Surface are represented by security coercions, which enforce the flow of
information while satisfying the gradual guarantee.

I mechanize the proofs of type safety and the gradual guarantee for \Surface in
the Agda proof assistant. I prove noninterference for \Surface by simulating
\Surface with its dynamic extreme.

In summary, my thesis is that it is possible to design a gradual IFC programming
language that satisfies noninterference and the gradual guarantee while
supporting type-based reasoning, by excluding the unknown label from run-time
security labels and using security coercions to represent casts.

\ifdefined\committeeMemberFourTypedName

\null\hfill \myRule\\
\null\hfill \committeeChairpersonTypedName, \committeeChairpersonPostNominalInitials\\
\null\hfill \myRule\\
\null\hfill \committeeMemberTwoTypedName, \committeeMemberTwoPostNominalInitials\\
\null\hfill \myRule\\
\null\hfill \committeeMemberThreeTypedName, \committeeMemberThreePostNominalInitials\\
\null\hfill \myRule\\
\null\hfill \committeeMemberFourTypedName, \committeeMemberFourPostNominalInitials\\

\ifdefined\committeeMemberFiveTypedName
\null\hfill \myRule\\
\null\hfill \committeeMemberFiveTypedName, \committeeMemberFivePostNominalInitials\\
\fi

\fi
\restoregeometry


%%%%%%%%%%%%%%%%%%%%%%%%%%%%%%%%%%%%%
% Table of Contents
%%%%%%%%%%%%%%%%%%%%%%%%%%%%%%%%%%%%%

% Format for Table of Contents
\renewcommand{\cftchapdotsep}{\cftdotsep}  %add dot separators
\renewcommand{\cftchapfont}{\bfseries}  %set title font weight
\renewcommand{\cftchappagefont}{}  %set page number font weight
\renewcommand{\cftchappresnum}{Chapter }
\renewcommand{\cftchapaftersnum}{:}
\renewcommand{\cftchapnumwidth}{5em}
\renewcommand{\cftchapafterpnum}{\vskip\baselineskip} %set correct spacing for entries in single space environment
\renewcommand{\cftsecafterpnum}{\vskip\baselineskip}  %set correct spacing for entries in single space environment
\renewcommand{\cftsubsecafterpnum}{\vskip\baselineskip} %set correct spacing for entries in single space environment
\renewcommand{\cftsubsubsecafterpnum}{\vskip\baselineskip} %set correct spacing for entries in single space environment

%format title font size and position (this also applys to list of figures and list of tables)
\titleformat{\chapter}[display]
{\normalfont\bfseries\filcenter}{\chaptertitlename\ \thechapter}{0pt}{\MakeUppercase{#1}}

\renewcommand\contentsname{Table of Contents}
\currentpdfbookmark{Table of Contents}{TOC}
\begin{singlespace}
\tableofcontents
\end{singlespace}


\clearpage

%%%%%%%%%%%%%%%%%%%%%%%%%%%%%%%%%%%%%
% List of figures and tables
%%%%%%%%%%%%%%%%%%%%%%%%%%%%%%%%%%%%%
\phantomsection
\addcontentsline{toc}{chapter}{List of Tables}
\begin{singlespace}
\setlength\cftbeforetabskip{\baselineskip}  %manually set spacing between entries
\listoftables
\end{singlespace}

\clearpage

\phantomsection
\addcontentsline{toc}{chapter}{List of Figures}
\begin{singlespace}
\setlength\cftbeforefigskip{\baselineskip}  %manually set spacing between entries
\listoffigures
\end{singlespace}

\clearpage

%%%%%%%%%%%%%%%%%%%%%%%%%%%%
%
% Chapters
%
%%%%%%%%%%%%%%%%%%%%%%%%%%%%

%%%%%%%%%%%%%%%%%%%%%%
% formatting
%%%%%%%%%%%%%%%%%%%%%%

% resume page numbering for rest of document
\clearpage
\pagenumbering{arabic}
\setcounter{page}{1} % set the page number appropriately

% Adjust chapter title formatting
\titleformat{\chapter}[display]
{\normalfont\bfseries\filcenter}{\MakeUppercase\chaptertitlename\ \thechapter}{0pt}{\MakeUppercase{#1}}  %spacing between titles
\titlespacing*{\chapter}
  {0pt}{0pt}{30pt}	%controls vertical margins on title

% Adjust section title formatting
\titleformat{\section}{\normalfont\bfseries}{\thesection}{1em}{#1}

% Adjust subsection title formatting
\titleformat{\subsection}{\normalfont\bfseries}{\thesubsection}{1em}{#1}

% Adjust subsubsection title formatting
\titleformat{\subsubsection}{\normalfont\itshape}{\thesubsubsection}{1em}{#1}

%%%%%%%%%%%%%%%%
% Main text
%%%%%%%%%%%%%%%%

\chapter{Introduction}

\section{Motivation}

With the development of digital society, people are increasingly
concerned about the confidentiality of their personal data and
the integrity of their online assets. Increasingly relying on
computing devices and the Internet in their daily life, people fear that
sensitive personal information, such as social security numbers,
medical records, bank account balances... may be revealed to malicious
third parties. People also worry that their digital photo albums,
signatures on online legal documents, spreadsheets in cloud storage...
may be tampered and manipulated by potential attackers.

Indeed, the fears are justified by recent new events.
In 2018, the Cambridge Analytica scandal hit the world headlines,
where the data collected from 87 million social media users was misused
without their consent
~\parencite{cadwalladr2018facebook,kitchgaessner2017cambridge,gonzalez2019global,hinds2020wouldn}.
In the healthcare sector, from 2005 to 2019, 249.09 million individuals
were affected by data breaches that caused exposure of sensitive medical
data~\parencite{seh2020healthcare}. In respect of data integrity, researchers
have found ways to tamper the analytics APIs~\parencite{pfeffer2018tampering} and
the metadata (such as the numbers of likes, follows, and views)~\parencite{paquet2017can}
of major social media platforms. To deal with the security and privacy
challenges of the increasingly digitalized world, the European Union introduced
the General Data Protection Regulation (GDPR) to reform and regulate
the collection and processing of personal data. However, studies show
that business entities experience challenges in complying with GDPR
or auditing for compliance~\parencite{smirnova2024understanding},
particularly small-to-medium enterprises~\parencite{sirur2018we,freitas2018gdpr,harting2021impacts}.

\begin{figure}[tbp]
  \small
  \begin{align*}
    \langle RECORD \rangle ::= & {\color{blue} \textbf{\{FirstName=}} \langle ID \rangle {\color{blue} \textbf{;}} \\
                               & \enspace {\color{blue} \textbf{LastName=}} \langle ID \rangle {\color{blue} \textbf{;}} \\
                               & \enspace {\color{blue} \textbf{SSN=}} \langle SSN \rangle {\color{blue} \textbf{\}}} \\
    \langle ID \rangle     ::= & w , w \in \{ {\color{blue} \textbf{A}}, ... {\color{blue} \textbf{Z}}, {\color{blue} \textbf{a}}, ... {\color{blue} \textbf{z}} \}^{+} \\
    \langle SSN \rangle    ::= & \langle D \rangle \langle D \rangle \langle D \rangle {\color{blue} \textbf{-}}
                                 \langle D \rangle \langle D \rangle {\color{blue} \textbf{-}}
                                 \langle D \rangle \langle D \rangle \langle D \rangle \langle D \rangle \\
    \langle D \rangle      ::= & d , d \in \{ {\color{red} \textbf{0}}, ... {\color{red} \textbf{9}} \}
  \end{align*}
  \caption{The user input grammar for a hypothetical application}
  \label{fig:grammar}
\end{figure}

From a technical perspective, ensuring the security and privacy of
personal data typically involves tracking and checking
the flow of information.
Modern software applications often accept user input where
selected fields are sensitive, whose confidentiality is required
during both parsing and processing. To rule out information leaks,
neither the sensitive fields, nor any data that depends on those fields,
is allowed to be revealed to a low-privilege observer.
Consider a web application that receives three fields from its user:
1) first name 2) last name 3) social security number, the grammar of which
is defined in figure \ref{fig:grammar}, where terminals are divided into
low-security and high-security.

\section{Problem Statement}

\section{Thesis Statement}

%\section{\Surface}

\section{Outline}


\chapter{Gradual Information-Flow Control (IFC) in \Surface}
\label{ch:examples}

In this chapter, we first define the gradual IFC surface language \Surface, by
presenting its syntax and type system in Section~\ref{sec:surface-def}. After
that, we put \Surface into action. We present example programs to demonstrate
how \Surface enables a gradual, smooth transition between static and dynamic
IFC, while supporting type-based reasoning and satisfying the gradual guarantee.
We review the basics of gradual IFC using \Surface programs in
Section~\ref{sec:example1}. In Section~\ref{sec:example2}, we show that the
tension between security and the gradual guarantee can be achieved by removing
\unk from the runtime security labels. In Section~\ref{sec:example3}, we
demonstrate that \Surface enables the same type-based reasoning capabilities as
\GSLRef.

\section{The Gradual IFC Language \Surface}
\label{sec:surface-def}

We first define the gradual language \Surface. It is similar to \GSLRef with
respect to syntax and typing rules. The main syntactic difference is that in
\Surface, the security labels of literals and newly created memory cells default
to a specific label such as \low, while in \GSLRef they default to a runtime
unknown security level \unk.

\section{The Gradual Transition Between Static and Dynamic IFC in \Surface}
\label{sec:example1}

For simplicity, we use the security lattice $\langle \{\high, \low\}
, \preccurlyeq, \curlyvee , \curlywedge \rangle$, where $\high$ is for
private data and $\low$ is for publicly disclosable data. The ordering
is standard: $\low \preccurlyeq \high$ and
$\high \npreccurlyeq \low$. So information is allowed to flow from
public sources to private sinks but not the other way around.
We refer to $\{\high,\low\}$ as \emph{specific security labels}.

\section{Implicit Flow, NSU Checks, Unknown Security, and the Gradual Guarantee}
\label{sec:example2}

\section{Type-Based Reasoning in \Surface}
\label{sec:example3}


\chapter{The Definition of the Cast Calculus \CC}
\label{ch:sem}

\todo[inline]{Add sentences about why using coercions as the cast representation}

In this chapter, I first present a coercion calculus for security labels in
Section~\ref{sec:coercion-calc-labels}. I show that coercion on labels can model
both explicit and implicit information flows. I then define a second coercion
calculus whose purpose is to cast a program value from one type to another type
in Section~\ref{sec:coercion-calc-values}. I use this second coercion calculus
as the representation of casts in the intermediate language \CC. With this cast
representation in hand, I am going to present the full definition of \CC in
Section~\ref{sec:cc}. Finally, I show that \Surface can be compiled into \CC in
Section~\ref{sec:compile}. As a result, the semantics of \Surface is given by
the semantics of \CC.

\section{A Coercion Calculus for Security Labels}
\label{sec:coercion-calc-labels}

In this section, I describe a coercion calculus on security labels. This
coercion calculus is an important stepping stone to our representation of casts
between security types. I first explain the motivation of using coercions to
represent information flows (Section~\ref{sec:motiv}). I then demonstrate that
we can use coercion composition to model explicit flows and use coercion
stamping to model implicit flows (Section~\ref{sec:cexpr-comp-stamp}). Finally,
I define how these coercions act on security labels by defining a language of
label expressions whose meaning is defined by a reduction relation
(Section~\ref{sec:lexpr}). Label expressions are used to model security checks
for the heap policy in \CC.

\subsection{Why Coercions?}
\label{sec:motiv}

\begin{figure}[tbp]
\raggedright
  \[
  \begin{array}{rcll}
    \text{specific security labels} & \ell & \in & \{ \low , \high \} \\
    \text{security labels}  & g    & ::= & \unk \MID \ell \\
    \text{blame labels}         & \bl{p}, \bl{q}     &      & \\
    \text{security coercions}            & c, d     & ::=  & \id{g} \MID \up \MID \inj{\ell} \MID \proj{\ell}{p} \MID \bot^{\bl{p}} \\
    \text{coercion sequences} & \bar{c}, \bar{d} & ::=  & \id{g} \MID \err{g_1}{g_2}{p} \MID \bar{c} \seq c
  \end{array}
  \]
  \fbox{$\vdash c : g_1 \Rightarrow g_2$}
  {\small
  \begin{gather*}
    \inference{}{\vdash \id{g} : g \Rightarrow g}
    \quad
    \inference{}
              {\vdash \,\up\, : \low \Rightarrow \high}
    \quad
    \inference{}
              {\vdash \inj{\ell} : \ell \Rightarrow \unk}
    \\[1ex]
    \inference{}
              {\vdash \proj{\ell}{p} : \unk \Rightarrow \ell}
    \qquad
    \inference{}
              {\vdash \bot^{\bl{p}} : \high \Rightarrow \low}
  \end{gather*}}
  \fbox{$\vdash \bar{c} : g_1 \Rightarrow g_2$}
  {\small
  \begin{gather*}
    \inference{}{\vdash \id{g} : g \Rightarrow g}
    \quad
    \inference{\vdash \bar{c} : g_1 \Rightarrow g_2 & \vdash c : g_2 \Rightarrow g_3}
              {\vdash \bar{c} \seq c : g_1 \Rightarrow g_3}
    \quad
    \inference{}
              {\vdash \err{g_1}{g_2}{p} : g_1 \Rightarrow g_2}
  \end{gather*}}
  \fbox{$\mathbf{NF}\; \bar{c}$}
  {\small
  \begin{gather*}
  \inference{}{\mathbf{NF}\; \id{g}}
  \quad
  \inference{}{\mathbf{NF}\; \id{\unk} \seq \proj{\ell}{p}}
  \quad
  \inference{\mathbf{NF}\; \bar{c}}{\mathbf{NF}\; \bar{c}\seq\inj{\ell}}
  \quad
  \inference{\mathbf{NF}\; \bar{c}}{\mathbf{NF}\; \bar{c}\seq\up}
  \end{gather*}}
  \fbox{$c \mathrel{;} c \longrightarrow c$}
  {\small
  \begin{gather*}
  \textit{?-id}~
  \inference{}
            {\inj{\ell} \seq \proj{\ell}{p} \longrightarrow \id{\ell}}
  \quad
  \textit{?-}\uparrow~
  \inference{}
            {\inj{\low} \seq \proj{\high}{p} \longrightarrow \;\;\up}
  \quad
  \textit{?-}\bot~
  \inference{}
            {\inj{\high} \seq \proj{\low}{p} \longrightarrow \bot^{\bl{p}}}
  \end{gather*}}
  \fbox{$\bar{c} \longrightarrow \bar{d}$}
  {\small
  \begin{gather*}
  \textit{id}~
  \inference{\mathbf{NF}\; \bar{c}}
            {\bar{c} \seq \id{g} \longrightarrow \bar{c}}
  \qquad
  \bot~
  \inference{\mathbf{NF}\; \bar{c} & \vdash \bar{c} : g_1 \Rightarrow g_2}
            {\bar{c} \seq \bot^{\bl{p}} \longrightarrow \err{g_1}{\low}{p} }
  \qquad
  \xi\textit{-}\bot~
  \inference{\vdash c : g_2 \Rightarrow g_3}
    {\err{g_1}{g_2}{p} \seq c \longrightarrow \err{g_1}{g_3}{p}}
  \\[1ex]
  \xi_L~
  \inference{\bar{c} \longrightarrow \bar{d}}{\bar{c} \seq c \longrightarrow \bar{d} \seq c}
  \qquad
  \xi_R~
  \inference{\mathbf{NF}\; \bar{c} & c \seq d \longrightarrow c'}
            {\bar{c} \seq c \seq d \longrightarrow \bar{c}; c'}
  \end{gather*}}
  \caption{Syntax, typing, normal forms, and semantics of security coercions and coercion sequences}
  \label{fig:cexpr}
\end{figure}

As we have seen in Section~\ref{sec:examples}, gradual information flows can be
modeled as casts. For example, the cast sequence $\high \Rightarrow \unk
\Rightarrow \low$ should be statically accepted but dynamically rejected, while
the sequence $\low \Rightarrow \unk \Rightarrow \high$ should be statically and
dynamically accepted, promoting the security of data to high. Such sequences of
casts can be arbitrarily long (for example, $\low \Rightarrow \high \Rightarrow
\unk \Rightarrow \unk \Rightarrow \low$), which motivates us to model the casts
on security labels as coercions.

In \Surface, the source security label of a coercion sequence comes from
literals, while the sink is whatever security level that the observer has: for
example, the \texttt{publish} function of Section~\ref{sec:examples} is of \low.
Coercions can be easily sequenced and composed. Checking information flow at
runtime is accomplished by reducing coercion sequences to their normal forms.

There are two noteworthy benefits of the coercion representation for IFC. First,
coercions can be used to represent NSU checking while satisfying the gradual
guarantee. In brief, whenever a memory location is written to, the current PC is
coerced to the security level of that location. We are going to formally
introduce label expressions as our representation for PC in
Section~\ref{sec:lexpr} and discuss NSU in detail in
Section~\ref{sec:semantics}. Second, the coercion representation benefits
mechanization because it enables modular reasoning. The main simulation lemma
(Lemma~\ref{lem:sim}) depends on the simulation results of coercion sequences
and label expressions, which are stated as separate lemmas and reasoned
independently in our Agda code.

The syntax and typing for security coercions and coercion sequences is defined
in Figure~\ref{fig:cexpr}. A security label is either $\low$, $\high$, or
statically unknown (\unk). There are five security coercions: identity (\id{g}),
subtype (\up), injection (\inj{\ell}), and projection (\proj{\ell}{p}), and
blame ($\boldsymbol{\bot}^{\bl{p}}$). Projection, which corresponds to the
notion of a runtime check, is the only one responsible for blame, so it carries
a blame label \bl{p}. A coercion sequence $\bar{c}$ starts with either success
\id{g} or failure (\err{g_1}{g_2}{p}). Each coercion has a source and target
type $g_1 \Rightarrow g_2$. The \id{g} casts the label $g$ to itself;
\up~promotes security from \low to \high; injection casts to \unk from a
specific label $\ell$ and projection does the opposite. Appending a single
coercion to a coercion sequence makes the target security label that of the
single coercion.

Information flow is enforced in the reduction semantics of security coercions,
shown in Figure~\ref{fig:cexpr}. Injection followed by projection to the same
label collapses to the identity (\textit{?-id}). Flowing from \low to \high is
allowed, so an injection from \low followed by a projection to \high collapses
into the \up{} coercion~($\textit{?-}\uparrow$). An information flow from \high
to \low is prohibited, so an injection to \high followed by a projection to \low
triggers an error that blames the projection ($\textit{?-}\bot$). The predicate
$\mathbf{NF}$ that specifies the normal forms of coercion sequences. The
reduction rules for coercion sequences are also defined in
Figure~\ref{fig:cexpr}. Appending \id{g} onto a coercion sequence reduces to the
that sequence (\textit{id}). The failure coercions annihilate the other
coercions in the sequence ($\bot$ and $\xi\textit{-}\bot$). We choose the
evaluation order in a coercion sequence to be from left to right ($\xi_L$ and
$\xi_R$), because that corresponds to the direction of information flow from
source to sink: in the example above, $\low \Rightarrow \high \Rightarrow \unk
\Rightarrow \unk \Rightarrow \low$, we validate that \low can flow to \high
before we check the flow from \high through \unk to \low.

\subsection{Monitoring Explicit and Implicit Flows}
\label{sec:cexpr-comp-stamp}

\begin{figure}[tbp]
\raggedright
  \fbox{$\bar{c}\mdoubleplus\bar{c} = \bar{c}$}
  {\small
\begin{align*}
\bar{c} \mdoubleplus \err{g_2}{g_3}{p} &= \err{g_1}{g_3}{p}  \quad\text{where $\vdash \bar{c} : g_1 \Rightarrow g_2$} \\
\bar{c} \mdoubleplus \id{g} &= \bar{c} \seq \id{g} \\
\bar{c}_1 \mdoubleplus (\bar{c}_2 \seq c) &= (\bar{c}_1 \mdoubleplus \bar{c}_2) \seq c
\end{align*}
  }

  \fbox{$\mathit{stamp} \; \bar{c} \; \ell = \bar{c}$}\hfill
  {\small
\begin{align*}
\mathit{stamp} \; \bar{c} \; \low &= \bar{c} \\
\mathit{stamp} \; \id{\low} \; \high &= \id{\low}\seq\up \\
\mathit{stamp} \; \id{\high} \; \high &= \id{\high} \\
\mathit{stamp} \; (\id{\low}\seq\inj{\low}) \; \high &= \id{\low}\seq\up\seq\inj{\high} \\
\mathit{stamp} \; (\id{\high}\seq\inj{\high}) \; \high &= \id{\high}\seq\inj{\high} \\
\mathit{stamp} \; (\id{\low}\seq\up\seq\inj{\high}) \; \high &= \id{\low}\seq\up\seq\inj{\high} \\
\mathit{stamp} \; (\id{\low}\seq\up) \; \high &= \id{\low}\seq\up
\end{align*}
}

  \fbox{$\mathit{stamp!} \; \bar{c} \; \ell = \bar{c}$}
  {\small
\begin{align*}
\mathit{stamp!} \; \bar{c} \; \low &=
  \begin{cases}
  \bar{c} & \text{~if} \vdash \bar{c} : \ell \Rightarrow \unk \\
  \bar{c}\seq\inj{\ell_2} & \text{~if} \vdash \bar{c} : \ell_1 \Rightarrow \ell_2
  \end{cases}
  \\
\mathit{stamp!} \; \id{\low} \; \high &= \id{\low}\seq\up\seq\inj{\high} \\
\mathit{stamp!} \; \id{\high} \; \high &= \id{\high}\seq\inj{\high} \\
\mathit{stamp!} \; (\id{\low}\seq\inj{\low}) \; \high &= \id{\low}\seq\up\seq\inj{\high} \\
\mathit{stamp!} \; (\id{\high}\seq\inj{\high}) \; \high &= \id{\high}\seq\inj{\high} \\
\mathit{stamp!} \; (\id{\low}\seq\up\seq\inj{\high}) \; \high &= \id{\low}\seq\up\seq\inj{\high} \\
\mathit{stamp!} \; (\id{\low}\seq\up) \; \high &= \id{\low}\seq\up\seq\inj{\high}
\end{align*}
}
  \caption{Composing and stamping coercions}
  \label{fig:comp-stampc}
\end{figure}

We model explicit flow using security coercions. We can compose two coercion
sequences ($\bar{c} \mdoubleplus \bar{d}$), where $\bar{c} : g_1 \Rightarrow
g_2$ and $\bar{d} : g_2 \Rightarrow g_3$, to form a flow from $g_1$ to $g_3$,
which is defined in Figure~\ref{fig:comp-stampc}.

The stamping operation captures the intuition of an implicit flow from the
security level $\ell'$ to a coercion sequence $\bar{c}$. We define the stamping
operation in Figure~\ref{fig:comp-stampc} as two functions,
$\mathit{stamp}(\bar{c},\ell)$ and $\mathit{stamp!}(\bar{c},\ell)$. Both
function require $\bar{c}$ to be in normal form and that its source label is not
\unk. The $\mathit{stamp!}$ operator promotes the security of the coercion
$\bar{c}$ to be at least $\ell$ and then injects the coercion if necessary,
while $\mathit{stamp}$ only promotes the security but does not inject. These
stamping operations satisfy the gradual guarantee, because when stamping on a
more precise coercion sequence and a less precise coercion sequence, stamping
preserves the precision relation (Section~\ref{sec:sim-cexpr}) between them
(Lemma~\ref{lem:cexpr-stamp-sim}). The stamping operations of coercion sequences
are used in the stamping operations of (1) label expressions, which are our
representation of PC and (2) values in the cast calculus. Those three types of
stamping together formalize the notion of implicit flow in \Surface.

\subsection{Security Label Expressions}
\label{sec:lexpr}

In this section we introduce security label expressions, which we use
to model the security level of the PC. Security label expressions are
crucial for implementing NSU checking in a way that satisfies the
gradual guarantee.

\begin{figure}[tbp]
\raggedright
  \[
  \begin{array}{rcll}
    \text{label expressions} & e & ::= & \ell \MID \blame{p} \MID \cccast{e}{\bar{c}} \\
  \end{array}
  \]
  \fbox{$\vdash e \Leftarrow g$}
  {\small
  \begin{gather*}
    {\vdash}\textit{l}~
    \inference{}{\vdash \ell \Leftarrow \ell}
    \quad
    {\vdash}\textit{lcast}~
    \inference{\vdash e \Leftarrow g_1 & \vdash \bar{c} : g_1 \Rightarrow g_2}
    {\vdash \cccast{e}{\bar{c}} \Leftarrow g_2}
    \\[1ex]
    {\vdash}\textit{lblame}~
    \inference{}
    {\vdash \blame{p} \Leftarrow g}
  \end{gather*}}
  \fbox{$\mathbf{Irreducible}\; \bar{c}$}
  {\small
    \begin{gather*}
      \inference{\mathbf{NF} \; \bar{c} & \vdash \bar{c} : g_1 \Rightarrow g_2 & g_1 \neq g_2}
                {\mathbf{Irreducible} \; \bar{c}}
  \end{gather*}}
  \fbox{$\mathbf{NF} \; e$}
  {\small
  \begin{gather*}
  \inference{}{\mathbf{NF} \; \ell}
  \qquad
  \inference{\mathbf{Irreducible} \; \bar{c}}{\mathbf{NF} \; (\cccast{\ell}{\bar{c}})}
  \end{gather*}}
  \fbox{$e_1 \longrightarrow e_2$}
  {\small
  \begin{gather*}
  \xi\textit{-l}~
  \inference{e_1 \longrightarrow e_2}
  {\cccast{e_1}{\bar{c}} \longrightarrow \cccast{e_2}{\bar{c}}}
  \qquad
  \xi\textit{-lblame}~
  \inference{}{\cccast{\blame{p}}{\bar{c}} \longrightarrow \blame{p}}
  \\[2ex]
  \beta\textit{-id}~
  \inference{}{\cccast{\ell}{\id{\ell}} \longrightarrow \ell}
  \qquad
  \textit{lcast}~
  \inference{\bar{c} \longrightarrow^{+} \bar{d} & \mathbf{NF} \; \bar{d}}
  {\cccast{\ell}{\bar{c}} \longrightarrow \cccast{\ell}{\bar{d}}}
  \\[2ex]
  \textit{lblame}~
  \inference{\bar{c} \longrightarrow^{*} \err{\ell}{g}{p}}
  {\cccast{\ell}{\bar{c}} \longrightarrow \blame{p}}
  \qquad
  \textit{lcomp}~
  \inference{\mathbf{Irreducible} \; \bar{c}}
  {\cccast{\cccast{\ell}{\bar{c}}}{\bar{d}} \longrightarrow \cccast{\ell}{\bar{c} \mdoubleplus \bar{d}}}
  \end{gather*}}
  \caption{Syntax, typing, normal forms, and semantics of label expressions}
  \label{fig:lexpr}
\end{figure}

A label expression is either (1) a specific security label, (2) blame (to
signify an error), or (3) a coercion applied to a label expression
(Figure~\ref{fig:lexpr}).
%
A label expression is in normal form ($\mathbf{NF}$) if it is either (1) a
specific security label or (2) an irreducible coercion applied to a specific
security label. (A coercion is irreducible if it is a non-identity coercion in
normal form). \PC ranges over label expressions in normal form.
%
The reduction relation for label expressions steps a label expression towards
its normal form. The idea is that given a label expression of the form
$\cccast{e}{\bar{d}}$, we first reduce $e$ to normal form and then apply the
coercion $\bar{d}$. For example, if $e$ reduces to a label wrapped in coercion
$\cccast{\ell}{\bar{c}}$, then the \textit{lcomp} rule says to reduce by
composing the two coercions, producing $\cccast{\ell}{\bar{c} \mdoubleplus
  \bar{d}}$. Furthermore, in a label expression of the form
$\cccast{e}{\bar{d}}$, the coercion $\bar{d}$ may also need to be reduced, which
is accomplished by the \textit{lcast} rule that refers to the reduction relation
for coercion sequences (Figure~\ref{fig:cexpr}). If the coercion reduces to an
identity, then the coercion application goes away ($\beta$\textit{-id}), whereas
if the coercion reduces to a failure, then the label expression reduces to blame
(\textit{lblame}).

The stamping and security level operators for label expressions are defined in
Figure~\ref{fig:stamp-lval} of the Appendix. They both require their input to be
in normal form, which can be either (1) a specific security label $\ell$, or (2)
a label wrapped with an irreducible coercion sequence $\cccast{\ell}{\bar{c}}$.
For (1), stamping \low with \high results in \cccast{\low}{\up}, otherwise the
label expression remains unchanged; for (2), we directly stamp the coercion
sequence using \textit{stamp} for coercion sequences defined in
Figure~\ref{fig:comp-stampc}. The definition of \textit{stamp!} is analogous,
except that it turns to the \textit{stamp!} operator of coercion sequences. The
security level operator $|{-}|$ is defined such that (1) a specific security
label indicates the security level for itself and (2) the security of the
coercion sequence $|\bar{c}|$ records the security level for
$\cccast{\ell}{\bar{c}}$.

In Section~\ref{sec:cc}, I am going to describe how label expressions are used
to implement NSU checks, which enforce the heap policy for write operations.

\section{A Coercion Calculus on Values}
\label{sec:coercion-calc-values}

TBA

\section{The Cast Calculus \CC: An Intermediate Language For Gradual IFC}
\label{sec:cc}

TBA

\todo[inline]{Put compilation into another chapter}
\section{Compiling from \Surface to \CC}
\label{sec:compile}


\chapter{Compiling From \Surface to \CC}
\label{ch:compile}

\begin{figure}[tbp]
\raggedright
\fbox{$\compile{M} = N$}
{\footnotesize
\begin{align*}
    \small
    % App
    \intertext{Let $L, M$ be well-typed programs $\Gamma; g' \vdash L : (\Fun{A}{\gc}{B})_g, \Gamma; g' \vdash M : A'$}
    \compile{\app{L}{M}{p}} =&
    \begin{cases}
      \ccapp{(\cccast{\compile{L}}{\bm{c_1}})}{(\cccast{\compile{M}}{\bm{c_2}})}{A}{B}{g} \qquad\text{if $g$, $g'$ and $\gc$ are all specific} \\
      \qquad\text{where}~ \bm{c_1} = (\Fun{A}{\gc}{B})_g \Rightarrow^{\bl{p}} (\Fun{A}{g' \curlyvee g}{B})_g, \bm{c_2} = A' \Rightarrow^{\bl{p}} A \\
      \cccast{(\ccappstar{(\cccast{\compile{L}}{\bm{c_1}})}{(\cccast{\compile{M}}{\bm{c_2}})}{A}{T})}{\bm{d}} \quad~\text{otherwise} \\
      \qquad\text{where}~ B = T_{g''}, \bm{c_1} = (\Fun{A}{\gc}{T_{g''}})_g \Rightarrow^{\bl{p}} (\Fun{A}{\unk}{T_{\unk}})_{\unk} \\
      \qquad\qquad~ \bm{c_2} = A' \Rightarrow^{\bl{p}} A, \bm{d} = T_{\unk} \Rightarrow^{\bl{p}} (\textit{stamp}\;T_{g''}\;g)
    \end{cases}
    % Assign
    \intertext{Let $L, M$ be well-typed programs $\Gamma; g' \vdash L : (\Refer{T_{\hat{g}}})_g, \Gamma; g' \vdash M : A$}
    \compile{\assign{L}{M}{p}} =&
    \begin{cases}
    \ccassign{(\compile{L})}{(\cccast{(\compile{M})}{\bm{c_2}})}{T}{\hat{g}}{g} & ~\text{if $g$, $g'$ and $\hat{g}$ are all specific} \\
    \ccassignproj{(\cccast{(\compile{L})}{\bm{c_1}})}{(\cccast{(\compile{M})}{\bm{c_2}})}{T}{\hat{g}}{\bl{p}} & ~\text{otherwise}
    \end{cases} \\
    &\text{where}~
    \bm{c_1} = (\Refer{T_{\hat{g}}})_g \Rightarrow^{\bl{p}} (\Refer{T_{\hat{g}}})_{\unk} , \bm{c_2} = A \Rightarrow^{\bl{p}} T_{\hat{g}}
\end{align*}}
\caption{Compilation from \Surface to \CC}
\label{fig:compile}
\end{figure}

The compile function takes the form $\compile{M} = M'$, where $M$ is a \Surface
program and $M'$ is a \CC term. Consider the case for assignment. If $g$, $g'$,
and $\hat{g}$ are specific, the check for heap policy can be statically
justified. We recursively compile both $L$ and $M$ and generate a static
\texttt{assign}. We cast $M'$ using coercion $\bm{c_2}$, which casts from the
type of $M'$ to the type of the memory cell. The coercion is produced by the
coerce function, $({-}\Rightarrow^{-}{-})$, which takes two types and a blame
label, returning a coercion on values by calling the coerce function of labels
on each pair of security labels inside those two types. If at least one of the
three security labels is \unk, the typing information is insufficient to justify
the assignment. The compilation produces an \texttt{assign?} instead, whose
semantics performs NSU checking at runtime.


\chapter{Type Safety of \Surface}
\label{ch:type-safety}

\section{Type Safety of \CC by Progress and Preservation}
\label{sec:cc-type-safety}

We first prove that substitution preserves types:

\begin{lemma}[Substitution preserves types]
  \label{lem:subst-pres}
  If $(\Gamma,x{:}A);\Sigma;g;\ell\vdash N \Leftarrow B$ and \\
  $\forall g'\,\ell'.\,\Gamma;\Sigma;g';\ell' \vdash V \Leftarrow A$, then
  $\Gamma;\Sigma;g\;\ell \vdash N[x:=V] \Leftarrow B$.
\end{lemma}
\begin{proof}
  The proof is fully mechanized in \texttt{/src/CC2/SubstPreserve.agda}.
\end{proof}

We show that \CC is type safe by proving progress and preservation. Progress
says that a well-typed \CC term does not get stuck. The term is either a value
or a blame, which does not reduce, or the term takes one reduction step.

\begin{theorem}[Progress]
\label{thm:progress}
Suppose \PC is well-typed: $\vdash \PC \Leftarrow g$,
$M$ is well-typed:
\[
\emptyset ; \Sigma ; g ; | \PC | \vdash M \Leftarrow A
\]
and the heap $\mu$ is also well-typed: $\Sigma \vdash \mu$.
Then either (1) $M$ is a value or (2) $M$ is a blame: {\normalfont $M = \blame{p}$}
or (3) $M$ can take a reduction step:
$\reduce{M}{\mu}{\PC}{N}{\mu'}$ for some $N$ and $\mu'$.
\end{theorem}
\begin{proof}
  The proof is fully mechanized in \texttt{progress} of \texttt{/src/CC2/Progress.agda}.
\end{proof}

Heap well-typedness is defined point-wise (in \texttt{/src/Memory/HeapTyping.agda}):

\begin{align*}
\Sigma \vdash \mu \triangleq & \forall n\,\ell\,T.\,\text{if}~\Sigma(\ell,n)=T \text{, then}~\mu(\ell,n)=V \\
& \text{and}~\emptyset;\Sigma;\low;\low\vdash V \Leftarrow T_{\ell} ~\text{for some}~ V
\end{align*}

The operation semantics of \CC preserves types and the well-typedness of heap:

\begin{theorem}[Preservation]
\label{thm:preservation}
Suppose \PC is well-typed:  $\vdash \PC \Leftarrow g$,
$M$ is well-typed: $\emptyset ; \Sigma ; g ; |\PC| \vdash M \Leftarrow A$
and the heap $\mu$ is also well-typed: $\Sigma \vdash \mu$.
If $\reduce{M}{\mu}{\PC}{N}{\mu'}$, there exists $\Sigma'$ s.t
$\Sigma' \supseteq \Sigma$, $\emptyset ; \Sigma' ; g ; |\PC| \vdash N \Leftarrow A$,
and $\Sigma' \vdash \mu'$.
\end{theorem}
\begin{proof}
  The proof is fully mechanized in \texttt{pres} of \texttt{/src/CC2/Preservation.agda}.
\end{proof}

In addition, the compilation from \Surface to \CC preserves types:

\begin{theorem}[Compilation preserves types]
  \label{thm:compile-pres}
  If $M$ is a well-typed \Surface program: $\Gamma ; g \vdash M : A$, then the
  compiled \CC term is also well-typed: $\Gamma ; \emptyset ; g ; \low \vdash
  \compile{M} \Leftarrow A$.
\end{theorem}
\begin{proof}
  The proof is fully mechanized in
  \texttt{/src/Compile/CompilationPresTypes.agda}.
\end{proof}

\section{Type Safety of \Surface}
\label{sec:surface-type-safety}

\begin{figure}[tbp]
  \raggedright
  \fbox{$\vdash r : A$}
  \begin{gather*}
    \inference{b \in \{\true, \false\}}{\vdash b : \Bool_{\ell}}
    \quad
    \inference{}{\vdash \mathit{diverge} : A}
    \quad
    \inference{}{\vdash \mathit{error}\;\bl{p} : A}
  \end{gather*}
  \caption{Well-typed evaluation result}
  \label{fig:wt-result}
\end{figure}

\begin{theorem}[Type Safety of \Surface]
  \label{thm:type-safety}
  If $M$ is a well-typed \Surface program: $(x{:}\Bool_{\high});\low \vdash M : \Bool_{\low}$
  and $\mathit{eval}(M,b)=r$,
  then the result is also well-typed: $\vdash r : \Bool_{\low}$.
\end{theorem}
%% \begin{proof}
%%   Casing on $r$ yields three cases:
%%   \begin{itemize}
%%     \item $r=b$.
%%     \item $r=\mathit{error}\;\bl{p}$.
%%     \item $r=\mathit{diverge}$.
%%   \end{itemize}
%% \end{proof}


\chapter{Gradual Guarantee of \Surface}
\label{ch:gg}

In this chapter, I prove that \Surface satisfies that gradual guarantee: if a
\Surface program successfully evaluates to some output value, then a variant of
the program in which the type annotations are less precise must also evaluate to
the same output value.

\section{Simulation Between More and Less Precise Coercion Sequences}
\label{sec:sim-cexpr}

\begin{figure}[tbp]
\raggedright
  \fbox{$\vdash c \sqsubseteq d$, $\vdash c \sqsubseteq g$ , and $\vdash g \sqsubseteq d$}
  {\small
  \begin{gather*}
    {\sqsubseteq}\textit{-c} ~
    \inference{g_1 \sqsubseteq g_1' & g_2 \sqsubseteq g_2' \\ \vdash c : g_1 \Rightarrow g_2 & \vdash d : g_1' \Rightarrow g_2'}
              {\vdash c \sqsubseteq d}
    \\[2ex]
    {\sqsubseteq}\textit{-cl} ~
    \inference{g_1 \sqsubseteq g & g_2 \sqsubseteq g \\ \vdash c : g_1 \Rightarrow g_2}
              {\vdash c \sqsubseteq g}
    \qquad
    {\sqsubseteq}\textit{-cr} ~
    \inference{g \sqsubseteq g_1 & g \sqsubseteq g_2 \\ \vdash d : g_1 \Rightarrow g_2}
              {\vdash g \sqsubseteq d}
  \end{gather*}}
  \fbox{$\vdash \bar{c} \sqsubseteq \bar{d}$}
  {\small
  \begin{gather*}
    {\sqsubseteq}\textit{-id} ~
    \inference{g \sqsubseteq g'}
              {\vdash \id{g} \sqsubseteq \id{g'}}
    \quad
    {\sqsubseteq}\textit{-cast} ~
    \inference{\vdash \bar{c} \sqsubseteq \bar{d} & \vdash c \sqsubseteq d}
              {\vdash \bar{c} \seq c \sqsubseteq \bar{d} \seq d}
    \\[2ex]
    {\sqsubseteq}\textit{-castl} ~
    \inference{\vdash \bar{c} \sqsubseteq \bar{d} & \vdash c \sqsubseteq g_2 \\ \vdash \bar{d} : g_1 \Rightarrow g_2}
              {\vdash \bar{c} \seq c \sqsubseteq \bar{d}}
    \quad
    {\sqsubseteq}\textit{-castr} ~
    \inference{\vdash \bar{c} \sqsubseteq \bar{d} & \vdash g_2 \sqsubseteq d \\ \vdash \bar{c} : g_1 \Rightarrow g_2}
              {\vdash \bar{c} \sqsubseteq \bar{d} \seq d}
    \\[2ex]
    {\sqsubseteq}\textit{-}\bot ~
    \inference{g_1 \sqsubseteq g_3 & g_2 \sqsubseteq g_4 & \vdash \bar{c} : g_1 \Rightarrow g_2}
              {\vdash \bar{c} \sqsubseteq \err{g_3}{g_4}{p}}
  \end{gather*}}
  \fbox{$\vdash \bar{c} \sqsubseteq_l g$}
  {\small
  \begin{gather*}
    {\sqsubseteq}_l\textit{-id} ~
    \inference{g \sqsubseteq g'}
              {\vdash \id{g} \sqsubseteq_l g'}
    \qquad\quad
    {\sqsubseteq}_l\textit{-cast} ~
    \inference{\vdash \bar{c} \sqsubseteq_l g & \vdash c \sqsubseteq_l g}
              {\vdash \bar{c} \seq c \sqsubseteq_l g}
  \end{gather*}}
 \fbox{$\vdash g \sqsubseteq_r \bar{d}$}
 {\small
  \begin{gather*}
    {\sqsubseteq}_r\textit{-id} ~
    \inference{g \sqsubseteq g'}
              {\vdash g \sqsubseteq_r \id{g'}}
    \quad
    {\sqsubseteq}_r\textit{-cast} ~
    \inference{\vdash g \sqsubseteq_r \bar{d} & \vdash g \sqsubseteq_r d}
              {\vdash g \sqsubseteq_r \bar{d} \seq d}
    \\[2ex]
    {\sqsubseteq}_r\textit{-}\bot ~
    \inference{g \sqsubseteq g_1 & g \sqsubseteq g_2}
              {\vdash g \sqsubseteq_r \err{g_1}{g_2}{p}}
  \end{gather*}}
  \caption{Precision relation of the coercion calculus}
  \label{fig:cexpr-prec}
\end{figure}

Our end goal is to prove the gradual guarantee for \Surface.
The proof depends on a simulation lemma between more and less precise terms of
the cast calculus \CC . We use coercion sequences as the IFC monitor in \CC.
Reducing a coercion sequence can result in a blame which errors the program. So
we would like to prove that the simulation lemma holds for the coercion calculus
on security labels.
%
The precision relation on security coercions is defined in
Figure~\ref{fig:cexpr-prec}. The precision relation between two coercion
sequences $\bar{c},\bar{d}$ takes the form $\vdash \bar{c} \sqsubseteq \bar{d}$.
Recall that the gradual guarantee states that replacing type annotations with
\unk (decreasing type precision) should result in the same value for a correctly
running program while adding annotations (increasing type precision) may trigger
more runtime errors. The precision relation is a syntactical characterization of
the runtime behaviors of programs of different type precision. We explain the
intuition with two examples.

\begin{example}
\label{ex:prec}
\normalfont

Consider the following two programs that are related by precision
because the first one has a $\unk$ annotation where the second one
has \high.
\[
\true_{\low} : \Bool_{\unk} : \Bool_{\unk} \qquad\text{ and }\qquad
\true_{\low} : \Bool_{\high} : \Bool_{\unk}
\]
At runtime, the less precise program on the left produces value
$(\cccast{\true}{\id{\low} \seq \inj{\low}})$ and the more precise program
on the right produces $(\cccast{\true}{\id{\low} \seq \up \seq \inj{\high}})$.
The \texttt{true}s are straightforwardly related; we need to show the two
coercion sequences are also related:
\[
\vdash \id{\low} \seq \inj{\low} \sqsubseteq \id{\low} \seq \up \seq \inj{\high}
\]
Starting at the beginning of the
two sequences, we have $\id{\low} \sqsubseteq \id{\low}$ because
$\sqsubseteq$ is reflexive. Next we have
$\inj{\low} \sqsubseteq \;\up$, which makes sense because the source
and targets of the two coercions are related by precision:
$\low \sqsubseteq \low$ and $\unk \sqsubseteq \high$.  Finally, the
coercion $\inj{\high}$ can be added to the end of the more precise
sequence because both its source and target type or more precise than
the target of the left-hand sequence.  That is,
$\unk \sqsubseteq \high$ and $\unk \sqsubseteq \unk$.
%
The injections at the ends of the two sequences, \inj{\low}
and \inj{\high}, cannot be directly related via precision because
$\low \not\sqsubseteq \high$.  Instead, \inj{\low} is related
with \up{}. This underlines the indispensability of explicit subtype
coercion \up{} for the purposes of proving the gradual guarantee.

\end{example}

\begin{example}
\label{ex:prec-error}
\normalfont

Next we consider an example where the less precise program produces a
value but the more precise program encounters an error. This situation
is allowed by the gradual guarantee, but the opposite one is not. We
extend the example with a cast to $\low$ on the more precise side.
\[
\true_{\low} : \Bool_{\unk} : \Bool_{\unk} : \Bool_{\unk}
\qquad\text{ and }\qquad
\true_{\low} : \Bool_{\high} : \Bool_{\unk} : \Bool_{\low}
\]
The first program again produces value $(\cccast{\true}{\id{\low}\seq\inj{\low}})$.
The second, on the other hand, reduces to
$(\cccast{\true}{\id{\low} \seq \up \seq \highlightred{\inj{\high} \seq \proj{\low}{p}}}) \longrightarrow^{*}
(\cccast{\true}{\err{\low}{\low}{p}})$, because of the contradicting
annotations \high and \low (note that \high is in the middle of the sequence and both the
source and target labels of the blame coercion are \low so that types are preserved).
The failure is then propagated out and the term further reduces to \blame{p}.
The precision of coercion sequences relates $\boldsymbol{\bot}$ on the right-hand side
to any coercion sequence on the left so long as the respective source and target types are
related via precision, in this case $\low \sqsubseteq \low$ and $\unk \sqsubseteq \low$.
\end{example}

Consider the security levels (Definition~\ref{def:sec}) of both sides
of Example~\ref{ex:prec}, which are \low on the less precise side and
\high on the more precise side.
We observe that \Surface terms related by precision may produce
values of different security: a less precise value may have lower
security than a more precise value. Indeed, we prove the following for
coercions in normal form:

\begin{lemma}[Security is monotonic with respect to precision]
\label{lem:prec-sec}
Suppose $\mathbf{NF}\; \bar{c}$ and $\mathbf{NF}\; \bar{d}$. If $\vdash \bar{c} \sqsubseteq \bar{d}$, then $| \bar{c} | \preccurlyeq | \bar{d} |$.
\end{lemma}
\begin{proof}
  The proof is mechanized in \texttt{/src/CoercionExpr/SecurityLevel.agda}.
\end{proof}

Next we prove a catch-up lemma for coercion sequences, where the less
precise side catches up with a more precise sequence that is in normal
form. The proof is by casing on $\mathbf{NF}\;\bar{d}$ first and then
performing induction on the precision relation in each case.

\begin{lemma}[Catching up to a more precise coercion sequence]
\label{lem:cexpr-catchup}
If $\mathbf{NF}\; \bar{d}$ and $\vdash \bar{c} \sqsubseteq \bar{d}$,
there exists $\bar{c}'$ such that $\bar{c} \longrightarrow^{*} \bar{c}'$,
$\mathbf{NF}\; \bar{c}'$, and $\vdash \bar{c}' \sqsubseteq \bar{d}$.
\end{lemma}
\begin{proof}
  The proof is mechanized in \texttt{/src/CoercionExpr/CatchUp.agda}.
\end{proof}

\noindent Using Lemma~\ref{lem:cexpr-catchup}, we then prove the following
simulation lemma for coercion sequences:

\begin{lemma}[Simulation between related coercion sequences]
\label{lem:cexpr-sim}
If $\vdash \bar{c} \sqsubseteq \bar{d}$ and $\bar{d} \longrightarrow \bar{d}'$,
there exists $\bar{c}'$ such that $\bar{c} \longrightarrow^{*} \bar{c}'$
and $\vdash \bar{c}' \sqsubseteq \bar{d}'$.
\end{lemma}
\begin{proof}
  The proof is mechanized in \texttt{sim} of \texttt{/src/CoercionExpr/Simulation.agda}.
\end{proof}

We also prove that stamping on coercion sequences preserves precision:

\begin{lemma}[Stamping preserves precision of coercion sequences]
\label{lem:cexpr-stamp-sim}
If $\vdash \bar{c} \sqsubseteq \bar{d}$, then \\ $\vdash \mathit{stamp}\;\bar{c}\;\ell \sqsubseteq \mathit{stamp}\;\bar{d}\;\ell$
and $\vdash \mathit{stamp!}\;\bar{c}\;\ell_1 \sqsubseteq \mathit{stamp!}\;\bar{d}\;\ell_2$ and
$\vdash \mathit{stamp!}\;\bar{c}\;\ell_1 \sqsubseteq \mathit{stamp}\;\bar{d}\;\ell_2$ if $\ell_1 \preccurlyeq \ell_2$.
\end{lemma}
\begin{proof}
  The proof is mechanized in \texttt{/src/CoercionExpr/Stamping.agda}.
\end{proof}

\section{Simulation Between \CC Terms of Different Precision}
\label{sec:simulation}

\begin{figure}[tbp]
\raggedright
  \fbox{$\precctx{\Gamma}{\Gamma'}{\Sigma}{\Sigma'}{g}{g'}{\ell}{\ell'}\ccprec{M}{M'}{A}{A'}$}
  {\small
  \begin{gather*}
  {\sqsubseteq}\textit{-var}~
  \inference{\Gamma \ni x : A & \Gamma' \ni x : A'}
            {\precctx{\Gamma}{\Gamma'}{\Sigma}{\Sigma'}{g}{g'}{\ell}{\ell'}\ccprec{x}{x}{A}{A'}}
  \\[3ex]
  {\sqsubseteq}\textit{-const}~
  \inference{}
            {\precctx{\Gamma}{\Gamma'}{\Sigma}{\Sigma'}{g}{g'}{\ell}{\ell'}\ccprec{\ccconst{k}}{\ccconst{k}}{\iota_{\ell}}{\iota_{\ell}}}
  \\[3ex]
  {\sqsubseteq}\textit{-addr}~
  \inference{\Sigma(\hat{\ell}, n) = T & \Sigma'(\hat{\ell}, n) = T'}
  {\precctx{\Gamma}{\Gamma'}{\Sigma}{\Sigma'}{g}{g'}{\ell}{\ell'}\ccprec{\ccaddr{n}}{\ccaddr{n}}{(\Refer{T_{\hat{\ell}}})_\ell}{(\Refer{T'_{\hat{\ell}}})_\ell}}
  \\[3ex]
    {\sqsubseteq}\textit{-lam}~
    \inference{g_2 \sqsubseteq g_2' & A \sqsubseteq A' \\
               \forall \ell\,\ell'.\, \precctx{(\Gamma,x{:}A)}{(\Gamma',x{:}A')}{\Sigma}{\Sigma'}{g_2}{g_2'}{\ell}{\ell'}\ccprec{N}{N'}{B}{B'}}
              {\precctx{\Gamma}{\Gamma'}{\Sigma}{\Sigma'}{g_1}{g_1'}{\ell_1}{\ell_1'}\ccprec{\cclam{x}{N}}{\cclam{x}{N'}}{(\Fun{A}{g_2}{B})_{\ell_2}}{(\Fun{A'}{g_2'}{B'})_{\ell_2}}}
  \\[3ex]
    {\sqsubseteq}\textit{-app}~
    \inference{\precctx{\Gamma}{\Gamma'}{\Sigma}{\Sigma'}{\ell_1}{\ell_1}{\ell_2}{\ell_2'}\ccprec{L}{L'}{(\Fun{A}{\ell_1 \curlyvee \ell_3}{B})_{\ell_3}}{(\Fun{A'}{\ell_1 \curlyvee \ell_3}{B'})_{\ell_3}} \\
    \precctx{\Gamma}{\Gamma'}{\Sigma}{\Sigma'}{\ell_1}{\ell_1}{\ell_2}{\ell_2'}\ccprec{M}{M'}{A}{A'} \\
    C = \mathit{stamp}\;B\;\ell_3 & C' = \mathit{stamp}\;B'\;\ell_3}
              {\precctx{\Gamma}{\Gamma'}{\Sigma}{\Sigma'}{\ell_1}{\ell_1}{\ell_2}{\ell_2'}\ccprec{\ccapp{L}{M}{A}{B}{\ell_3}}{\ccapp{L'}{M'}{A'}{B'}{\ell_3}}{C}{C'}}
  \\[3ex]
  {\sqsubseteq}\textit{-app}{\star}~
    \inference{\precctx{\Gamma}{\Gamma'}{\Sigma}{\Sigma'}{g}{g'}{\ell}{\ell'}\ccprec{L}{L'}{(\Fun{A}{\unk}{T_{\unk}})_{\unk}}{(\Fun{A'}{\unk}{T'_{\unk}})_{\unk}} \\
      \precctx{\Gamma}{\Gamma'}{\Sigma}{\Sigma'}{g}{g'}{\ell}{\ell'}\ccprec{M}{M'}{A}{A'}}
              {\precctx{\Gamma}{\Gamma'}{\Sigma}{\Sigma'}{g}{g'}{\ell}{\ell'}\ccprec{\ccappstar{L}{M}{A}{T}}{\ccappstar{L'}{M'}{A'}{T'}}{T_{\unk}}{T'_{\unk}}}
  \\[3ex]
    {\sqsubseteq}\textit{-app${\star}$l}~
    \inference{\precctx{\Gamma}{\Gamma'}{\Sigma}{\Sigma'}{g}{\ell_1}{\ell_2}{\ell_2'}\ccprec{L}{L'}{(\Fun{A}{\unk}{T_{\unk}})_{\unk}}{(\Fun{A'}{\ell_1 \curlyvee \ell_3}{B'})_{\ell_3}} \\
                  \precctx{\Gamma}{\Gamma'}{\Sigma}{\Sigma'}{g}{\ell_1}{\ell_2}{\ell_2'}\ccprec{M}{M'}{A}{A'} \\
                  C' = \mathit{stamp}\;B'\;\ell_3}
              {\precctx{\Gamma}{\Gamma'}{\Sigma}{\Sigma'}{g}{\ell_1}{\ell_2}{\ell_2'}\ccprec{\ccappstar{L}{M}{A}{T}}{\ccapp{L'}{M'}{A'}{B'}{\ell_3}}{T_{\unk}}{C'}}
  \\[3ex]
  {\sqsubseteq}\textit{-let}~
  \inference{\precctx{\Gamma}{\Gamma'}{\Sigma}{\Sigma'}{g}{g'}{\ell_1}{\ell_1'}\ccprec{M}{M'}{A}{A'} \\
  \forall \ell_2\,\ell_2'.\, \precctx{(\Gamma,x{:}A)}{(\Gamma',x{:}A')}{\Sigma}{\Sigma'}{g}{g'}{\ell_2}{\ell_2'}\ccprec{N}{N'}{B}{B'}}
  {\precctx{\Gamma}{\Gamma'}{\Sigma}{\Sigma'}{g}{g'}{\ell_1}{\ell_1'}\ccprec{\cclet{x}{M}{A}{N}}{\cclet{x}{M'}{A'}{N'}}{B}{B'}}
  \end{gather*}}
  \caption{Precision rules of \CC (Part I)}
  \label{fig:cc-prec-1}
\end{figure}

\begin{figure}[tbp]
\raggedright
  \fbox{$\precctx{\Gamma}{\Gamma'}{\Sigma}{\Sigma'}{g}{g'}{\ell}{\ell'}\ccprec{M}{M'}{A}{A'}$}
  {\small
  \begin{gather*}
  {\sqsubseteq}\textit{-if}~
  \inference{\precctx{\Gamma}{\Gamma'}{\Sigma}{\Sigma'}{\ell_1}{\ell_1}{\ell_2}{\ell_2'}\ccprec{L}{L'}{\Bool_{\ell_3}}{\Bool_{\ell_3}} \\
      \forall \ell\,\ell'.\,\precctx{\Gamma}{\Gamma'}{\Sigma}{\Sigma'}{\ell_1 \curlyvee \ell_3}{\ell_1 \curlyvee \ell_3}{\ell}{\ell'}\ccprec{M}{M'}{A}{A'} \\
      \forall \ell\,\ell'.\,\precctx{\Gamma}{\Gamma'}{\Sigma}{\Sigma'}{\ell_1 \curlyvee \ell_3}{\ell_1 \curlyvee \ell_3}{\ell}{\ell'}\ccprec{N}{N'}{A}{A'} \\
      B = \mathit{stamp}\;A\;\ell_3 & B' = \mathit{stamp}\;A'\;\ell_3}
            {\precctx{\Gamma}{\Gamma'}{\Sigma}{\Sigma'}{\ell_1}{\ell_1}{\ell_2}{\ell_2'}\ccprec{\ccif{L}{A}{\ell_3}{M}{N}}{\ccif{L'}{A'}{\ell_3}{M'}{N'}}{B}{B'}}
  \\[3ex]
  {\sqsubseteq}\textit{-if}{\star}~
  \inference{\precctx{\Gamma}{\Gamma'}{\Sigma}{\Sigma'}{g}{g'}{\ell_1}{\ell_1'}\ccprec{L}{L'}{\Bool_{\unk}}{\Bool_{\unk}} \\
      \forall \ell_2\,\ell_2'.\,\precctx{\Gamma}{\Gamma'}{\Sigma}{\Sigma'}{\unk}{\unk}{\ell_2}{\ell_2'}\ccprec{M}{M'}{T_{\unk}}{T'_{\unk}} \\
      \forall \ell_2\,\ell_2'.\,\precctx{\Gamma}{\Gamma'}{\Sigma}{\Sigma'}{\unk}{\unk}{\ell_2}{\ell_2'}\ccprec{N}{N'}{T_{\unk}}{T'_{\unk}} }
            {\precctx{\Gamma}{\Gamma'}{\Sigma}{\Sigma'}{g}{g'}{\ell_1}{\ell_1'}\ccprec{\ccifstar{L}{T}{M}{N}}{\ccifstar{L'}{T'}{M'}{N'}}{T_{\unk}}{T'_{\unk}}}
  \\[3ex]
  {\sqsubseteq}\textit{-if${\star}$l}~
  \inference{\precctx{\Gamma}{\Gamma'}{\Sigma}{\Sigma'}{g}{\ell_1}{\ell_2}{\ell_2'}\ccprec{L}{L'}{\Bool_{\unk}}{\Bool_{\ell_3}} \\
    \forall \ell\,\ell'.\,\precctx{\Gamma}{\Gamma'}{\Sigma}{\Sigma'}{\unk}{\ell_1 \curlyvee \ell_3}{\ell}{\ell'}\ccprec{M}{M'}{T_{\unk}}{A'} \\
    \forall \ell\,\ell'.\,\precctx{\Gamma}{\Gamma'}{\Sigma}{\Sigma'}{\unk}{\ell_1 \curlyvee \ell_3}{\ell}{\ell'}\ccprec{N}{N'}{T_{\unk}}{A'} \\
    B' = \mathit{stamp}\;A'\;\ell_3}
            {\precctx{\Gamma}{\Gamma'}{\Sigma}{\Sigma'}{g}{\ell_1}{\ell_2}{\ell_2'}\ccprec{\ccifstar{L}{T}{M}{N}}{\ccif{L'}{A'}{\ell_3}{M'}{N'}}{T_{\unk}}{B'}}
  \\[3ex]
    {\sqsubseteq}\textit{-ref}~
    \inference{\precctx{\Gamma}{\Gamma'}{\Sigma}{\Sigma'}{\ell_1}{\ell_1}{\ell_2}{\ell_2'}\ccprec{M}{M'}{T_{\ell_3}}{T'_{\ell_3}} &
               \ell_1 \preccurlyeq \ell_3}
              {\precctx{\Gamma}{\Gamma'}{\Sigma}{\Sigma'}{\ell_1}{\ell_1}{\ell_2}{\ell_2'}\ccprec{\ccref{\ell_3}{M}}{\ccref{\ell_3}{M'}}{(\Refer{T_{\ell_3}})_{\low}}{(\Refer{T'_{\ell_3}})_{\low}}}
  \\[3ex]
    {\sqsubseteq}\textit{-ref?}~
    \inference{\precctx{\Gamma}{\Gamma'}{\Sigma}{\Sigma'}{\unk}{\unk}{\ell_1}{\ell_1'}\ccprec{M}{M'}{T_{\ell_2}}{T'_{\ell_2}}}
              {\precctx{\Gamma}{\Gamma'}{\Sigma}{\Sigma'}{\unk}{\unk}{\ell_1}{\ell_1'}\ccprec{\ccrefproj{\ell_2}{M}{p}}{\ccrefproj{\ell_2}{M'}{q}}{(\Refer{T_{\ell_2}})_{\low}}{(\Refer{T'_{\ell_2}})_{\low}}}
  \\[3ex]
  {\sqsubseteq}\textit{-ref?l}~
  \inference{\precctx{\Gamma}{\Gamma'}{\Sigma}{\Sigma'}{\unk}{\ell_1}{\ell_2}{\ell_3}\ccprec{M}{M'}{T_\ell}{T'_\ell} &
             \ell_1 \preccurlyeq \ell}
  {\precctx{\Gamma}{\Gamma'}{\Sigma}{\Sigma'}{\unk}{\ell_1}{\ell_2}{\ell_3}\ccprec{\ccrefproj{\ell}{M}{p}}{\ccref{\ell}{M'}}{(\Refer{T_\ell})_{\low}}{(\Refer{T'_\ell})_{\low}}}
  \\[3ex]
  {\sqsubseteq}\textit{-deref}~
  \inference{\precctx{\Gamma}{\Gamma'}{\Sigma}{\Sigma'}{g}{g'}{\ell_1}{\ell_1'}\ccprec{M}{M'}{(\Refer{A})_{\ell_2}}{(\Refer{A'})_{\ell_2}} \\
  B = \mathit{stamp}\;A\;\ell_2 & B' = \mathit{stamp}\;A'\;\ell_2}
  {\precctx{\Gamma}{\Gamma'}{\Sigma}{\Sigma'}{g}{g'}{\ell_1}{\ell_1'}\ccprec{\ccderef{M}{A}{\ell_2}}{\ccderef{M'}{A'}{\ell_2}}{B}{B'}}
  \\[3ex]
    {\sqsubseteq}\textit{-deref}{\star}~
    \inference{\precctx{\Gamma}{\Gamma'}{\Sigma}{\Sigma'}{g}{g'}{\ell}{\ell'}\ccprec{M}{M'}{(\Refer{T_{\unk}})_{\unk}}{(\Refer{T'_{\unk}})_{\unk}}}
              {\precctx{\Gamma}{\Gamma'}{\Sigma}{\Sigma'}{g}{g'}{\ell}{\ell'}\ccprec{\ccderefstar{M}{T}}{\ccderefstar{M'}{T'}}{T_{\unk}}{T'_{\unk}}}
  \\[3ex]
    {\sqsubseteq}\textit{-deref${\star}$l}~
    \inference{\precctx{\Gamma}{\Gamma'}{\Sigma}{\Sigma'}{g}{g'}{\ell}{\ell'}\ccprec{M}{M'}{(\Refer{T_{\unk}})_{\unk}}{(\Refer{A'})_{\ell_2}} \\
               B' = \mathit{stamp}\;A'\;\ell_2}
              {\precctx{\Gamma}{\Gamma'}{\Sigma}{\Sigma'}{g}{g'}{\ell_1}{\ell_1'}\ccprec{\ccderefstar{M}{T}}{\ccderef{M'}{A'}{\ell_2}}{T_{\unk}}{B'}}
  \end{gather*}}
  \caption{Precision rules of \CC (Part II)}
  \label{fig:cc-prec-2}
\end{figure}

\begin{figure}[tbp]
\raggedright
  \fbox{$\precctx{\Gamma}{\Gamma'}{\Sigma}{\Sigma'}{g}{g'}{\ell}{\ell'}\ccprec{M}{M'}{A}{A'}$}
  {\small
  \begin{gather*}
  {\sqsubseteq}\textit{-assign}~
  \inference{\precctx{\Gamma}{\Gamma'}{\Sigma}{\Sigma'}{\ell_1}{\ell_1}{\ell_2}{\ell_2'}\ccprec{L}{L'}{(\Refer{T_{\hat{\ell}}})_\ell}{(\Refer{T'_{\hat{\ell}}})_\ell} \\
             \precctx{\Gamma}{\Gamma'}{\Sigma}{\Sigma'}{\ell_1}{\ell_1}{\ell_2}{\ell_2'}\ccprec{M}{M'}{T_{\hat{\ell}}}{T'_{\hat{\ell}}} \\
             \ell_1 \preccurlyeq \hat{\ell} & \ell \preccurlyeq \hat{\ell}}
            {\begin{align*}\precctx{\Gamma}{\Gamma'}{\Sigma}{\Sigma'}{\ell_1}{\ell_1}{\ell_2}{\ell_2'}
                \vdash \ccassign{L}{M}{T}{\hat{\ell}}{\ell} \sqsubseteq \ccassign{L'}{M'}{T'}{\hat{\ell}}{\ell} \Leftarrow \\
                \Unit_{\low} \sqsubseteq \Unit_{\low}
            \end{align*}}
  \\[4ex]
  {\sqsubseteq}\textit{-assign?}~
    \inference{\precctx{\Gamma}{\Gamma'}{\Sigma}{\Sigma'}{g}{g'}{\ell}{\ell'}\ccprec{L}{L'}{(\Refer{T_{\hat{g}}})_{\unk}}{(\Refer{T'_{\hat{g}'}})_{\unk}} \\
      \precctx{\Gamma}{\Gamma'}{\Sigma}{\Sigma'}{g}{g'}{\ell}{\ell'}\ccprec{M}{M'}{T_{\hat{g}}}{T'_{\hat{g}'}} }
              {\begin{align*}\precctx{\Gamma}{\Gamma'}{\Sigma}{\Sigma'}{g}{g'}{\ell}{\ell'}
                  \vdash \ccassignproj{L}{M}{T}{\hat{g}}{p} \sqsubseteq \ccassignproj{L'}{M'}{T'}{\hat{g}'}{q} \Leftarrow \\
                  \Unit_{\low} \sqsubseteq \Unit_{\low}
              \end{align*}}
  \\[4ex]
    {\sqsubseteq}\textit{-assign?l}~
    \inference{\precctx{\Gamma}{\Gamma'}{\Sigma}{\Sigma'}{g}{\ell_1}{\ell_2}{\ell_2'}\ccprec{L}{L'}{(\Refer{T_{\hat{g}}})_{\unk}}{(\Refer{T'_{\hat{\ell}}})_\ell} \\
      \precctx{\Gamma}{\Gamma'}{\Sigma}{\Sigma'}{g}{\ell_1}{\ell_2}{\ell_2'}\ccprec{M}{M'}{T_{\hat{g}}}{T'_{\hat{\ell}}} \\
      \ell_1 \preccurlyeq \hat{\ell} & \ell \preccurlyeq \hat{\ell}}
              {\begin{align*}\precctx{\Gamma}{\Gamma'}{\Sigma}{\Sigma'}{g}{\ell_1}{\ell_2}{\ell_2'}
                  \vdash \ccassignproj{L}{M}{T}{\hat{g}}{p} \sqsubseteq \ccassign{L'}{M'}{T'}{\hat{\ell}}{\ell} \Leftarrow \\
                  \Unit_{\low} \sqsubseteq \Unit_{\low}
              \end{align*}}
  \\[4ex]
  {\sqsubseteq}\textit{-prot}~
    \inference
        {\precctx{\Gamma}{\Gamma'}{\Sigma}{\Sigma'}{g_1}{g_1'}{|\PC|}{|\PC'|}\ccprec{M}{M'}{A}{A'} &
          \ccprec{\PC}{\PC'}{g_1}{g_1'} \\
          \ell_1 \curlyvee \ell_2 \preccurlyeq |\PC| & \ell'_1 \curlyvee \ell_2 \preccurlyeq |\PC'| &
          B = \mathit{stamp}\;A\;\ell_2 & B' = \mathit{stamp}\;A'\;\ell_2}
        {\precctx{\Gamma}{\Gamma'}{\Sigma}{\Sigma'}{g_2}{g_2'}{\ell_1}{\ell'_1}\ccprec{\ccprot{\PC}{\ell_2}{M}{A}}{\ccprot{\PC'}{\ell_2}{M'}{A'}}{B}{B'}}
  \\[4ex]
  {\sqsubseteq}\textit{-prot!}~
  \inference
  {\precctx{\Gamma}{\Gamma'}{\Sigma}{\Sigma'}{g_1}{g_1'}{|\PC|}{|\PC'|}\ccprec{M}{M'}{T_{\unk}}{T'_{\unk}} &
   \ccprec{\PC}{\PC'}{g_1}{g_1'} \\
   \ell_1 \curlyvee \ell_2 \preccurlyeq |\PC| & \ell'_1 \curlyvee \ell'_2 \preccurlyeq |\PC'| &
   \highlight{highlight}{\ell_2 \preccurlyeq \ell_2'}}
  {\precctx{\Gamma}{\Gamma'}{\Sigma}{\Sigma'}{g_2}{g_2'}{\ell_1}{\ell'_1}\ccprec{\ccprot{\PC}{\ell_2}{M}{T_{\unk}}}{\ccprot{\PC'}{\ell'_2}{M'}{T'_{\unk}}}{T_{\unk}}{T'_{\unk}}}
  \\[4ex]
  {\sqsubseteq}\textit{-prot!l}~
  \inference
      {\precctx{\Gamma}{\Gamma'}{\Sigma}{\Sigma'}{g_1}{g_1'}{|\PC|}{|\PC'|}\ccprec{M}{M'}{T_{\unk}}{A} &
          \ccprec{\PC}{\PC'}{g_1}{g_1'} \\
          \ell_1 \curlyvee \ell_2 \preccurlyeq |\PC| & \ell'_1 \curlyvee \ell'_2 \preccurlyeq |\PC'| &
          B = \mathit{stamp}\;A\;\ell_2' &
          \highlight{highlight}{\ell_2 \preccurlyeq \ell_2'}}
      {\precctx{\Gamma}{\Gamma'}{\Sigma}{\Sigma'}{g_2}{g_2'}{\ell_1}{\ell'_1}\ccprec{\ccprot{\PC}{\ell_2}{M}{T_{\unk}}}{\ccprot{\PC'}{\ell'_2}{M'}{A}}{T_{\unk}}{B}}
  \end{gather*}}
  \caption{Precision rules of \CC (Part III)}
  \label{fig:cc-prec-3}
\end{figure}

\begin{figure}[tbp]
\raggedright
  \fbox{$\precctx{\Gamma}{\Gamma'}{\Sigma}{\Sigma'}{g}{g'}{\ell}{\ell'}\ccprec{M}{M'}{A}{A'}$}
  {\small
  \begin{gather*}
  {\sqsubseteq}\textit{-cast}~
  \inference
  {\precctx{\Gamma}{\Gamma'}{\Sigma}{\Sigma'}{g}{g'}{\ell}{\ell'}\ccprec{M}{M'}{A}{A'} & \bm{c} \sqsubseteq \bm{c'} \\
   \vdash \bm{c} : A \Rightarrow B & \vdash \bm{c'} : A' \Rightarrow B'}
  {\precctx{\Gamma}{\Gamma'}{\Sigma}{\Sigma'}{g}{g'}{\ell}{\ell'}\ccprec{\cccast{M}{\bm{c}}}{\cccast{M'}{\bm{c'}}}{B}{B'}}
  \\[4ex]
  {\sqsubseteq}\textit{-castl}~
  \inference
  {\precctx{\Gamma}{\Gamma'}{\Sigma}{\Sigma'}{g}{g'}{\ell}{\ell'}\ccprec{M}{M'}{A}{A'} &
   \bm{c} \sqsubseteq A' & \vdash \bm{c} : A \Rightarrow B}
  {\precctx{\Gamma}{\Gamma'}{\Sigma}{\Sigma'}{g}{g'}{\ell}{\ell'}\ccprec{\cccast{M}{\bm{c}}}{M'}{B}{A'}}
  \\[4ex]
  {\sqsubseteq}\textit{-castr}~
  \inference
  {\precctx{\Gamma}{\Gamma'}{\Sigma}{\Sigma'}{g}{g'}{\ell}{\ell'}\ccprec{M}{M'}{A}{A'} &
   A \sqsubseteq \bm{c'} & \vdash \bm{c'} : A' \Rightarrow B'}
  {\precctx{\Gamma}{\Gamma'}{\Sigma}{\Sigma'}{g}{g'}{\ell}{\ell'}\ccprec{M}{\cccast{M'}{\bm{c'}}}{A}{B'}}
  \\[4ex]
  {\sqsubseteq}\textit{-blame}~
  \inference
  {\Gamma; \Sigma; g; \ell \vdash M \Leftarrow A & A \sqsubseteq A'}
  {\precctx{\Gamma}{\Gamma'}{\Sigma}{\Sigma'}{g}{g'}{\ell}{\ell'}\ccprec{M}{\blame{p}}{A}{A'}}
  \end{gather*}}
  \caption{Precision rules of \CC (Part IV)}
  \label{fig:cc-prec-4}
\end{figure}

The main simulation lemma says that if two terms are related by
\textit{precision} and the more precise side takes one step, then the less
precise side is able to multi-step and get back in sync. The precision relation
is in form
$\precctx{\Gamma}{\Gamma'}{\Sigma}{\Sigma'}{g}{g'}{\ell}{\ell'}\ccprec{M}{M'}{A}{A'}$,
where $\Gamma;\Sigma;g;\ell$ corresponds to the typing context, heap context,
type of \PC, and security of \PC of the less precise term $M$ and
$\Gamma';\Sigma';g';\ell'$ is for those of the more precise term $M'$. The types
of the two terms, $A$ and $A'$, are related by precision between types. The
intuition between this precision relation is that casts are allowed to appear in
different places between the more precise and the less precise \CC terms.
Moreover, the casts must be in shapes that preserve the precision of \Surface
(more or fewer static type annotations provided by the programmer). According to
the gradual guarantee, the more precise side is allowed to signal more blames,
so there is a rule (${\sqsubseteq}\textit{-blame}$) that relates \blame{p} to
any term $M$ on the less precise side as long as their types are in sync. We
list the precision rules for the cast calculus \CC in
Figure~\ref{fig:cc-prec-1},~\ref{fig:cc-prec-2},~\ref{fig:cc-prec-3},
and~\ref{fig:cc-prec-4}.

With the precision relation of \CC defined, we first state the catch-up lemma,
which catches up to a more-precise value by multi-stepping on the less-precise
side:

\begin{lemma}[Catching up to more precise]
\label{lem:catchup}
If term $M$ and value $V'$ are related by precision:
$$\precctx{\Gamma}{\Gamma'}{\Sigma}{\Sigma'}{g}{g'}{\ell}{\ell'}\ccprec{M}{V'}{A}{A'}$$
then there exists value $V$ s.t $M\mid\mu\mid\PC \longrightarrow^{*} V\mid\mu$ and
$\precctx{\Gamma}{\Gamma'}{\Sigma}{\Sigma'}{g}{g'}{\ell}{\ell'}\ccprec{V}{V'}{A}{A'}$.
\end{lemma}
\begin{proof}
  The proof is mechanized in \texttt{/src/Simulation/CatchUp.agda}.
\end{proof}

It is straightforward to show that substitution preserves precision for \CC
(typing context precision is defined point-wise):

\begin{lemma}[Substitution preserves precision]
  \label{lem:subst-pres-prec}
  Suppose the typing contexts are related by precision: $\Gamma \sqsubseteq
  \Gamma$ and $\Sigma \sqsubseteq \Sigma'$. If
  $$\precctx{(\Gamma,x{:}A)}{(\Gamma',x{:}A')}{\Sigma}{\Sigma'}{g}{g'}{\ell}{\ell'}\ccprec{N}{N'}{B}{B'}$$
  and
  $$\forall g\,g'\,\ell\,\ell'.\, \precctx{\Gamma}{\Gamma'}{\Sigma}{\Sigma'}{g}{g'}{\ell}{\ell'}\ccprec{V}{V'}{A}{A'}$$
  then
  $\precctx{\Gamma}{\Gamma'}{\Sigma}{\Sigma'}{g}{g'}{\ell}{\ell'}\ccprec{N[x:=V]}{N'[x:=V']}{B}{B'}$.
\end{lemma}
\begin{proof}
  The proof is mechanized in \texttt{/src/CC2/SubstPrecision.agda}.
\end{proof}

We then prove the main simulation lemma using Lemma~\ref{lem:catchup}
("catch-up") and Lemma~\ref{lem:subst-pres-prec} (substitution preserves
precision). Heap precision is defined point-wise, similar to the definition of
heap well-typedness.

\begin{lemma}[Simulation between more precise and less precise \CC terms]
\label{lem:sim}
Suppose \PC, $\PC'$ are related by precision: $\ccprec{\PC}{\PC'}{g}{g'}$.
Moreover suppose $M$, $M'$ are related by precision:
$$\precctx{\emptyset}{\emptyset}{\Sigma_1}{\Sigma_1'}{g}{g'}{|\PC|}{|\PC'|}\ccprec{M}{M'}{A}{A'}$$
heap contexts $\Sigma_1$, $\Sigma_1'$ are related by precision:
$\Sigma_1 \sqsubseteq \Sigma_1'$, the initial heaps $\mu_1$, $\mu_1'$
are also related by precision: $\Sigma_1 ; \Sigma_1' \vdash \mu_1 \sqsubseteq \mu_1'$. \\
If \reduce{M'}{\mu_1'}{\PC'}{N'}{\mu_2'}, there exists $\Sigma_2$,
$\Sigma_2'$, $N$, $\mu_2$ s.t $\Sigma_2 \supseteq \Sigma_1$,
$\Sigma_2' \supseteq \Sigma_1'$, $\Sigma_2 \sqsubseteq \Sigma_2'$ ,
$$M\mid\mu_1\mid\PC \longrightarrow^{*} N\mid\mu_2$$
the resulting terms are related by precision:
$\precctx{\emptyset}{\emptyset}{\Sigma_2}{\Sigma_2'}{g}{g'}{|\PC|}{|\PC'|}\ccprec{N}{N'}{A}{A'}$
and the resulting heaps are also related by precision: $\Sigma_2 ; \Sigma_2' \vdash \mu_2 \sqsubseteq \mu_2'$.
\end{lemma}
\begin{proof}
  The proof is mechanized in \texttt{/src/Simulation/Simulation.agda}.
\end{proof}

\section{The Gradual Guarantee of \Surface}
\label{sec:gg}

\begin{figure}[tbp]
  \raggedright
  \fbox{$\vdash M_1 \sqsubseteq M_2$}
  {\small
  \begin{gather*}
    % const
    {\sqsubseteq}^G\textit{-const}~
    \inference{}{\vdash \const{k}{\ell} \sqsubseteq \const{k}{\ell}}
    \qquad
    % var
    {\sqsubseteq}^G\textit{-var}~
    \inference{}{\vdash x \sqsubseteq x}
    \\[2ex]
    % lam
    {\sqsubseteq}^G\textit{-lam}~
    \inference{g_1 \sqsubseteq g_2 & A_1 \sqsubseteq A_2 & \vdash N_1 \sqsubseteq N_2 }
              {\vdash \lam{g_1}{x}{A_1}{N_1}{\ell} \sqsubseteq \lam{g_2}{x}{A_2}{N_2}{\ell}}
    \quad
    % app
    {\sqsubseteq}^G\textit{-app}~
    \inference{\vdash L_1 \sqsubseteq L_2 & \vdash M_1 \sqsubseteq M_2}
              {\vdash \app{L_1}{M_1}{p} \sqsubseteq \app{L_2}{M_2}{p}}
    \\[2ex]
    % if
    {\sqsubseteq}^G\textit{-if}~
    \inference{\vdash L_1 \sqsubseteq L_2 & \vdash M_1 \sqsubseteq M_2 & \vdash N_1 \sqsubseteq N_2}
              {\vdash \ifexp{L_1}{M_1}{N_1}{p} \sqsubseteq \ifexp{L_2}{M_2}{N_2}{p}}
    \\[2ex]
    % ann
    {\sqsubseteq}^G\textit{-ann}~
    \inference{\vdash M_1 \sqsubseteq M_2 & A_1 \sqsubseteq A_2}
              {\vdash \ann{M_1}{A_1}{p} \sqsubseteq \ann{M_2}{A_2}{p}}
    \\[2ex]
    % let
    {\sqsubseteq}^G\textit{-let}~
    \inference{\vdash M_1 \sqsubseteq M_2 & \vdash N_1 \sqsubseteq N_2}
              {\vdash \letexp{x}{M_1}{N_1} \sqsubseteq \letexp{x}{M_2}{N_2}}
    \\[2ex]
    % ref
    {\sqsubseteq}^G\textit{-ref}~
    \inference{\vdash M_1 \sqsubseteq M_2}
              {\vdash \refexp{\ell}{M_1}{p} \sqsubseteq \refexp{\ell}{M_2}{p}}
    \quad
    % deref
    {\sqsubseteq}^G\textit{-deref}~
    \inference{\vdash M_1 \sqsubseteq M_2}
              {\vdash \deref{M_1}{p} \sqsubseteq \deref{M_2}{p}}
    \\[2ex]
    % assign
    {\sqsubseteq}^G\textit{-assign}~
    \inference{\vdash L_1 \sqsubseteq L_2 & \vdash M_1 \sqsubseteq M_2}
              {\vdash \assign{L_1}{M_1}{p} \sqsubseteq \assign{L_2}{M_2}{p}}
  \end{gather*}}
  \caption{Precision rules of \Surface}
  \label{fig:surface-precision}
\end{figure}

The definition of term precision for \Surface is in
Figure~\ref{fig:surface-precision}. We first prove that the compilation from
\Surface to \CC preserves types:

\begin{lemma}[Compilation preserves precision]
  \label{lem:compile-pres-prec}
  Suppose the tying contexts are related by precision: $\Gamma \sqsubseteq
  \Gamma'$, and the PC labels are also related: $g \sqsubseteq g'$. If
  well-typed \Surface terms $M, M'$ are related by precision: $\vdash M
  \sqsubseteq M'$ and $\Gamma ; g \vdash M : A$ and $\Gamma'; g' \vdash M' :
  A'$, then
  $$\forall \ell\, \ell'.\, \precctx{\Gamma}{\Gamma'}{\emptyset}{\emptyset}{g}{g'}{\ell}{\ell'}\ccprec{\compile{M}}{\compile{M'}}{A}{A'}$$
\end{lemma}
\begin{proof}
  The proof is mechanized in \texttt{/src/Compile/Precision/CompilePrecision.agda}.
\end{proof}

We prove the following lemma about \Surface that is a corollary of
Lemma~\ref{lem:sim}:

\begin{lemma}
\label{lem:gg}
Suppose $M$ and $M'$ are well-typed terms in \Surface that are related by
precision, that is $\vdash M \sqsubseteq M'$ and $(x{:}\Bool_{\high}); \low \vdash
M : A$ and $(x{:}\Bool_{\high}); \low \vdash M' : A'$, so $A \sqsubseteq A'$.
%
If the compilation of $M'$ substituted with input $b$ reduces to a value:
$$(\compile{M'})[x:=\ccconst{b}]\mid\emptyset\mid\low \longrightarrow^{*} V'\mid\mu'$$
there exists some value $V$ and heap $\mu$ s.t. the compilation of $M$
substituted with $b$ reduces to $V$:
$$(\compile{M})[x:=\ccconst{b}]\mid\emptyset\mid\low \longrightarrow^{*} V\mid\mu$$
and the resulting values are related by precision for some $\Sigma$, $\Sigma'$:
$$\precctx{\emptyset}{\emptyset}{\Sigma}{\Sigma'}{\low}{\low}{\low}{\low}\ccprec{V}{V'}{A}{A'}$$
\end{lemma}
\begin{proof}
  By Lemma~\ref{lem:compile-pres-prec} (compilation preserves precision), we have
  $$\precctx{(x{:}\Bool_{\high})}{(x{:}\Bool_{\high})}{\emptyset}{\emptyset}{\low}{\low}{\low}{\low}\ccprec{\compile{M}}{\compile{M'}}{A}{A'}$$
  Substitution preserves precision (Lemma~\ref{lem:subst-pres-prec}), so
  $$\precctx{\emptyset}{\emptyset}{\emptyset}{\emptyset}{\low}{\low}{\low}{\low}\ccprec{(\compile{M})[x:=\ccconst{b}]}{(\compile{M'})[x:=\ccconst{b}]}{A}{A'}$$
  We then proceed by induction on the reduction of $(\compile{M'})[x:=\ccconst{b}]$ to a value
  $V'$, using Lemma~\ref{lem:sim} to show that $(\compile{M})[x:=\ccconst{b}]$ reduces to a
  corresponding term at each step. So we have $(\compile{M})[x:=\ccconst{b}] \longrightarrow^{*}
  N$ where $N \sqsubseteq V'$ for some $N$. We then apply
  Lemma~\ref{lem:catchup} to show that $N$ reduces to a value $V$ where $V
  \sqsubseteq V'$.
\end{proof}

Finally, we state and prove the gradual guarantee of \Surface as a corollary of Lemma~\ref{lem:gg}:

\begin{theorem}[Gradual guarantee of \Surface]
  Suppose $M$ and $M'$ are \Surface programs related by precision: $\vdash M \sqsubseteq M'$.
  If $\mathit{eval}(M', b_1)=b_2$, then $\mathit{eval}(M, b_1)=b_2$.
\end{theorem}
\begin{proof}
  By Definition~\ref{def:surface-program}, $(x{:}\Bool_{\high});\low\vdash M :
  \Bool_{\low}$ and $(x{:}\Bool_{\high});\low\vdash M' : \Bool_{\low}$. By
  $\mathit{eval}(M', b_1)=b_2$ and inversion on the definition of \textit{eval}
  (Figure~\ref{fig:eval}), we have
  $\reducemult{(\compile{M'})[x:=\ccconst{b_1}]}{\emptyset}{\low}{\ccconst{b_2}}{\mu'}$
  for some $\mu'$. By Lemma~\ref{lem:gg}, there exists some $V, \mu$ such that
  $$(\compile{M})[x:=\ccconst{b_1}]\mid\emptyset\mid\low \longrightarrow^{*} V\mid\mu$$
  and
  $$\precctx{\emptyset}{\emptyset}{\Sigma}{\Sigma'}{\low}{\low}{\low}{\low}\ccprec{V}{\ccconst{b_2}}{\Bool_{\low}}{\Bool_{\low}}$$
  By inversion on the precision of \CC and by the canonical form of a value of $\Bool_{\low}$, $V = \ccconst{b_2}$.
  By the definition of \textit{eval}, $\mathit{eval}(M, b_1)=b_2$.
\end{proof}


\chapter{Noninterference of \Surface}
\label{ch:noninterference}

{\color{NavyBlue} %%% new text
In this chapter, I prove that \Surface satisfies noninterference: high-security
input never flows into low-security output.

I show that security checks can be modeled by reducing security coercion
sequences to their normal form in Section~\ref{sec:norm-IF}. After that, I
present the full proof of noninterference for \Surface. This proof employs a
three-step approach.

First, in Section~\ref{sec:NI-dynifc}, I prove that the dynamic extreme of
\Surface, \DynIFC, satisfies noninterference. This proof is a straightforward
adaptation of the noninterference proof from \textcite{Chen:2022aa}. The proof
(Lemma~\ref{lem:NI-old}) uses a standard erasure-based
approach~\parencite{LI20101974, stefan2011flexible, stefan2012flexible,
  Fennell:2013ab, STEFAN:2017ta}, where the high-security parts of a program are
erased to an opaque value.

Second, in Section~\ref{sec:sim-cc-dynifc}, I prove a simulation lemma between
\CC and \DynIFC (Lemma~\ref{lem:sim-leq}). The main intuition of the simulation
relation (Figure~\ref{fig:sim-rel-1} and~\ref{fig:sim-rel-2}) is that a \CC term
always produces a value that is as secure as the one produced by its related
\DynIFC term. I translate \CC terms to \DynIFC by (1) getting rid of all the
casts and (2) converting static heap enforcement (\texttt{ref} and
\texttt{assign}) to dynamic enforcement (NSU) (Figure~\ref{fig:cast-erase}). The
noninterference property of \CC (Lemma~\ref{lem:NI-CC}) follows directly from
the multi-step simulation lemma (Lemma~\ref{lem:sim-mult}) and the
noninterference result of \DynIFC (Lemma~\ref{lem:NI-old}).

Third, in Section~\ref{sec:NI}, I prove the noninterference theorem of \Surface
(Theorem~\ref{thm:NI-Surface}) as a corollary of the noninterference property of
\CC.

} %%% end new text

Similar to \GSLRef and GLIO, the statements of noninterference for
both \DynIFC and \CC are termination-insensitive.  Thus, I will only
consider successful executions that produce values. In other words,
I will not consider reduction rules that trigger IFC monitor failures,
such as NSU errors in \DynIFC or cast errors in \CC.  As I explained
in Section~\ref{sec:example1}, the programming language runtime can
force the program to diverge whenever blame is detected, possibly
sending a private error message to the software developer.

The Agda proofs cited in the section are available at:
\begin{center}
  \url{https://github.com/Gradual-Typing/LambdaIFCStar}
\end{center}

\section{The Normalization of Coercions Checks Information Flow}
\label{sec:norm-IF}

I show that reducing coercion sequences to normal form models IFC checks because
the normalization either succeeds or fails. If a coercion sequences successfully
reduces to normal form, the IFC check succeeds and the flow is justified. If it
reduces to a failure, then an illegal flow is detected and the program errors.

\begin{lemma}[Strong normalization of coercion sequences]
If $\vdash \bar{c} : g_1 \Rightarrow g_2$, then either
(1) $\bar{c} \longrightarrow^{*} \bar{d}$ and $\mathbf{NF}\;\bar{d}$ or
(2) $\bar{c} \longrightarrow^{*} \err{g_1}{g_2}{p}$.
\end{lemma}
\begin{proof}
  The proof is in \texttt{cexpr-sn} of
  \texttt{/src/CoercionExpr/CoercionExpr.agda}
\end{proof}

\noindent Normalization of coercion sequences is deterministic:

\begin{lemma}[Reduction of coercion sequences is deterministic]
If $\bar{c} \longrightarrow \bar{d}_1$ and $\bar{c} \longrightarrow \bar{d}_2$,
then $\bar{d}_1 = \bar{d}_2$.
\end{lemma}
\begin{proof}
  The proof is in \texttt{det} of \texttt{/src/CoercionExpr/CoercionExpr.agda}.
\end{proof}

\begin{lemma}[Normalization of coercion sequences is deterministic]
Suppose $\bar{c} \longrightarrow^{*} \bar{d}_1$
and $\bar{c} \longrightarrow^{*} \bar{d}_2$.
If $\mathbf{NF}\;\bar{d}_i$ or $\bar{d}_i = \err{g_1}{g_2}{p}$,
then $\bar{d}_1 = \bar{d}_2$.
\end{lemma}
\begin{proof}
  The proof is in \texttt{det-mult} of \texttt{/src/CoercionExpr/CoercionExpr.agda}.
\end{proof}

Recall that in Section~\ref{sec:cexpr-comp-stamp}, we mention that coercion
composition models explicit flows, while coercion stamping models implicit flow.
Here we prove lemmas that formalize that intuition.

We reason about explicit flows first. If we compose one coercion
sequence with another and then reduce the result to normal form, the
security of the resulting coercion sequence should be greater than or
equal to that of the first sequence:

\begin{lemma}[Composition models explicit flow]\ \\
  \label{lem:comp-explicit}
  If $\mathbf{NF} \; \bar{c}$
  and $\bar{c} \mdoubleplus \bar{d} \longrightarrow^{*} \bar{c}'$
  and $\mathbf{NF} \; \bar{c}'$,
  then $|\bar{c}| \preccurlyeq |\bar{c}'|$.
\end{lemma}
\begin{proof}
  The proof is in \texttt{comp-security} of \texttt{/src/CoercionExpr/SecurityLevel.agda}.
\end{proof}

Next we show that stamping models implicit flow correctly, promoting
the security of the stamped coercion by joining it with the stamped
label:

\begin{lemma}[Stamping models implicit flow]\ \\
  \label{lem:stamp-implicit}
  If $\;\mathbf{NF}\; \bar{c}$,
  then $| \mathit{stamp} \; \bar{c} \; \ell| = |\bar{c}| \curlyvee \ell$
  and $| \mathit{stamp!} \; \bar{c} \; \ell| = |\bar{c}| \curlyvee \ell$.
\end{lemma}
\begin{proof}
  The proof is in \texttt{stamp$_{\mathtt{l}}$-security} of \texttt{/src/CoercionExpr/Stamping.agda}.
\end{proof}

\section{Noninterference of \DynIFC}
\label{sec:NI-dynifc}

\begin{figure}[tbp]
  \raggedright
  \fbox{\bigstep{\mu}{\pc}{M}{V}{\mu'}}
  {\small
  \begin{gather*}
  {\Downarrow}\textit{-val}~
  \inference{}{\bigstep{\mu}{\pc}{V}{V}{\mu}}
  \qquad
  {\Downarrow}\textit{-app}~
  \inference
  {\bigstep{\mu}{\pc}{L}{\dynlam{x}{N}{\ell}}{\mu_1} \\
   \bigstep{\mu_1}{\pc}{M}{V}{\mu_2} \\
   \bigstep{\mu_2}{\pc \curlyvee \ell}{N[ x := V ]}{W}{\mu_3}}
  {\bigstep{\mu}{\pc}{L \; M}{W \curlyvee \ell}{\mu_3}}
  \\[2ex]
  {\Downarrow}\textit{-if-true}~
  \inference
  {\bigstep{\mu}{\pc}{L}{\dynconst{\true}{\ell}}{\mu_1} \\
   \bigstep{\mu_1}{\pc \curlyvee \ell}{M}{V}{\mu_2}}
  {\bigstep{\mu}{\pc}{\dynif{L}{M}{N}}{V \curlyvee \ell}{\mu_2}}
  \quad
  {\Downarrow}\textit{-if-false}~
  \inference
  {\bigstep{\mu}{\pc}{L}{\dynconst{\false}{\ell}}{\mu_1} \\
   \bigstep{\mu_1}{\pc \curlyvee \ell}{N}{V}{\mu_2}}
  {\bigstep{\mu}{\pc}{\dynif{L}{M}{N}}{V \curlyvee \ell}{\mu_2}}
  \\[2ex]
    {\Downarrow}\textit{-ref?}~
    \inference
        {\bigstep{\mu}{\pc}{M}{V}{\mu_1} \\
          n \; \mathbf{FreshIn} \; \mu_1(\ell) & \highlightred{\pc \preccurlyeq \ell}}
        {\bigstep{\mu}{\pc}{\dynref{?}{\ell}{M}}{\dynaddr{n_{\ell}}{\low}}{(\mu_1, \ell \mapsto n \mapsto (V \curlyvee \ell))}}
  \\[2ex]
  {\Downarrow}\textit{-deref}~
  \inference
  {\bigstep{\mu}{\pc}{M}{\dynaddr{n_{\hat{\ell}}}{\ell}}{\mu_1} \\
    \mathit{\mu_1(\hat{\ell},n) = V}}
  {\bigstep{\mu}{\pc}{\dynderef{M}}{V \curlyvee \ell}{\mu_1}}
  \\[2ex]
  {\Downarrow}\textit{-assign?}~
  \inference
  {\bigstep{\mu}{\pc}{L}{\dynaddr{n_{\hat{\ell}}}{\ell}}{\mu_1} \\
   \bigstep{\mu_1}{\pc}{M}{V}{\mu_2} &
   \highlightred{\pc \curlyvee \ell \preccurlyeq \hat{\ell}}}
  {\bigstep{\mu}{\pc}{\dynassign{L}{?}{M}}{\dynconst{\unit}{\low}}{[\hat{\ell} \mapsto n \mapsto (V \curlyvee \hat{\ell})] \mu_2}}
  \end{gather*} }
  \caption{Big-step operational semantics (successful cases) of \DynIFC}
  \label{fig:big-step-dyn}
\end{figure}


%% \begin{figure}[tbp]
%%   \raggedright
%%   \fbox{$\reduce{M}{\mu}{\pc}{N}{\mu'}$}
%%   {\small
%%   \begin{gather*}
%%   \xi~
%%   \inference
%%   {\reduce{M}{\mu}{\pc}{M'}{\mu'}}
%%   {\reduce{\mathit{plug}\;M\;F}{\mu}{\pc}{\mathit{plug}\;M'\;F}{\mu'}}
%%   \\[1ex]
%%   \textit{prot-val}~
%%   \inference{}{\reduce{\dynprot{\ell}{V}}{\mu}{\pc}{V \curlyvee \ell}{\mu}}
%%   \quad
%%   \textit{prot-ctx}~
%%   \inference
%%   {\reduce{M}{\mu}{\pc \curlyvee \ell}{M'}{\mu'}}
%%   {\reduce{\dynprot{\ell}{M}}{\mu}{\pc}{\dynprot{\ell}{M'}}{\mu'}}
%%   \\[1ex]
%%   \beta~
%%   \inference{}{\reduce{\dynlam{x}{N}{\ell} \; V}{\mu}{\pc}{\dynprot{\ell}{(N [x := V])}}{\mu}}
%%   \\[1ex]
%%   \beta\textit{-if-true}~
%%   \inference{}{\reduce{\dynif{\dynconst{\true}{\ell}}{M}{N}}{\mu}{\pc}{\dynprot{\ell}{M}}{\mu}}
%%   \\[1ex]
%%   \beta\textit{-if-false}~
%%   \inference{}{\reduce{\dynif{\dynconst{\false}{\ell}}{M}{N}}{\mu}{\pc}{\dynprot{\ell}{N}}{\mu}}
%%   \\[1ex]
%%   \textit{ref?-ok}~
%%   \inference
%%   {\highlightred{\pc \preccurlyeq \ell} & n \; \mathbf{FreshIn} \; \mu(\ell)}
%%   {\reduce{\dynref{?}{\ell}{V}}{\mu}{\pc}{\dynaddr{n_{\ell}}{\low}}{(\mu, \ell \mapsto n \mapsto (V \curlyvee \ell))}}
%%   \\[1ex]
%%   \textit{deref}~
%%   \inference
%%   {\mathit{\mu(\hat{\ell},n) = V}}
%%   {\reduce{\dynderef{\dynaddr{n_{\hat{\ell}}}{\ell}}}{\mu}{\pc}{\dynprot{\ell}{V}}{\mu}}
%%   \\[1ex]
%%   \textit{assign?-ok}~
%%   \inference
%%   {\highlightred{\pc \curlyvee \ell \preccurlyeq \hat{\ell}}}
%%   {\reduce{\dynassign{\dynaddr{n_{\hat{\ell}}}{\ell}}{?}{V}}{\mu}{\pc}{\dynconst{\unit}{\low}}{[\hat{\ell} \mapsto n \mapsto (V \curlyvee \hat{\ell})] \mu}}
%%   \end{gather*}}
%%   \caption{Small-step operational semantics (successful cases) of \DynIFC}
%%   \label{fig:sem-succ-dyn}
%% \end{figure}

We obtain the big-step semantics in Figure~\ref{fig:big-step-dyn} by a
mechanical conversion from the successful (non-erroring) cases of the small-step
semantics of \DynIFC in Figure~\ref{fig:dyn-ifc}. I conjecture that small-step
and big-step semantics for \DynIFC coincide; in particular, multi-stepping to a
value should imply big-step:

\begin{lemma}
  \label{lem:mult-step-impl-big-step}
  If $\reducemult{M}{\mu}{\pc}{V}{\mu'}$, then $\bigstep{\mu}{\pc}{M}{V}{\mu'}$
\end{lemma}

\noindent The proof technique for Lemma~\ref{lem:mult-step-impl-big-step} should
be standard. We could first follow~\textcite{Streicher:2006aa} (the first
exercise about PCF on p. 17) and prove a lemma that if
$\reduce{M}{\mu_1}{\pc}{N}{\mu_2}$ and $\bigstep{\mu_2}{\pc}{N}{V}{\mu_3}$, then
$\bigstep{\mu_1}{\pc}{M}{V}{\mu_3}$. We then perform induction on
$\reducemult{M}{\mu}{\pc}{V}{\mu'}$: if $M$ is already a value, the goal is
proved directly; otherwise, $M$ takes at least one reduction step, so we use the
induction hypothesis and apply the aforementioned lemma.

\begin{lemma}[Noninterference of \DynIFC]
\label{lem:NI-old}
If
$\bigstep{\emptyset}{\low}{M [ x := \dynconst{b_1}{\high}]}{\dynconst{b_3}{\low}}{\mu_1}$
and
$\bigstep{\emptyset}{\low}{M [ x := \dynconst{b_2}{\high}]}{\dynconst{b_4}{\low}}{\mu_2}$
then $b_3 = b_4$.
\end{lemma}
\begin{proof}
The proof is fully mechanized in \texttt{/src/Dyn/Noninterference.agda}. The
structure of the proof directly follows the noninterference proof of
\CCOld~\parencite{Chen:2022aa}.
\end{proof}

\section{Simulation Between \CC and \DynIFC}
\label{sec:sim-cc-dynifc}

\begin{figure}[tbp]
  \raggedright
  \fbox{$\simrel{\gc}{M}{N}{A}$}
  \small
  \begin{gather*}
    % var
    {\leq}\textit{-var}~
    \inference{}{\simrel{\gc}{x}{x}{A}}
    \qquad
    % const
    {\leq}\textit{-const}~
    \inference{\ell' \preccurlyeq \ell}
              {\gc \vdash \dynconst{k}{\ell'} \leq \ccconst{k} \Leftarrow \iota_{\ell}}
    \\[2ex]
    {\leq}\textit{-wrapped-const}~
    \inference{\ell' \preccurlyeq |\bar{c}| & \vdash \bar{c} : \ell \Rightarrow g & \mathbf{NF}\;\bar{c} & \ell \neq g}
    {\gc \vdash \dynconst{k}{\ell'} \leq \cccast{\ccconst{k}}{\coerc{\id{\iota}}{\bar{c}}} \Leftarrow \iota_g}
    \\[2ex]
    % lambda
    {\leq}\textit{-lam}~
    \inference{\simrel{g}{N'}{N}{B} & \ell' \preccurlyeq \ell}
              {\simrel{\gc}{\dynlam{x}{N'}{\ell'}}{\cclam{x}{N}}{(\Fun{A}{g}{B})_{\ell}}}
    \\[2ex]
    {\leq}\textit{-wrapped-lam}~
    \inference{\simrel{g_3}{N'}{N}{D} & \ell \preccurlyeq |\bar{c}| \\ \vdash \bar{d} : g_1 \Rightarrow g_3 & \vdash \bm{c} : A \Rightarrow C & \vdash \bm{d} : D \Rightarrow B & \mathbf{NF}\;\bar{c}}
              {\simrel{\gc}{\dynlam{x}{N'}{\ell}}{\cccast{(\cclam{x}{N})}{\coerc{\funco{\bar{d}}{\bm{c}}{\bm{d}}}{\bar{c}}}}{(\Fun{A}{g_1}{B})_{g_2}}}
    \\[2ex]
    % addr
    {\leq}\textit{-addr}~
    \inference{\ell' \preccurlyeq \ell}
              {\simrel{\gc}{\dynaddr{n_{\hat{\ell}}}{\ell'}}{\ccaddr{n}}{\Refer{(T_{\hat{\ell}})}_{\ell}}}
    \\[2ex]
    {\leq}\textit{-wrapped-addr}~
    \inference{\ell \preccurlyeq |\bar{c}| & \vdash \bm{c} : T_{g_1} \Rightarrow S_{\hat{\ell}} & \vdash \bm{d} : S_{\hat{\ell}} \Rightarrow T_{g_1} & \mathbf{NF}\;\bar{c}}
              {\simrel{\gc}{\dynaddr{n_{\hat{\ell}}}{\ell}}{\cccast{(\ccaddr{n})}{\coerc{\refco{\bm{c}}{\bm{d}}}{\bar{c}}}}{(\Refer{(T_{g_1})})_{g_2}}}
    \\[2ex]
    % app
    {\leq}\textit{-app}~
    \inference{\simrel{\ell^c}{M'}{M}{(\Fun{A}{\ell^c \curlyvee \ell}{B})_\ell} & \simrel{\ell^c}{N'}{N}{A}}
              {\simrel{\ell^c}{M'\;N'}{\ccapp{M}{N}{A}{B}{\ell}}{C}}
    \\[2ex]
    % app*
    {\leq}\textit{-app}\star~
    \inference{\simrel{\gc}{M'}{M}{(\Fun{A}{\unk}{T_{\unk}})_{\unk}} & \simrel{\gc}{N'}{N}{A}}
              {\simrel{\gc}{M'\;N'}{\ccappstar{M}{N}{A}{T}{}}{T_{\unk}}}
    \\[2ex]
    % if
    {\leq}\textit{-if}~
    \inference{\simrel{\ell^c}{L'}{L}{\Bool_\ell} & \simrel{\ell^c \curlyvee \ell}{M'}{M}{A} & \simrel{\ell^c \curlyvee \ell}{N'}{N}{A}}
              {\simrel{\ell^c}{\dynif{L'}{M'}{N'}}{\ccif{L}{A}{\ell}{M}{N}}{B}}
    \\[2ex]
    % if*
    {\leq}\textit{-if}\star~
    \inference{\simrel{\gc}{L'}{L}{\Bool_{\unk}} & \simrel{\unk}{M'}{M}{T_{\unk}} & \simrel{\unk}{N'}{N}{T_{\unk}}}
              {\simrel{\gc}{\dynif{L'}{M'}{N'}}{\ccifstar{L}{T}{M}{N}}{T_{\unk}}}
    \\[2ex]
    % cast
    {\leq}\textit{-cast}~
    \inference{\simrel{\gc}{M}{N}{A} & \vdash \bm{c} : A \Rightarrow B}
              {\simrel{\gc}{M}{\cccast{N}{\bm{c}}}{B}}
    \\[2ex]
    % prot
    {\leq}\textit{-prot}~
    \inference{\ell' \preccurlyeq \ell & g_2 \vdash M' \leq M \Leftarrow A & \vdash PC \Leftarrow g_2}
    {g_1 \vdash \dynprot{\ell'}{M'} \leq \ccprot{\PC}{\ell}{M}{A} \Leftarrow B}
  \end{gather*}
  \caption{Simulation relation between \CC and \DynIFC (Part I)}
  \label{fig:sim-rel-1}
\end{figure}

\begin{figure}[tbp]
  \raggedright
  \fbox{$\simrel{\gc}{M}{N}{A}$}
  \small
  \begin{gather*}
    % ref
    {\leq}\textit{-ref}~
    \inference{\simrel{\ell^c}{M'}{M}{T_\ell}}
              {\simrel{\ell^c}{\dynref{?}{\ell}{M'}}{\ccref{\ell}{M}}{(\Refer{T_\ell})_{\low}}}
    \\[2ex]
    % ref?
    {\leq}\textit{-ref?}~
    \inference{\simrel{\unk}{M'}{M}{T_\ell}}
              {\simrel{\unk}{\dynref{?}{\ell}{M'}}{\ccrefproj{\ell}{M}{p}}{(\Refer{T_\ell})_{\low}}}
    \\[2ex]
    % deref
    {\leq}\textit{-deref}~
    \inference{\simrel{\gc}{M'}{M}{(\Refer{A})_\ell}}
              {\simrel{\gc}{\dynderef{M'}}{\ccderef{M}{A}{\ell}}{B}}
    \quad
    % deref*
    {\leq}\textit{-deref}\star~
    \inference{\simrel{\gc}{M'}{M}{(\Refer{T_{\unk}})_{\unk}}}
              {\simrel{\gc}{\dynderef{M'}}{\ccderefstar{M}{T}}{T_{\unk}}}
    \\[2ex]
    % assign
    {\leq}\textit{-assign}~
    \inference{\simrel{\ell^c}{L'}{L}{(\Refer{T_{\hat{\ell}}})_{\ell}} & \simrel{\ell^c}{M'}{M}{T_{\hat{\ell}}}}
              {\simrel{\ell^c}{\dynassign{L'}{?}{M'}}{\ccassign{L}{M}{T}{\hat{\ell}}{\ell}}{\Unit_{\low}}}
    \\[2ex]
    {\leq}\textit{-assign?}~
    \inference{\simrel{\gc}{L'}{L}{(\Refer{T_g})_{\unk}} & \simrel{\gc}{M'}{M}{T_g}}
              {\simrel{\gc}{\dynassign{L'}{?}{M'}}{\ccassignproj{L}{M}{T}{g}{p}}{\Unit_{\low}}}
  \end{gather*}
  \caption{Simulation relation between \CC and \DynIFC (Part II)}
  \label{fig:sim-rel-2}
\end{figure}

\begin{definition}[Simulation between heaps]
\label{def:sim-heap}
$\Sigma \vdash \mu' \leq \mu \triangleq \forall \ell, n.$ if $\mu(\ell,n) = V$, then there exists
$V'$ s.t $\mu'(\ell,n) = V'$ and $\simrel{\low}{V'}{V}{T_{\ell}}$ where $T = \Sigma(\ell,n)$.
\end{definition}

\begin{lemma}[Casting preserves simulation]
  \label{lem:cast-sim}
  If $\simrel{\gc}{W'}{V}{A}$, $\vdash \bm{c} : A \Rightarrow B$, \\
  and $\cccast{V}{\bm{c}} \longrightarrow^{*} W$, then $\simrel{\gc}{W'}{W}{B}$.
\end{lemma}
\begin{proof}
By induction on the multi-step cast reduction.
\begin{description}
\item[Zero step] Directly proved by applying rule ${\leq}\textit{-cast}$.
\item[One or more steps] Casing on the first reduction step yields three sub-cases:
\begin{description}
\item[\textit{cast}]
\begin{align}
\simrel{\gc}{W'}{V_r}{A} \\
\cccast{V_r}{\coerc{c_r}{\bar{c}}} \longrightarrow \cccast{V_r}{\coerc{c_r}{\bar{d}}} \longrightarrow^{*} W
\end{align}
By induction hypothesis, \simrel{\gc}{W'}{W}{B}.
\item[\textit{cast-id}]
\begin{align}
\simrel{\gc}{W'}{V_r}{A} \\
\cccast{V_r}{\coerc{\id{\iota}}{\id{g}}} \longrightarrow V_r
\end{align}
The goal is proved directly.
\item[\textit{cast-comp}]
\begin{align}
\simrel{\gc}{W'}{\cccast{V_r}{\bm{c}}}{A} \\
\cccast{\cccast{V_r}{\bm{c}}}{\bm{d}} \longrightarrow \cccast{V_r}{\bm{c} \mdoubleplus \bm{d}} \longrightarrow^{*} W
\end{align}
We further reason about $\cccast{V_r}{\bm{c} \mdoubleplus \bm{d}} \longrightarrow^{*} W$. The coercion
after composition $\bm{c} \mdoubleplus \bm{d}$ is reducible and not identity, so the reduction must
take one step by \textit{cast} and reduce the top-level coercion sequence to its normal form $\bar{c}_n$:
\[
\cccast{\cccast{V_r}{\coerc{c_{r1}}{\bar{c}}}}{\coerc{c_{r2}}{\bar{d}}} \longrightarrow \cccast{V_r}{\coerc{c_r}{\bar{c} \mdoubleplus \bar{d}}}
\longrightarrow \cccast{V_r}{\coerc{c_r}{\bar{c}_n}} \longrightarrow^{*} W
\]
We know $|\bar{c}| \preccurlyeq |\bar{c}_n|$ because composition models explicit flow (Lemma~\ref{lem:comp-explicit}).
\begin{itemize}
\item If $V_r = \ccconst{k}$ and $\bar{c}_n = \id{\ell}$. $W' = \dynconst{k}{\ell'}$
and $\ell' \preccurlyeq |\bar{c}|$ (Lemma~\ref{lem:sim-wrapped-const}).
We know $\simrel{\gc}{\dynconst{k}{\ell'}}{\ccconst{k}}{\iota_\ell}$ by rule ${\leq}\textit{-const}$,
because $\ell' \preccurlyeq |\bar{c}| \preccurlyeq |\bar{c}_n| = |\id{\ell}| = \ell$.
\item Otherwise, \cccast{V_r}{\coerc{c_r}{\bar{c}_n}} is already a value.
Apply rule ${\leq}\textit{-wrapped-const}$, ${\leq}\textit{-wrapped-lam}$, or ${\leq}\textit{-wrapped-addr}$
depending on whether $V_r$ is a constant, a $\lambda$, or an address.
\end{itemize}
\end{description}
\end{description}
\end{proof}

\begin{lemma}[Substitution preserves simulation]
  \label{lem:subst-sim}
  If $\simrel{g}{N'}{N}{A}$, $(\Gamma, x:B);\Sigma;g;\ell \vdash N \Leftarrow A$, and $\forall g. \simrel{g}{M'}{M}{B}$, then $\simrel{g}{N'[x:=M']}{N[x:=M]}{A}$.
\end{lemma}
\begin{proof}
  The proof is fully mechanized in \texttt{/src/Security/SubstPres.agda}. The
  simulation relation is defined in \texttt{/src/Security/SimRel.agda}.
\end{proof}

\begin{lemma}[Simulation with wrapped constant]
\label{lem:sim-wrapped-const}
If $\simrel{\gc}{M}{\cccast{(\ccconst{k})}{\coerc{\id{\iota}}{\bar{c}}}}{\iota_g}$
and $(\coerc{\id{\iota}}{\bar{c}})$ is irreducible,
then there exists $\ell$ s.t $M = \dynconst{k}{\ell}$, and $\ell \preccurlyeq |\bar{c}|$.
\end{lemma}
\begin{proof}
Inversion on the simulation relation:
\begin{description}
\item[Related by ${\leq}\textit{-wrapped-const}$] The goal is proved directly.
\item[Related by ${\leq}\textit{-cast}$] We know \simrel{\gc}{M}{\ccconst{k}}{\iota_\ell}
              and $\bar{c} : \ell \Rightarrow g$. Note that $(\coerc{\id{\iota}}{\bar{c}})$ is irreducible,
              so $\bar{c}$ can be \up, $\inj{\ell}$, or $\up\seq\inj{\high}$, so $\ell \preccurlyeq |\bar{c}|$.
              By rule ${\leq}\textit{-const}$, we know $M = \dynconst{k}{\ell'}$ for some $\ell'$
              and $\ell' \preccurlyeq \ell$.
              Thus $\ell' \preccurlyeq \ell \preccurlyeq |\bar{c}|$.
\end{description}
\end{proof}

\begin{lemma}[Simulation with reference proxy]
\label{lem:sim-ref-proxy}
If $\simrel{\gc}{M}{\cccast{(\ccaddr{n})}{\coerc{\refco{\bm{c}}{\bm{d}}}{\bar{c}}}}{(\Refer{T_{g_1}})_{g_2}}$
$\vdash \bm{c} : T_{g_1} \Rightarrow S_{\hat{\ell}}$, $\vdash \bm{d} : S_{\hat{\ell}} \Rightarrow T_{g_1}$,
and $\mathbf{NF}\;\bar{c}$, then there exists $\ell$ s.t $M = \dynaddr{n_{\hat{\ell}}}{\ell}$,
and $\ell \preccurlyeq |\bar{c}|$.
\end{lemma}
\begin{proof}
Analogous to Lemma~\ref{lem:sim-wrapped-const}. The only extra case to consider
is $\bar{c} = \id{\ell}$, which also satisfies $\ell \preccurlyeq |\bar{c}|$.
\end{proof}

\begin{lemma}[Simulation with function proxy]
\label{lem:sim-func-proxy}
If $\simrel{\gc}{M}{\cccast{(\cclam{x}{N})}{\coerc{\funco{\bar{d}}{\bm{c}}{\bm{d}}}{\bar{c}}}}{(\Fun{A}{g_1}{B})_{g_2}}$,
$\vdash \bar{d} : g_1 \Rightarrow g_3$,
and $\mathbf{NF}\;\bar{c}$, then there exists $N', \ell$ s.t $M = \dynlam{x}{N'}{\ell}$,
$\simrel{g_3}{N'}{N}{B}$, and $\ell \preccurlyeq |\bar{c}|$.
\end{lemma}
\begin{proof}
Analogous to Lemma~\ref{lem:sim-wrapped-const} and Lemma~\ref{lem:sim-ref-proxy}.
\end{proof}

\begin{lemma}[Simulation with \CC value]
\label{lem:sim-val}
If $\simrel{\gc}{M}{V}{A}$, then $M$ is a value.
\end{lemma}
\begin{proof}
Inversion on the value $V$ and the simulation relation $\simrel{\gc}{M}{V}{A}$ yields 6 cases.
We consider the two cases for constants; the cases for $\lambda$s and addresses are analogous.
\begin{description}
\item[Related by ${\leq}\textit{-const}$] We know $M = \dynconst{k}{\ell}$ for some $k, \ell$,
so $M$ is a value.
\item[Related by ${\leq}\textit{-wrapped-const}$] By Lemma~\ref{lem:sim-wrapped-const},
$M = \dynconst{k}{\ell}$ for some $k, \ell$, so $M$ is a value.
\end{description}
\end{proof}

\begin{lemma}[Stamping preserves simulation]
\label{lem:sim-stamp}
If $\simrel{\gc}{V}{W}{A}$ and $\ell' \preccurlyeq \ell$, then
$\simrel{\gc}{V \curlyvee \ell'}{\mathit{stamp}\;W\;\ell}{\mathit{stamp}\;A\;\ell}$.
\end{lemma}
\begin{proof}
Casing on $\ell' \preccurlyeq \ell$:
\begin{description}
\item[Case 1] $\ell'=\low, \ell=\low$: Straightforward because stamping low returns the same value.
\item[Case 2] $\ell'=\low, \ell=\high$: Casing on value $W$ and the simulation relation $\simrel{\gc}{V}{W}{A}$
produces 6 sub-cases. We consider the two cases for constants below (corresponding to ${\leq}\textit{-const}$ and
${\leq}\textit{-wrapped-const}$), because the cases for $\lambda$s and addresses are analogous:
\begin{itemize}
\item $\simrel{\gc}{\dynconst{k}{\ell_1'}}{\ccconst{k}}{\iota_{\ell_1}}$
      \begin{itemize}
      \item[-] If $\ell_1 = \low$, $\simrel{\gc}{\dynconst{k}{\ell_1'}}{\cccast{\ccconst{k}}{\coerc{\id{\iota}}{\up}}}{\iota_{\high}}$
      because $\ell_1' \preccurlyeq |\up| = \high$ (rule ${\leq}\textit{-wrapped-const}$).
      \item[-] If $\ell_1 = \high$, $\simrel{\gc}{\dynconst{k}{\ell_1'}}{\ccconst{k}}{\iota_{\high}}$
      because $\ell_1' \preccurlyeq \high$ (rule ${\leq}\textit{-const}$).
      \end{itemize}
\item $\simrel{\gc}{\dynconst{k}{\ell_1'}}{\cccast{\ccconst{k}}{\coerc{\id{\iota}}{\bar{c}}}}{-}$
      \begin{itemize}
      \item[-]
        $\simrel{\gc}{\dynconst{k}{\ell_1'}}{\cccast{k}{\coerc{\id{\iota}}{\mathit{stamp}\;\bar{c}\;\high}}}{-}$
        because \newline $\ell_1' \preccurlyeq |\mathit{stamp}\;\bar{c}\;\high|
        = |\bar{c}| \curlyvee \high = \high$ by Lemma~\ref{lem:stamp-implicit}
        (Stamping models implicit flow) and rule
        ${\leq}\textit{-wrapped-const}$.
      \end{itemize}
\end{itemize}
\item[Case 3] $\ell'=\high, \ell=\high$: Analogous to Case 2; the only difference is that
           instead of $\dynconst{k}{\ell_1'}$, the left side is always $\dynconst{k}{\high}$
           because it is stamped with $\ell' = \high$.
\end{description}
\end{proof}

\begin{lemma}[Casting a label expression models explicit flow]
\label{lem:cast-lexpr-explicit}
If $\mathbf{NF}\;e_1$ and $\cccast{e_1}{\bar{c}} \longrightarrow^{*} e_2$ and $\mathbf{NF}\;e_2$,
$|e_1| \preccurlyeq |e_2|$.
\end{lemma}
\begin{proof}
The proof is in \texttt{cast-security} of \texttt{/src/LabelExpr/Security.agda}.
The proof is by inversion on the multi-step reduction.
\end{proof}

\begin{lemma}[Stamping a label expression models implicit flow]
\label{lem:stamp-lexpr-implicit}
If $\mathbf{NF}\;e$, $|\mathit{stamp}\;e\;\ell| = |e| \curlyvee \ell$ and
$|\mathit{stamp!}\;e\;\ell| = |e| \curlyvee \ell$.
\end{lemma}
\begin{proof}
The proof is fully mechanized in $\mathtt{stamp_e\texttt{-}security}$ and
$\mathtt{stamp!_e\texttt{-}security}$ of \texttt{/src/LabelExpr/Security.agda}.
The proof is by casing on $\mathbf{NF}\;e$.
\end{proof}

\begin{lemma}
\label{lem:cast-leq}
Suppose $\Gamma ; \Sigma ; g ; \ell \vdash V \Leftarrow S_{\ell_1}$ and
$\vdash \bm{c} : S_{\ell_1} \Rightarrow T_{\ell_2}$. If $\cccast{V}{\bm{c}} \longrightarrow^{*} W$,
then $\ell_1 \preccurlyeq \ell_2$.
\end{lemma}
\begin{proof}
By induction on the multi-step reduction.
\begin{description}
\item[Zero step] $\bm{c} = (\coerc{c_r}{\bar{c}})$ is irreducible, so
$\bar{c} : \ell_1 \Rightarrow \ell_2$ is in its normal form. Thus $\ell_1 \preccurlyeq \ell_2$.
\item[One or more steps] Case on the first reduction step:
\begin{itemize}
\item Rule \textit{cast}: $\cccast{V_r}{\coerc{c_r}{\bar{c}}} \longrightarrow \cccast{V_r}{\coerc{c_r}{\bar{d}}} \longrightarrow^{*} W$.
  By preservation of coercion sequences, $\bar{d} : \ell_1 \Rightarrow \ell_2$. By induction hypothesis, $\ell_1 \preccurlyeq \ell_2$.
\item Rule \textit{cast-id}: $\cccast{V_r}{\coerc{\id{\iota}}{\id{g}}} \longrightarrow V_r$. We directly know $\ell_1 = g = \ell_2$.
\item Rule \textit{cast-comp}: $\cccast{\cccast{V_r}{\bm{c}}}{\bm{d}} \longrightarrow \cccast{V_r}{\bm{c} \mdoubleplus \bm{d}} \longrightarrow^{*} W$.
      Suppose $\bm{c} = (\coerc{-}{\bar{c}}), \bar{c} : \ell_0 \Rightarrow \ell_1$ and
      $\bm{d} = (\coerc{-}{\bar{d}}), \bar{d} : \ell_1 \Rightarrow \ell_2$.
      We know $\cccast{V_r}{\coerc{-}{\bar{c} \mdoubleplus \bar{d}}} \longrightarrow \cccast{V_r}{\bar{c}_n} \longrightarrow^{*} W$.
      By Lemma~\ref{lem:comp-explicit} (Composition models explicit flow),
      $\ell_1 = |\bar{c}| \preccurlyeq |\bar{c}_n| = \ell_2$.
\end{itemize}
\end{description}
\end{proof}

\begin{lemma}
\label{lem:leq-value-pc}
If $\simrel{g_1}{M}{V}{A}$ then $\simrel{g_2}{M}{V}{A}$.
\end{lemma}
\begin{proof}
The proof is fully mechanized in Agda and can be found in the
supplementary material. The proof is by casing on value and the
simulation relation.
\end{proof}


\begin{lemma}[Simulation between \CC and \DynIFC]
  \label{lem:sim-leq}
  Suppose $M_1$ is a well-typed \CC term: $\Gamma ; \Sigma_1 ; \gc ; |\PC| \vdash M_1 \Leftarrow A$,
  $PC$ is a well-typed label expression: $\vdash \PC \Leftarrow \gc$, and $\mu_1$ is a well-typed
  heap: $\Sigma_1 \vdash \mu_1$.
  Suppose $\gc \vdash M_1' \leq M_1 \Leftarrow A$, $\Sigma_1 \vdash \mu_1' \leq \mu_1$, and $\ell \preccurlyeq |\PC|$.
  If $\reduce{M_1}{\mu_1}{\PC}{M_2}{\mu_2}$, then there exists $M_3, \mu_3, M_2', \mu_2'$ s.t
  $\reducemult{M_2}{\mu_2}{\PC}{M_3}{\mu_3}$,
  $\reducemult{M_1'}{\mu_1'}{\ell}{M_2'}{\mu_2'}$,
  $\gc \vdash M_2' \leq M_3 \Leftarrow A$,
  and $\Sigma_2 \vdash \mu_2' \leq \mu_3$ for some $\Sigma_2$.
\end{lemma}
\begin{proof}
  The proof is by induction on the reduction relation
  $\reduce{M_1}{\mu_1}{\PC}{M_2}{\mu_2}$
  and inversion on the simulation relation
  $\vdash M_1' \leq M_1 \Leftarrow A$.
  We only consider successful cases of the reduction (which do not produce errors),
  because the theorem statement of noninterference (Theorem~\ref{thm:NI-Surface})
  for \Surface
  is termination-insensitive.
  \begin{description}
    \item[Case~$\xi$:]
      \begin{align}
      \reduce{N_1}{\mu_1}{\PC}{N_2}{\mu_2} \\
      \mathit{plug}\;N_1\;F = M_1 \\
      \mathit{plug}\;N_2\;F = M_2
      \end{align}
      We case on the frame $F$. We consider $F=\ccapp{\Box}{M}{A}{B}{\ell}$ below;
      the cases for other frames all follow the same pattern.
      By inversion of the simulation relation, the left side must be a function application:
      \[
      \simrel{\ell^c}{N_1'\;M'}{\ccapp{N_1}{M}{A}{B}{\ell}}{C}
      \]
      We know $\simrel{\ell^c}{N_1'}{N_1}{(\Fun{A}{\ell^c \curlyvee \ell}{B})_\ell}$ and $\simrel{\ell^c}{M'}{M}{A}$.
      By the induction hypothesis, there exists $N_3, \mu_3, N_2', \mu_2'$ such that
      $\reducemult{N_2}{\mu_2}{\PC}{N_3}{\mu_3}$, $\reducemult{N_1'}{\mu_1'}{\ell}{N_2'}{\mu_2'}$,
      $\simrel{\ell^c}{N_2'}{N_3}{(\Fun{A}{\ell^c \curlyvee \ell}{B})_\ell}$, and $\Sigma_2 \vdash \mu_2' \leq \mu_3$ for some $\Sigma_2$.
      Choose $M_3 = \mathit{plug}\;N_3\;F = \ccapp{N_3}{M}{A}{B}{\ell}$, $\mu_3 = \mu_3$, $M_2' = N_2'\;M'$,
      and $\mu_2' = \mu_2'$.
      We know $\reducemult{\ccapp{N_2}{M}{A}{B}{\ell}}{\mu_2}{\PC}{\ccapp{N_3}{M}{A}{B}{\ell}}{\mu_3}$,
      $\reducemult{N_1'\;M'}{\mu_1'}{\ell}{N_2'\;M'}{\mu_2'}$, and $\Sigma_2 \vdash \mu_2' \leq \mu_3$.
      We still need to show $\simrel{\ell^c}{N_2'\;M'}{\ccapp{N_3}{M}{A}{B}{\ell}}{C}$, which is proved by applying
      rule ${\leq}\mathit{-app}$.
  \item[Case~\textit{prot-val}:]
    By inversion on the simulation relation, the left side must be a protection term.
    By Lemma~\ref{lem:sim-val}, the body $V'$ must be a value:
    \begin{align}
    g_1 \vdash \dynprot{\ell'}{V'} &\leq \ccprot{\PC}{\ell}{V}{A} \Leftarrow B  \\
    g_2 \vdash V' &\leq V \Leftarrow A \label{eqn:117} \\
    \ell' &\preccurlyeq \ell \label{eqn:118} \\
    \vdash \PC &\Leftarrow g_2
    \end{align}
    We take a step by \textit{prot-val} on the left side, we need to relate:
    \[
    \simrel{g_1}{V' \curlyvee \ell'}{\mathit{stamp}\;V\;\ell}{B}
    \]
    which is proved by applying Lemma~\ref{lem:sim-stamp} on~\eqref{eqn:117} (with Lemma~\ref{lem:leq-value-pc} applied)
    and~\eqref{eqn:118}.
  \item[Case~\textit{prot-ctx}:]
  \[
  \reduce{M}{\mu_1}{\PC_2}{N}{\mu_2}
  \]
  By inversion on the simulation relation, the left side must be protection:
  \begin{align}
    g_1 \vdash \dynprot{\ell'}{M'} &\leq \ccprot{\PC_2}{\ell}{M}{A} \Leftarrow B  \\
    g_2 \vdash M' &\leq M \Leftarrow A \\
    \ell' &\preccurlyeq \ell \\
    \vdash \PC_2 &\Leftarrow g_2
  \end{align}
  By inversion on the typing of the protection term on the right:
  \begin{equation}
  \label{eqn:119}
  |\PC_1| \curlyvee \ell \preccurlyeq |\PC_2|
  \end{equation}
  where $\PC_1$ is the original PC to reduce the protection term on the right.
  To get the induction hypothesis, we need to show $\ell'' \curlyvee \ell' \preccurlyeq |PC_2|$
  where $\ell''$ is the PC to reduce the protection term on the left.
  We know $\ell'' \preccurlyeq |PC_1|$, thus:
  \[
  \ell'' \curlyvee \ell' \preccurlyeq \ell' \curlyvee \ell \preccurlyeq |\PC_1| \curlyvee \ell \preccurlyeq |\PC_2|
  \]
  The induction hypothesis shows that there exists $L, \mu_3, N', \mu_2'$ s.t
  $\reducemult{N}{\mu_2}{\PC_2}{L}{\mu_3}$,
  $\reducemult{M'}{\mu_1'}{\ell'' \curlyvee \ell'}{N'}{\mu_2'}$,
  $\simrel{g_2}{N'}{L}{A}$, and $\Sigma_2 \vdash \mu_2' \leq \mu_3$ for some $\Sigma_2$.
  We step the left side to $\dynprot{\ell'}{N'}$ and step the right side to
  $\ccprot{\PC_2}{\ell}{L}{A}$ using the congruence of \textit{prot-ctx}.
  We then relate the protection terms on both sides using rule ${\leq}\mathit{-prot}$.
  \item[Case~\textit{cast}:] There are three sub-cases: \textit{cast}, \textit{cast-id},
  and \textit{cast-comp}.
    \begin{description}
    \item[Sub-case~\textit{cast}:]
    We know $\simrel{\gc}{M'}{\cccast{V_r}{\coerc{c_r}{\bar{c}}}}{B}$. \cccast{V_r}{\coerc{c_r}{\bar{c}}}
    is not a value, so $\simrel{\gc}{M'}{V_r}{A}$. We need to relate
    $\simrel{\gc}{M'}{\cccast{V_r}{\coerc{c_r}{\bar{d}}}}{B}$, which is directly proved by applying
    ${\leq}\textit{-cast}$.
    \item[Sub-case~\textit{cast-id}:]
    We know $\simrel{\gc}{M'}{\cccast{V_r}{\coerc{\id{\iota}}{\id{g}}}}{A}$. Again, note that
    \cccast{V_r}{\coerc{\id{\iota}}{\id{g}}} is not a value, so $\simrel{\gc}{M'}{V_r}{A}$.
    \item[Sub-case~\textit{cast-comp}:]
    We know $\simrel{\gc}{M'}{\cccast{\cccast{V_r}{\bm{c}}}{\bm{d}}}{B}$. By inversion using
    rule ${\leq}\mathit{-cast}$, $\simrel{\gc}{M'}{\cccast{V_r}{\bm{c}}}{A}$; $\bm{c}$ is irreducible,
    by Lemma~\ref{lem:sim-val} $M'$ is a value. We need to show there exists $M$ such that
    $\reducemult{\cccast{V_r}{\bm{c}\mdoubleplus\bm{d}}}{\mu_1}{\PC}{M}{\mu_1}$ and $\simrel{\gc}{M'}{M}{B}$.
    Casing on $V_r$:
    \begin{itemize}
    \item \textbf{If $V_r = \ccconst{k}$.} Suppose
    $\bm{c}\mdoubleplus\bm{d} = \coerc{d_r}{\bar{c} \mdoubleplus \bar{d}}$ for some $d_r$.
    We take a step using \textit{cast} by reducing the composed coercion sequence to its
    normal form: $\bar{c} \mdoubleplus \bar{d} \longrightarrow^{+} \bar{c}_n$.
    By Lemma~\ref{lem:comp-explicit} (Composition models explicit flow), $|\bar{c}| \preccurlyeq |\bar{c}_n|$.
    By Lemma~\ref{lem:sim-wrapped-const}, we know $M' = \dynconst{k}{\ell'}$ and $\ell' \preccurlyeq |\bar{c}|$.
    By transitivity, $\ell' \preccurlyeq |\bar{c}_n|$.
    (1) If $\bar{c}_n$ is not an identity coercion, we relate
    $\simrel{\gc}{\dynconst{k}{\ell'}}{\cccast{\ccconst{k}}{\coerc{d_r}{\bar{c}_n}}}{B}$ directly using
    rule ${\leq}\textit{-wrapped-const}$. (2) If $\bar{c}_n = \id{\ell}$, we take one additional step by
    \textit{cast-id}. We know $\ell' \preccurlyeq |\id{\ell}| = \ell$,
    thus $\simrel{\gc}{\dynconst{k}{\ell'}}{\ccconst{k}}{\iota_\ell}$
    by rule ${\leq}\textit{-const}$.
    \item If $V_r = \cclam{x}{N}$. Similar to the constant case above.
    The only difference is that we do not need to specially handle $\bar{c}_n=\id{\ell}$,
    because the function coercion $(\coerc{\funco{\bar{d}}{\bm{c}}{\bm{d}}}{\id{\ell}})$
    is already irreducible.
    \item If $V_r = \ccaddr{n}$. Same as the lambda case above.
    \end{itemize}
    \end{description}
  \item[Case~$\beta$:]
      By inversion on the simulation relation, the left side must be a function application:
      \[
      \simrel{\ell^c}{L\;M}{\ccapp{(\cclam{x}{N})}{V}{A}{B}{\ell}}{C}
      \]
      where $C=\textit{stamp}\;B\;\ell$.
      We know that $L$ must be a $\lambda$ by rule ${\leq}\textit{-lam}$ and
      $M$ must be a value by Lemma~\ref{lem:sim-val}:
      \begin{align}
      \ell^c \vdash (\dynlam{x}{N'}{\ell'})\;V' &\leq \ccapp{(\cclam{x}{N})}{V}{A}{B}{\ell}
      \Leftarrow C \\
      \ell' &\preccurlyeq \ell       \label{eqn:105} \\
      \ell^c \curlyvee \ell \vdash N' &\leq N \Leftarrow B \label{eqn:106} \\
      \ell^c \vdash V' &\leq V \Leftarrow A \label{eqn:107}
      \end{align}
      We take one step by $\beta$ on the left side of the simulation relation:
      \[
        \reduce{(\dynlam{x}{N'}{\ell'})\;V'}{\mu'}{\ell''}{\dynprot{\ell'}{(N'[x:=V'])}}{\mu'}
      \]
      Now we need to show
      \[
      \ell^c \vdash \dynprot{\ell'}{(N'[x:=V'])} \leq \ccprot{(\mathit{stamp}\;\PC\;\ell)}{\ell}{(N[x:=V])}{B}
      \Leftarrow C
      \]
      Applying ${\leq}\textit{-prot}$ yields three sub-goals:
      (1) $\ell' \preccurlyeq \ell$ is directly proved by~\eqref{eqn:105}
      (2) $\simrel{\ell^c \curlyvee \ell}{N'[x:=V']}{N[x:=V]}{B}$ is proved by applying Lemma~\ref{lem:subst-sim}
          on~\eqref{eqn:106} and~\eqref{eqn:107}.
      (3) $\vdash \mathit{stamp}\;\PC\;\ell \Leftarrow \ell^c \curlyvee \ell$ because stamping is well-typed.
  \item[Case~\textit{app-cast}:]
      \begin{align}
      \cccast{(\mathit{stamp}\;\PC_1\;\ell)}{\bar{d}} &\longrightarrow^{*} \PC_2  \label{eqn:108} \\
      \cccast{V}{\bm{c}} &\longrightarrow^{*} W  \label{eqn:109}
      \end{align}
      By inversion on the simulation relation, the left side must be a function application:
      \[
      \ell^c \vdash L\;M \leq \ccapp{(\cccast{\cclam{x}{N}}{\coerc{\funco{\bar{d}}{\bm{c}}{\bm{d}}}{\bar{c}}})}{V}{C}{D}{\ell}
      \Leftarrow E
      \]
      where $E = \textit{stamp}\;D\;\ell$, $\vdash \bar{d} : \ell^c \curlyvee \ell \Rightarrow \gc$,
      $\vdash \bm{c} : C \Rightarrow A$, and $\vdash \bm{d} : B \Rightarrow D$.

      We know that $L$ must be a $\lambda$ by Lemma~\ref{lem:sim-func-proxy} and
      $M$ must be a value by Lemma~\ref{lem:sim-val}:
      \begin{align}
      \ell^c \vdash (\dynlam{x}{N'}{\ell'})\;V' &\leq \ccapp{(\cccast{\cclam{x}{N}}{\coerc{\funco{\bar{d}}{\bm{c}}{\bm{d}}}{\bar{c}}})}{V}{C}{D}{\ell}
      \Leftarrow E \\
      \ell' &\preccurlyeq |\bar{c}| \label{eqn:110} \\
      \gc \vdash N' &\leq N \Leftarrow B \label{eqn:111} \\
      \ell^c \vdash V' &\leq V \Leftarrow C \label{eqn:112}
      \end{align}
      We take a step by $\beta$ on the left side of simulation relation:
      \[
        \reduce{(\dynlam{x}{N'}{\ell'})\;V'}{\mu'}{\ell''}{\dynprot{\ell'}{(N'[x:=V'])}}{\mu'}
      \]
      Now we need to show
      \[
      \ell^c \vdash \dynprot{\ell'}{(N'[x:=V'])} \leq \ccprot{\PC_2}{\ell}{(\cccast{(N[x:=W])}{\bm{d}})}{D}
      \Leftarrow E
      \]
      By rule ${\leq}$\textit{-prot}, it is equivalent to showing:
      \begin{align}
      \ell' &\preccurlyeq \ell \label{eqn:115} \\
      \gc \vdash N'[x:=V'] &\leq \cccast{(N[x:=W])}{\bm{d}} \Leftarrow D \label{eqn:116} \\
      \vdash \PC_2 &\Leftarrow \gc
      \end{align}
      We know $\vdash \PC_2 \Leftarrow \gc$ because reducing label expressions preserves types.
      By~\eqref{eqn:110} and $|\bar{c}| = \ell$ we prove \eqref{eqn:115}.
      Apply Lemma~\ref{lem:cast-sim} on~\eqref{eqn:112} and~\eqref{eqn:109}, we get
      $\simrel{\ell^c}{V'}{W}{A}$.
      By Lemma~\ref{lem:subst-sim}, Lemma~\ref{lem:leq-value-pc}, and~\eqref{eqn:111},
      $\simrel{\gc}{N'[x:=V']}{N[x:=W]}{B}$.
      Apply rule ${\leq}$\textit{-cast} and we prove~\eqref{eqn:116}.
  \item[Case~\textit{app$\star$-cast}:] Similar to \textit{app-cast}.
  The only noteworthy difference is that on the right side, we use $|\bar{c}|$ instead of the
  $\ell$ from the syntax of the function application, both in the protection term
  and when stamping the PC.
  \item[Cases~$\beta$\textit{-if-true} and~$\beta$\textit{-if-false}:]
  By inversion on the simulation relation, the left side must also
  be an if-conditional:
  \[
  \simrel{\ell^c}{\dynif{\dynconst{\true}{\ell'}}{M'}{N'}}{\ccif{\ccconst{\true}}{A}{\ell}{M}{N}}{B}
  \]
  We know:
  \begin{align}
  \ell' &\preccurlyeq \ell \\
  \ell^c \curlyvee \ell \vdash M' &\leq M \Leftarrow A \\
  \ell^c \curlyvee \ell \vdash N' &\leq N \Leftarrow A
  \end{align}
  Take one step on the left side. We need to relate
  \[
  \simrel{\ell^c}{\dynprot{\ell'}{M'}}{\ccprot{(\mathit{stamp}\;\PC\;\ell)}{\ell}{M}{A}}{B}
  \]
  The goal is directly proved by applying rule ${\leq}\textit{-prot}$.
  The case for $\beta$\textit{-if-false} is analogous.
  \item[Cases \textit{if-true-cast} and \textit{if-false-cast}:]
  Note that the label partial order trivially holds because $\ell' \preccurlyeq \high$
  for any $\ell'$. The rest is analogous to $\beta$\textit{-if-true} and~$\beta$\textit{-if-false}.
  \item[Cases \textit{if$\star$-true-cast} and \textit{if$\star$-false-cast}:]
  By Lemma~\ref{lem:sim-wrapped-const} we know $\ell' \preccurlyeq |\bar{c}|$.
  The rest is analogous to $\beta$\textit{-if-true} and~$\beta$\textit{-if-false}.
  \item[Case~\textit{ref}:]
  %% \[
  %% n\;\mathbf{FreshIn}\;\mu(\ell)
  %% \]
  From the typing derivation (rule ${\vdash}\textit{-ref}$), we also know:
  \begin{align}
  \vdash \PC &\Leftarrow \ell^c \\
  \ell^c &\preccurlyeq \ell
  \end{align}
  which implies
  \begin{equation}
  |\PC| \preccurlyeq \ell
  \end{equation}
  We know $\ell' \preccurlyeq |\PC| \preccurlyeq \ell$, where $\ell'$ is the PC that the left side is reduced with.
  We take a step by \textit{ref?-ok} on the left side by choosing the same fresh address $n$. We need to show:
  \begin{align}
  \simrel{\ell^c}{\dynaddr{n_\ell}{\low}}{\ccaddr{n}}{(\Refer{T_\ell})_{\low}}  \label{eqn:120} \\
  (\Sigma_1, \ell \mapsto n \mapsto T) \vdash (\mu', \ell \mapsto n \mapsto (V' \curlyvee \ell)) \leq (\mu, \ell \mapsto n \mapsto V)  \label{eqn:121}
  \end{align}

  \eqref{eqn:120} is proved directly by ${\leq}\textit{-addr}$.
  To prove \eqref{eqn:121} we need to show:
  \begin{equation}
  \label{eqn:122}
  \simrel{\low}{V' \curlyvee \ell}{V}{T_\ell}
  \end{equation}
  By inversion on the simulation relation and Lemma~\ref{lem:leq-value-pc},
  we know $\simrel{\low}{V'}{V}{T_\ell}$.
  From the definition of stamping we know $\textit{stamp}\;V\;\ell = V$
  if $V \Leftarrow T_\ell$;
  \eqref{eqn:122} is thus proved by Lemma~\ref{lem:sim-stamp}.
  \item[Case~\textit{ref?}:]
  \[
  \cccast{\PC_1}{\unk \Rightarrow^{\bl{p}} \ell} \longrightarrow^{*} \PC_2
  \]
  Reducing label expressions preserves types, so $\vdash \PC_2 \Leftarrow \ell$.
  We know $|\PC_2| = \ell$ and $\ell' \preccurlyeq |\PC_1|$, where $\ell'$ is the
  PC that the left side is reduced with. Casting models explicit flow
  (Lemma~\ref{lem:cast-lexpr-explicit}), so $|\PC_1| \preccurlyeq |\PC_2|$.
  We thus have $\ell' \preccurlyeq |\PC_1| \preccurlyeq |\PC_2| = \ell$. Take one step
  by \textit{ref?-ok} on the left side by choosing the same fresh address $n$.
  The rest of the proof follows that of \textit{ref}.
  %% We need to show:
  %% \begin{align}
  %% \simrel{\dynaddr{n_\ell}{\low}}{\ccaddr{n}}{(\Refer{T_\ell})_{\low}}  \\
  %% (\mu', \ell \mapsto n \mapsto (V' \curlyvee \ell)) \leq (\mu, \ell \mapsto n \mapsto V)
  %% \end{align}
  \item[Case~\textit{deref}:]
  \[
  \mu(\hat{\ell},n) = V
  \]
  By $\Sigma \vdash \mu' \leq \mu$ and Definition~\ref{def:sim-heap},
  there exists $V'$ s.t $\mu'(\hat{\ell}, n) = V'$ and $\simrel{\low}{V'}{V}{T_{\hat{\ell}}}$
  where $T = \Sigma(\hat{\ell}, n)$.
  By inversion on the simulation relation, the left side must be a dereference:
  \[
  \simrel{\gc}{\dynderef{\dynaddr{n_{\hat{\ell}}}{\ell'}}}{\ccderef{(\ccaddr{n})}{T_{\hat{\ell}}}{\ell}}{B}
  \]
  where $\ell' \preccurlyeq \ell$. Take one step on the left using rule \textit{deref},
  we need to relate:
  \[
  \simrel{\gc}{\dynprot{\ell'}{V'}}{\ccprot{\high}{\ell}{V}{T_{\hat{\ell}}}}{B}
  \]
  which is directly proved by rule ${\leq}\textit{-prot}$ and Lemma~\ref{lem:leq-value-pc}.
  \item[Case~\textit{deref$\star$-cast}:]
  \[
  \mu(\hat{\ell}, n) = V
  \]
  By $\Sigma \vdash \mu' \leq \mu$ and Definition~\ref{def:sim-heap},
  there exists $V'$ s.t $\mu'(\hat{\ell}, n) = V'$ and $\simrel{\low}{V'}{V}{S_{\hat{\ell}}}$
  where $S = \Sigma(\hat{\ell}, n)$.
  By inversion on the simulation relation, the left side must be a dereference:
  \[
  \simrel{\gc}{\dynderef{\dynaddr{n_{\hat{\ell}}}{\ell}}}{\ccderefstar{(\cccast{\ccaddr{n}}{\coerc{\refco{\bm{c}}{\bm{d}}}{\bar{c}}})}{T}}{T_{\unk}}
  \]
  where $\vdash \bm{c} : T_{\unk} \Rightarrow S_{\hat{\ell}} , \vdash \bm{d} : S_{\hat{\ell}} \Rightarrow T_{\unk}$.
  By Lemma~\ref{lem:sim-ref-proxy}, $\ell \preccurlyeq |\bar{c}|$.
  We take a step on the left side using \textit{deref}. We need to relate:
  \[
  \simrel{\gc}{\dynprot{\ell}{V'}}{\ccprot{\high}{|\bar{c}|}{(\cccast{V}{\bm{d}})}{T_{\unk}}}{B}
  \]
  which is proved directly by applying Lemma~\ref{lem:leq-value-pc}, ${\leq}\textit{-prot}$,
  and then ${\leq}\textit{-cast}$.
  \item[Case~\textit{deref-cast}:] Analogous to \textit{deref$\star$-cast}.
  \item[Case~\textit{$\beta$-assign}:]
  By inversion on the simulation relation, the left side is also an assignment:
  \[
  \simrel{\ell^c}{\dynassign{\dynaddr{n_{\hat{\ell}}}{\ell'}}{?}{V'}}{\ccassign{(\ccaddr{n})}{V}{T}{\hat{\ell}}{\ell}}{\Unit_{\low}}
  \]
  where $\ell' \preccurlyeq \ell$. From the typing derivation we know $\ell^c \curlyvee \ell \preccurlyeq \hat{\ell}$,
  $\vdash \PC \Leftarrow \ell^c$. We know $\ell'' \preccurlyeq |\PC| = \ell^c$ where $\ell''$ is the PC on the left,
  thus $\ell'' \curlyvee \ell' \preccurlyeq \hat{\ell}$. We take one step on the left side using \textit{assign?-ok}.
  The \texttt{unit}s on both sides relate straightforwardly. We need to relate:
  \[
  \Sigma \vdash [\hat{\ell} \mapsto n \mapsto (V' \curlyvee \hat{\ell})] \mu' \leq [\hat{\ell} \mapsto n \mapsto V] \mu
  \]
  Given $\Sigma(\hat{\ell}, n) = T$, we need to show:
  \[
  \simrel{\low}{V' \curlyvee \hat{\ell}}{V}{T_{\hat{\ell}}}
  \]
  which can be proved similar to~\eqref{eqn:122} because $\vdash V \Leftarrow T_{\hat{\ell}}$.
  \item[Case~\textit{assign-cast}:]
  \[
  \cccast{V}{\bm{c}} \longrightarrow^{*} W
  \]
  where $\vdash \bm{c} : T_{\hat{\ell}_2} \Rightarrow S_{\hat{\ell}_1}$. By Lemma~\ref{lem:cast-leq},
  $\hat{\ell}_2 \preccurlyeq \hat{\ell}_1$. From the typing derivation, $\vdash \PC \Leftarrow \ell^c$ and
  $\ell^c \curlyvee \ell \preccurlyeq \hat{\ell}_2$. By inversion on the simulation relation,
  the left side is also an assignment:
  \[
  \simrel{\ell^c}{\dynassign{\dynaddr{n_{\hat{\ell}_1}}{\ell'}}{?}{V'}}{\ccassign{(\cccast{\ccaddr{n}}{\coerc{\refco{\bm{c}}{\bm{d}}}{\bar{c}}})}{V}{T}{\hat{\ell}_2}{\ell}}{\Unit_{\low}}
  \]
  We know $\ell' \preccurlyeq |\bar{c}| = \ell$. $\ell'' \preccurlyeq |\PC| = \ell^c$ where $\ell''$
  is the PC that the left side is reduced with. Thus we have
  $\ell'' \curlyvee \ell' \preccurlyeq \hat{\ell}_2 \preccurlyeq \hat{\ell}_1$.
  The check succeeds so we take one step on the left side by \textit{assign?-ok}.
  The \texttt{unit}s on both sides relate straightforwardly. We need to relate:
  \[
  \Sigma \vdash [\hat{\ell}_1 \mapsto n \mapsto (V' \curlyvee \hat{\ell}_1)] \mu' \leq [\hat{\ell}_1 \mapsto n \mapsto W] \mu
  \]
  Need to show:
  \[
  \simrel{\low}{V' \curlyvee \hat{\ell}_1}{W}{S_{\hat{\ell}_1}}
  \]
  which can be proved similar to~\eqref{eqn:122}.

  \item[Case~\textit{assign?-cast}:]
  \begin{align}
  \cccast{(\mathit{stamp!}\;\PC_1\;|\bar{c}|)}{\unk \Rightarrow^{\bl{p}} \hat{\ell}} &\longrightarrow^{*} \PC_2 \\
  \cccast{V}{\bm{c}} &\longrightarrow^{*} W
  \end{align}

  By inversion on the simulation relation, the left side is also an assignment:
  \[
  \simrel{\gc}{\dynassign{\dynaddr{n_{\hat{\ell}}}{\ell}}{?}{V'}}{\ccassignproj{\left(\cccast{\ccaddr{n}}{\coerc{\refco{\bm{c}}{\bm{d}}}{\bar{c}}}\right)}{V}{T}{g}{p}}{\Unit_{\low}}
  \]
  where $\vdash \bm{c} : T_g \Rightarrow S_{\hat{\ell}}, \vdash \bm{d} : S_{\hat{\ell}} \Rightarrow T_g$.
  By Lemma~\ref{lem:sim-ref-proxy}, $\ell \preccurlyeq |\bar{c}|$.
  Stamping models implicit flow (Lemma~\ref{lem:stamp-lexpr-implicit}),
  so $|\mathit{stamp}\;\PC_1\;|\bar{c}|| = |\PC_1| \curlyvee |\bar{c}|$;
  casting models explicit flow (Lemma~\ref{lem:cast-lexpr-explicit}), so
  $|\PC_1| \curlyvee |\bar{c}| \preccurlyeq |\PC_2|$.
  Reduction of label expressions preserves types, so $\vdash \PC_2 \Leftarrow \hat{\ell}$.
  Thus $|\PC_2| = \hat{\ell}$, $|\PC_1| \curlyvee |\bar{c}| \preccurlyeq \hat{\ell}$.
  We know $\ell' \preccurlyeq |\PC_1|$ where $\ell'$ is the PC the left side reduces with.
  Thus $\ell' \curlyvee \ell \preccurlyeq \hat{\ell}$; the check succeeds so we take one
  step on the left side by \textit{assign?-ok}.
  The \texttt{unit}s on both sides relate straightforwardly. We need to relate:
  \[
  \Sigma \vdash [\hat{\ell} \mapsto n \mapsto (V' \curlyvee \hat{\ell})] \mu' \leq [\hat{\ell} \mapsto n \mapsto W] \mu
  \]
  Need to show:
  \[
  \simrel{\low}{V' \curlyvee \hat{\ell}}{W}{S_{\hat{\ell}}}
  \]
  which again can be proved similar to~\eqref{eqn:122}.

  \end{description}
\end{proof}

\begin{lemma}[Multi-step simulation]
\label{lem:sim-mult}
Suppose $M$ is a well-typed \CC term $\Gamma ; \Sigma ; \gc ; |\PC| \vdash M \Leftarrow A$,
$PC$ is a well-typed label expression: $\vdash \PC \Leftarrow \gc$, and $\mu_1$ is a well-typed
heap: $\Sigma \vdash \mu_1$.
Suppose $\simrel{\gc}{M'}{M}{A}$, $\Sigma \vdash \mu_1' \leq \mu_1$ ,and $\ell \preccurlyeq |\PC|$.
If $\reducemult{M}{\mu_1}{\PC}{V}{\mu_2}$, then there exists $V',\mu_2'$ such that
$\reducemult{M'}{\mu_1'}{\ell}{V'}{\mu_2'}$
and $\simrel{\gc}{V'}{V}{A}$.
\end{lemma}
\begin{proof}
By induction on multi-step reduction $\reducemult{M}{\mu_1}{\PC}{V}{\mu_2}$.
\begin{description}
\item[Zero step] If $M$ is already a value, choose $V'$ to be $M'$.
By Lemma~\ref{lem:sim-val}, $M'$ is also a value.
The values are in sync because $\simrel{\gc}{M'}{M}{A}$.
\item[One or more steps]
\begin{align}
\reduce{M}{\mu_1}{\PC}{N}{\mu_3} \\
\reducemult{N}{\mu_3}{\PC}{V}{\mu_2}
\end{align}
Apply Lemma~\ref{lem:sim-leq} and then use the induction hypothesis.
\end{description}
\end{proof}

\begin{figure}[tbp]
  \raggedright
  \fbox{$\epsilon\;M\;A = M'$}
  {\small
  \begin{align*}
  \epsilon\;x\;{-} &= x \\
  \epsilon\;(\ccconst{k})\;(\iota_\ell) &= \dynconst{k}{\ell} \\
  \epsilon\;(\cclam{x}{N})\;((\Fun{A}{-}{B})_{\ell}) &= \dynlam{x}{\epsilon\;N\;B}{\ell} \\
  \epsilon\;(\ccaddr{n})\;(\Refer{(T_{\hat{\ell}})}_{\ell}) &= \dynaddr{n_{\hat{\ell}}}{\ell} \\
  \epsilon\;(\ccapp{M}{N}{A}{B}{\ell})\;{-} &= (\epsilon\;M\;(\Fun{A}{\unk}{B})_\ell)\;(\epsilon\;N\;A) \\
  \epsilon\;(\ccappstar{M}{N}{A}{T}{})\;{-} &= (\epsilon\;M\;(\Fun{A}{\unk}{T_{\unk}})_{\unk})\;(\epsilon\;N\;A) \\
  \epsilon\;(\ccif{L}{A}{\ell}{M}{N})\;{-} &= \dynif{(\epsilon\;L\;\Bool_\ell)}{(\epsilon\;M\;A)}{(\epsilon\;N\;A)} \\
  \epsilon\;(\ccifstar{L}{T}{M}{N})\;{-} &= \dynif{(\epsilon\;L\;\Bool_{\unk})}{(\epsilon\;M\;T_{\unk})}{(\epsilon\;N\;T_{\unk})} \\
  \epsilon\;(\ccref{\ell}{M})\;(\Refer{T_\ell})_{\low} &= \dynref{?}{\ell}{(\epsilon\;M\;T_\ell)} \\
  \epsilon\;(\ccrefproj{\ell}{M}{p})\;(\Refer{T_\ell})_{\low} &= \dynref{?}{\ell}{(\epsilon\;M\;T_\ell)} \\
  \epsilon\;(\ccderef{M}{A}{\ell})\;{-} &= \dynderef{(\epsilon\;M\;(\Refer{A})_\ell)} \\
  \epsilon\;(\ccderefstar{M}{T})\;{-} &= \dynderef{(\epsilon\;M\;(\Refer{T_{\unk}})_{\unk})} \\
  \epsilon\;(\ccassign{L}{M}{T}{\hat{\ell}}{\ell})\;{-} &= \dynassign{(\epsilon\;L\;(\Refer{T_{\hat{\ell}}})_\ell)}{?}{(\epsilon\;M\;T_{\hat{\ell}})} \\
  \epsilon\;(\ccassignproj{L}{M}{T}{\hat{g}}{p})\;{-} &= \dynassign{(\epsilon\;L\;(\Refer{T_{\hat{g}}})_{\unk})}{?}{(\epsilon\;M\;T_{\hat{g}})} \\
  \epsilon\;(\cccast{N}{\bm{c}})\;{-} &= \epsilon\;N\;A \quad \text{, where }\vdash c : A \Rightarrow B \\
  \epsilon\;(\ccprot{\PC}{\ell}{M}{A})\;{-} &= \dynprot{\ell}{(\epsilon\;M\;A)}
  \end{align*}}
  \caption{Erasure from \CC to \DynIFC}
  \label{fig:cast-erase}
\end{figure}

We then prove that \CC satisfies termination-insensitive noninterference. The
statement of termination-insensitive noninterference says that if we run a
program with different high-security inputs in two executions, then their
low-security output values should be the related (e.g the same boolean):

\begin{lemma}[Noninterference for \CC]
\label{lem:NI-CC}
If $M$ is well-typed:
$(x{:}\Bool_{\high}) ; \emptyset ; \low ; \low \vdash M \Leftarrow \Bool_{\low}$
and
{\normalfont
\begin{equation*}
\reducemult{M [ x := \ccconst{b_1}]}{\emptyset}{\low}{V_1}{\mu_1}
\quad\text{ and }\quad
\reducemult{M [ x := \ccconst{b_2}]}{\emptyset}{\low}{V_2}{\mu_2}
\end{equation*}}
then $V_1 = V_2$.
\end{lemma}
\begin{proof}
From the definition of $\epsilon$ (Figure~\ref{fig:cast-erase}), we know
$\simrel{\low}{\epsilon(M)[x:= \dynconst{b_i}{\high}]}{M [ x :=
    \ccconst{b_i}]}{\Bool_{\low}}$. By Lemma~\ref{lem:sim-mult}, there exists
$V_i', \mu_i'$ s.t $\reducemult{\epsilon(M)[x:=
    \dynconst{b_i}{\high}]}{\emptyset}{\low}{V_i'}{\mu_i'}$ and
$\simrel{\low}{V_i'}{V_i}{\Bool_{\low}}$. The simulation relation must be of
form $\simrel{\low}{\dynconst{a_i}{\low}}{\ccconst{a_i}}{\Bool_{\low}}$ where
$a_i$ is the output boolean. By Lemma~\ref{lem:mult-step-impl-big-step} and
Lemma~\ref{lem:NI-old} , $a_1 = a_2$, thus $V_1 = \ccconst{a_1} = \ccconst{a_2}
= V_2$.
\end{proof}

\section{Noninterference of \Surface}
\label{sec:NI}

The noninterference lemma of \Surface is a straightforward corollary of the
noninterference lemma of \CC and compilation preserves types:

\begin{lemma}
\label{lem:NI-Surface}
Suppose a \Surface term $M$ is well-typed:
\[
(x{:}\Bool_{\high}) ; \low \vdash M : \Bool_{\low}
\]
If for any boolean inputs $b_1,b_2$
{\normalfont
\begin{equation*}
\reducemult{(\compile{M}) [ x := \ccconst{b_1}]}{\emptyset}{\low}{V_1}{\mu_1}
\quad\text{ and }\quad
\reducemult{(\compile{M}) [ x := \ccconst{b_2}]}{\emptyset}{\low}{V_2}{\mu_2}
\end{equation*}}
then the resulting values $V_1 = V_2$.
\end{lemma}
\begin{proof}
By Theorem~\ref{thm:compile-pres} (Compilation preserves types),
$(x{:}\Bool_{\high}) ; \emptyset ; \low ; \low \vdash M' \Leftarrow \Bool_{\low}$.
By Lemma~\ref{lem:NI-CC} (Noninterference for \CC), $V_1 = V_2$.
\end{proof}

{\color{NavyBlue} %%% new text
Finally, we prove noninterference for \Surface as a direct corollary of
Lemma~\ref{lem:NI-Surface}. Noninterference says that for any high-security,
sensitive user input $b_1, b_2$, the low-security, publicly visible output will
always be the same:

\begin{theorem}[Noninterference for \Surface]
  \label{thm:NI-Surface}
  If $M$ is a \Surface program and $\mathit{eval}(M,b_1)=b_1'$ and $\mathit{eval}(M,b_2)=b_2'$,
  then $b_1' = b_2'$.
\end{theorem}
\begin{proof}
  By Definition~\ref{def:surface-program} (whole programs of \Surface),
  $(x{:}\Bool_{\high}) ; \low \vdash M : \Bool_{\low}$. By the definition of
  \textit{eval} and the canonical form of a value of $\Bool_{\low}$,
  $\reducemult{(\compile{M}) [ x := \ccconst{b_1}]}{\emptyset}{\low}{\ccconst{b_1'}}{\mu_1}$ and
  $\reducemult{(\compile{M}) [ x := \ccconst{b_2}]}{\emptyset}{\low}{\ccconst{b_2'}}{\mu_2}$.
  Applying Lemma~\ref{lem:NI-Surface}, $b_1'=b_2'$.
\end{proof}

} %%% end new text


%%%%%%%%%%%%%%%%
% Appendices
%%%%%%%%%%%%%%%%

%\input{appendix.tex}

%%%%%%%%%%%%%%%%
% References
%%%%%%%%%%%%%%%%
%\begin{singlespace}  % use single-line spacing for multi-line text within a single reference
%	\setlength\bibitemsep{\baselineskip}  %manually set separataion betwen items in bibliography to double space
%	\printbibliography[heading=bibintoc,title={References}]
%\end{singlespace}
\printbibliography[heading=bibintoc,title={References}]
%\addcontentsline{toc}{chapter}{References}  %add References section to Table of Contents

%%%%%%%%%%%%%%%%
% Vita
% Only for PhD students
% Masters students remove this line
%%%%%%%%%%%%%%%%
\chapter*{Vita}
\addtocontents{toc}{
 \unexpanded{\unexpanded{\renewcommand{\cftchapdotsep}{\cftnodots}}}
}
\addcontentsline{toc}{chapter}{Curriculum Vitae}

\doublespacing

\textbf{\Huge Tianyu Chen}


\section*{\sc Contact Information}
\vspace{.05in}
\begin{tabular}{@{}p{3in}p{3in}}
  3025B. Luddy Hall, Indiana University        & {\it E-mail:} chen512@iu.edu                          \\
  700 N. Woodlawn Ave.    & {\it Website:} \href{https://homes.luddy.indiana.edu/chen512/}{homes.luddy.indiana.edu/chen512}  \\
  Bloomington, IN 47408   &
\end{tabular}


\section*{\sc Education}

\begin{itemize}
\item {\em Ph.D.}, Computer Science, Indiana University \hfill {\bf August 2016 -- May 2025}
  \begin{itemize}
    \item Advisor: Professor Jeremy G. Siek
  \end{itemize}
\item {\em B.Eng.}, Tsinghua University \hfill {\bf August 2012 -- July 2016}
  \begin{itemize}
  \item Major: Computer Science and Technology
  \end{itemize}
\end{itemize}


\section*{\sc Experience}

{\bf Indiana University}, Bloomington, Indiana USA \\
{\em Instructor-of-Record (IOR)} \hfill {\bf August 2024 -- Present} \\
{\em Research Assistant and Associate Instructor} \hfill {\bf August 2016 -- July 2024} \\
{\bf CertiK}, Remote, New York USA \\
{\em Software Verification Intern on Coq and CompCert} \hfill {\bf May 2021 -- August 2021} \\
{\bf Columbia University}, New York City, New York USA \\
{\em Visiting Student on Transparent Paxos} \hfill {\bf August 2015 -- October 2015} \\


\section*{\sc Publications}

{\bf Quest Complete: the Holy Grail of Gradual Security} \\
Tianyu Chen, Jeremy G. Siek. \\
In Proceedings of the 45th ACM SIGPLAN Conference on Programming Language Design and
Implementation (\textbf{PLDI 2024}). \\
{\bf Parameterized Cast Calculi and Reusable Meta-theory for Gradually Typed Lambda Calculi } \\
Jeremy G. Siek, Tianyu Chen. \\
In Journal of Functional Programming (\textbf{JFP}), November 2021. \\
{\bf Mechanized Type Safety for Gradual Information Flow} \\
Tianyu Chen, Jeremy G. Siek. \\
In the 7th Workshop on Language-Theoretic Security (\textbf{LangSec 2021}). \\
{\bf Racing in Hyperspace: Closing Hyper-threading Side Channels on SGX with Contrived Data Races} \\
Guoxing Chen, Wenhao Wang, Tianyu Chen, Sanchuan Chen, Yinqian Zhang, XiaoFeng Wang, Ten-Hwang Lai, Dongdai Lin. \\
In Proceedings of 2018 IEEE Symposium on Security and Privacy (\textbf{Oakland 2018}). \\
{\bf Characterizing Smartwatch Usage in the Wild} \\
Xing Liu, Tianyu Chen, Feng Qian, Zhixiu Guo, Felix Xiaozhu Lin, Xiaofeng Wang, Kai Chen. \\
In Proceedings of the 15th Annual International Conference on Mobile Systems, Applications,
and Services (\textbf{MobiSys 2017}). \\
{\bf Paxos Made Transparent} \\
Heming Cui, Rui Gu, Cheng Liu, Tianyu Chen, Junfeng Yang. \\
In Proceedings of the 25th Symposium on Operating Systems Principles (\textbf{SOSP 2015}). \\


\section*{\sc Drafts and Posters}

{\bf Mechanized Noninterference for Gradual Security } \\
Tianyu Chen, Jeremy G. Siek. \\
Draft, November 2022. \\
{\bf Generic Blame-Subtyping Theorem in Agda Using Abstract Binding Trees } \\
Tianyu Chen. \\
In POPL 2022 Student Research Competition (\textbf{POPL SRC 2022}). \\

\section*{\sc Teaching}

{\bf System Programming With C and Unix} (CSCI-C291 and ENGR-E111) \\
{\em Instructor} \hfill {\bf Spring 2025, Fall 2024} \\
{\bf Data Structures} (CSCI-C343, 200 students) \\
{\em Lead Associate Instructor} \hfill {\bf Spring 2024} \\
{\bf Data Structures, Honors} (CSCI-H343) \\
{\em Associate Instructor} \hfill {\bf Fall 2023} \\
{\bf Secure Protocols} (CSCI-B433 and INFO-I433) \\
{\em Associate Instructor} \hfill {\bf Spring 2020, Spring 2019} \\
{\bf Malware: Threat and Defense} (CSCI-B546 and INFO-I521) \\
{\em Associate Instructor} \hfill {\bf Fall 2019} \\

\section*{\sc Awards}

{\bf Luddy PhD Instructor Award}
\hfill {\bf 2024 -- 2025} \\
{\bf Research Assistant of the Year} (Computer Science)
\hfill {\bf 2023 -- 2024} \\

\section*{\sc Service}

{\bf \href{https://wonks.github.io/}{PL Wonks}} \\
{\em Organizer, Video Chair} \hfill {\bf Fall 2023 -- Present} \\
{\bf \href{https://wonks.github.io/plrg/}{The PL Reading Group (PLRG)}} \\
{\em Organizer} \hfill {\bf Fall 2021 -- Spring 2023}

\section*{\sc Software}

{\bf \href{https://github.com/Gradual-Typing/LambdaIFCStar}{Mechanized Gradual Information-Flow Control}}
\hfill {\bf December 2023}
\begin{itemize}
\item Designed two gradual information-flow control programming languages,
  with and without type-guided classification.
\item Mechanized the proofs of type safety, noninterference, and the gradual
  guarantee in Agda.
\end{itemize}
{\bf \href{https://github.com/jsiek/gradual-typing-in-agda}{Gradual Typing in Agda}}
\hfill {\bf August 2020}
\begin{itemize}
\item Contributed to the mechanized compendium of the gradually-typed
  $\lambda$-calculus (GTLC) and a variety of cast calculi.
\end{itemize}
{\bf \href{https://github.com/Gradual-Typing/lambda-sec/tree/master/glio}{Mechanization of GLIO}}
\hfill {\bf May 2020}
\begin{itemize}
\item Implemented a definitional interpreter for GLIO, an experimental
  gradual security programming language proposed by researchers from Carnegie
  Mellon University.
\item Proved type safety for GLIO and mechanized the proof in Agda.
\end{itemize}


\section*{\sc Programming Skills}

{\bf Languages: } C, Java, Python, Racket, Agda, Coq \\
{\bf Tools: } Emacs, Shell scripting, Unix sysadmin, virtualization, Linux containers

\pagenumbering{gobble}


\end{document}
