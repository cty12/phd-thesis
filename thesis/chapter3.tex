\chapter{The Definition of the Cast Calculus \CC}
\label{ch:sem}

%\todo[inline]{Add sentences about why using coercions as the cast representation}

As we have seen in Section~\ref{sec:examples}, gradual information flows can be
modeled as casts. For example, the cast sequence $\high \Rightarrow \unk
\Rightarrow \low$ should be statically accepted but dynamically rejected, while
the sequence $\low \Rightarrow \unk \Rightarrow \high$ should be statically and
dynamically accepted, promoting the security of data to high. Such sequences of
casts can be arbitrarily long (for example, $\low \Rightarrow \high \Rightarrow
\unk \Rightarrow \unk \Rightarrow \low$), which motivates us to represent the
casts on security labels as \textit{coercions}.

In this chapter, I first present a coercion calculus for security labels in
Section~\ref{sec:coercion-calc-labels}. I show that coercion on labels can model
both explicit and implicit information flows. I then define a second coercion
calculus whose purpose is to cast a program value from one type to another type
in Section~\ref{sec:coercion-calc-values}. I use this second coercion calculus
as the representation of casts in the intermediate language \CC. With this cast
representation in hand, I am going to present the full definition of \CC in
Section~\ref{sec:cc}. Finally, I show that \Surface can be compiled into \CC in
Section~\ref{sec:compile}. As a result, the semantics of \Surface is given by
the semantics of \CC.

\section{A Coercion Calculus for Security Labels}
\label{sec:coercion-calc-labels}

In this section, I describe a coercion calculus for security labels. This
coercion calculus is an important stepping stone to our representation of casts
between security types. I first define the syntax, the type system, and the
semantics of the coercion calculus for security labels
(Section~\ref{sec:coercion-calc-def}). I then demonstrate that we can use
coercion composition to model explicit flows and use coercion stamping to model
implicit flows (Section~\ref{sec:cexpr-comp-stamp}). Finally, I define how these
coercions act on security labels by defining a language of label expressions
whose meaning is defined by a reduction relation (Section~\ref{sec:lexpr}).
Label expressions are used to model security checks that enforce the heap policy
in \CC.

\subsection{Syntax, Typing, and Semantics of the Coercion Calculus for Security Labels}
\label{sec:coercion-calc-def}

\begin{figure}[tbp]
\raggedright
  \[
  \begin{array}{rcll}
    \text{specific security labels} & \ell & \in & \{ \low , \high \} \\
    \text{security labels}  & g    & ::= & \unk \MID \ell \\
    \text{blame labels}         & \bl{p}, \bl{q}     &      & \\
    \text{security coercions}            & c, d     & ::=  & \id{g} \MID \up \MID \inj{\ell} \MID \proj{\ell}{p} \MID \bot^{\bl{p}} \\
    \text{coercion sequences} & \bar{c}, \bar{d} & ::=  & \id{g} \MID \err{g_1}{g_2}{p} \MID \bar{c} \seq c
  \end{array}
  \]
  \fbox{$\vdash c : g_1 \Rightarrow g_2$}
  {\small
  \begin{gather*}
    \inference{}{\vdash \id{g} : g \Rightarrow g}
    \quad
    \inference{}
              {\vdash \,\up\, : \low \Rightarrow \high}
    \quad
    \inference{}
              {\vdash \inj{\ell} : \ell \Rightarrow \unk}
    \\[1ex]
    \inference{}
              {\vdash \proj{\ell}{p} : \unk \Rightarrow \ell}
    \qquad
    \inference{}
              {\vdash \bot^{\bl{p}} : \high \Rightarrow \low}
  \end{gather*}}
  \fbox{$\vdash \bar{c} : g_1 \Rightarrow g_2$}
  {\small
  \begin{gather*}
    \inference{}{\vdash \id{g} : g \Rightarrow g}
    \quad
    \inference{\vdash \bar{c} : g_1 \Rightarrow g_2 & \vdash c : g_2 \Rightarrow g_3}
              {\vdash \bar{c} \seq c : g_1 \Rightarrow g_3}
    \quad
    \inference{}
              {\vdash \err{g_1}{g_2}{p} : g_1 \Rightarrow g_2}
  \end{gather*}}
  \fbox{$\mathbf{NF}\; \bar{c}$}
  {\small
  \begin{gather*}
  \inference{}{\mathbf{NF}\; \id{g}}
  \quad
  \inference{}{\mathbf{NF}\; \id{\unk} \seq \proj{\ell}{p}}
  \quad
  \inference{\mathbf{NF}\; \bar{c}}{\mathbf{NF}\; \bar{c}\seq\inj{\ell}}
  \quad
  \inference{\mathbf{NF}\; \bar{c}}{\mathbf{NF}\; \bar{c}\seq\up}
  \end{gather*}}
  \fbox{$c \mathrel{;} c \longrightarrow c$}
  {\small
  \begin{gather*}
  \textit{?-id}~
  \inference{}
            {\inj{\ell} \seq \proj{\ell}{p} \longrightarrow \id{\ell}}
  \quad
  \textit{?-}\uparrow~
  \inference{}
            {\inj{\low} \seq \proj{\high}{p} \longrightarrow \;\;\up}
  \quad
  \textit{?-}\bot~
  \inference{}
            {\inj{\high} \seq \proj{\low}{p} \longrightarrow \bot^{\bl{p}}}
  \end{gather*}}
  \fbox{$\bar{c} \longrightarrow \bar{d}$}
  {\small
  \begin{gather*}
  \textit{id}~
  \inference{\mathbf{NF}\; \bar{c}}
            {\bar{c} \seq \id{g} \longrightarrow \bar{c}}
  \qquad
  \bot~
  \inference{\mathbf{NF}\; \bar{c} & \vdash \bar{c} : g_1 \Rightarrow g_2}
            {\bar{c} \seq \bot^{\bl{p}} \longrightarrow \err{g_1}{\low}{p} }
  \qquad
  \xi\textit{-}\bot~
  \inference{\vdash c : g_2 \Rightarrow g_3}
    {\err{g_1}{g_2}{p} \seq c \longrightarrow \err{g_1}{g_3}{p}}
  \\[1ex]
  \xi_L~
  \inference{\bar{c} \longrightarrow \bar{d}}{\bar{c} \seq c \longrightarrow \bar{d} \seq c}
  \qquad
  \xi_R~
  \inference{\mathbf{NF}\; \bar{c} & c \seq d \longrightarrow c'}
            {\bar{c} \seq c \seq d \longrightarrow \bar{c}; c'}
  \end{gather*}}
  \caption{Syntax, typing, normal forms, and semantics of security coercions and coercion sequences}
  \label{fig:cexpr}
\end{figure}

In \Surface, the source security label of a coercion sequence comes from
literals, while the sink is whatever security level that the observer has: for
example, the \texttt{publish} function of Section~\ref{sec:examples} is of \low.
Coercions can be easily sequenced and composed. Checking information flow at
runtime is accomplished by reducing coercion sequences to their normal forms.

There are two noteworthy benefits of using coercions to represent IFC. First,
coercions can be used to represent NSU checking while satisfying the gradual
guarantee. In brief, whenever a memory location is written to, the current PC is
coerced to the security level of that location. We are going to formally
introduce label expressions as our representation for PC in
Section~\ref{sec:lexpr} and discuss NSU in detail in Section~\ref{sec:cc}.
Second, the coercion representation benefits mechanization because it enables
modular reasoning. The main simulation lemma (Lemma~\ref{lem:sim}) depends on
the simulation results of coercion sequences and label expressions, which are
stated as separate lemmas and reasoned independently in our Agda code.

The syntax and typing for security coercions and coercion sequences is defined
in Figure~\ref{fig:cexpr}. A security label is either $\low$, $\high$, or
statically unknown (\unk). There are five security coercions: identity (\id{g}),
subtype (\up), injection (\inj{\ell}), and projection (\proj{\ell}{p}), and
blame ($\boldsymbol{\bot}^{\bl{p}}$). Projection, which corresponds to the
notion of a runtime check, is the only one responsible for blame, so it carries
a blame label \bl{p}. A coercion sequence $\bar{c}$ starts with either success
\id{g} or failure (\err{g_1}{g_2}{p}). Each coercion has a source and target
type $g_1 \Rightarrow g_2$. The \id{g} casts the label $g$ to itself;
\up~promotes security from \low to \high; injection casts to \unk from a
specific label $\ell$ and projection does the opposite. Appending a single
coercion to a coercion sequence makes the target security label that of the
single coercion.

Information flow is enforced in the reduction semantics of security coercions,
shown in Figure~\ref{fig:cexpr}. Injection followed by projection to the same
label collapses to the identity (\textit{?-id}). Flowing from \low to \high is
allowed, so an injection from \low followed by a projection to \high collapses
into the \up{} coercion~($\textit{?-}\uparrow$). An information flow from \high
to \low is prohibited, so an injection to \high followed by a projection to \low
triggers an error that blames the projection ($\textit{?-}\bot$). The predicate
$\mathbf{NF}$ that specifies the normal forms of coercion sequences. The
reduction rules for coercion sequences are also defined in
Figure~\ref{fig:cexpr}. Appending \id{g} onto a coercion sequence reduces to the
that sequence (\textit{id}). The failure coercions annihilate the other
coercions in the sequence ($\bot$ and $\xi\textit{-}\bot$). We choose the
evaluation order in a coercion sequence to be from left to right ($\xi_L$ and
$\xi_R$), because that corresponds to the direction of information flow from
source to sink: in the example above, $\low \Rightarrow \high \Rightarrow \unk
\Rightarrow \unk \Rightarrow \low$, we validate that \low can flow to \high
before we check the flow from \high through \unk to \low.

\subsection{Monitoring Explicit and Implicit Flows}
\label{sec:cexpr-comp-stamp}

\begin{figure}[tbp]
\raggedright
  \fbox{$\bar{c}\mdoubleplus\bar{c} = \bar{c}$}
  {\small
\begin{align*}
\bar{c} \mdoubleplus \err{g_2}{g_3}{p} &= \err{g_1}{g_3}{p}  \quad\text{where $\vdash \bar{c} : g_1 \Rightarrow g_2$} \\
\bar{c} \mdoubleplus \id{g} &= \bar{c} \seq \id{g} \\
\bar{c}_1 \mdoubleplus (\bar{c}_2 \seq c) &= (\bar{c}_1 \mdoubleplus \bar{c}_2) \seq c
\end{align*}
  }

  \fbox{$\mathit{stamp} \; \bar{c} \; \ell = \bar{c}$}\hfill
  {\small
\begin{align*}
\mathit{stamp} \; \bar{c} \; \low &= \bar{c} \\
\mathit{stamp} \; \id{\low} \; \high &= \id{\low}\seq\up \\
\mathit{stamp} \; \id{\high} \; \high &= \id{\high} \\
\mathit{stamp} \; (\id{\low}\seq\inj{\low}) \; \high &= \id{\low}\seq\up\seq\inj{\high} \\
\mathit{stamp} \; (\id{\high}\seq\inj{\high}) \; \high &= \id{\high}\seq\inj{\high} \\
\mathit{stamp} \; (\id{\low}\seq\up\seq\inj{\high}) \; \high &= \id{\low}\seq\up\seq\inj{\high} \\
\mathit{stamp} \; (\id{\low}\seq\up) \; \high &= \id{\low}\seq\up
\end{align*}
}

  \fbox{$\mathit{stamp!} \; \bar{c} \; \ell = \bar{c}$}
  {\small
\begin{align*}
\mathit{stamp!} \; \bar{c} \; \low &=
  \begin{cases}
  \bar{c} & \text{~if} \vdash \bar{c} : \ell \Rightarrow \unk \\
  \bar{c}\seq\inj{\ell_2} & \text{~if} \vdash \bar{c} : \ell_1 \Rightarrow \ell_2
  \end{cases}
  \\
\mathit{stamp!} \; \id{\low} \; \high &= \id{\low}\seq\up\seq\inj{\high} \\
\mathit{stamp!} \; \id{\high} \; \high &= \id{\high}\seq\inj{\high} \\
\mathit{stamp!} \; (\id{\low}\seq\inj{\low}) \; \high &= \id{\low}\seq\up\seq\inj{\high} \\
\mathit{stamp!} \; (\id{\high}\seq\inj{\high}) \; \high &= \id{\high}\seq\inj{\high} \\
\mathit{stamp!} \; (\id{\low}\seq\up\seq\inj{\high}) \; \high &= \id{\low}\seq\up\seq\inj{\high} \\
\mathit{stamp!} \; (\id{\low}\seq\up) \; \high &= \id{\low}\seq\up\seq\inj{\high}
\end{align*}
}
  \caption{Composing and stamping coercions}
  \label{fig:comp-stampc}
\end{figure}

We model explicit flow using security coercions. We can compose two coercion
sequences ($\bar{c} \mdoubleplus \bar{d}$), where $\bar{c} : g_1 \Rightarrow
g_2$ and $\bar{d} : g_2 \Rightarrow g_3$, to form a flow from $g_1$ to $g_3$,
which is defined in Figure~\ref{fig:comp-stampc}.

The stamping operation captures the intuition of an implicit flow from the
security level $\ell'$ to a coercion sequence $\bar{c}$. We define the stamping
operation in Figure~\ref{fig:comp-stampc} as two functions,
$\mathit{stamp}(\bar{c},\ell)$ and $\mathit{stamp!}(\bar{c},\ell)$. Both
function require $\bar{c}$ to be in normal form and that its source label is not
\unk. The $\mathit{stamp!}$ operator promotes the security of the coercion
$\bar{c}$ to be at least $\ell$ and then injects the coercion if necessary,
while $\mathit{stamp}$ only promotes the security but does not inject. These
stamping operations satisfy the gradual guarantee, because when stamping on a
more precise coercion sequence and a less precise coercion sequence, stamping
preserves the precision relation (Section~\ref{sec:sim-cexpr}) between them
(Lemma~\ref{lem:cexpr-stamp-sim}). The stamping operations of coercion sequences
are used in the stamping operations of (1) label expressions, which are our
representation of PC and (2) values in the cast calculus. Those three types of
stamping together formalize the notion of implicit flow in \Surface.

\subsection{Security Label Expressions}
\label{sec:lexpr}

In this section we introduce security label expressions, which we use
to model the security level of the PC. Security label expressions are
crucial for implementing NSU checking in a way that satisfies the
gradual guarantee.

\begin{figure}[tbp]
\raggedright
  \[
  \begin{array}{rcll}
    \text{label expressions} & e & ::= & \ell \MID \blame{p} \MID \cccast{e}{\bar{c}} \\
  \end{array}
  \]
  \fbox{$\vdash e \Leftarrow g$}
  {\small
  \begin{gather*}
    {\vdash}\textit{l}~
    \inference{}{\vdash \ell \Leftarrow \ell}
    \quad
    {\vdash}\textit{lcast}~
    \inference{\vdash e \Leftarrow g_1 & \vdash \bar{c} : g_1 \Rightarrow g_2}
    {\vdash \cccast{e}{\bar{c}} \Leftarrow g_2}
    \\[1ex]
    {\vdash}\textit{lblame}~
    \inference{}
    {\vdash \blame{p} \Leftarrow g}
  \end{gather*}}
  \fbox{$\mathbf{Irreducible}\; \bar{c}$}
  {\small
    \begin{gather*}
      \inference{\mathbf{NF} \; \bar{c} & \vdash \bar{c} : g_1 \Rightarrow g_2 & g_1 \neq g_2}
                {\mathbf{Irreducible} \; \bar{c}}
  \end{gather*}}
  \fbox{$\mathbf{NF} \; e$}
  {\small
  \begin{gather*}
  \inference{}{\mathbf{NF} \; \ell}
  \qquad
  \inference{\mathbf{Irreducible} \; \bar{c}}{\mathbf{NF} \; (\cccast{\ell}{\bar{c}})}
  \end{gather*}}
  \fbox{$e_1 \longrightarrow e_2$}
  {\small
  \begin{gather*}
  \xi\textit{-l}~
  \inference{e_1 \longrightarrow e_2}
  {\cccast{e_1}{\bar{c}} \longrightarrow \cccast{e_2}{\bar{c}}}
  \qquad
  \xi\textit{-lblame}~
  \inference{}{\cccast{\blame{p}}{\bar{c}} \longrightarrow \blame{p}}
  \\[2ex]
  \beta\textit{-id}~
  \inference{}{\cccast{\ell}{\id{\ell}} \longrightarrow \ell}
  \qquad
  \textit{lcast}~
  \inference{\bar{c} \longrightarrow^{+} \bar{d} & \mathbf{NF} \; \bar{d}}
  {\cccast{\ell}{\bar{c}} \longrightarrow \cccast{\ell}{\bar{d}}}
  \\[2ex]
  \textit{lblame}~
  \inference{\bar{c} \longrightarrow^{*} \err{\ell}{g}{p}}
  {\cccast{\ell}{\bar{c}} \longrightarrow \blame{p}}
  \qquad
  \textit{lcomp}~
  \inference{\mathbf{Irreducible} \; \bar{c}}
  {\cccast{\cccast{\ell}{\bar{c}}}{\bar{d}} \longrightarrow \cccast{\ell}{\bar{c} \mdoubleplus \bar{d}}}
  \end{gather*}}
  \caption{Syntax, typing, normal forms, and semantics of label expressions}
  \label{fig:lexpr}
\end{figure}

A label expression is either (1) a specific security label, (2) blame (to
signify an error), or (3) a coercion applied to a label expression
(Figure~\ref{fig:lexpr}).
%
A label expression is in normal form ($\mathbf{NF}$) if it is either (1) a
specific security label or (2) an irreducible coercion applied to a specific
security label. (A coercion is irreducible if it is a non-identity coercion in
normal form). \PC ranges over label expressions in normal form.
%
The reduction relation for label expressions steps a label expression towards
its normal form. The idea is that given a label expression of the form
$\cccast{e}{\bar{d}}$, we first reduce $e$ to normal form and then apply the
coercion $\bar{d}$. For example, if $e$ reduces to a label wrapped in coercion
$\cccast{\ell}{\bar{c}}$, then the \textit{lcomp} rule says to reduce by
composing the two coercions, producing $\cccast{\ell}{\bar{c} \mdoubleplus
  \bar{d}}$. Furthermore, in a label expression of the form
$\cccast{e}{\bar{d}}$, the coercion $\bar{d}$ may also need to be reduced, which
is accomplished by the \textit{lcast} rule that refers to the reduction relation
for coercion sequences (Figure~\ref{fig:cexpr}). If the coercion reduces to an
identity, then the coercion application goes away ($\beta$\textit{-id}), whereas
if the coercion reduces to a failure, then the label expression reduces to blame
(\textit{lblame}).

\begin{figure}[tbp]
  \raggedright
  \fbox{$\mathit{stamp}\,e\,\ell = e$}
       {\small
         \begin{align*}
           \mathit{stamp} \; \ell \; \low &= \ell \\
           \mathit{stamp} \; \low \; \high &= \cccast{\low}{\id{\low}\seq\up} \\
           \mathit{stamp} \; \high \; \high &= \high \\
           \mathit{stamp} \; (\cccast{\ell}{\bar{c}}) \; \ell' &= \cccast{\ell}{\mathit{stamp} \; \bar{c} \; \ell'}
         \end{align*}
       }
  \fbox{$\mathit{stamp!}\,e\,\ell = e$}
       {\small
         \begin{align*}
           \mathit{stamp!} \; \ell \; \ell' &= \cccast{\ell}{\mathit{stamp!}\;\id{\ell}\;\ell'} \\
           \mathit{stamp!} \; (\cccast{\ell}{\bar{c}}) \; \ell' &= \cccast{\ell}{\mathit{stamp!} \; \bar{c} \; \ell'}
         \end{align*}
       }
  \fbox{$|e| = \ell$}
       {\small
         \begin{align*}
           | \ell | &= \ell \\
           | \cccast{\ell}{\bar{c}} | &= | \bar{c} |
         \end{align*}
       }
  \caption{Stamping and security level operators for security label expressions}
  \label{fig:stamp-lval}
\end{figure}

The stamping and security level operators for label expressions are defined in
Figure~\ref{fig:stamp-lval}. They both require their input to be in normal form,
which can be either (1) a specific security label $\ell$, or (2) a label wrapped
with an irreducible coercion sequence $\cccast{\ell}{\bar{c}}$. For (1),
stamping \low with \high results in \cccast{\low}{\up}, otherwise the label
expression remains unchanged; for (2), we directly stamp the coercion sequence
using \textit{stamp} for coercion sequences defined in
Figure~\ref{fig:comp-stampc}. The definition of \textit{stamp!} is analogous,
except that it turns to the \textit{stamp!} operator of coercion sequences. The
security level operator $|{-}|$ is defined such that (1) a specific security
label indicates the security level for itself and (2) the security of the
coercion sequence $|\bar{c}|$ records the security level for
$\cccast{\ell}{\bar{c}}$.

In Section~\ref{sec:cc}, I am going to describe how label expressions are used
to implement NSU checks, which enforce the heap policy for write operations.

\section{A Coercion Calculus on Values}
\label{sec:coercion-calc-values}

\begin{figure}[tbp]
\raggedright
  \[
  \begin{array}{rcll}
    \text{base types}               & \iota     & ::= & \Unit \MID \Bool \\
    \text{raw types}                & T, S      & ::= & \iota \MID \Fun{A}{\gc}{B} \mid \Refer{(T_g)} \\
    \text{types}                    & A, B      & ::= & T_g \\
    \text{raw coercions}            & c_r, d_r  & ::=  & \id{\iota} \MID \refco{\bm{c}}{\bm{d}} \MID \left( \funco{\bar{d}}{\bm{c}}{\bm{d}} \right) \\
    \text{coercions}                & \bm{c}, \bm{d} & ::= & \coerc{c_r}{\bar{c}}
  \end{array}
  \]
  \fbox{$\cccast{V}{\bm{c}} \longrightarrow M$}
  { \small
  \begin{gather*}
  \textit{cast}~
  \inference{\bar{c} \longrightarrow^{+} \bar{d} & \mathbf{NF} \; \bar{d}}
  {\cccast{V_r}{\coerc{c_r}{\bar{c}}} \longrightarrow \cccast{V_r}{\coerc{c_r}{\bar{d}}}}
  \qquad
  \textit{cast-blame}~
  \inference{\bar{c} \longrightarrow^{*} \err{g_1}{g_2}{p}}
  {\cccast{V_r}{\coerc{c_r}{\bar{c}}} \longrightarrow \blame{p}}
  \\
  \textit{cast-id}~
  \inference{}{\cccast{V_r}{\coerc{\id{\iota}}{\id{g}}} \longrightarrow V_r}
  \qquad
  \textit{cast-comp}~
  \inference{\mathbf{Irreducible} \; \bm{c}}
  {\cccast{\cccast{V_r}{\bm{c}}}{\bm{d}} \longrightarrow \cccast{V_r}{\bm{c} \mdoubleplus \bm{d}}}
  \end{gather*}}
  \caption{Syntax and semantics of coercions on values.}
  \label{fig:coercions}
\end{figure}

In this section, we define a second coercion calculus whose purpose is to cast a
program value from one type to another type. We use this coercion calculus as
the representation of casts in the intermediate language \CC. These coercions on
values make use of the coercions on security labels that we defined in
Section~\ref{sec:coercion-calc-labels}, because the types in \CC are annotated
with security labels, as is usual for a static and gradually-typed IFC
languages.

We begin with the definition of types in Figure~\ref{fig:coercions}, which is
standard for gradual security type systems: each type has a security label
ascription on it, which is either a specific label $\ell$ or $\unk$. Function
types have one extra label \gc, which is a static approximation of the security
level of the PC while executing the function body. In a reference type
$(\Refer{T_{\hat{g}}})_g$, the label $\hat{g}$ of the referenced type also
doubles as the security level of the memory cell.

\begin{figure}[tbp]
\raggedright
  \fbox{$\bm{c}\mdoubleplus \bm{c} = \bm{c}$}
  {\small
\begin{align*}
\left( \coerc{\id{\iota}}{\bar{c}} \right) \mdoubleplus \left( \coerc{\id{\iota}}{\bar{d}} \right) &= \left( \coerc{\id{\iota}}{\bar{c} \mdoubleplus \bar{d}} \right) \\
\left( \coerc{\refco{\bm{c_1}}{\bm{c_2}}}{\bar{c}} \right) \mdoubleplus \left( \coerc{\refco{\bm{d_1}}{\bm{d_2}}}{\bar{d}} \right) &= \left( \coerc{\refco{(\bm{d_1} \mdoubleplus \bm{c_1})}{(\bm{c_2} \mdoubleplus \bm{d_2})}}{\bar{c} \mdoubleplus \bar{d}} \right) \\
\left( \coerc{\funco{\bar{c}_1}{\bm{c_1}}{\bm{c_2}}}{\bar{c}_2} \right) \mdoubleplus \left( \coerc{\funco{\bar{d}_1}{\bm{d_1}}{\bm{d_2}}}{\bar{d}_2} \right) &= \left( \coerc{\funco{\bar{d}_1 \mdoubleplus \bar{c}_1}{(\bm{d_1} \mdoubleplus \bm{c_1})}{(\bm{c_2} \mdoubleplus \bm{d_2})}}{\bar{c}_2 \mdoubleplus \bar{d}_2} \right)
\end{align*}
  }
  \fbox{$\mathbf{Irreducible} \; \bm{c}$}
  {\small
  \begin{gather*}
    \inference{\mathbf{NF} \; \bar{c} & \ell \neq g}
    {\mathbf{Irreducible} \; \left( \coerc{\id{\iota}}{\bar{c}} \right)}
    ~{\vdash \bar{c} : \ell \Rightarrow g}
    \\[1ex]
    \inference{\mathbf{NF} \; \bar{c}}
    {\mathbf{Irreducible} \; \left( \coerc{\refco{\bm{c}}{\bm{d}}}{\bar{c}} \right)}
    \quad
    \inference{\mathbf{NF} \; \bar{c}}
    {\mathbf{Irreducible} \; \left( \coerc{\funco{\bar{d}}{\bm{c}}{\bm{d}}}{\bar{c}} \right)}
  \end{gather*}}
  \caption{Composition of coercions on values.}
  \label{fig:coercion-composition}
\end{figure}

The syntax and semantics for coercions on values is defined in
Figure~\ref{fig:coercions}. Each coercion $\bm{c}$ consists of a raw coercion
$c_r$ that casts the type of the value and the label coercion $\bar{c}$ that
casts the security label of the value. There are three kinds of raw coercions:
identity coercions $\id{g}$, coercions between reference types
$(\refco{\bm{c}}{\bm{d}})$, and coercions between function types $\left(
\funco{\bar{d}}{\bm{c}}{\bm{d}} \right)$. In the coercion on functions, the
$\bar{d}$ casts the PC of the function.
%
The syntax for values is not defined until the next section, so here we remark
that $V$ ranges over values, which can either be a raw value $V_r$ (constant,
address, or $\lambda$) or an irreducible coercion applied to a raw value:
\cccast{V_r}{\bm{c}}, where there is no $M$ such that $\cccast{V_r}{\bm{c}}
\longrightarrow M$. The definition of irreducible coercion is in
Figure~\ref{fig:coercion-composition}.

The reduction rules in Figure~\ref{fig:coercions} apply a coercion to a value,
yielding a value or triggering blame. The \textit{cast} rule normalizes the
coercion $\bar{c}$ on the security label. If it normalizes to a failure
coercion, the rule \textit{cast-blame} triggers blame. We reduce identity
coercions using rule \textit{cast-id}. Finally, if the value is wrapped with an
irreducible coercion, we compose the coercion with the coercion being applied
(rule \textit{cast-comp}). The composition operator ${-}\mdoubleplus{-}$ is also
defined in Figure~\ref{fig:coercion-composition}; the intuition of the
composition operator is that $\cccast{\cccast{V_r}{\bm{c}}}{\bm{d}}$ must be
contextual equivalent to $\cccast{V_r}{\bm{c} \mdoubleplus \bm{d}}$. There are
no reduction rules specific to reference coercions $\refco{\bm{c}}{\bm{d}}$ or
function coercions $\left( \funco{\bar{d}}{\bm{c}}{\bm{d}} \right)$ because they
are irreducible coercions that wrap a value. Their action occurs when the value
is used in an elimination form such as in a function call or a read or write to
the reference, which we explain in the next section.

\section{The Cast Calculus \CC: An Intermediate Language For Gradual IFC}
\label{sec:cc}

In this section, we define the cast calculus \CC by presenting its syntax and
type system in Section~\ref{sec:cc-typing} and its operational semantics in
Section~\ref{sec:cc-semantics}.

\subsection{Syntax and Typing of \CC}
\label{sec:cc-typing}

\begin{figure}[tbp]
\raggedright
  {\small \[
  \begin{array}{rcll}
    \text{variables}            & x, y, z   &      & \\
    \text{constants}            & k         & \in  & \{ \unit , \true , \false \} \\
    \text{terms}  & L , M , N & ::=  & x \MID \ccconst{k} \MID \ccaddr{n} \MID \cclam{x}{N} \MID
                                       \ccapp{L}{M}{A}{B}{\ell} \MID \ccappstar{L}{M}{A}{T} \\
                           &  & \MID & \ccif{L}{A}{\ell}{M}{N} \MID \ccifstar{L}{T}{M}{N} \MID
                                       \cclet{x}{M}{A}{N} \\
                           &  & \MID & \ccref{\ell}{M} \MID \ccrefproj{\ell}{M}{p} \MID \ccderef{M}{A}{\ell} \MID \ccderefstar{M}{T} \\
                           &  & \MID & \ccassign{L}{M}{T}{\hat{\ell}}{\ell} \MID \ccassignproj{L}{M}{T}{\hat{g}}{p} \\
                           &   & \MID & \ccprot{\PC}{\ell}{M}{A} \MID \cccast{M}{\bm{c}} \MID \blame{p} \\
    \text{raw values} & V_r, W_r & ::=  &  \ccconst{k} \MID \ccaddr{n} \MID \cclam{x}{N} \\
    \text{values}     & V, W     & ::=  & V_r \MID \cccast{V_r}{\bm{c}} \text{, where } \mathbf{Irreducible}\;\bm{c}
  \end{array}
  \]}
  \caption{Syntax of the cast calculus \CC}
  \label{fig:cc-syntax}
\end{figure}

As usual, the cast calculus \CC is a statically-typed language that includes an
explicit term for casts, written \cccast{M}{\bm{c}}, where $M$ is a term and
$\bm{c}$ is a coercion to be applied to the value of $M$. Furthermore, many of
the operators in \CC have two variants, a ``static'' one for when the pertinent
security label is statically known and the ``dynamic'' one for when the security
label is statically unknown. The operational semantics of the ``dynamic''
variants involve runtime checking. The syntax and typing rules for \CC are shown
in Figure~\ref{fig:cc-typing-full} and described in the following paragraphs.

A value is a raw value (constant, address or $\lambda$) or an
irreducible coercion applied to a raw value.

\begin{figure}[tbp]
\raggedright
  \fbox{$\Gamma ; \Sigma ; g ; \ell \vdash M \Leftarrow A$}
  {\small
  \begin{gather*}
    {\vdash}\mathit{var} ~
    \inference{\Gamma \ni x : A}
              {\Gamma; \Sigma; g; \ell \vdash x \Leftarrow A}
    \qquad
    {\vdash}\mathit{const} ~
    \inference{k : \iota}
              {\Gamma; \Sigma; g; \ell \vdash \ccconst{k} \Leftarrow \iota_\ell}
    \\[1ex]
    {\vdash}\mathit{addr} ~
    \inference{\Sigma(\hat{\ell}, n) = T}
              {\Gamma; \Sigma; g; \ell' \vdash \ccaddr{n} \Leftarrow (\Refer{T_{\hat{\ell}}})_\ell}
    \\[1ex]
    {\vdash}\mathit{lam} ~
    \inference{\forall \ell'' . (\Gamma , x{:}A); \Sigma; g; \ell'' \vdash N \Leftarrow B}
              {\Gamma; \Sigma; g'; \ell' \vdash \cclam{x}{N} \Leftarrow (\Fun{A}{g}{B})_\ell}
    \quad
    {\vdash}\mathit{let} ~
    \inference{\Gamma; \Sigma; g; \ell \vdash M \Leftarrow A \\
               \forall \ell' . (\Gamma, x{:}A); \Sigma; g; \ell' \vdash N \Leftarrow B}
              {\Gamma; \Sigma; g; \ell \vdash \cclet{x}{M}{A}{N} \Leftarrow B}
    \\[1ex]
    {\vdash}\mathit{app} ~
    \inference{\Gamma; \Sigma; \ell'; \ell'' \vdash L \Leftarrow (\Fun{A}{\ell' \curlyvee \ell}{B})_{\ell} \\
               \Gamma; \Sigma; \ell'; \ell'' \vdash M \Leftarrow A \\
               C = \mathit{stamp} \; B \; \ell}
              {\Gamma; \Sigma; \ell'; \ell'' \vdash \ccapp{L}{M}{A}{B}{\ell} \Leftarrow C}
    \quad
    {\vdash}\mathit{app}{\star}~
    \inference{\Gamma; \Sigma; g; \ell \vdash L \Leftarrow (\Fun{A}{\unk}{(T_{\unk})})_{\unk} \\
               \Gamma; \Sigma; g; \ell \vdash M \Leftarrow A \\
               %C = \mathit{stamp} \; B \; \unk
               }
              {\Gamma; \Sigma; g; \ell \vdash \ccappstar{L}{M}{A}{T} \Leftarrow T_{\unk}}
    \\[1ex]
    {\vdash}\mathit{if} ~
    \inference{\Gamma; \Sigma; \ell'; \ell'' \vdash L \Leftarrow \Bool_{\ell} \\
               \forall \ell_1 . \Gamma; \Sigma; \ell' \curlyvee \ell; \ell_1 \vdash M \Leftarrow A \\
               \forall \ell_2 . \Gamma; \Sigma; \ell' \curlyvee \ell; \ell_2 \vdash N \Leftarrow A \\
               B = \mathit{stamp} \; A \; \ell}
              {\Gamma; \Sigma; \ell'; \ell'' \vdash \ccif{L}{A}{\ell}{M}{N} \Leftarrow B}
    \qquad
    {\vdash}\mathit{if}{\star}~
    \inference{\Gamma; \Sigma; g; \ell \vdash L \Leftarrow \Bool_{\unk} \\
               \forall \ell_1 . \Gamma; \Sigma; \unk; \ell_1 \vdash M \Leftarrow T_{\unk} \\
               \forall \ell_2 . \Gamma; \Sigma; \unk; \ell_2 \vdash N \Leftarrow T_{\unk} \\
               %B = \mathit{stamp} \; A \; \unk
               }
              {\Gamma; \Sigma; g; \ell \vdash \ccifstar{L}{T}{M}{N} \Leftarrow T_{\unk}}
    \\[1ex]
    {\vdash}\mathit{ref} ~
    \inference{\Gamma; \Sigma; \ell'; \ell'' \vdash M \Leftarrow T_\ell \\
               \highlightblue{\ell' \preccurlyeq \ell}}
              {\Gamma; \Sigma; \ell'; \ell'' \vdash \ccref{\ell}{M} \Leftarrow (\Refer{T_\ell})_{\low}}
    \qquad
    {\vdash}\mathit{ref?} ~
    \inference{\Gamma; \Sigma; \unk; \ell' \vdash M \Leftarrow T_\ell}
              {\Gamma; \Sigma; \unk; \ell' \vdash \ccrefproj{\ell}{M}{p} \Leftarrow (\Refer{T_\ell})_{\low}}
    \\[1ex]
    {\vdash}\mathit{deref} ~
    \inference{\Gamma; \Sigma; g; \ell' \vdash M \Leftarrow (\Refer{A})_\ell \\
               B = \mathit{stamp} \; A \; \ell}
              {\Gamma; \Sigma; g; \ell' \vdash \ccderef{M}{A}{\ell} \Leftarrow B}
    \qquad
    {\vdash}\mathit{deref}{\star}~
    \inference{\Gamma; \Sigma; g; \ell \vdash M \Leftarrow (\Refer{(T_{\unk})})_{\unk} \\
               %B = \mathit{stamp} \; A \; \unk
               }
              {\Gamma; \Sigma; g; \ell \vdash \ccderefstar{M}{T} \Leftarrow T_{\unk}}
    \\[1ex]
    {\vdash}\mathit{assign} ~
    \inference{\Gamma; \Sigma; \ell'; \ell'' \vdash L \Leftarrow (\Refer{T_{\hat{\ell}}})_\ell &
               \Gamma; \Sigma; \ell'; \ell'' \vdash M \Leftarrow T_{\hat{\ell}} &
               \highlightblue{\ell' \curlyvee \ell \preccurlyeq \hat{\ell}} }
              {\Gamma; \Sigma; \ell'; \ell'' \vdash \ccassign{L}{M}{T}{\hat{\ell}}{\ell} \Leftarrow \Unit_{\low}}
    \\[1ex]
    {\vdash}\mathit{assign?} ~
    \inference{\Gamma; \Sigma; g; \ell \vdash L \Leftarrow (\Refer{T_{\hat{g}}})_{\unk} &
               \Gamma; \Sigma; g; \ell \vdash M \Leftarrow T_{\hat{g}}}
              {\Gamma; \Sigma; g; \ell \vdash \ccassignproj{L}{M}{T}{\hat{g}}{p} \Leftarrow \Unit_{\low}}
    \\[2ex]
    {\vdash}\mathit{prot} ~
    \inference{\Gamma; \Sigma; g'; | \PC | \vdash M \Leftarrow A &
               \vdash \PC \Leftarrow g' &
               \ell' \curlyvee \ell \preccurlyeq | \PC | &
               B = \mathit{stamp} \; A \; \ell}
              {\Gamma; \Sigma; g; \ell' \vdash \ccprot{\PC}{\ell}{M}{A} \Leftarrow B}
    \\[2ex]
    {\vdash}\mathit{cast} ~
    \inference{\Gamma; \Sigma; g; \ell \vdash M \Leftarrow A & \vdash \bm{c} : A \Rightarrow B}
              {\Gamma; \Sigma; g; \ell \vdash \cccast{M}{\bm{c}} \Leftarrow B}
    \qquad
    {\vdash}\mathit{blame} ~
    \inference{}
              {\Gamma; \Sigma; g; \ell \vdash \blame{p} \Leftarrow A}
  \end{gather*}}
  \caption{Typing rules of the cast calculus \CC. The side conditions that enforce the
  heap policy statically during memory write operations are \highlightblue{\text{highlighted}}}
  \label{fig:cc-typing-full}
\end{figure}

The typing rules are syntax-directed.  The typing judgment is of the
form $\Gamma ; \Sigma ; g ; \ell \vdash M \Leftarrow A$, which says
that we are type-checking \CC term $M$ against the expected type $A$,
where $\Gamma$ is the typing context, $\Sigma$ is the heap typing
context, $g$ is the security label that PC is typed at, and $\ell$ is
the security level of PC.  Both $\Sigma$ and the security level $\ell$
play a role during runtime.  The security level of the PC is constrained in
rule ${\vdash}\textit{prot}$, which is for the protection term that
arises during reduction (we are going to discuss this rule momentarily).
In the premises for sub-terms that do not immediately
reduce, such as the body of a $\lambda$ and the branches of an
\texttt{if}, we universally quantify the security level (as in
$\forall \ell$), which helps us prove that
compilation preserves types (Theorem~\ref{thm:compile-pres}).  The heap
context $\Sigma$ is mostly standard: looking up
$\Sigma(\hat{\ell}, n)$, where $n$ is the index in part of the heap
with security $\hat{\ell}$, gives us a raw type. Each memory cell is
associated with a specific security label $\hat{\ell}$, which is
specified by the programmer when that cell is
created.

As the typing rules always stay in checking mode, constants,
addresses, and $\lambda$s do not carry any label. The security of
these raw values is in their types: for example,
$\ccaddr{n} \Leftarrow (\Refer{T_{\hat{\ell}}})_\ell$ says that the
security of the address $n$ itself is $\ell$ and it points to a memory
cell labeled $\hat{\ell}$.

The typing rules for the static variants are similar to the typing
rules in a static security type system.  For example, in the static
function application rule ${\vdash}\textit{app}$, both top-level
labels on the function type as well as the security label that the
current PC is typed at are static and the rule mirrors one in a static
system.  On the other hand, in the dynamic version of application rule
${\vdash}\textit{app}\star$, both top-level labels as well as the label
of the co-domain type are \unk and PC is allowed to be typed at \unk,
indicating the presence of injections.
As another example, in the static version of memory assignment rule
${\vdash}\textit{assign}$, all labels to perform the heap policy
check, including the security of the memory address itself ($\ell$),
the security of the memory cell that the address references
($\hat{\ell}$), and the security of PC ($\ell'$) are known statically
and satisfy $\highlightblue{\ell' \curlyvee \ell \preccurlyeq \hat{\ell}}$.
At runtime, this static assignment
can happen directly without any runtime overhead (as is shown in the
example of Section~\ref{sec:example2}). Its dynamic counterpart rule
${\vdash}\textit{assign?}$ does not maintain these static security
invariants and thus requires runtime NSU checking, which is
implemented as a projection on PC.

As we shall see in the reduction rules, the semantics of the
protection term \texttt{prot} performs two things: (1) \texttt{prot} promotes
the security of the value reduced from its body by level $\ell$
(2) it uses a new PC to reduce its body.
However, the new PC cannot be any label expression. It has to be one with
higher security than both the current PC and the security level $\ell$.
We capture this invariant in side condition $\ell' \curlyvee \ell \preccurlyeq |\PC|$
in rule ${\vdash}\textit{prot}$, where $\ell'$ is the security of the current PC
and $|\PC|$ is the security for the new PC.
The invariant is used in the proof of noninterference.

\subsection{Operational Semantics for \CC}
\label{sec:cc-semantics}

\begin{figure}[tbp]
\raggedright
  \fbox{\reduce{M}{\mu}{\PC}{N}{\mu'}}
  {\footnotesize
  \begin{gather*}
  \textit{prot-ctx}
  \inference
  {\reduce{M}{\mu}{\PC'}{M'}{\mu'}}
  {\reduce{\ccprot{\PC'}{\ell}{M}{A}}{\mu}{\PC}{\ccprot{\PC'}{\ell}{M'}{A}}{\mu'}}
  \;
  \textit{prot-val}
  \inference{}{\reduce{\ccprot{\PC'}{\ell}{V}{A}}{\mu}{\PC}{\textit{stamp-val}\,V A\, \ell}{\mu}}
  \\[1ex]
  \textit{cast}~
  \inference{\cccast{V}{\bm{c}} \longrightarrow M}{\reduce{\cccast{V}{\bm{c}}}{\mu}{\PC}{M}{\mu}}
  \quad
  \beta~
  \inference{}{\reduce{\ccapp{(\cclam{x}{N})}{V}{A}{B}{\ell}}{\mu}{\PC}{\ccprot{(\mathit{stamp}\;\PC\;\ell)}{\ell}{(N [x := V])}{B}}{\mu}}
  \\[1ex]
  \textit{app-cast}~
  \inference{\mathbf{NF}\;\bar{c} & \cccast{(\textit{stamp}\;\PC\;\ell)}{\bar{d}} \longrightarrow^{*} \PC' & \cccast{V}{\bm{c}} \longrightarrow^{*} W }
  {\reduce{\ccapp{\left(\cccast{\cclam{x}{N}}{\coerc{\funco{\bar{d}}{\bm{c}}{\bm{d}}}{\bar{c}}}\right)}{V}{C}{D}{\ell}}{\mu}{\PC}{\ccprot{\PC'}{\ell}{(\cccast{(N [x := W])}{\bm{d}})}{D}}{\mu}}
  \\[1ex]
  \textit{app}{\star}\textit{-cast}~
  \inference{\mathbf{NF}\;\bar{c} & \cccast{(\textit{stamp!}\;\PC\;|\bar{c}|)}{\bar{d}} \longrightarrow^{*} \PC' & \cccast{V}{\bm{c}} \longrightarrow^{*} W }
  {\reduce{\ccappstar{\left(\cccast{\cclam{x}{N}}{\coerc{\funco{\bar{d}}{\bm{c}}{\bm{d}}}{\bar{c}}}\right)}{V}{C}{T}}{\mu}{\PC}{\ccprot{\PC'}{|\bar{c}|}{(\cccast{(N [x := W])}{\bm{d}})}{(T_{\unk})}}{\mu}}
  \\[1ex]
  \textit{if-true}~
  \inference{}{\reduce{\ccif{\ccconst{\true}}{A}{\ell}{M}{N}}{\mu}{\PC}{\ccprot{(\mathit{stamp}\;\PC\;\ell)}{\ell}{M}{A}}{\mu}}
  \\[1ex]
  \textit{if}{\star}\textit{-true-cast}~
  \inference{\mathbf{NF}\;\bar{c}}{\reduce{\ccifstar{\left(\cccast{\ccconst{\true}}{\coerc{\id{\Bool}}{\bar{c}}}\right)}{T}{M}{N}}{\mu}{\PC}{\ccprot{(\mathit{stamp!}\;\PC\;|\bar{c}|)}{|\bar{c}|}{M}{(T_{\unk})}}{\mu}}
  \\[1ex]
  \textit{ref}~
  \inference{n \; \mathbf{FreshIn} \; \mu(\ell)}
  {\reduce{\ccref{\ell}{V}}{\mu}{\PC}{\ccaddr{n}}{(\mu , \ell \mapsto n \mapsto V)}}
  \quad
  \textit{ref?}~
  \inference{n \; \mathbf{FreshIn} \; \mu(\ell) & \highlightred{\cccast{\PC}{\unk \Rightarrow^{\bl{p}} \ell} \longrightarrow^{*} \PC'} }
  {\reduce{\ccrefproj{\ell}{V}{p}}{\mu}{\PC}{\ccaddr{n}}{(\mu , \ell \mapsto n \mapsto V)}}
  \\[1ex]
  \textit{ref?-blame}
  \inference{\cccast{\PC}{\unk \Rightarrow^{\bl{p}} \ell} \longrightarrow^{*} \blame{q}}
  {\reduce{\ccrefproj{\ell}{V}{p}}{\mu}{\PC}{\blame{q}}{\mu}}
  \;
  \textit{assign}
  \inference{}{\reduce{\ccassign{(\ccaddr{n})}{V}{T}{\hat{\ell}}{\ell}}{\mu}{\PC}{\ccconst{\unit}}{[\hat{\ell} {\mapsto} n {\mapsto} V] \, \mu}}
  \\[1ex]
  \textit{assign?-cast}~
  \inference{\mathbf{NF}\;\bar{c} & \vdash \bm{c} : T_g \Rightarrow S_{\hat{\ell}} & \vdash \bm{d} : S_{\hat{\ell}} \Rightarrow T_g & \highlightred{\cccast{(\mathit{stamp!}\;\PC\;|\bar{c}|)}{\unk \Rightarrow^{\bl{p}} \hat{\ell}} \longrightarrow^{*} \PC'} & \cccast{V}{\bm{c}} \longrightarrow^{*} W}
  {\reduce{\ccassignproj{\left(\cccast{\ccaddr{n}}{\coerc{\refco{\bm{c}}{\bm{d}}}{\bar{c}}}\right)}{V}{T}{g}{p}}{\mu}{\PC}{\ccconst{\unit}}{[\hat{\ell} \mapsto n \mapsto W] \; \mu}}
  \\[1ex]
  \textit{deref}~
  \inference{\mu(\hat{\ell}, n) = V}
  {\reduce{\ccderef{(\ccaddr{n})}{T_{\hat{\ell}}}{\ell}}{\mu}{\PC}{\ccprot{\_}{\ell}{V}{T_{\hat{\ell}}}}{\mu}}
  \\[1ex]
  \textit{deref}{\star}\textit{-cast}~
  \inference{\mathbf{NF}\;\bar{c} & \vdash \bm{c} : S_{\unk} \Rightarrow T_\ell & \vdash \bm{d} : T_\ell \Rightarrow S_{\unk} & \mu(\ell, n) = V}
  {\reduce{\ccderefstar{\left(\cccast{\ccaddr{n}}{\coerc{\refco{\bm{c}}{\bm{d}}}{\bar{c}}}\right)}{S}}{\mu}{\PC}{\ccprot{\_}{|\bar{c}|}{(\cccast{V}{\bm{d}})}{(S_{\unk})}}{\mu}}
  \end{gather*}}
  \caption{Selected semantics rules for \CC. NSU checks are represented using label expressions (\highlightred{\text{highlighted}})}
  \label{fig:reduction}
\end{figure}

We show the interesting rules of the operational semantics of \CC in
Figure~\ref{fig:reduction} (full version in Figure~\ref{fig:reduction-full-1}
and~\ref{fig:reduction-full-2} of the Appendix).
The reduction relation takes the form $\reduce{M}{\mu}{\PC}{N}{\mu'}$,
which reduces the configuration of term $M$ and heap $\mu$ under the
label expression \PC to another configuration $N$ and $\mu'$. The heap is
a map $(\ell, n) \mapsto V$, where a cell is indexed by its security
level $\ell$ and by index $n$ among the cells of $\ell$.
The predicate $(n\;\mathbf{FreshIn}\;\mu(\ell))$ means that the index $n$ is fresh
(not already in use) among all cells with security $\ell$; when
performing a lookup, $\mu(\ell, n) = V$ retrieves the value $V$ at
index $n$ whose security level is $\ell$.

\textbf{\textit{Protection terms. }}
Following standard approaches to IFC, a protection
term \ccprot{\PC'}{\ell}{M}{A} has two functionalities: (1) it ensures
that the reduction inside $M$ does not leak information through heap
write operations (2) it promotes the security level of the computation
result of $M$ to at least level $\ell$. The first functionality is
achieved by switching to \PC' from the current $\PC$ when reducing the
body $M$ (rule \textit{prot-ctx}). Recall Section~\ref{sec:cc-typing},
the typing of \texttt{prot} makes sure that \PC' has the correct
security that is at least as secure as both $\PC$ and $\ell$.
The second functionality is achieved by stamping the value produced by
the body of \texttt{prot} (rule \textit{prot-val}).  The stamping of
values in \CC (Figure~\ref{fig:stamp-val} in the Appendix) is analogous to
stamping of label expressions and turns to the stamping operation for
coercions on labels (Figure~\ref{fig:comp-stampc}) in a similar
way. Again there are two cases, because the value is either (1) a raw value or
(2) a coercion-wrapped value. Suppose the value is a raw value, if its type has \low security
and is stamped with \high, the value becomes wrapped with a subtype coercion;
otherwise the value stays unchanged. Otherwise, if the value is wrapped
with an irreducible coercion, we stamp the top-level coercion sequence.

\textbf{\textit{Function Application. }} The $\beta$ rule is standard for IFC languages.
It generates a \texttt{prot} term with the specific security label $\ell$ that
comes from the label on the $\lambda$, preventing implicit
flow from the function being applied through both the computation
result and the heap. The \textit{app-cast} rule applies a function
wrapped in a function coercion to a value $V$. The application is
``static'', so the security level of \texttt{prot} comes from the
function type just like $\beta$. The function coercion is distributed
into its domain coercion $\bm{c}$, its co-domain coercion $\bm{d}$,
and the coercion on PC $\bar{d}$. The coercion $\bar{c}$ is not used
because the function type is fully static, so its security is already
indicated by its type.  The domain coercion $\bm{c}$ casts the input
of the function $V$ to $W$ and $W$ is substituted into the body of the
$\lambda$. The substituted body goes through the co-domain cast
$\bm{d}$, and is then protected by $\ell$ using \texttt{prot}.  The
stamped PC casts to $\PC'$ by $\bar{d}$ and $\PC'$ is used as the PC
for \texttt{prot}.  The rule $\textit{app}{\star}\textit{-cast}$ is similar
to \textit{app-cast} except for two things: (1) the PC is stamped
and then injected using \textit{stamp!} to preserve types
(2) the security level of the function proxy used in protection is indicated
by $|\bar{c}|$ instead, because the top-level security label of the
function is statically unknown (\unk).

\textbf{\textit{If-conditional. }} The static rule \textit{if-true} is standard;
the \texttt{if} term reduces a \texttt{prot} whose security $\ell$
comes from the type of the branch condition, guarding against implicit
flow. The rationale of $\textit{if}{\star}\textit{-true-cast}$ follows that
of $\textit{app}{\star}\textit{-cast}$: (1) a \textit{stamp!} is generated to
stamp and then inject the PC and (2) the security of \texttt{prot} is retrieved
from the coercion in the branch condition.

\textbf{\textit{NSU and heap operations. }}
Let us consider reference creation first.
A ``static'' reference creation is secure because its typing (rule
${\vdash}\textit{-ref}$, Figure~\ref{fig:cc-typing-full} of the Appendix)
already enforces the heap policy. Consequently, the allocation can happen directly
(rule \textit{ref}) without any runtime checking. Rule \textit{ref?}
does the same reference creation but with NSU checking, by casting the
current PC to the security $\ell$ of the newly created memory
cell.
The coerce function (${-}\Rightarrow^{-}{-}$) takes two
security labels and a blame label to generate a coercion on labels.
In this case, $\unk \Rightarrow^{\bl{p}} \ell$ generates
$\id{\unk}\seq\proj{\ell}{p}$, which performs a projection whose
target is $\ell$. If the projected PC reduces to a blame, it means
that NSU checking fails so we lift the blame to \CC. Assignment
follows the same pattern: a static assignment can happen directly,
while \texttt{assign?} requires NSU checking, by stamping the current
PC with the security indicated in the coercion and then projecting to
$\hat{\ell}$, where $\hat{\ell}$ is the security of the heap cell.
The input coercion $\bm{c}$ is applied before the value is stored into
the cell.

The rule for static dereferencing \textit{deref} looks up index $n$ in
all memory cells with security level $\hat{\ell}$. The value from the
lookup is protected with $\ell$, the top-level security label of the
reference type. The PC of the \texttt{prot} does not matter, because
$V$ is already a value and will not reduce.
The rule $\textit{deref}{\star}\textit{-cast}$ dereferences a reference proxy.
It looks up the value $V$ in the heap, applies the output coercion $\bm{d}$,
and generates a \texttt{prot} with the security of the coercion $|\bar{c}|$.
The PC of \texttt{prot} does not matter in this case either, because
applying a coercion is pure and does not produce side effects.

\todo[inline]{Define a down arrow and an up arrow (diverge)}

\section{Compiling from \Surface to \CC}
\label{sec:compile}

\todo[inline]{Put compilation into another chapter}
